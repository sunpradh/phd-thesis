%----------------------------------------
% CHAPTER: INTRODUCTION TO LATTICE GAUGE THEORIES
%----------------------------------------
\chapter{Introduction to Lattice Gauge Theories}
\label{chap:introduction_to_lattice_gauge_theories}

One of the most important open questions in high-energy physics is \emph{confinement} in \ac{qcd},
or in general in a non-Abelian gauge theory.
%which is an example of non-Abelian gauge theory and it is believed to be the correct theory that describe the strong interaction.
The best evidence for confinement comes from the Wilson formulation of gauge theories on a lattice \cite{wilson1974confinement}, which, at first glance, can appear odd because the vacuum is not a crystal \cite{creutz1985book}.
Indeed, there have not been experimental proofs so far that show any deviations from the symmetries of the Lorentz group.

From the point of view of particle physics, the lattice represents a mathematical trick.
It provides a cutoff, which removes the ultraviolet infinities that often appear in \acp{qft}.
It is just a regulator and as such it must be removed after renormalization.
Physical results can only be extracted in the continuum limit, where the lattice spacing goes to zero.

But why do we need such a regulator?
Infinities has always been present in \acp{qft} since its conception.
Consider the case of \ac{qed}.
It had an immense success without ever using a discrete space-time, thanks to \emph{perturbation theory}.
The most conventional calculation schemes are based on Feynman expansions,
where a given observable is expressed as a power series in the interaction coupling.
The terms are computed until a divergence is met in a particular diagram.
Then, these divergences can be removed using some regularization scheme or methods provided by the Renormalization Group \cite{peskin1995qft}.

The reason why this methodology may fail in non-Abelian theories lies in the fact that some phenomena, like confinement, are inherently \emph{non-perturbative}.
Roughly speaking, perturbation theory relies on the fact that the true interacting theory is just a slight modification of the free theory.
In other words, it works only when the coupling constants are small.
In the case of \ac{qcd}, the free theory with vanishing coupling constant has no resemblance to the observed phenomenon.

In order to go beyond the diagrammatic approach of Feynman expansions, one needs a non-perturbative cutoff.
This is the main strength of the lattice, it eliminates all the wavelengths smaller than the lattice spacing before any kind of expansions is done.
Furthermore, on a lattice a field theory is \emph{mathematically well-defined}, in contrast with many standard formulations of \ac{qft}s (like the path-integral approach).

\Acp{lgt} are just a reformulation of \acp{qft} on a lattice, which exposes a close connection with \emph{\ac{sm}}.
For example, it can be showed that a path-integral in \ac{qft} is equivalent to a partition function in \ac{sm}.
Furthermore, it can be showed that the coupling constant in \ac{qft} corresponds directly to the temperature, and a strong coupling expansion becomes equivalent to a high temperature expansion.
Thus, a lattice formulation of \ac{qft}s allows a particle physicist to use the full technology of \ac{sm} and \acf{cm}.
Notice that this connection between \ac{qft} and \ac{sm} does not require a lattice, but it is only made more transparent by it.

While in particle physics the lattice is a useful trick, this is not true in \ac{cm}.
Indeed, in this field a lattice structure can emerge naturally, or rather it can even be required.
It is sufficient to think that many materials have a crystalline structure, hence the mathematical models describing these materials have to be formulated on a lattice.
Then, field theories can be used for probing the physics at scale lengths much larger that the lattice spacing.
Therefore, in \ac{cm} the roles are reversed: continuous fields are ``approximations'' of what really happens on a lattice.


\bigskip

In this section we first briefly review Yang-Mills theory, which is the generalization of \ac{qed} to any non-Abelian compact gauge group, like $\SU(N)$.
Then, we move onto the Wilson formulation of \acp{lgt} in the path-integral approach.

% SECTION: Review of Yang-Mills theory
%----------------------------------------
% SECTION: YANG MILLS THEORY
%----------------------------------------
\section{Review of Yang-Mills theory}
\label{sec:yang_mills_theory}

\todo{Ovviamente da riscrivere}.
A Yang-Mills theory is a gauge field theory on Minkowski space $\R^{1,d}$, where the gauge group $U(1)$ or $SU(N)$, with matter fields, which are defined by a representation of the gauge group.
For example, \emph{Quantum Chromodynamics} (QCD) is an $SU(3)$ gauge theory with Dirac spinors in the fundamental representation.
Keeping in mind the example of QCD, the Lagrangian of the theory is
\begin{equation}
    \mathcal{L} = - \frac{1}{2 g^2} \tr(F_{\mu \nu} F^{\mu \nu}) + \overline{\psi} \qty( i \gamma^{\mu} D_{\mu} - m ) \psi,
\end{equation}
where the fermions $\psi$ are taken in the fundamental representation of $SU(N)$ and the covariant derivative is $D_{\mu} = \partial_{\mu} - i A_{\mu}$.
We choose the convention where the Lie algebra generators $T^a$ are Hermitian and $[T^a, T^b] = i f^{abc} T^c$, with real structure constants $f^{abc}$.
Furthermore, the generators are such that $\tr(T^a T^b) = \frac{1}{2} \delta^{ab}$.
The strength-field tensor $F_{\mu \nu}$ is given by
\begin{equation}
    F_{\mu \nu} = \partial_{\mu} A_{\nu} - \partial_{\nu} A_{\mu} - i [A_{\mu}, A_{\nu}]
\end{equation}
and transforms in the adjoint representation of $SU(N)$.
Both the gauge field $A_{\mu}$ and the curvature tensor $F_{\mu \nu}$ live in the Lie algebra $\mathfrak{su}(N)$.

Under a gauge transformation given by a group-valued function $g(x) \in SU(N)$, such that $\psi \mapsto g(x) \psi(x)$, then
\begin{equation}
    A_{\mu}(x) \mapsto g(x) A_{\mu}(x) g(x)^{-1} + i g(x) \partial_{\mu} g(x)^{-1},
\end{equation}
so that $D_{\mu} \psi(x) \mapsto g(x) D_{\mu} \psi(x)$, while
\begin{equation}
    F_{\mu \nu} \mapsto g(x) F_{\mu \nu} g(x)^{-1},
\end{equation}
leaving the action invariant.

The action of the theory in $d+1$ dimensions is the given by
\begin{equation}
    S[A, \psi, \overline{\psi}] = \int \dd^{d+1} \mathcal{L}
\end{equation}
and the path integral
\begin{equation}
    Z = \int \mathcal{D} A \mathcal{D} \overline{\psi} \mathcal{D} \psi \, e^{i S[A, \psi, \overline{\psi}]}.
\end{equation}


%----------------------------------------
% SUBSECTION: EUCLIDEAN FIELD THEORY
%----------------------------------------
\subsection{Euclidean field theory}
\label{sub:euclidean_field_theory}

In order to work in a Euclidean space-time, we need first to perform a \emph{Wick rotation}, where the time coordinate $x_0$ is mapped a forth space coordinate $x_4$, through $x_0 = - i x_4$.
The has the effect of changing the path-integral integrand from $e^{i S}$, which is oscillatory, to $e^{-S}$, which is positive and can be interpreted as a probability distribution of the configurations of the fields.

The Euclidean path integral is
\begin{equation}
    Z_E = \int \mathcal{D} A \mathcal{D} \overline{\psi} \mathcal{D} \psi e^{-S},
\end{equation}
so that the Minkowski action and the Euclidean action satisfy $i S_M = - S_E$.
The respective actions are given by
\begin{equation}
    S_M = \int \dd^{d+1} x_M \mathcal{L}_M, \qquad
    S_E = \int \dd^{d+1} x_E \mathcal{L}_E.
\end{equation}
The rotation $x_0 = -i x_4$ leads to $\mathcal{L}_E = - \mathcal{L}_M$.

The Wick rotation does not change the form of the gauge kinetic term, i.e.,
\begin{equation}
    - \frac{1}{2 g^2} \tr(F_{\mu \nu} F^{\mu \nu}).
\end{equation}
The sum is a simple Euclidean sum, where there are no minus signs when raising or lowering indices and $\mu = 1, \dots, d+1$.

Considering now the fermionic part of the Yang-Mills Lagrangian, we need to perform the Wick rotation on the Dirac operator.
In Minkowski space-time
\begin{equation}
    \overline{\psi} \qty( i \gamma^{\mu}_M D_{\mu} - m ) \psi
    = \overline{\psi} \qty( i \gamma^{\mu}_M \partial_{\mu} + \gamma^{\mu}_M A_{\mu} - m ) \psi,
\end{equation}
where $\gamma^{\mu}_M$ are the gamma matrices of the Clifford algebra, and they satisfy $\{\gamma^{\mu}_M, \gamma^{\nu}_M\} = 2 \eta^{\mu \nu}$.
In the Euclidean Clifford algebra instead the gamma matrices $\gamma^{\mu}_E$ satisfy $\{\gamma^{\mu}_E, \gamma^{\nu}_E\} = 2 \delta^{\mu \nu}$.
Given the fact that we have $\partial_0 = i \partial_4$ and $A_0 = i A_4$, in order to obtain the correct form we have to put $\gamma^0_M = \gamma^4_E$.
This procedure yields
\begin{equation}
    \overline{\psi} \qty( i \gamma^{\mu}_M \partial_{\mu} + \gamma^{\mu}_M A_{\mu} - m ) \psi
    =
    - \overline{\psi} \qty( \gamma^{\mu}_E \partial_{\mu} + i\gamma^{\mu}_E A_{\mu} + m ) \psi.
\end{equation}

Since $\mathcal{L}_E = - \mathcal{L}_M$, we finally arrive at
\begin{equation}
    \mathcal{L}_E
    = \frac{1}{2 g^2} \tr (F_{\mu \nu} F^{\mu \nu}) + \overline{\psi} \qty( \gamma^{\mu} D_{\mu} ) \psi,
\end{equation}
where the indices are all Euclidean and $D_{\mu} = \partial_{\mu} + i A_{\mu}$.



%----------------------------------------
% SUBSECTION: HAMILTONIAN FORMULATION
%----------------------------------------
\subsection{Hamiltonian formulation}
\label{sub:hamiltonian_formulation}

The Hamiltonian formulation of a Yang-Mills theory can be tricky, especially the part about the gauge field.
Usually, one has to procede by computing the conjugate momenta and performing a Legendre transform.
The main issue here is that the gauge field component $A_0$, does not have a conjugate momentum:
\begin{equation}
    \pdv{\mathcal{L}}{\dot{A}_0} = 0.
\end{equation}
Hence, the transformation is not invertible.
The easiest way to remedy to the situation is to impose the gauge condition $A_0 = 0$,
which is called \emph{canonical gauge} or \emph{temporal gauge}.
With this condition, the kinetic term for the gauge fields can be written as
\begin{equation}
    \mathcal{L}
    = - \frac{1}{2 g^2} \tr(F_{\mu \nu} F^{\mu \nu})
    = \frac{1}{g^2} \qty( \mathbf{E}^2 - \mathbf{B}^2 )
    = \frac{1}{g^2} \qty(E^a_i E^a_i - B^a_i B^a_i),
    \label{eq:YM_lagrang_temporal_gauge}
\end{equation}
where $\mathbf{E}$ and $\mathbf{B}$ are, respectively, the ``chromoelectric'' and the ``chromomagnetic'' fields.
In the temporal gauge $\mathbf{E} = \dot{\mathbf{A}}$, the time derivative of the spatial components $\mathbf{A}$ of the gauge field $A_{\mu}$, while $\mathbf{B}$ corresponds to the spatial components of the strength-field tensor $F^{\mu \nu}$ and does not involve any time derivative.
From the Legendre transformation of \eqref{eq:YM_lagrang_temporal_gauge} we obtain the Hamiltonian density:
\begin{equation}
    \mathcal{H}
    = \frac{1}{g^2} E^a_i \dot{A}^a_i - \frac{1}{2 g^2} \qty( E^a_i E^a_i - B^a_i B^a_i )
    = \frac{1}{2g^2} \tr ( \mathbf{E}^2 + \mathbf{B}^2 ),
\end{equation}
hence the Hamiltonian in $d$ spatial dimensions is
\begin{equation}
    H = \int \dd^{d} x \frac{1}{2 g^2} \tr(\mathbf{E}^2 + \mathbf{B}^2)
\end{equation}



% SECTION: Path integral approach to lattice field theory
%----------------------------------------
% SECTION: Quantum simulation
%----------------------------------------
\section{Lattice Field Theory}
\label{sec:lattice_field_theory}

Starting from the path integral formulation, the first step in the formulation of a \emph{lattice field theory} (LFT) is the discretization of space-time, where a discrete $d+1$-dimensional lattice substitutes the continuum space-time.
The simplest choice in this regard is a hypercubic lattice with lattice spacing $a$, but in theory an LFT can be defined on any type of lattice.
An immediate advantage of using a lattice instead of a continuum is the natural ultraviolet cutoff given by the inverse of the lattice spacing.

Formally a lattice $\Lambda$ is defined as
\begin{equation}
    \Gamma = \qty{
        x \in \R^4 :
        x = \sum_{\mu = 1}^{d+1} a n_{\mu} \hat{\mu} \quad
        n_{\mu} \in \Z
    },
\end{equation}
where $\mu = 1, \dots, d+1$ and $\hat{\mu}$ is the unit vector in the direction $\mu$.
The edges will be labeled by a pair $(x,\hat{\mu})$, meaning that we are referring to the edge in the $\hat{\mu}$ direction from the vertex $x$.
It is important to fix an orientation for each direction in the lattice.
The most natural choice is to choose, obviously, $+ \hat{\mu}$ for each $\hat{\mu}$.
So, even though $(x,\hat{\mu})$ and $(x + \hat{\mu}, - \hat{\mu})$ refers to the same link, the former is traversed in the positive direction while the latter in the negative direction.

In an LFT, both the vertices and edges (also called links) host degrees of freedom (d.o.f).
In particular, the matter fields lives on the vertices while the gauge fields live on the links between vertices.
However, the definition of these d.o.f.~will need some care, because we have two main requirements:
\begin{itemize}
    \item The lattice action should reduce to the continuum action in the continuum limit, i.e., $a \to 0$;
    \item The lattice action should respect the gauge symmetry.
\end{itemize}
Lorentz invariance is naturally broken on a lattice but we expect to recover it in the continuum limit.


\subsection{Gauge fields on a lattice}
\label{sub:gauge_fields_on_a_lattice}


\subsection{Fermions on a lattice}
\label{sub:fermions_on_a_lattice}

