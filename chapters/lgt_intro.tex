%----------------------------------------
% CHAPTER: INTRODUCTION TO LATTICE GAUGE THEORIES
%----------------------------------------
\chapter{Introduction to Lattice Gauge Theories}
\label{chap:introduction_to_lattice_gauge_theories}

One of the most important open questions regarding high-energy physics is confinement in Quantum Chromodynamics (QCD), or in general in a non-Abelian gauge theory.
The best evidence for confinement comes from the Wilson formulation of gauge theories on a lattice \cite{wilson1974confinement}, which, at first glance, can appear odd because the vacuum is not a crystal.
Indeed, there have not been experimental proofs so far that show any deviations for the symmetries of the Lorentz group.

From the point of view of particle physics, the lattice represents a mathematical trick.
It provides a cutoff, which removes the ultraviolet infinities that infest quantum field theory (QFT).
It is just a regulator and as such it must be removed after renormalization.
Physical results can only be extracted in the continuum limit, where the lattice spacing goes to zero.

But why do we need such a regulator?
Infinities has always been present in quantum field theories since its conception.
Consider the case of Quantum Electrodynamics.
It had an immense success without ever using a discrete space-time, thanks to \emph{perturbation theory}.
The most conventional calculation schemes are based on Feynman expansions,
where a given observable is expressed as a power series in the interaction coupling.
The terms are computed until a divergence is met in a particular diagram.
These divergences can then be removed with some regularization method.

The reason why this methodology does not work in non-Abelian theories lies in the fact that some phenomena, like confinement, are inherently \emph{non-perturbative}.
Roughly speaking, perturbation theory relies on the fact that the true interacting theory is just a slight modification of the free theory.
In other words, it works only when the coupling constants are small.
In the case of QCD, the free theory with vanishing coupling constant has no resemblance to the observed phenomenon.

In order to go beyond the diagrammatic approach of Feynman expansions, one needs a non-perturbative cutoff.
This is the main strength of the lattice, it eliminates all the wavelengths smaller than the lattice spacing before any kind of expansions is done.
Furthermore, on a lattice a field theory is \emph{mathematically well-defined}, in contrast with many standard formulations of QFTs (like the path-integral approach).

A lattice formulation of QFTs exposes a close connection with \emph{statistical mechanics} (SM), especially in Euclidean space-time.
In fact, it can be showed that a path-integral in QFT is equivalent to a partition function in SM.
The square of the coupling constant in QFT corresponds directly to the temperature, and a strong coupling expansion becomes equivalent to a high temperature expansion.
Thus, with a lattice formulation of QFTs allows a particle physicist to use the full technology of SM and condensed matter theory.

\todo{revisionare ciò che è stato scritto}

\medskip

In this section we first briefly review Yang-Mills theory, which is main formalism for QFTs with gauge symmetries.
Then, we move onto the Wilson formulation of lattice gauge theories in the path-integral approach.

\todo{c'è altro da aggiungere?}

% SECTION: Review of Yang-Mills theory
%----------------------------------------
% SECTION: YANG MILLS THEORY
%----------------------------------------
\section{Review of Yang-Mills theory}
\label{sec:yang_mills_theory}

A \ac{ym} theory is a gauge field theory on Minkowski space $\R^{1,d}$ coupled to matter.
    The gauge group $G$ is usually chosen to be a compact Lie group like $\U(1)$ or $\SU(N)$ and the matter fields are defined by a representation of $G$.
    For example, \ac{qcd} is an $SU(3)$ gauge theory with Dirac spinors in the fundamental representation.
We choose to keep the dimension $d$ of space completely general and to use $D$ to denote the full dimension of space-time, i.e., $D=d+1$.

We start from the Lagrangian.
Considering that \ac{ym} theory can be seen as the generalization of \ac{qed}, a $\U(1)$ gauge theory, to any compact Lie group, the Lagrangian looks exactly like the one from \ac{qed}:
\begin{equation}
    \LagrangYM = - \frac{1}{2 g^2} \tr(F_{\mu \nu} F^{\mu \nu}) + \overline{\psi} \qty( i \gamma^{\mu} D_{\mu} - m ) \psi,
    \label{eq:yang_mills_lagrangian}
\end{equation}
with some differences that will be explained later.
Hereafter, the Einstein summation rule is implied.
Notice that in \eqref{eq:yang_mills_lagrangian} we have only considered one fermionic species.
In more realistic cases we would have a some over the different fermion flavors but for simplicity and ease of exposition we will ignore flavors and consider only type of fermion.




\paragraph*{Gauge fields}

The symbol $D_{\mu}$ in \eqref{eq:yang_mills_lagrangian} denote the \emph{covariant derivative}:
\begin{equation}
    D_{\mu} = \partial_{\mu} - i q A_{\mu},
    \label{eq:covariant_derivative}
\end{equation}
where $A_{\mu}$ are the space-time components of the gauge fields.
Each component is Lie algebra valued function of space-time:
\begin{equation}
    A_{\mu}(x) = \sum_{a} A_{\mu}^a(x) T^a,
\end{equation}
where the sum is over the generators $T^a$ of the Lie algebra $\g$, corresponding to the group $G$.
In the following, we will use Greek indices for space-time coordinates, and Latin indices for the algebra structure.
We choose the convention where the generators $T^a$ are Hermitian with
\begin{equation}
    [T^a, T^b] = i f^{abc} T^c
    \qand
    \tr(T^a T^b) = \frac{1}{2} \delta^{ab},
\end{equation}
with real structure constants $f^{abc}$.

The dynamics of the gauge fields is given by $F_{\mu \nu}$, which is the \emph{strength-field tensor} and defined as
\begin{equation}
    F_{\mu \nu} = \partial_{\mu} A_{\nu} - \partial_{\nu} A_{\mu} - i [A_{\mu}, A_{\nu}].
    \label{eq:strength_field_tensor}
\end{equation}
and transforms in the adjoint representation of $SU(N)$.
Notice that when $G = U(1)$, i.e.~Abelian, the commutator term in \eqref{eq:strength_field_tensor} vanishes and we re-obtain the strength-field tensor of \ac{qed}.
Like the gauge field $A_{\mu}$, also the tensor $F_{\mu \nu}$ lives in the Lie algebra $\g$.
Therefore, the product $F^{\mu \nu} F_{\mu \nu}$ is actually a matrix.
Only scalar terms are allowed in the Lagrangian \eqref{eq:yang_mills_lagrangian}, hence we have to take the trace:
\begin{equation*}
    \tr F^{\mu \nu} F_{\mu \nu} = \sum_{a} F^{\mu \nu a} F_{\mu \nu}^a.
\end{equation*}



\paragraph*{Fermions}

Given the fact that we will not look at dynamical matter fields in this manuscript, we only give a very brief introduction to fermions.
The fermion field $\psi(x)$ lives in the \emph{fundamental representation} of $G$.
The Lagrangian of a free fermion field $\psi$ is
\begin{equation}
    \Lagrangian_{\psi} = \overline{\psi} \qty( \partial_{\mu} \gamma^{\mu} - m ).
\end{equation}
The matrices $\gamma^{\mu}$ form a \emph{Clifford algebra}, which is defined by the relations
\begin{equation}
    \acomm{\gamma^{\mu}}{\gamma^{\nu}} = 2 \eta^{\mu \nu},
\end{equation}
where $\eta^{\mu \nu}$ is the space-time metric.
For the latter we choose the convention $\eta^{\mu \nu} = \mathrm{diag}(+1, -1, -1, -1)$.
The conjugate field $\overline{\psi}$ is defined as
\begin{equation*}
    \overline{\psi} = \psi^{\dagger}\gamma^0.
\end{equation*}
The interaction with the gauge field can be obtained with a simple minimal coupling, where the derivative $\partial_{\mu}$ is substituted with the covariant derivative $D_{\mu}$ in \eqref{eq:covariant_derivative}.




\paragraph*{Gauge transformations}

A gauge transformation is defined by a group valued function of space-time $g: \R^{1,d} \to G$.
It transforms the fermion field as
\begin{equation}
    \psi(x) \mapsto g(x) \psi(x).
    \label{eq:fermion_gauge_transf}
\end{equation}
Here we have been a bit sloppy with notation, by writing $g(x) \psi(x)$ in \eqref{eq:fermion_gauge_transf} we actually mean the action of the element $g(x) \in G$ in the same representation of $\psi(x)$.

% Under a gauge transformation given by a group-valued function $g(x) \in SU(N)$, such that $\psi \mapsto g(x) \psi(x)$, then
In order to have an invariant Lagrangian, the gauge fields $A_{\mu}$ have to undergo a transformation induced by the function $g$:
\begin{equation}
    A_{\mu}(x) \mapsto g(x) A_{\mu}(x) g(x)^{-1} + i g(x) \partial_{\mu} g(x)^{-1},
\end{equation}
so that $D_{\mu} \psi(x) \mapsto g(x) D_{\mu} \psi(x)$, while
\begin{equation}
    F_{\mu \nu} \mapsto g(x) F_{\mu \nu} g(x)^{-1}.
\end{equation}


\paragraph*{Path-integral}

For a path integral formulation we need to first define the action.
This is just the integral of the Lagrangian in \eqref{eq:yang_mills_lagrangian}, over the $D$-dimensional space-time:
% The action of the theory in $D=d+1$ dimensions is the given by
\begin{equation}
    \Action[A, \psi, \overline{\psi}] = \int_{\R^{1,d}}  \dd^{D} x \Lagrangian.
    \label{eq:action_def}
\end{equation}
The action \eqref{eq:action_def} defines the ``weight'' in the path-integral.
Now we can write down the partition function for a \ac{ym} theory:
\begin{equation}
    \PartFunc = \int \DD A\, \DD \overline{\psi}\, \DD \psi \, e^{i \Action[A, \psi, \overline{\psi}]}.
    \label{eq:partition_function_mink}
\end{equation}
The measures in the path-integral can be physically interpreted as a measure over all the possible configurations of the fields $A_{\mu}$, $\overline{\psi}$ and $\psi$, but are still lacking a rigorous mathematical definition \cite{peskin1995qft}.
We will not elaborate further on this topic.


%----------------------------------------
% SUBSECTION: EUCLIDEAN FIELD THEORY
%----------------------------------------
\subsection{Euclidean field theory}
\label{sub:euclidean_field_theory}

In the previous section we introduced \ac{ym} theory in Minkowski space-time.
We now need to move onto Euclidean space-time, which is the starting point for \ac{lgt}.
This is true for various reasons.
As stated previously, Euclidean formulation allows to bridge into the territory of \ac{sm}, meaning we can use its full technology.
Second, there are also some advantages from the operational point of view.
The weight in \eqref{eq:partition_function_mink} is a complex phase, which can be problematic from the computational point of view because a priori convergence is not guaranteed.
We will see that in Euclidean space-time the weight will become a positive-defined function, which makes it clear it is a probability distribution.

In order to pass to a Euclidean space-time $\R^{d+1}$, we need to perform a \emph{Wick rotation},
where the time coordinate $x_0$ is mapped to a forth space coordinate $x_4$:
\begin{equation*}
    x_0 \mapsto - i x_4.
    \label{eq:wick_rotation}
\end{equation*}
To distinguish quantities in Euclidean or Minkowski space-time we use the subscripts $E$ and $M$, respectively.
The rotation \eqref{eq:wick_rotation} affects both space-time measures,
\begin{equation*}
    \dd^D x_M = \dd x_0 \dd^d x_i \qand
    \dd^D x_E = \dd^d x_i \dd x_4,
\end{equation*}
and the time-components of the quantities that enters the Lagrangian, which leads to an overall effect
\begin{equation*}
    \Lagrangian_E = - \Lagrangian_M.
\end{equation*}
The action is defined in the same way in both types of space-time,
\begin{equation*}
    \Action_M = \int \dd^D x_M \Lagrangian_M
    \qand
    \Action_E = \int \dd^D x_E \Lagrangian_E,
\end{equation*}
and due to the Wick rotation \eqref{eq:wick_rotation}, we obtain that they satisfy
\begin{equation}
    i \Action_M = - \Action_E.
\end{equation}
This leads to the following definition of the Euclidean path-integral:
\begin{equation}
    \PartFunc_E = \int \DD A\, \DD \overline{\psi}\, \DD \psi\, e^{-\Action_E[A, \overline{\psi}, \psi]}.
    \label{eq:euclidean_path_integral}
\end{equation}
Notice that the weight $e^{-\Action_E}$ in \eqref{eq:euclidean_path_integral} is now a positive-valued function, given that $\Action_E$ is a real function, and has the form of a \emph{Boltzmann weight}.
In other words, following the spirit of \ac{sm}, we have now a probability distribution $e^{- \Action_E}$ over the configurations of the fields $A_{\mu}$, $\psi$ and $\overline{\psi}$.

% It has the effect of changing the integrand in \eqref{eq:partition_function_mink} from $e^{i S}$, which is oscillatory, to $e^{-S}$, which is positive and can be interpreted as a probability distribution of the configurations of the fields.

% The Euclidean path integral is
% \begin{equation}
%     \PartFunc_E = \int \mathcal{D} A \mathcal{D} \overline{\psi} \mathcal{D} \psi e^{-S},
% \end{equation}
% so that the Minkowski action and the Euclidean action satisfy $i S_M = - S_E$.
% The respective actions are given by
% \begin{equation}
%     \Action_M = \int \dd^{d+1} x_M \Lagrangian_M, \qquad
%     \Action_E = \int \dd^{d+1} x_E \Lagrangian_E.
% \end{equation}
% The rotation $x_0 = -i x_4$ leads to $\Lagrangian_E = - \Lagrangian_M$.
%
Let's describe this procedure in more details.
The Wick rotation does not change the aspect of the gauge kinetic term,
\begin{equation}
    - \frac{1}{2 g^2} \tr(F_{\mu \nu} F^{\mu \nu}),
\end{equation}
but the sum is now a simple Euclidean sum, where there are minus signs appearing when raising or lowering indices, and $\mu = 1, \dots, d+1$.

Considering now the fermionic part of the \ac{ym} Lagrangian, we need to perform the Wick rotation on the Dirac fields $\psi$ and $\overline{\psi}$.
In Minkowski space-time
\begin{equation}
    \overline{\psi} \qty( i \gamma^{\mu}_M D_{\mu} - m ) \psi
    = \overline{\psi} \qty( i \gamma^{\mu}_M \partial_{\mu} + \gamma^{\mu}_M A_{\mu} - m ) \psi,
\end{equation}
where $\gamma^{\mu}_M$ denotes the gamma matrices in Minkowski space-time:
\begin{equation*}
    \{\gamma^{\mu}_M, \gamma^{\nu}_M\} = 2 \eta^{\mu \nu}.
\end{equation*}
The Euclidean Clifford algebra instead uses gamma matrices $\gamma^{\mu}_E$ that instead satisfy
\begin{equation*}
    \{\gamma^{\mu}_E, \gamma^{\nu}_E\} = 2 \delta^{\mu \nu}.
\end{equation*}
Given that we have $\partial_0 = i \partial_4$ and $A_0 = i A_4$, the correct form can only be achieved by putting $\gamma^0_M = \gamma^4_E$.
This procedure yields
\begin{equation}
    \overline{\psi} \qty( i \gamma^{\mu}_M \partial_{\mu} + \gamma^{\mu}_M A_{\mu} - m ) \psi
    =
    - \overline{\psi} \qty( \gamma^{\mu}_E \partial_{\mu} + i\gamma^{\mu}_E A_{\mu} + m ) \psi.
\end{equation}
Since $\Lagrangian_E = - \Lagrangian_M$, we can conclude
\begin{equation}
    \Lagrangian_E
    = \frac{1}{2 g^2} \tr (F_{\mu \nu} F^{\mu \nu}) + \overline{\psi} \qty( \gamma^{\mu} D_{\mu} + m) \psi,
\end{equation}
where the indices are all Euclidean and $D_{\mu} = \partial_{\mu} + i A_{\mu}$.



%----------------------------------------
% SUBSECTION: HAMILTONIAN FORMULATION
%----------------------------------------
\subsection{Hamiltonian formulation}
\label{sub:hamiltonian_formulation}

Even though Wilson's formulation \cite{wilson1974confinement} is in the path-integral and Lagrangian language, we will also review the Hamiltonian formulation of non-Abelian \ac{qft} because its connection to Quantum Simulation.
Expressing a \ac{ym} theory in the Hamiltonian language can be tricky, especially in the presence of gauge symmetries.
Usually, one has to procede by computing the conjugate momenta and perform a Legendre transform in order to obtain the Hamiltonian.
In the presence of gauge symmetries, the time-component $A_0$ of the gauge fields does \emph{not} have a conjugate momentum.
Instead it leads to a \emph{constraint}:
% The main issue here is that the gauge field component $A_0$, does not have a conjugate momentum:
\begin{equation}
    \pdv{\Lagrangian}{\dot{A}_0} = 0,
\end{equation}
which means that the Legendre transform is not invertible.

The easiest remedy is to \emph{fix the gauge} beforehand, by imposing $A_0 = 0$, which is called \emph{canonical} or \emph{temporal gauge}.
With this condition, the gauge fields Lagrangian can be written as
\begin{equation}
    - \frac{1}{2 g^2} \tr(F_{\mu \nu} F^{\mu \nu})
    = \frac{1}{g^2} \qty( \mathbf{E}^2 - \mathbf{B}^2 )
    = \frac{1}{g^2} \qty(E^a_i E^a_i - B^a_i B^a_i),
    \label{eq:YM_lagrang_temporal_gauge}
\end{equation}
where $\mathbf{E}$ and $\mathbf{B}$ are, respectively, the corresponding electric and magnetic fields for a non-Abelian theory.
In the temporal gauge we only have the spatial components $\mathbf{A}$ of the gauge field $A_{\mu}$.
The electric field $\mathbf{E}$ is the time derivative of $\mathbf{A}$, i.e.~$\mathbf{E} = \dv{\mathbf{A}}{t}$,
which means that $\mathbf{E}$ is the conjugate momentum to $\mathbf{A}$.
Meanwhile, the magnetic field $\mathbf{B}$ can be obtained from the spatial components of the strength-field tensor $F^{\mu \nu}$, with $B_i = - \frac{1}{2} \varepsilon_{ijk} F^{jk}$, where $\varepsilon_{ijk}$ is the Levi-Civita symbol.
Once the gauge is fixed, the Hamiltonian can be finally be obtained with a Legendre transform:
% From the Legendre transformation of \eqref{eq:YM_lagrang_temporal_gauge} we obtain the Hamiltonian density:
\begin{equation}
    \mathcal{H}
    = \frac{1}{g^2} E^a_i \dot{A}^a_i - \frac{1}{2 g^2} \qty( E^a_i E^a_i - B^a_i B^a_i )
    = \frac{1}{2g^2} \tr ( \mathbf{E}^2 + \mathbf{B}^2 ).
\end{equation}

In the Hamiltonian formulation, the fields $\mathbf{A}$ and $\mathbf{E}$ have to be elevated to operators, by imposing the following commutation relations:
\begin{equation}
    \begin{split}
        \comm{A^a_i(x)}{E^b_j(y)} & = i g^2 \delta_{ij} \delta_{ab} \delta(x-y) \\
        \comm{E^a_i(x)}{E^b_j(y)} & = \comm{A^a_i(x)}{A^b_j(y)} = 0.
    \end{split}
    \label{eq:comm_rel_E_A_continuum}
\end{equation}
A careful reader will notice that \eqref{eq:comm_rel_E_A_continuum} are completely analogous to a position-momentum commutation relation, similar to $\comm{x_i}{p_j} = i \delta_{ij}$.
In fact, like in the latter case, where the momentum $p_i$ is the generator of translations of $x_i$, the electric field $\mathbf{E}$ is the generator of translation of $\mathbf{A}$.
To be more precise, it is the canonical momentum $\mathbf{E}/g^2$ that generates translations of $\mathbf{A}$.
In other words, $\mathbf{E}/g^2$ \emph{generates infinitesimal gauge transformations}.
This point of view will be rather useful when treating the gauge fields on a lattice.


In order to impose the canonical gauge $A_0 = 0$, the equation of motion for $A_0$ has to be satisfied:
\begin{equation}
    \partial_{\mu} \qty( \pdv{\Lagrangian}{(\partial_{\mu} A_0)} ) - \pdv{\Lagrangian}{A_0} = 0.
\end{equation}
In the absence of sources, this leads to
\begin{equation}
    D_i E_i = 0,
    \label{eq:non_abelian_gauss_law}
\end{equation}
where $D_i$ and $E_i$ are the spatial components of the covariant derivative and electric field, respectively.
What we obtained is basically the generalization of \emph{Gauss law} to non-Abelian theories.
In fact, the condition \eqref{eq:non_abelian_gauss_law} for a $U(1)$ theory reduces to the well known $\nabla \cdot \mathbf{E} = 0$.
Unfortunately, the equation \eqref{eq:non_abelian_gauss_law} is inconsistent with the commutation relations \eqref{eq:comm_rel_E_A_continuum}, so it cannot be implemented as an operator equation.
The easiest solution, or loophole, to this empasse is to impose to consider \emph{physical} or \emph{gauge-invariant} only states that satisfy
\begin{equation}
    D_i E_i \ket{\psi_{\phys}} = 0.
\end{equation}
This constraint select a subspace of the overall Hilbert space $\HilbertSpace$, which will be labeled as the \emph{physical Hilbert space} $\Hphys$.


%----------------------------------------
% SUBSECTION: The Sign Problem
%----------------------------------------
\subsection{The Sign Problem}
\label{sub:the_sign_problem}

A number of interesting phases have been predicted for \ac{qcd} in the $\mu - T$ plane \cite{aarts2016qcd}, where $\mu$ is the chemical potential and $T$ the temperature, such as \emph{quark-gluon plasma} \cite{detar2009qcdthermo} or \emph{color superconductivity} \cite{alford2001coloursc} (see Fig.~\ref{fig:qcd_phase_diagram}).
Unfortunately, detailed quantitative analysis of \ac{qcd} has been limited to the $\mu = 0$ region only \cite{aarts2016qcd}.
This is mainly due to the difficulty of studying \ac{qcd} in the low energy regime, where the perturbative approach fails \cite{peskin1995qft, creutz1985book}.
Moreover, even \ac{lgt}s, at least in the path-integral formulation, is not applicable for $\mu \neq 0$ due to the infamous \emph{sign problem}.


\begin{figure}[t]
    \centering
    \begin{tikzpicture}[
    asse/.style = {-stealth, very thick},
    transizio/.style = {-Circle, very thick},
    align=center
    ]
    % Axis
    \draw[asse] (0, 0) -- (7, 0) node [below left, font=\small] {chemical potential};
    \draw[asse] (0, 0) -- (0, 5) node [above left, rotate=90, font=\small] {temperature};
    % transition lines
    % \draw[transizio] (1.5, 0) arc (0:40:1.5);
    \draw[transizio] (3.5, 0) arc (0:60:3)
        node[above] {critical point};
    % etichette
    \node[above] at (0.75, 0) {vacuum};
    \node[above] at (2.25, 0) {nuclear\\ matter};
    \node[above right] at (3.5, 0) {neutron\\ stars};
    \node at (6.5, 1) {color\\ superconductor?};
    \node at (4, 4) {quark-gluon\\ plasma};
    \node[rotate=90] at (0.5, 3.5) {early universe};
\end{tikzpicture}

    \caption{A (very) rough sketch of the phase diagram of \ac{qcd} \cite{aarts2016qcd}.}
    \label{fig:qcd_phase_diagram}
\end{figure}


In the Hamiltonian formulation, the chemical potential in introduced in the same manner as standard \ac{sm}.
If $H$ is the Hamiltonian density operator and $N$ a fermion density operator, then one can simply replace $H$ with $H - \mu N$.
In the case of a \ac{ym} theory, the fermion density operator would correspond to the fermion number operator $N = \psi^{\dagger} \psi$.

In the path integral formalism, fermions are introduced as Grassmann variables.
This means that they can easily be integrated out \cite{peskin1995qft, aarts2016qcd}:
\begin{equation}
    \int \DD \overline{\psi}\, \DD \psi \exp(-\int d^{D} \overline{\psi} K \psi) = \det K,
\end{equation}
where $K$ is the kinetic operator for the fermions.
If the fermions are coupled to $A_{\mu}$, as it happens in \ac{ym} theory, then $K$ has some complicated dependence on the fields $A_{\mu}$.
If one includes the chemical potential term $\mu \psi^{\dagger} \psi$ in the Lagrangian, then the fermion determinant $\det K$ turns out to be complex \cite{aarts2016qcd}, with a non-trivial phase factor.

As a result, the integrand of the path-integral is no longer positive-definite, and it cannot be interpreted as a probability distribution.
Furthermore, a complex weight in the path-integral makes the integrand oscillatory, which does not help with convergence.
This is summarizes the so-called \emph{sign problem}, which poses severe limitation to, for example, \ac{mc} simulations in the finite $\mu$ region.


% SECTION: Path integral approach to lattice field theory
%----------------------------------------
% SECTION: Wilson approach to gauge theories
%----------------------------------------
\section{Wilson approach to \acl{lgt}}
\label{sec:wilson_approach_to_lft}

Starting from the path integral formulation, the first step in the formulation of a \ac{lgt} is the discretization of space-time, where a discrete $d+1$-dimensional lattice substitutes the continuum space-time.
The simplest choice in this regard is a hypercubic lattice with lattice spacing $a$, but in theory an \ac{lgt} can be defined on any type of lattice.
An immediate advantage of using a lattice instead of a continuum is the natural ultraviolet cutoff given by the inverse of the lattice spacing.

Formally, a lattice $\Lattice$ is defined as
\begin{equation}
    \Lattice = \qty{
        x \in \R^{D} :
        x = \sum_{\mu = 1}^{D} a n_{\mu} \hat{\mu} \quad
        n_{\mu} \in \Z
    },
\end{equation}
where $\mu = 1, \dots, D$, and $\hat{\mu}$ is the unit vector in the $\hat{\mu}$-th direction.
The edges, or \emph{links}, will be labeled by a pair $(x,\hat{\mu})$, meaning that we are referring to the link in the $\hat{\mu}$ direction from the vertex, or \emph{site}, $x$.
It is important to fix an orientation for each direction in the lattice.
The most natural choice is to choose $+ \hat{\mu}$ for each $\hat{\mu}$.
So, even though $(x,\hat{\mu})$ and $(x + \hat{\mu}, - \hat{\mu})$ refers to the same link, the former is traversed in the positive direction while the latter in the negative direction.

In an \ac{lgt}, both the sites and links host \ac{dof}.
In particular, the matter fields lives on the sites while the gauge fields live on the links.
The definition of these \ac{dof} will need some care, because we have two main requirements, especially if we are interested in \ac{ym} theory:
\begin{itemize}
    \item The lattice action should reduce to the continuum action in the continuum limit, i.e., $a \to 0$;
    \item The lattice action should respect the gauge symmetry.
\end{itemize}
Lorentz invariance is naturally broken on a lattice but we expect to recover it in the continuum limit.


%
% SUBSECTION: Gauge fields on a lattice
%
\subsection{Gauge fields on a lattice}
\label{sub:gauge_fields_on_a_lattice}

The simplest way to define a \ac{ym} action on a lattice would be to consider a continuum action, substitute finite-difference approximations for derivatives, and replace the space-time integral by a sum over the lattice sites.
However, the result of this is an action which is not-gauge invariant for non-zero lattice spacing \cite{wilson1974confinement}.
This is likely to mean that the theory would still lack gauge-invariance in the $a \to 0$ limit.
The alternative, outlined in \cite{wilson1974confinement, creutz1985book}, would be to formulate gauge invariance for a lattice theory then modify its action until it is gauge invariant for any $a$.

We start by considering a general group $G$.
We associate an element $U_{\mu}(x) \in G$ to each link $(x, \mu)$.
If one traverse the link in the opposite direction, one should consider the inverse element $U^{-1}$.
In the case of $\SU(N)$, we take $U_{\mu}(x)$ to be the matrices in the fundamental representation and a vector potential can be obtained in the continuum limit by writing
\begin{equation}
    U_{\mu}(x) = e^{i a g A_{\mu}(x)}.
\end{equation}


It is necessary to discuss about gauge invariance before moving to the dynamics of these gauge fields.
% Before moving on the dynamics of these gauge fields we have talk on the meaning of gauge invariance on the lattice.
A gauge transformation is described by a group-valued function $g(x)$ (in the appropriate representation), which acts on the vertices $x$.
The variable $U_{\mu}(x)$ sits in the middle of the site $x$ and $x + \hat{\mu}$, therefore it transforms as
\begin{equation}
    U_{\mu}(x) \mapsto g(x) U_{\mu}(x) g(x + a \hat{\mu})^{\dagger}
    \label{eq:gauge_transf_field_lattice}.
\end{equation}

The action of the gauge fields $U_{\mu}(x)$ has to satisfy two requirements:
% In order to introduce a dynamics for the gauge fields $U_{\mu}(x)$, we need to define their action which need two satisfy two requirements:
it has to be \emph{gauge-invariant} and reduce to the pure gauge \ac{ym} action in the continuum limit.
From \eqref{eq:gauge_transf_field_lattice}, we can immediately deduce that taking the product of $U_{\mu}(x)$ along a closed curve will yield a gauge-invariant quantity.
The simplest close curve we can consider is a \emph{plaquette}, i.e., the smallest square face.

Consider a plaquette $\plaquette$ sitting in the $(\mu, \nu)$-plane at site $x$.
We define the \emph{single plaquette \acl{wl}} $\WilsonPlaq$ as
\begin{equation}
    \WilsonPlaq =
    U_{\mu}(x) U_{\nu}(x + a \hat{\mu}) U_{\mu}(x + a \hat{\nu})^{\dagger} U_{\nu}(x)^{\dagger}.
    \label{eq:single_plaquette_Wilson_loop}
\end{equation}
Notice that we do not have any sum in the indices $\mu$ and $\nu$ because they are not Lorentz indices.
% The quantity in \eqref{eq:single_plaquette_Wilson_loop} is called a single plaquette \emph{\ac{wl}}.
Only scalar quantities are allowed in the action, so we need to take the trace of $\WilsonPlaq$.
Then, our lattice action will be defined as the sum over the plaquettes of $\tr \WilsonPlaq$ (and its Hermitian conjugate):
\begin{equation}
    \Action_{\text{W}} = - \frac{1}{g^2} \sum_{\square} \qty( \tr \WilsonPlaq + \tr \WilsonPlaq^{\dagger} ).
    \label{eq:wilson_action}
\end{equation}
This is known as the \emph{Wilson action} \citneeded.
The quantity $\tr \WilsonPlaq$ behaves as expected in the continuum limit, where we have to work with the strength field $F^{\mu \nu}$:
\begin{equation}
    \tr \WilsonPlaq \approx N - \frac{a^{4}}{2} \tr F_{\mu \nu} F^{\mu \nu} + \mathcal{O}(a^6),
\end{equation}

The lattice action is not unique.
The Wilson action in \eqref{eq:wilson_action} is the simplest choice that one can make that satisfy our requirement.
Some other modification, for example, can include other types of closed loops and these modification can have their place.
However, they will not be considered here.

Obviously, in a path-integral formulation of \ac{lgt}s we need to define the path-integral in order to have a quantum theory:
\begin{equation}
    Z = \int \prod_{(x, \hat{\mu})}  \dd U_{\mu}(x) e^{- \Action_{\text{W}}}.
    \label{eq:path_intelgral_Wilson_lgt}
\end{equation}
Here we integrate over all possible values for the gauge variables.
Due to the fact that $U_{\mu}$ are elements of a group $G$ that for most physical applications is compact, the most natural choice is to the invariant group measure also known as \emph{Haar measure}.
% where the integration measure $\dd U_{\mu}(x)$ is understood to be the Haar measure.
% In case of a compact group, like $\SU(N)$, it is well defined and yields a finite value.
Notice that \eqref{eq:path_intelgral_Wilson_lgt} is now a well defined mathematical quantity, unlike the path-integral in continuum theory where a clear mathematical definition is still lacking.
Now that the path-integral measure has been defined, we can compute the average of an observable $\mathcal{O}$ with
\begin{equation}
    \ev*{\mathcal{O}} = \frac{1}{Z} \int \prod_{(x, \hat{\mu})} \dd U_{\mu}(x) \mathcal{O} e^{- \Action_{\text{W}}}
\end{equation}


%
% SUBSECTION: Order parameters and gauge invariance
%
\subsection{Order parameters and gauge invariance}
\label{sub:wilson_confinement_test}

The Wilson formulation of \ac{lgt}s can resemble spin models studied in statistical mechanics.
The link variables $U_{\mu}(x)$ can be thought as some sort of generalization of the spin \ac{dof}.
They are distributed in a crystal-like structure and interact with their nearest neighbours, in this case through a four-body interaction (the plaquette action), instead of two-body interaction (like the Ising model).
If one wants to pursue this analogy, then it is reasonable to look at order parameters that behaves like the spontaneous magnetization, where a non-vanishing expectation value signals a phase transition.
The analogue of such an order parameter in \ac{lgt} would be something like
\begin{equation}
    \ev*{U_{\mu}(x)} \neq 0,
    \label{eq:non_zero_link_var_expt_value}
\end{equation}
but it has been shown that this is impossible in Wilson theory \cite{elitzur1975theorem}.

In standard spin models, a non-zero magnetization represents a spontaneous breaking of the global symmetry of the system.
Consider the simplest case of the classical Ising model, where the \ac{dof} are binary variables $\sigma = \pm 1$.
Without an external field, the energy is given by the interaction of nearest neighbouring spins, i.e., $\sigma_i \sigma_j$.
This system has an obvious global $\Z_2$ symmetry, that corresponds to the inversion $\sigma_i \mapsto -\sigma_i$ off all the spins.
A ferromagnetic phase is, by definition, signaled by $\ev*{\sigma} \neq 0$, which necessarily breaks the global $\Z_2$ symmetry of the model.
Once a direction is selected by $\ev*{\sigma} \neq 0$, it remains stable under thermal fluctuations because they cannot coherently shift the magnetization of a large (or infinite) number of spins.

In a \ac{lgt}, an expectation value like \eqref{eq:non_zero_link_var_expt_value} would \emph{break gauge invariance}, which is a \emph{local symmetry}, not a global one.
As explained previously, gauge invariance means that the action is unchanged under local arbitrary ``rotations'' of the link variables $U_{\mu}(x)$ (see \eqref{eq:gauge_transf_field_lattice}).
Hence, thermal fluctuations will induce such rotations and in the long rung it will average on all the possible gauges, which leads to
\begin{equation}
    \ev*{U_{\mu}(x)} = \int \dd U_{\mu}(x) U_{\mu}(x) = 0
\end{equation}
if $U_{\mu}$ contains only non-trivial irreducible representations of the group (see Th.~\ref{th:orthogonality_compact_groups} and \eqref{eq:corollary_orthogonality_th}).
This means that ``magnetization'' is always vanishing in a \ac{lgt} and gauge invariance cannot be spontaneously broken, which is the contents of the Elitzur theorem \cite{elitzur1975theorem}.

The conclusion of this brief discussion may seem rather grim, as magnetization in spin models is the most convenient and used order parameter.
But this does not mean that there are no other order parameters in a \ac{lgt}.
We have just showed that the problem when considering something like $\ev*{U_{\mu}}$ is gauge invariance.
So, the most reasonable step forward is to consider \emph{gauge-invariant quantities} as order parameters.
We have already seen that tracing over a product of $U_{\mu}$ variables along a closed curve is a gauge-invariant quantity, called \emph{\ac{wl}}.

In so far, we have considered only single plaquette loops but nothing restraints us from considering arbitrary large loops, indeed it serves as a \emph{confinement test} for pure gauge theories.
% One of the most important quantity that can be computed for a pure \ac{lgt} is the \ac{wl} for generic closed paths $\mathcal{C}$.
% It serves as \emph{confinement} test.
It has been shown \cite{wilson1974confinement} that confinement is equivalent to the \emph{area law} behaviour of \ac{wl}s, i.e.,
\begin{equation}
    \ev*{W(\mathcal{C})} \sim \exp(- \sigma A(\mathcal{C})),
    \label{eq:wilson_area_law}
\end{equation}
where $A(\mathcal{C})$ is the minimal area inside the closed path $\mathcal{C}$ and $\sigma$ the \emph{string tension} (the coefficient of the linear potential between two quarks).
On the other hand, in the absence of confinement one finds instead the \emph{perimeter law}:
\begin{equation}
    \ev*{W(\mathcal{C})} \sim \exp(- k P(\mathcal{C})),
    \label{eq:wilson_perimeter_law}
\end{equation}
where $P(\mathcal{C})$ is the perimeter of the curve $\mathcal{C}$ and $k$ just some constant.

The reason behind this behaviour can be seen with a simple qualitative picture \cite{creutz1985book, wilson1974confinement}.
A closed timelike \ac{wl} basically represents a process in which a quark-antiquark pair is produced, moved along the sides of the loop and annihilated.
If we are in a confining phase we can then expect a liner potential between the quark and antiquark.
We can imagine a flux tube \emph{binding} the two charges, which swoops the whole inside area of the loop.
Then, it is easy to image that the energy of this whole process will necessarily depend on the area of the loop.
On the other hand, if we are in a deconfined phase then there is no potential binding the two quarks.
In this case the energy of the whole process depends only on the self-energy of quarks, which move along the sides of the loop.
Therefore, the leading energy contribution of this process depends on the perimeter, instead of the area.
Obviously, this picture is no longer valid when dynamical matter is involved.
In a confining phase, pair production is always preferred when separating two quarks at large distances.
\todo{inserire immagine}

From \eqref{eq:wilson_area_law} and \eqref{eq:wilson_perimeter_law}, we can deduce that the string tension $\sigma$ can be used as an order parameter.
It is non-zero for a confining phase, while it vanishes for a deconfined phase.
But it is \emph{non-local} in nature, as it involves the asymptotic behaviour of potential, and therefore of the correlation functions of the theory.



\subsection{Fermions on a lattice}
\label{sub:fermions_on_a_lattice}

Defining fermions is not an easy task due to the known \emph{doubling problem}.
In simple terms, when introducing fermions on a lattice, instead of a continuous space, it leads to a extra spurious fermions, which are just lattice artifacts.

\todo{parlare della discretizzazione del fermione prima}

In order to briefly see this, consider the correlation function for a single fermionic species.
If $K$ is the kinetic matrix for the fermions, then $G = K^{-1}$ gives their correlation matrix.
One finds \citneeded that the correlation function between two sites $x$ and $y$ has the form
\begin{equation}
    (G)_{x,y} = \frac{1}{a^d L^d} \sum_{k} \tilde{G}_{k} e^{2 \pi i k \cdot (x - y) / L},
    \label{eq:fermionic_corr_func_real_space}
\end{equation}
where $a$ is the lattice spacing, $L^d$ the total volume and $\tilde{G}_k$ the correlation function in momentum space:
\begin{equation}
    \tilde{G}^{-1}_k = m + \frac{i}{a} \sum_{\mu} \gamma_{\mu} \sin(2 \pi k_{\mu} / L).
    \label{eq:fermionic_corr_func_latt}
\end{equation}
It involves a trigonometric function because the derivative term involves nearest neighbouring sites.
One can then take the model to a large lattice, which justifies in substituting the discrete sums with integrals:
\begin{equation}
    \frac{2 \pi k_{\mu}}{La}  \; \to \; q_{\mu}
    \qquad \text{and} \qquad
    \frac{1}{a^d L^d} \sum_{k} \; \to \; \int \frac{\dd q^d}{(2 \pi)^d},
    \label{eq:limit_large_lattice}
\end{equation}
where the $q_{\mu}$'s are continuous momentum variables.
This substitution maps \eqref{eq:fermionic_corr_func_latt} into
\begin{equation}
    \tilde{G}_k^{-1} = m + \frac{i}{a} \sum_{\mu} \gamma_{\mu} \sin(a q_{\mu}).
    \label{eq:fermionic_corr_func_large_latt}
\end{equation}

One can naively think of taking the limit $a \to 0$ and expand $\sin(a q_{\mu})$ around the zero and obtain something that look like the correct continuum limit:
\begin{equation}
    \tilde{G}_{\mu}^{-1} = m + i \slashed{q} + \mathcal{O}(a^2).
\end{equation}
But one should not be fooled by this sloppy procedure just because it appears to give the wanted result.
Each component $q_{\mu}$ takes values in the region $[-\pi/a, +\pi/a]$, hence we have to integrate on the whole volume $[-\pi/a, +\pi/a]^d$.
Looking at \eqref{eq:fermionic_corr_func_large_latt}, it is clear that the major contributions to $G$ in \eqref{eq:fermionic_corr_func_real_space} comes from the zeros of $\tilde{G}_k^{-1}$.
This, not only vanishes in the region $q_{\mu} \sim 0$ but also for large momentum $q_{\mu} \sim \pi/a$.
The propagator has no suppression of momentum values near $\pi/a$.
We can isolate the large momenta region by considering
\begin{equation}
    \tilde{q}_{\mu} = q_{\mu} - \pi/a
\end{equation}
for each direction in space.
In this way, we de facto half the integration region,
\begin{equation}
    \int_{-\pi/a}^{\pi/a} \dd q_{\mu} \;\to\;
    \int_{-\pi/2a}^{\pi/2a} (\dd q_{\mu} + \dd \tilde{q}_{\mu})
\end{equation}
and now the limit $a \to 0$ can be taken safely, but it comes with a price to pay.
For each direction in space, we have two independent regions that gives a free fermion contribution to the propagator in the continuum limit.
We have effectively \emph{doubled} the number of fermions for each direction.
In a $d$-dimensional lattice we end up with $2^d$ independent fermions, even though we initially started with just one.

\todo{parlare del ruolo della chiralità}

There are many solutions to this fermion doubling problem\citneeded, but we will focus only on one in this manuscript: \emph{the staggered fermions}\citneeded.
We have seen that these fictitious fermions come from the large momenta regions, where $q_{\mu} \sim \pi/a$.
Brutally cutting out this large momenta regions spoils the completeness of the Fourier transform, so it is not a solution, but a smarter solution can give out the same effect.
The idea is to spread the fermionic \ac{dof} over multiple lattice sites, reducing effectively the momenta space.
For example, in two dimensions it would correspond to placing the \emph{particles} on \emph{even} sites and the \emph{antiparticles} on \emph{odd} sites.
A site is considered even or odd when $(-1)^x = (-1)^{x_1 + \dots + x_d} = +1$ or $-1$.

To obtain a staggered fermion, we define a new fermionic species $\chi(x)$ such that
\begin{equation}
    \psi(x) = \prod_{\mu} (\gamma^{\mu})^{n_{\mu}} \chi(x),
\end{equation}
where $x_{\mu} = an_{\mu}$.
Now, if we want to express the discredited covariant derivative, the term $\gamma^{\mu} \psi(x+a n_{\mu})$ have two extra powers of $\gamma^{\mu}$ compared to $\overline{\psi}$.
Since $(\gamma^{\mu})^2 = \pm 1$, we have therefore
\begin{equation}
    \overline{\psi}(x) \gamma^{\mu} \psi(x) = (-1)^{\eta_{\mu}(x)} \chi(x)^{\dagger} \chi(x + a n_{\mu}),
\end{equation}
where $\eta_{\mu}(x)$ is some sign function depending on the site $x$.
This function can be obtained from the commutation relations of the gamma matrices.
In particular, in two dimensions we have
\begin{equation}
    \eta_1(x) = 1
    \quad \text{and} \quad
    \eta_2(x) = (-1)^{n_1},
\end{equation}
while in four (Euclidean) dimension we have instead \todo{Citare Tong, gauge theories}
\begin{equation}
    \eta_1(x) = 1, \quad
    \eta_2(x) = (-1)^{n_1}, \quad
    \eta_3(x) = (-1)^{n_1 + n_2}, \quad
    \eta_4(x) = (-1)^{n_1 + n_2 + n_3}.
\end{equation}
A similar reasoning can be applied to the mass term $m \overline{\psi}(x) \psi(c)$, where it becomes
\begin{equation}
    m \overline{\psi}(x) \psi(x) = (-1)^{\eta(x)} \chi(x)^{\dagger} \chi(x),
\end{equation}
for some sign function $\eta(x)$ that can be obtained from the commutation relations of the gamma functions.

