%----------------------------------------
% SECTION: Quantum simulation
%----------------------------------------
\section{Quantum simulation}
\label{sec:quantum_simulation}

Simulating quantum mechanics is a very challenging task, in particular if one is interested in many-body systems.
The description of the state requires a large number of parameters, for keeping track of all the quantum amplitudes, that grows exponentially with the system size.
Hence, one would have an \emph{exponential explosion} in terms of \emph{classical} resources (i.e., computer memory).
If simulating a quantum system is not a task for classical computers, then it should be a task for \emph{quantum computers}.
This kind of devices, first envisioned from Feynman\citneeded, promises much more than simulating quantum mechanics.
Indeed, quantum computation and quantum information theory are, still today, very active research fields.

A quantum computer can encodes the large amount of information of a quantum system in its large number of amplitudes.
So, the size of a quantum computer would only be proportional to the size of the quantum system it intends to simulate, \emph{without} an exponential explosion in \emph{quantum} resources.
In fact, a quantum computer can indeed act as a \emph{universal quantum simulator} (Lloyd 1996\citneeded).
This is basically the idea behind \emph{digital quantum simulations}.

However, another approach is possible.
One can mimic the evolution of a given quantum system by means of another \emph{analogous} and \emph{controllable} quantum system.
Hence, we will only need a specific quantum machine for a specific class of problems.
This is, instead, the idea behind \emph{analog quantum simulations}.
In this case, for a specific set of problems the full implementation of a quantum computer may not be necessary.

In general, \emph{quantum simulation} can be (loosely) defined as simulating a quantum system by quantum mechanical means,
There are three paths that can be taken in this regard:
\begin{itemize}
    \item digital quantum simulation
    \item analog quantum simulation
    \item quantum-information inspired algorithms for classical simulation
\end{itemize}
We will discuss briefly each one of them.
By \emph{quantum simulator} we mean a \emph{controllable} quantum system used to simulate or emulate other quantum systems.
We see that only digital and analog quantum simulations employ a quantum simulator.
The last option employs techniques, inspired by quantum information theory, that make it possible to truncate and approximates quantum states in order to have efficient classical simulations.


\todo{Inserire figura schematica di un quantum simulator}

\subsection{Digital quantum simulations}
\label{sub:digital_quantum_simulations}

This approach employs the circuit model for quantum computation.
Generally, the quantum simulator is a collection of \emph{qubits} (i.e., two-level quantum systems).
This constitutes the \emph{quantum register}.
A wave function of the simulated system using the computation basis, in other words a superposition of binary bit string.
Each bit of a string refers to the state of a qubit in the registry, which can be either $\ket{0}$ or $\ket{1}$ in the computational basis.
On this register, any many-qubit unitary transformation $U$ is implemented through the application of a sequence of single- and two-qubit unitary operations, called \emph{quantum gates}.
An immediate example of relevant unitary transformation is the time-evolution operator, that solves the time-independent Schrödinger equation.

Even though it has been proven (Lloyd 1996\citneeded) that ``anything'' can be simulated on a quantum computer, not all unitary operations can be simulated \emph{efficiently}.
Therefore, there are \emph{mathematically possible} Hamiltonians that cannot be efficiently simulated with a circuit-based model.
Luckily, it is believed that \emph{physically relevant} Hamiltonians can indeed be simulated efficiently.
Furthermore, it should be stressed that the implemented unitary operations are often just approximations of the desired unitary operation.
With greater precision comes a greater number of gates.

The typical setup for a digital simulation is made of three steps:
\begin{itemize}
    \item
        \emph{Initial-state preparation} --- where the quantum register has to prepared in the state $\ket{\psi(0)}$.
        This step can be by itself difficult, and it is not always guaranteed that an efficient algorithm may exist.
    \item
        \emph{Unitary evolution} --- where the circuit has to reproduce or simulate the action of a unitary operator $U$.
        In case of a unitary time evolution and local Hamiltonian this can be achieved approximately with some ``trotterization'' scheme.
    \item
        \emph{Final measurement} --- after obtained the wanted state $\ket{\psi(t)} = U \ket{\psi(0)}$, a \emph{measurement} is needed in order to extract the relevant physical information.
        Instead of capturing the whole wave function $\ket{\psi(t)}$, with for example quantum tomography, one may proceed with the direct estimation of certain physical quantities, such as correlation functions or spectra of operators.
\end{itemize}


\subsection{Analog quantum simulations}
\label{sub:analog_quantum_simulations}

Analog quantum simulation (AQS) in another possible approach to quantum simulation, where a one quantum system mimics or emulate another.
The Hamiltonian of the system to be simulated $H_{\text{sys}}$ is directly mapped onto the Hamiltonian of the simulator $H_{\text{sim}}$.
Obviously, this can be done if there is a mapping between the system and the simulator.
Note that the simulator may only partly reproduce the dynamics of the system, or simulate some effective description of the system.

An important advantage of AQS is that it does not require a full quantum computer, even more the simulator does not even need to be a computer at all.
Finding the mapping in an AQS might look, at first, simpler than finding the most efficient gate decomposition of a Hamiltonian, but it is not always guaranteed and there are no recipes ready for such mappings.
The obvious drawback of AQS is that the quantum simulators are problem specific.

\todo{Aggiungere altro}


\subsection{Quantum-inspired algorithms}
\label{sub:quantum_inspired_algorithms}

\emph{Classical} numerical algorithms for the simulation of quantum many-body systems came out of research on quantum information theory in these later years.
The most important examples of quantum-inspired algorithms are \emph{tensor networks} methods.
Tensor networks make it possible to compress the information about a many-body wave function by expressing it as a contraction of a network of tensors (as suggested by the name).
For a large class of physically relevant models, the ground state is gapped and has, in a certain sense, a finite amount of entanglement.
This fact is expressed by the so-called \emph{area law}, where the entanglement between two partitions of the system grows with size of the boundary, the area between the two partitions, and not with the size of the partition itself.
The main advantage of tensor networks is their ability to capture this area law.

\todo{Aggiungere altro}
