\chapter*{Report on PhD activities}
\addcontentsline{toc}{chapter}{Report on PhD activities}

% No paragraph indentation
\setlength{\parindent}{0pt}
\newcommand{\PhD}{Ph.D.\xspace}
\newcommand{\MSc}{M.Sc.\xspace}
\newcommand{\BSc}{B.Sc.\xspace}

% arXiv ID
\newcommand{\arXiv}[1]{\href{https://arxiv.org/abs/#1}{\texttt{arXiv:#1}}}

% Common locations
\newcommand{\Unibo}{University of Bologna}
\newcommand{\GGI}{Galileo Galilei Institute, Florence, Italy}

% Section headings
%
% Arguments:
%   1. Section title
\newcommand{\Section}[1]{
    \vspace*{0.5cm}
    \noindent%
    % \textsc{\Large \lowercase{#1}}\vspace{1em} ~~~\textcolor{black!70}{\hrulefill}
    {\Large\PlexMedium #1}
    \vspace*{0.35cm}
}

% Item
%
% Arguments:
%   1. right block text (large block and normal text)
% Optional arguments:
%   1. left block (narrow block and italicized text)
% Lenghts:
%   \ItemLeftWidth  width of the left block
%   \ItemRightWidth width of the right block
\newcommand{\Item}[2][\null]{
    {
        \begin{tabular}{p{\ItemLeftWidth}p{\ItemRightWidth}}
            \raggedleft#1 & \parbox[t]{\linewidth}{#2}
        \end{tabular}
        \vspace{\ItemVspace}
    }
}

% Defines the widths
\newlength{\ItemRightWidth}
\newlength{\ItemLeftWidth}
\newlength{\ItemVspace}
\setlength{\ItemLeftWidth}{1cm}
\setlength{\ItemRightWidth}{\textwidth}
\addtolength{\ItemRightWidth}{-\ItemLeftWidth}
\setlength{\ItemVspace}{5pt}


% Pubblication
%
% Arguments:
%   1. Authors
%   2. Title
%   3. Other info (arXiv id, status, etc...)
%   4. Year
\newcommand{\Publication}[4]{
    \Item[{\normalfont #4}]{
        \frenchspacing #1\nonfrenchspacing\\
        \emph{#2}\\
        #3
    }
}

% Thesis (for cosupervisioning)
%
% Arguments:
%   1. Authors
%   2. Title
%   3. Year
\newcommand{\Thesis}[3]{
    \Item[{\normalfont #3}]{
        \frenchspacing #1\nonfrenchspacing\\
        \emph{#2}
    }
}

% Events (School, workshop, etc...)
%
% Arguments:
%   1. Event name
%   2. Localtion
%   3. Year
\newcommand{\Event}[3]{
    \Item[{\normalfont #3}]{
        #1 \\
        {\small #2}
    }
}


\vspace*{-1.25cm}

\Section{Publications}

\Publication{\underline{SP}, A. Maroncelli, E. Ercolessi}{Discrete Abelian lattice gauge theories on a ladder and their dualities with quantum clock models}{\arXiv{2208.04182} (in pubblication)}{2022}

\Publication{A. Mariani, \underline{SP}, E. Ercolessi}{Hamiltonians and gauge-invariant Hilbert space for Yang-Mills theories with finite gauge group }{(in preparation)}{2023}

\Publication{\underline{SP}, E. Ercolessi}{Long-range Kitaev chains and Toeplitz matrices}{(in preparation)}{2023}

\Publication{F. Dell'Anna, \underline{SP}, E. Ercolessi}{Fisher information and multipartite entanglement in long-range systems}{(in preparation)}{2023}



\Section{Co-supervisioning of \MSc~Thesis}

\Thesis{R. Cioli}{Digital quantum simulations of the $\mathbb{Z}_N$ Toric Code}{2022}

\Thesis{L. Pecorari}{Diagnosing criticality in symmetric and chiral clock models}{2022}

\Thesis{D. Rossi}{Fracton phases: analytical description and simulations of their thermal behavior}{2021}

\Thesis{L. Lumia}{Digital quantum simulations of Yang-Mills lattice gauge theories}{2021}

\Thesis{P. Baglioni}{Ergodicity and localization in $\mathbb{Z}_n$ lattice Schwinger model}{2020}

\Thesis{F. Rotella}{Topological characterization of two-dimensional p-wave superconductors}{2020}



\Section{PhD Schools}

\Event{Tensor$22$ School on Tensor Networks based approaches to Quantum Many-Body Systems}{Vienna, Austria}{2022}

\Event{SQMS/GGI Summer School on Quantum Simulation of Field Theories}{\GGI}{2022}

\Event{Quantum Sensing, Information Processing and Computing}{Bologna, Italy}{2022}

\Event{SFT$2022$ Lectures on Statistical Field Theories}{\GGI}{2022}

\Event{SFT$2021$ Lectures on Statistical Field Theories}{\GGI}{2021}

\Event{SCQM$20$ Winter school on strongly correlated quantum matter}{Max Planck Institute for the Physics of Complex Systems, Dresden, Germany}{2020}

\Event{Autumn School on Correlated Electrons: Topology, Entanglement, and Strong Correlations}{Forschungzentrum J\"ulich, Germany}{2020}



\Section{Workshops and Conferences}

\Event{Quantum Methods for Lattice Gauge Theories}{MITP, Mainz, Germany}{2022}

\vspace*{-3pt}

\Item{\underline{Talk}: ``\emph{Discrete Abelian lattice guage theories on a ladder and their dualities with quantum clock models}''}

\Event{IQIS$2022$ Italian Quantum Information Science Conference}{Palermo, Italy}{2022}

\vspace*{-3pt}

\Item{\underline{Poster}: ``\emph{Quantum Fisher Information and Topological Phases}''}

\Event{AQC$2022$ Conference on Adiabatic Quantum Computation / Quantum Annealing}{ICTP, Trieste, Italy}{2022}

\Event{Randomness, Integrability and Universality}{\GGI}{2022}

\Event{Topological properties of gauge theories and their applications to high-energy and condensed matter physics}{\GGI}{2021}

\Event{The Hitchhiker's Guide to Condensed Matter and Statistical Physics: Topological Phenomena in Condensed Matter}{ICTP, Trieste, Italy}{2021}

\Event{Gravity and Emergent Gauge Fields in Condensed and Synthetic Matter}{MITP, Mainz, Germany}{2021}

\Event{The Hitchhiker's Guide to Condensed Matter and Statistical Physics: Machile Learning for Condensed Matter}{ICTP, Trieste, Italy}{2021}



\Section{Visiting periods}

\Event{Visiting period at the Center for Quantum Devices under the supervision of Prof.~Michele Burrello, from $17/10/21$ to $4/01/22$ ($3$ months)}{Niels Bohr Institute, Copenhagen, Denmark}{2021}

\vspace*{-3pt}

\Item{\underline{Talk}: ``Toeplitz Matrices for the Long-range Kitaev Model''}



\Section{Teaching experience}

\Event{Tutor for the School \emph{Quantum Sensing, Information Processing and Computing}}{\small Dept.~of Physics and Astronomy, \Unibo}{2022}

\Event{Tutor for \emph{General Physics} (\BSc in Mathematics)}{\small Dept.~of Mathematics, \Unibo}{2022}



\Section{Courses}

\setlength{\ItemVspace}{-8pt}

\Item[2020]{Quantum mechanics and entanglements (24h)}

\Item[2020]{Higgs and electroweak physics at LHC (24h)}

\Item[2020]{Dark matter phenomenology and gravitational waves (24h)}

\Item[2020]{Collider physics (12h)}

\Item[2020]{Fundamentals of photonics (12h)}

\Item[2021]{Renormalization group in QFTs and Critical theories (12h)}

\Item[2021]{Solitons in Modern Perspective (12h)}

\Item[2021]{European strategies and funding opportunities (12h)}


