%======================================================================
% CHAPTER: DUALITIES IN LATTICE GAUGE THEORIES
%======================================================================
\chapter{Dualities in Abelian Models}
\label{chap:dualities_in_abelian_models}

In this chapter we present the findings of \cite{pradhan2022ladder}, where a duality transformation from the gauge-invariant subspace of a $\Z_N$ \ac{lgt} on a ladder geometry to an $N$-clock model on a single chain.
The main feature of this mapping is the emergence of a longitudinal field in the clock model, whose value depends on the super-selection sector of the gauge model.
% , implying that the different sectors of the gauge theory can show quite different phase diagrams.
In order to investigate this and see if confined phases might emerge, we perform a numerical analysis for $N = 2, 3,$ and $4$, using \acl{ed}.


% SECTION: Toric Code and its features
%----------------------------------------
% SECTION: Toric Code and its features
%----------------------------------------
\section{Toric Code and its features}
\label{sec:toric_code_and_its_features}

The Toric code is two-dimensional model of spin-\onehalf degrees of freedom (d.o.f).
It can be regarded as an example of a pure $\mathbb{Z}_2$ lattice gauge theory.
In particular we focus on a $L \times L$ square lattice (with periodic boundary conditions), and the d.o.f.~are defined on the links the lattice.
The local Hilbert space is $\mathbb{C}^2$ and as a basis we can use the computation basis (or $Z$-basis) $\qty{ \ket{0}, \ket{1} }$ for which the Pauli matrix $\sigma^z$ (shortened as $Z$) is diagonal.
The main local operators (that enters the Hamiltonian) are defined on the \emph{stars} (the links attached to a given site) and \emph{plaquettes} (links around a face) of the lattice:
\begin{equation}
    A_v = \prod_{j \in \text{star}(s)} \sigma^z_j, \qquad
    B_p = \prod_{j \in \partial p} \sigma^x_j.
\end{equation}
One can easily prove the following commutation relations of the $A_s$s and $B_p$s operators
\begin{equation}
    [ A_v, A_{v^{\prime} } ] = 0, \quad
    [ B_p, B_{p^{\prime} } ] = 0, \quad
    [ A_v, B_p ] = 0
    \label{eq:star_plaq_op_comm}
\end{equation}
for all $v$, $v^{\prime} $ and $p$, $p^{\prime} $.
The eigenvalues of the Pauli matrices are just $pm 1$, so the same holds true for $A_s$ and $B_p$.
Moreover, like the Pauli matrices, also $A_s^2 = \One$ and $B_p^2 = \One$.


Now, we proceed with writing the Toric Code Hamiltonian:
\begin{equation}
    H = - \sum_{v} A_v - \sum_{p} B_p
    \label{eq:toric_code_hamiltonian}
\end{equation}
which is \emph{exactly solvable}.
Indeed, given \eqref{eq:star_plaq_op_comm}, one can find a ground state $\ket{\Omega}$  by simply imposing a set of constraints
\begin{equation}
    A_v \ket{\Omega} = \ket{\Omega}, \qquad
    B_p \ket{\Omega} = \ket{\Omega}, \qquad \forall v,
    \label{eq:ground_state_constraints}
\end{equation}
Even more, it is possible to define the set of ground states
\begin{equation}
    \mathcal{L} = \qty{ \ket{\Omega} : A_s \ket{\Omega} = \ket{\Omega}, \quad B_p \ket{\Omega} = \ket{\Omega} \quad \forall s, p }
\end{equation}
which contents \emph{depends on the topology of the lattice}.
For example, if the lattice has periodic boundary conditions, then there are $4$ degenerate ground states $\ket{\Omega_{u,v}}$ with $u, v = \pm 1$, which can be distinguished only by non-local operators.
We will see how later.

\begin{figure}[t]
    \centering
    \input{assets/figures/star_plaq_operators_TC.tex}
    \caption{Graphical representation of the Toric Code operators $A_s$ and $B_p$}
\end{figure}

\subsection{Ground states}
\label{sub:ground_states}


Consider a torus geometry for the lattice, i.e.~periodic boundary conditions in both directions, of size $L \times L$.
From \eqref{eq:ground_state_constraints}, we have $2L^2$ constraints but these are not all independent.
In fact on a torus, one can see that
\begin{equation}
    \prod_{v} A_{v} = \One, \qquad
    \prod_{p} B_p = \One
\end{equation}
which actually means that there are $2L^2 - 2$ independent conditions.
The total Hilbert space has dimension $2^{2L^2}$, therefore the space $\mathcal{L}$ has dimension $2^{2L^2 - 2L^2 + 2} = 4$, which means that the Toric Code has $4$ degenerate ground states.
These states all satisfy the same set \eqref{eq:ground_state_constraints} of equations, which means that they\emph{cannot be distinguished by local operators}.
This is typical of topological phases in two dimensions \todo{ref}.

These ground states can only be distinguished by non-local operators that have to commute with Hamiltonian in \eqref{eq:toric_code_hamiltonian}, hence with all $A_v$ and $B_p$ but cannot be expressed in terms of $A_v$ and $B_p$.
The only operators that satisfy this requirements are defined on closed paths, on the direct or dual lattice, that cannot be reduced to a single plaquette loop (on the dual lattice the plaquettes are the stars of the direct lattice).
In other words, they are defined along \emph{non-contractible loops}.
The reason being that any product of $\sigma^x$ or $\sigma^z$ along a closed curve $\mathcal{C}$ that commutes with the Hamiltonian can be expressed as product of $A_v$ or $B_p$ of the stars or plaquettes enclosed by $\mathcal{C}$.
Consider now, two non-contractible curves $\mathcal{C}_{1}$ and $\mathcal{C}_{2}$ along the $\hat{1}$ and $\hat{2}$ direction respectively.
On these paths we can define the \emph{string operators}
\begin{equation}
    \overline{X}_1 = \prod_{j \in \mathcal{C}_1}  \sigma^x_j, \qquad
    \overline{X}_2 = \prod_{j \in \mathcal{C}_2}  \sigma^x_j,
    \label{eq:nonlocal_X_operators}
\end{equation}
it can be proved that they commute with all the local operators but cannot be expressed as a product of them.
The same can be repeated on the dual lattice, by considering dual non-contractible paths $\tilde{\mathcal{C}}_1$ and $\tilde{\mathcal{C}}_2$ and defining
\begin{equation}
    \overline{Z}_1 = \prod_{j \in \mathcal{\tilde{C}}_1}, \qquad
    \overline{Z}_2 = \prod_{j \in \mathcal{\tilde{C}}_2}
    \label{eq:nonlocal_Z_operators}
\end{equation}
Likewise, the operators in \eqref{eq:nonlocal_Z_operators} commutes with all the vertex and plaquettes operators but they do not commute with the X-operators in \eqref{eq:nonlocal_X_operators}.

In fact, \eqref{eq:nonlocal_X_operators} and \eqref{eq:nonlocal_Z_operators} have the same (anti)commutation relations of two qubits:
\begin{equation}
    \qty{ \overline{X}_1, \overline{Z}_2 } = 0, \qquad
    \qty{ \overline{X}_2, \overline{Z}_1 } = 0,
\end{equation}
Therefore, the Toric Code (on a torus) has a protected subspace $\mathcal{L}$ (the space of the ground states) that behaves like the Hilbert space of two qubits and the operators \eqref{eq:nonlocal_X_operators} and \eqref{eq:nonlocal_Z_operators} acts like unitary gates on this space.

\begin{figure}[t]
    \centering
    \input{assets/figures/nonlocal_operators_TC.tex}
    \caption{Non-contractible paths}
\end{figure}

\todo{revisionare}



\subsection{Particle excitations}%
\label{sub:particle_excitations}

Until now we have only discussed the ground states of the Toric Code, without touching the rest of the low energy sectors.
In other words, how do we describe the excitations of this model?
As we said, \eqref{eq:ground_state_constraints} are the set of constraints that defines the ground states.
Therefore, everytime a given state $\ket{\Psi}$ violates these equations, we will say that it contains \emph{particles}, which can be of different types.
If $A_v \ket{\Psi} = - \ket{\Psi}$, then we will say that the vertex $v$ contains a $z$-type particle.
Likewise, if $B_p \ket{\Psi} = - \ket{\Psi}$, then the plaquette $p$ contains a $x$-type particle.

Now the question: starting from a ground state $\ket{\Omega}$, how can we introduce some particles?
The answer is \emph{string operators}.
We are not considering closed strings, like we did in Sec.~\ref{sub:ground_states}, but any open string.
The shortest open string that we can consider is a single link.
So a $Z$-string on a single link is just $Z_j$, where $j$ is a label of a generic link.
Consider now the state
\begin{equation}
    \ket*{\Psi^Z} = Z_j \ket{\Omega},
\end{equation}
This state hosts particles at the ``boundaries'' of the $j$-th link, i.e.~the vertices touching $j$ which we call $v_0$ and $v_1$.
This can be proved by simply showing that $[ A_v, Z_j ] = 0$ for $v \neq v_0$ and $v \neq v_1$ and $\{ A_{v_0}, Z_j \} = \{ A_{v_1}, Z_j \} = 0$.
Which immediately implies that
\begin{equation}
    A_{v_0} \ket*{\Psi^Z} = A_{v_1} \ket*{\Psi^Z} = - \ket*{\Psi^Z}
\end{equation}



% SECTION: Generalization to ZN
%----------------------------------------
% SECTION:
%----------------------------------------
\section{\texorpdfstring{$\Z_N$}{Z\_N} Model}%
\label{sec:model}
In this section we are going to consider a class of Abelian lattice gauge theories (LGTs) on a two-dimensional lattice
with a discrete symmetry $\mathbb Z_N$. Then we will restrict these models on a \emph{ladder geometry},
which will be defined more precisely later.

\subsection{Schwinger-Weyl algebra}%
\label{sub:schwinger_weyl_algebra}

According to Wilson's Hamiltonian approach to lattice gauge theories \cite{wilson1974confinement}, $U(1)$ gauge fields are defined on
the links of a lattice $\lattice$  either in a pair of conjugate variables,
the electric field  $E_\ell$ and either the vector potential $A_\ell$, satisfying  $[E_\ell, A_{\ell^{\prime}} ] = i \delta_{\ell , \ell^{\prime} }$, or equivalently the magnetic operator, also called comparator,
$U_\ell = e^{-i A_{\ell} }$, such that $[E_\ell, U_{\ell^{\prime} } ] =  \delta_{\ell , \ell^{\prime} } \, U_{\ell}$, all acting on an infinite dimensional Hilbert space defined on each link.
This form of the canonical commutation relations represents the infinitesimal version of the relations: $ e^{i\xi E} e^{-i\eta A } e^{-i\xi E} = e^{i\xi \eta} e^ {-i\eta A }$,
for any $\xi, \eta \in \mathbb{R}$,
that define the Schwinger-Weyl group \cite{notarnicola2015discrete, ercolessi2018znmodels, schwinger1960unitary}.

For a discrete group like $\mathbb Z_N$, the notion of infinitesimal generators loses any meaning and we are led to directly consider, for each link $\ell \in \mathbb L$, two unitary operators
$V_\ell, \, U_\ell$, such that \cite{schwinger1960unitary, schwinger2001symbolism}
\begin{equation}
    V_\ell U_\ell V_\ell^{\dagger}=e^{2\pi i/N}U_\ell, \qquad
    U_\ell^N=\identity_N, \qquad
    V_\ell^N=\identity_N.
    \label{eq:schwinger_weyl_algebra}
\end{equation}
while on different links they commute.
Thus, by representing $\Z_N$  with the set of the $N$ roots of unity $e^{i 2 \pi k/N}$\, ($k=1, \cdots, N$), commonly referred to as the discretized circle,
we see that $V$ plays the role of a ``position operator'' on the discretized circle, while $U$ that of a ``momentum operator''.

These algebraic relations admit a faithful finite-dimensional representation of dimension $N$ \cite{weyl1950theory}, for any integer $N$, which is obtained as follows.
To each link $\ell$, we can associate an $N$-dimensional Hilbert space $\mathcal{H}_\ell$ generated by an orthonormal basis $\{\ket{v_{k,\ell}}\}$ ($k=1, \dots,N$), called the \emph{electric basis},
that diagonalizes $V_\ell$.
With this choice, we can promptly write the actions of $U_\ell$ and $V_\ell$:
\begin{equation}
    \begin{split}
        U\ket{v_{k,\ell}}           = \ket{v_{k+1,\ell}}, & \qquad
        U\ket{v_{N,\ell}}           = \ket{v_{1,\ell}}\\
        U^{\dagger}\ket{v_{k,\ell}} = \ket{v_{k-1,\ell}}, & \qquad
        U^\dagger\ket{v_{1,\ell}}   = \ket{v_{N,\ell}}\\
        V\ket{v_{k,\ell}}           = \omega^k \ket{v_{k,\ell}}, & \qquad
        V^\dagger\ket{v_{k,\ell}}   = \omega^{-k} \ket{v_{k,\ell}}.
    \end{split}
    \label{eq:elect_basis_op_action}
\end{equation}
where $\omega = e^{2 \pi i / N}$ and $k = 0, \dots, N-1$.
We choose to work in this particular basis and the various $k$ can be interpreted as the quantized values of the electric field on the links.

On a  two-dimensional square lattice of size $L \times L$, the links $\ell$ of the lattice can also be labeled with $(x, \pm\hat{i})$, where $x \in \lattice$ is a site and
$\hat{i}=\hat{1}, \hat{2}$ the two independent unit vectors.
In this way, $(x, \pm\hat{i})$ will refer to the link that start in $x$ and goes in the positive (negative) direction~$\hat{i}$ (see Fig.~\ref{fig:link_labels}).
This notation will be simplified when we reduce to the ladder case.


\begin{figure}
    \centering
    \input{assets/figures/link_labels.tex}
    \caption{
        Labelling of the sites and the links in the two dimensional lattice.
        A site is labeled simply with $x = (x_1, x_2)$, while $\hat{1} = (1,0)$ and $\hat{2} = (0,1)$ stand for the unit vectors of the lattice.
        A link $\ell$ is denoted with a pair $(x, \pm\hat{i})$, with $\hat{i} = \hat{1}, \hat{2}$.
    }%
    \label{fig:link_labels}
\end{figure}


\subsection{Gauge invariance and physical states}%
\label{sub:gauge_invariance}

Gauge transformations act on vector potentials while preserving the electric field.
For a $U(1)$ gauge theory, a local phase transformation is induced by a real function $\alpha_x$
defined on the vertices $x\in \mathbb L$, such that  $A_{\ell} \rightarrow A_{\ell} + (\alpha_{x_2} - \alpha_{x_1})$ or equivalently $U_{\ell} \rightarrow  e^{i(\alpha_{x_2} - \alpha_{x_1})
E_{\ell}}  U_{\ell}   e^{-i(\alpha_{x_2} - \alpha_{x_1} )E_{\ell}} $, where $x_1,x_2$ are the initial and final vertices of the (directed) link $\ell$.
In the case of a discrete symmetry, a gauge transformation at a site $x \in \lattice$ is a product of $V$'s (and $V^\dagger$'s) defined on the links which comes out (and enters) the vertex.
More specifically, for a two dimensional lattice,
if the link $\ell$ at site $x$ is oriented in the positive direction, i.e.~either $(x, +\hat{1})$ or $(x, +\hat{2})$, then $V$ is used, otherwise $V^\dagger$.
Thus,
the single local gauge transformation at the site $x$ is enforced by the operator:
\begin{equation}
    G_x =
    V_{(x, \hat{1})}^{\phantom{\dagger}}
    V_{(x, \hat{2})}^{\phantom{\dagger}}
    V^\dagger_{(x, -\hat{1})}
    V^\dagger_{(x, -\hat{2})},
    \label{eq:gauss_operator}
\end{equation}
as shown in the left part of in Fig. \ref{fig:star_plaq_operators}.

The whole operator algebra $\mathcal{A}$ of the theory is generated by the set of all $U_\ell$ and $V_\ell$ (and their Hermitian conjugates) of all the links of the lattice $\lattice$, while
the  \emph{gauge-invariant subalgebra} $\mathcal{A}_{\gi}$ consists of operators that commutes with all the $G_x$:
\begin{equation}
    \mathcal{A}_\gi = \{ O_{\gi} \in \mathcal{A} \;:\; [O_{\gi}, G_x] = 0 \quad \forall x \in \lattice \}.
\end{equation}
Using \eqref{eq:gauss_operator} and recalling \eqref{eq:schwinger_weyl_algebra}, it is possible to see that the $V_\ell$'s commute with $G_x$ (as expected), while the $U_\ell$'s do not.
In spite of that, we can build gauge-invariant operators out of the comparators $U_\ell$.
Consider a \emph{plaquette} $\square$ of the lattice $\lattice$ at $x$, by which we mean the face of the lattice with vertices $\{x, x+\hat{1}, x+\hat{1}+\hat{2}, x+\hat{2}\}$ in the counterclockwise order, as shown in the right part of Fig.~\ref{fig:star_plaq_operators}.
On this plaquette, the operator $U_{\square}$ is defined as
\begin{equation}
    U_{\square} =
    U_{(x, \hat{1})}
    U_{(x + \hat{1}, \hat{2})}
    U_{(x + \hat{1} + \hat{2}, -\hat{1})}^\dagger
    U_{(x + \hat{2}, -\hat{2})}^\dagger.
    \label{eq:plaq_operator}
\end{equation}
and one finds out that the $U_{\square}$'s commute with $G_x$, for all $x\in \mathbb L$, thus giving a generator of $\mathcal{A}_\gi$.

The set $\{U_{\square}, V_\ell\}$ (for all plaquettes $\square$ and all links $\ell$) may not be enough to generate the whole algebra $\mathcal{A}_\gi$, in case of periodic boundary conditions.
In order to prove this, consider a lattice $\lattice$, periodic in both dimensions, and denote with $\mathcal{C}_1$ and $\mathcal{C}_2$  any two \emph{non-contractible loops} around the lattice, that extends along the $\hat{1}$ and $\hat{2}$ direction respectively.
Then, define the (Wilson loop) operators $\overline{W}_1$ and $\overline{W}_2$ (pictured in blue in Fig.~\ref{fig:nonlocal_operators}):
\begin{equation}
    \overline{W}_i = \prod_{\ell \in \mathcal{C}_i} U_\ell, \qquad i=1,2.
    \label{eq:top_wilson_loop}
\end{equation}
A simple calculation shows that both $\overline{W}_1$ and $\overline{W}_2$ commute with all $G_x$, thus they are gauge-invariant, but one also finds out that none of them can be written as a product of $U_{\square}$ nor $V_\ell$.
Therefore they have to be added explicitly to the set of generators of $\mathcal{A}_\gi$ in order to obtain the whole algebra.
These operators $\overline{W}_1$ and $\overline{W}_2$ play a fundamental role in the model to define topological sectors of the theory, as we will see later.

The total Hilbert space $\mathcal{H}^{\text{tot}}$ is given by the $\otimes_{\ell} \mathcal{H}_{\ell}$.
A state of the whole lattice $\ket{\Psi_{\text{ph}}} \in \mathcal{H}^{\text{tot}}$ is said to be \emph{physical} if it is a \emph{gauge-invariant state}:
\begin{equation}
    G_x \ket{\Psi_{\text{ph}}} = \ket{\Psi_{\text{ph}}}, \qquad \forall x \in \lattice
    \label{eq:gauss_law}
\end{equation}
This condition can be translated into a constraint on the eigenvalues $v_{(x, \pm \hat{i})}= \omega^{k_{(x, \pm \hat{i})} }$  of the operators $V_\ell$ on the links $\ell = (x, \pm \hat{i})$ of the vertex $x$:
\begin{equation}
    v_{(x, \hat{1})}^{\phantom{\ast}}
    v_{(x, \hat{2})}^{\phantom{\ast}}
    v_{(x, -\hat{1})}^\ast
    v_{(x, -\hat{2})}^\ast = 1,
\end{equation}
or, because of (\ref{eq:elect_basis_op_action}):
\begin{equation}
    \sum_{i=1,2} \pqty{ k_{(x, \hat{i})} - k_{(x, -\hat{i})} } = 0 \quad \text{mod $N$}.
    \label{eq:gauss_law_elec_eigvals}
\end{equation}
Given the fact that the $k$ in \eqref{eq:schwinger_weyl_algebra} represent the values of the electric field, one can see that \eqref{eq:gauss_law_elec_eigvals} can be interpreted as a discretized version of the Gauss law $\nabla \cdot \vec{E} = 0$ in two dimensions,
for a pure gauge theory where there are no electric charges.


\begin{figure}[t]
    \centering
    \begin{tikzpicture}[scale=1.2]
    % Lattice
    \draw[Gray,thin] (-0.5,-0.5) grid (4.5,2.5);

    % Plaquette operator
    \draw[U] (3,1) -- (4,1) node [pos=0.5, below] {$U$};
    \draw[U] (4,1) -- (4,2) node [pos=0.5, right] {$U$};
    \draw[U] (4,2) -- (3,2) node [pos=0.5, above] {$U^\dagger$};
    \draw[U] (3,2) -- (3,1) node [pos=0.5, left]  {$U^\dagger$};
    \draw[Blue, ultra thick, pattern=north east lines, pattern color=Blue] (3,1) rectangle (4,2);
    \draw (3.5,1.5) node [fill=white, rounded corners] {$U_{\square}$};

    % Gauss operator
    \draw[V] (1, 1) -- (2, 1) node [pos=0.5, below right] {$V$};
    \draw[V] (1, 1) -- (1, 2) node [pos=0.5, above left] {$V$};
    \draw[V] (0, 1) -- (1, 1) node [pos=0.5, below left] {$V^\dagger$};
    \draw[V] (1, 0) -- (1, 1) node [pos=0.5, below right] {$V^\dagger$};

    \foreach \y in {0,1,2} \foreach \x in {0,1,...,4} \draw (\x,\y) node [site] {};

    \draw (1,1) node [above right] {$G_x$};
\end{tikzpicture}

    \caption{Pictorial representation of the Gauss operators $G_x$ in \eqref{eq:gauss_operator} (\emph{left}) and plaquette operator $U_{\square}$ in \eqref{eq:plaq_operator} (\emph{right}).}
    \label{fig:star_plaq_operators}
\end{figure}


\subsection{\texorpdfstring{$\Z_N$}{Z\_N} Hamiltonian and the Toric Code}%
\label{sub:hamiltonian}

The class of models we consider are described by the Hamiltonian \cite{tagliacozzo2011entanglement, hamma2008adiabatic, trebst2007topological}:
\begin{equation}
    H_{\Z_N}(\lambda) = - \sum_{\square} U_{\square} - \lambda \sum_{\ell} V_{\ell} + \text{h.c.},
    \label{eq:hamiltonian_base}
\end{equation}
where the first sum is over the plaquettes $\square$ of the lattice while the second sum is over the links $\ell$.
One can easily see that this Hamiltonian is local and gauge-invariant, hence the dynamics it describes it is fully contained in $\Hphys$.
Furthermore, the operator $U_{\square}$ plays the role of a \emph{magnetic} term, to be more precise it is the magnetic flux inside the plaquette $\square$, while $V$ is the \emph{electric} term.
The coupling $\lambda$ tunes the relative strength of the electric and magnetic energy contribution.

These models are akin to the Toric Code \cite{kitaev2003fault}, which can be thought as a prime example of a $\Z_2$ lattice gauge theory.
More precisely, $H_{\Z_2}$ in \eqref{eq:hamiltonian_base} can be thought as a \emph{deformation} of the former, where an external ``transverse'' field is added to it.
Indeed, using the notation used so far, the Toric Code can be formulated as:
\begin{equation}
    H_{\text{TC}} = - J_m \sum_{\square} U_{\square} - J_e \sum_{x} G_x.
    \label{eq:hamiltoniana_toric_code}
\end{equation}
whose ground states $\ket{\Psi}$ satisfies the constraints
\begin{equation}
    U_{\square} \ket{\Psi} = \ket{\Psi} \;\; \forall \; \square, \quad
    G_x \ket{\Psi} = \ket{\Psi} \;\; \forall x.
    \label{eq:constraints_gs_toric_code}
\end{equation}
Only elementary excitations above the ground state can violate these constraints and they can be of two type: a \emph{magnetic vortex} (which violates the plaquette constraint) or a \emph{electric charge} (which violates the Gauss law).
If one imposes $J_e \gg J_m$ to enforce Gauss law, in the low-energy sector there are no electric charges and one recovers the pure gauge $\Z_2$ model of \eqref{eq:hamiltonian_base} for $\lambda = 0$.
Therefore, in general the $\Z_N$ models described in \eqref{eq:hamiltonian_base} can be considered as generalization of the Toric Code, from the point of view of lattice gauge theories.


\subsection{Superselection sectors}
\label{sub:superselection_sectors}



\begin{figure}[t]
    \centering
    \begin{tikzpicture}
    % lattice grid
    \draw[lattice, step=1] (-0.5,-0.5) grid (4.5,4.5);

    % Wilson loops
    \draw[X] (-0.5, 1) -- (0, 1);
    \draw[X] (4, 1) -- (4.5, 1)
        node [pos=1, right] {$\mathcal{L}_1$}
        node [pos=0, above left, black] {$\Wilson_1$};
    \foreach \x in {0,...,3} \draw[U] (\x, 1) -- ++(1, 0);

    \draw[X] (3, -0.5) -- (3, 0)
        node [pos=0, below] {$\mathcal{L}_2$}
        node [pos=0.15, left, black] {$\Wilson_2$};
    \draw[X] (3, 4) -- (3, 4.5);
    \foreach \y in {0,...,3} \draw[U] (3, \y) -- ++(0, 1);

    % 't Hooft strings
    \draw[Z, dashed, ->-=0.4]
        (-0.5,3.5) -- (4.5,3.5)
        node[pos=0, left] {$\mathcal{C}_1$}
        ;
    \foreach \x in {0,...,4} { \draw[Z, ->-=0.40] (\x, 3) -- +(0, 1); }
    \draw (2,3) node [below right] {$\tHooft_1$};

    \draw[Z, dashed, ->-=0.2]
        (0.5,-0.5) -- (0.5,4.5)
        node[pos=1, above] {$\mathcal{C}_2$}
        ;
    \foreach \y in {0,...,4} { \draw[Z, ->-=0.40] (0, \y) -- +(1, 0); }
    \draw (1, 2) node [below right] {$\tHooft_2$};

    % Sites
    \DrawSites{0,...,4}{0,...,4};
\end{tikzpicture}

    \caption{Graphical representation of the non-local order parameters $\overline{W}_{1,2}$ (in blue) and $\overline{S}_{1,2}$ (in red) and their respective paths $\mathcal{C}_{1,2}$ and $\tilde{\mathcal{C}}_{1,2}$.}
    \label{fig:nonlocal_operators}
\end{figure}

Let us consider the Toric Code.
One of its main features is the presence of topologically protected degenerate ground states \cite{kitaev2003fault}.
In order to illustrate this, besides $\overline{W}_1$ and $\overline{W}_2$, defined in \eqref{eq:top_wilson_loop}, another type of non-local operators have to be introduced.
They are defined on \emph{cuts} of the lattice $\lattice$, i.e.~paths on the dual lattice $\tilde{\lattice}$.
Consider \emph{non-contractible} cuts $\tilde{\mathcal{C}}_1$ and $\tilde{\mathcal{C}}_2$ along the directions $\hat{1}$ and $\hat{2}$, respectively.
On this cuts, the ('t Hooft string) operators $\overline{S}_1$ and $\overline{S}_2$ are constructed as
\begin{equation}
    \overline{S}_i = \prod_{\ell \in \tilde{\mathcal{C}}_i} V_\ell, \qquad i=1,2,
    \label{eq:top_string_operators}
\end{equation}
in a similar fashion to \eqref{eq:top_wilson_loop}.
This is shown in red in Fig.~\ref{fig:nonlocal_operators}.
The operators $\overline{W}_i$ and $\overline{S}_i$ ($i=1,2$) commutes with all the operators $U_{\square}$ and $G_x$ in the Toric Code Hamiltonian $H_{\text{TC}}$ of \eqref{eq:hamiltoniana_toric_code}, but do not commute with each other.
In fact, we have $\overline{W}_i \overline{S}_j = - \overline{S}_j \overline{W}_i$ if $i \neq j$.
This means that $H_{\text{TC}}$ can be block-diagonalized with respect to the eigenvalues of $\overline{S}_i$ (or $\overline{W}_i$), while $\overline{W}_j$ (or $\overline{S}_j$) connects one block to the other.
Furthermore, since in the case of the  $\Z_2$ symmetry,
$\overline{S}_i$ (or $\overline{W}_i$) has only two eigenvalues (equal to $\pm 1$), there are a total of $2 \times 2 = 4$ degenerates ground states, which are topologically protected, thanks to the fact
that $\overline{W}_j$ (or $\overline{S}_j$) cannot be expressed in terms of the local operators $U_{\square}$ and $G_x$.
Notice that, as it can be easily seen, in the Toric Code the role of $\overline{W}_i$ and $\overline{S}_i$ can be interchanged.

Let us now turn to $\Z_N$ LGT models.
The operators $\overline{W}_i$ no longer commute with the Hamiltonian \eqref{eq:hamiltonian_base} which now contains an electric field term.
Thus, $\lambda \neq 0$, we have no degenerate ground states.
But we can still use the $\overline{S}_i$ operators to decompose the Hilbert space $\Hphys$, since they still commute with all the \emph{local operators} $U_{\square}$ and $V_{\ell}$ (thus also with $H_{\Z_N}$).
Now one can see that the operator $\overline{S}_i$ ($i=1,2$) of \eqref{eq:top_string_operators} has $N$ eigenvalues $\omega^n$, with $n=1, \dots, N-1$.
Hence, one can decompose $\Hphys$ as sum of superselection sectors
\begin{equation}
    \Hphys = \bigoplus_{n, m=0}^{N-1} \Hphys^{(n, m)},
    \label{eq:decomposizione_Hphys}
\end{equation}
where for each $\ket{\phi} \in \Hphys^{(n, m)}$ we have:
\begin{equation}
    \overline{S}_1 \ket{\phi} = \omega^{m}\ket{\phi}, \quad
    \overline{S}_2 \ket{\phi} = \omega^{n}\ket{\phi}.
\end{equation}
Let us consider now the role of the Wilson loops $\overline{W}_i$.
One can easily see that:
\begin{equation}
    \overline{W}_2 \overline{S}_1 = \omega \overline{S}_1 \overline{W}_2, \qquad
    \overline{W}_1 \overline{S}_2 = \omega \overline{S}_2 \overline{W}_1.
    \label{eq:algebra_op_nonlocali}
\end{equation}
It follows that $\overline{W}_{1,2}$ acts a shift operators for the eigenspaces of $\overline{S}_{2,1}$:
\begin{equation}
    \overline{W}_1 : \Hphys^{(n, m)} \to \Hphys^{(n + 1, m)}, \quad
    \overline{W}_2 : \Hphys^{(n, m)} \to \Hphys^{(n, m + 1)},
    \label{eq:azione_wilson_loop}
\end{equation}
where the integers $n + 1$ and $m + 1$ have to be taken $\mathrm{mod}\; N$.

From a physical point of view, the Wilson loops operators $\overline{W}_1$ and $\overline{W}_2$ create non-contractible electric loops around the lattice, while  the 't Hooft strings $\overline{S}_2$ and $\overline{S}_1$ detect the presence and the strength of these electric loops.
Therefore, it is clear that the Hilbert subspace $\Hphys^{(n, m)}$ is the subspace of all the states that contains an electric loop of strength $\omega^n$ and $\omega^{m}$ along the $\hat{1}$ and $\hat{2}$ direction, respectively.
Furthermore, the evolution of a state in $\Hphys^{(n,m)}$ with the Hamiltonian in \eqref{eq:hamiltonian_base} is confined in $\Hphys^{(n,m)}$.


% SECTION: Abelian models on the ladder
%--------------------------------------------------
% SECTION: Abelian models on the ladder
%--------------------------------------------------
\section{Abelian models on the ladder}
\label{sec:abelian_models_on_the_ladder}


\begin{figure}
    \centering
    \begin{tikzpicture}[
        scale=0.75,
        font=\small
    ]
    \draw[lattice] (-3,0) grid (9,2);

    \DrawSites{0,2,...,6}{0,2}

    \draw[Gray] (2,0) node [below] {$x^\uparrow$};
    \draw[Gray] (2,2) node [above] {$x^\downarrow$};

    \draw (2,1) node [left]  {$\runglink_x$};
    \draw (4,1) node [right]  {$\runglink_{x+1}$};
    \draw (3,0) node [below=4pt] {$\botlink_x$};
    \draw (3,2) node [above=4pt] {$\toplink_x$};
    \draw (3,1) node {$\square_x$};
\end{tikzpicture}

    \caption{Ladder geometry and labeling of its links.}%
    \label{fig:ladder_geometry}
\end{figure}


The main goal of this manuscript is the characterization of the phases of the model described above, but on a ladder geometry.
The peculiarity of this geometry is given by the fact that we can study a (quasi)one-dimensional non-trivial LGT with magnetic terms, which are not possible in pure one-dimensional systems.
Moreover, since the Hilbert space is highly constrained, we have the possibility to study systems of moderate size through exact diagonalization.
The latter will be analyzed in the last section.

% A $\Z_N$ LGT on a ladder geometry of dimensions $L\times 2$, with periodic boundary conditions (PBC) only along the $\hat{1}$ direction, is described as follows (see Fig.~\ref{fig:ladder_geometry}):
%
% \begin{itemize}
%     \item the $2L$ sites are indexed as $(x,y)$, where $x = 0, \dots, L-1$ and $y = 0,1$ with $(L,y) \equiv (0,y)$;
%     \item the $L$ plaquettes are indexed as $\square_x$, where $x = 0, \dots, L-1$ with $\square_L \equiv \square_0$;
%     \item the $3L$ links are subdivided in top links $\toplink_x$, bottom links $\botlink_x$ and vertical links $\runglink_x$.
%         For all of them $x = 0, \dots, L-1$ and $\ell^a_L \equiv \ell^a_0$ ($a = \uparrow, \downarrow, 0$).
% \end{itemize}
A \emph{ladder} is a lattice $\mathbb{L}$ made of two parallels chains, the \emph{legs}, coupled to each other by \emph{rungs} to form square plaquettes.
On the ladder, each rung is identified by a coordinate $i=1,\dots,L$, where $L$ is the length of the ladder, and the two vertices on the rung are denoted with $i^{\uparrow}$ and $i^{\downarrow}$ in the upper and lower leg, respectively.
Links are denoted by $\ell$.
On the legs they are labelled as $\toplink_i$ (upper leg) or $\botlink_i$ (lower leg), while on the rungs they are labelled $\runglink_i$.

In order to lighten our notation,
we use the symbols $V^0_i, \; U^0_i$ for the operators defined on the rung $i$, and  $V^{\rho}_i, \; U^{\rho}_i$ with $\rho = \uparrow, \downarrow$ for the operators on the horizontal links of the upper and lower leg to the right of the rung.
Also, the plaquette operators on the right of the rung $i$ will be labeled as $U_i$:
% \begin{equation*}
%     \begin{split}
%         U_{\runglink_i} & \equiv \Urung_i, \quad
%         U_{\botlink_i}   \equiv \Udown_i, \quad
%         U_{\toplink_i}   \equiv \Uup_i \\
%         V_{\runglink_i} & \equiv \Vrung_i, \quad
%         V_{\botlink_i}   \equiv \Vdown_i, \quad
%         V_{\toplink_i}   \equiv \Vup_i,
%     \end{split}
% \end{equation*}
% while for the plaquette operator
\begin{equation}
    U_i = \Udown_i \Urung_{i+1} (\Uup_i)^{\dagger} (\Urung_i)^{\dagger}.
    \label{eq:plaq_op_ladder}
\end{equation}
Moreover, on a ladder only three-legged vertices exist, so the Gauss operators are slightly modified:
% due to the geometric constraints (only three-legged vertices exist).
% The Gauss operators on the top and bottom vertices on the ladder become respectively
\begin{equation}
    \GaussUp_i
    = \Vup_i ( \Vup_{i-1} )^\dagger ( \Vrung_i )^\dagger \text{~and~}
    \GaussDown_i
    = \Vdown_i \Vrung_i ( \Vdown_{i-1} )^\dagger,
    \label{eq:gauss_law_ladder}
\end{equation}
where $\GaussUp_i$ and $\GaussDown_i$ refers, respectively, to the Gauss operators on the vertices $i^{\uparrow}$ and $i^{\downarrow}$.
% For later convenience, we write explicitly the plaquette operator $U_{x}$
% \begin{equation}
%     U_x = U_{\botlink_x} U_{\runglink_{x+1}} U_{\toplink_x}^\dagger U_{\runglink_x}^\dagger.
%     \label{eq:plaq_op_ladder}
% \end{equation}
Finally, we write explicitly the Hamiltonian for a $\Z_N$ LGT on a ladder:
\begin{equation}
    H_{\Z_N}^{\text{lad}}(\lambda) =
    - \sum_{i} \bqty{ U_i + \lambda \pqty{ \Vup_i + \Vdown_i + V^0_i } + \text{h.c.} }.
    \label{eq:ladder_hamiltonian}
\end{equation}


For what concerns topological sectors of the theory,
out of the Wilson loop operators in \eqref{eq:top_wilson_loop} only $\overline{W}_1$ is well defined, because we have periodic boundary conditions only along the $\hat{1}$ direction.
Hence, only $\overline{S}_2$ in \eqref{eq:top_string_operators} (the 't Hooft operator conjugate to $W_1$) can be used as a mean for distinguishing these different sectors.
Therefore the decomposition of the Hamiltonian is realized with blocks of the type $(\omega^k,1)$, with $k=0,\dots,N-1$.
One of the main features of this decomposition is that, once we have fixed the topological sector, it is possible to write the duality transformation of the block Hamiltonian, which leads to a one-dimensional quantum clock model, with a chiral longitudinal field.
The latter is the object of discussion of the following sections.


\begin{figure}
    \centering
    \input{assets/figures/ladder_operators.tex}
    \caption{Representation of the different ladder operators. On the right: plaquette operator $U_x$. On the left: the electric operators $\Vup$, $\Vdown$ and $V^0$.}
    \label{fig:ladder_operators}
\end{figure}


% SECTION: Bond-algebra approach to dualities
%----------------------------------------
% SECTION: Dualities in physics
%----------------------------------------
\section{Dualities in physics}%
\label{sec:dualities_in_physics}


Duality is a simple yet powerful idea in physics.
They can be intended as specific mathematical transformations connecting seemingly unrelated physical phenomena.
They have been know for a long time, indeed a first example would be the duality of the electromagnetic field in the absence of sources, noticed by Heaviside in 1884.
Generally in physics, the concept of duality is connected to ideas, like symmetries, mappings between different coupling regimes, perturbative expansions for strongly correlated systems, and the wave-particle duality of quantum mechanics \cite{savit1980duality, cobanera2011bond}.

They play a major role in statistical physics and condensed matter.
In statistical mechanics, dualities were introduced for the first time by Kramers and Wannier \cite{kramers1941statistics}, who found a relation between the high temperature and low temperature regimes of the two-dimensional Ising mode.
In this way, they were able to find the critical temperature years before Onsager solution \cite{onsager1944ising}.
In this case we speak of self-dualities, where the same model is mapped onto itself but in a different coupling regime.
The essential legacy of Kramers and Wannier is the fact that self-dualities can put constraints on the phase boundaries and the exact location of critical points.

Not all dualities are self-dualities.
In fact, it also possible to relate two apparently different physical models with a duality transformation.
A known example is the Jordan-Wigner transformation \cite{schultz1964ising, jordan1928pauli}, where spin degrees of freedom (which are bosonic in nature) are mapped onto fermionic degrees of freedom in one-dimension \todo{elaborare}.
This duality shows that, in fact, there is not much difference between bosonic and fermionic degrees of freedom.

\todo{forse aggiungere qualcosa in più}

%
% SUBSECTION: The bond-algebraic approach
%
\subsection{The bond-algebraic approach}
\label{sub:the_bond_algebraic_approach}

In the following section we will quickly review the bond-algebraic approach to dualities \cite{cobanera2011bond}, because it offers a powerful and convenient way for dealing with duality transformations, in particular when gauge symmetries are involved.
The concept of \emph{bond-algebra} was first introduced in \cite{nussinov2009bond} and it exploits the fact that most \emph{Hamiltonian are a sum of simple and (quasi)local terms}:
\begin{equation}
    H = \sum_{\Gamma} \lambda_{\Gamma} h_{\Gamma},
\end{equation}
where $\Gamma$ is a set of indices (e.g.~the lattice sites but can be completely general) and $\lambda_{\Gamma}$ are numbers (usually the couplings).
The terms $h_{\Gamma}$ are the \emph{bond operators} (or simply \emph{bonds}).
They involve at most few degrees of freedom which are locally near.
From the bonds $h_{\Gamma}$ we obtain a \emph{bond algebra} $\algebra\{h_{\Gamma}\}$, which is the algebra of all the operators generated by all the possible products and sums of the bonds $h_{\Gamma}$ and their Hermitian conjugates.
In practical terms, given a set of bonds $\{h_{\Gamma}\}$, the bond-algebra $\algebra\{h_{\Gamma}\}$ is the algebra spanned by
\begin{equation}
    \{
        \identity, h_{\Gamma}, \,
        h_{\Gamma}^{\dagger}, \,
        h_{\Gamma} h_{\Gamma^{\prime}}, \,
        h_{\Gamma}^{\dagger}  h_{\Gamma^{\prime}}, \,
        h_{\Gamma} h_{\Gamma^{\prime}}^{\dagger}, \,
        h_{\Gamma}^{\dagger}  h_{\Gamma^{\prime}}^{\dagger}, \,
        h_{\Gamma} h_{\Gamma^{\prime}} h_{\Gamma^{\prime\prime}},
        \dots
    \}
\end{equation}
By construction, $\algebra\{h_{\Gamma}\}$ is closed under the operation Hermitian conjugation, but since an Hamiltonian $H$ is Hermitian then $h_{\Gamma}^{\dagger} = h_{\Gamma^{\prime}}$ for some $\Gamma^{\prime}$.
Therefore, $\algebra \{h_{\Gamma}\}$ is simply spanned by
\begin{equation}
    \{
        \identity,
        h_{\Gamma}, \,
        h_{\Gamma} h_{\Gamma^{\prime}}, \,
        h_{\Gamma} h_{\Gamma^{\prime}} h_{\Gamma^{\prime\prime}}, \,
        \dots
    \}
\end{equation}

To make an example, consider the quantum Ising model with a transverse field, which is a chain of spin-\onehalf with the Hamiltonian
\begin{equation}
    \HIsing(h) = \sum_{i} \qty( \sigma^z_i \sigma^z_{i+1} + h \sigma^x_i ),
\end{equation}
where the sums runs over the sites of the chain and $h$ is the transverse field strength.
Notice that the Hamiltonian is sum of quasi-local terms.
In particular two types of terms: the interaction term $\sigma^z_i \sigma^z_{i+1}$ and the transverse field $\sigma^x_i$.
They are local or quasi-local because they involve at most two neighbouring sites.
These two set of terms are the bonds of the Hamiltonian $\HIsing$ and our bond-algebra $\algebra \{\sigma^z_i \sigma^z_{i+1}, \sigma^x_i\}$ is spanned by
\begin{equation}
    \{
        \identity, \,
        \sigma^z_i \sigma^z_{i+1}, \,
        \sigma^z_i \sigma^z_{i+1} \sigma^z_j \sigma^z_{j+1}, \,
        \dots, \,
        \sigma^x_i, \,
        \sigma^x_i \sigma^x_j, \,
        \dots, \,
        \sigma^z_i \sigma^z_{i+1} \sigma^x_i, \,
        \dots
    \}.
\end{equation}
\todo{è necessario?}


It is important to point out that a single Hamiltonian $H$ can have different bond algebras associated to it.
In fact, a bond algebra is determined by how the bonds of $H$ are decomposed or partitioned.
In principle, given any two decomposition of the same Hamiltonian,
\begin{equation*}
    H
    = \sum_{\Gamma} \lambda_{\Gamma} h_{\Gamma}
    = \sum_{\Sigma} \lambda^\prime_{\Sigma} h^\prime_{\Sigma},
\end{equation*}
one should expect $\mathcal{A}\{h_{\Gamma}\} \neq \mathcal{A}\{ h^{\prime}_{\Sigma} \}$ in general (see \cite{cobanera2011bond}).
Furthermore, the bonds $h_{\Gamma}$ that generate $\mathcal{A}\Bqty{ h_{\Gamma} }$ do not need to be independent.

In this framework, quantum dualities can be formulated as \emph{homomorphisms of bonds algebras}, i.e.~structure preserving mappings between bond algebras.
To be more precise, two Hamiltonian $H_1$ and $H_2$ that act on state spaces of the same dimensions are said to be \emph{dual} if there is some bond algebra $\mathcal{A}_{H_1}$ of $H_1$ that is homomorphic to some bond algebra $\mathcal{A}_{H_2}$ of $H_2$ and if the homomorphism $\Phi : \mathcal{A}_{H_1} \to \mathcal{A}_{H_2}$ maps $H_1$ onto $H_2$, $\Phi(H_1) = H_2$.
These mappings do not need to be isomorphisms, especially when gauge symmetries are involved, and we will explain why later.

Dualities, in this approach, are \emph{local} with respect to the bonds, i.e.~they map one bond $h_{\Gamma_1}$ of $H_1$ to one bond $h_{\Gamma_2}$ of $H_2$, which may translates to non-locality with respect to the elementary degrees of freedom.
This is due to the fact that the generators of a bond algebra are usually two- (or more) body operators and expressing the elementary degrees of freedom with these operators require large (if not infinite) products.

To make this approach clearer we now apply it to the 1D quantum Ising model with transverse field.
The Hamiltonian $\HIsing$ of this model
\begin{equation}
    \HIsing(\lambda) = \sum_{i}  \pqty{ \sigma_i^z \sigma_{i+1}^z + \lambda \sigma_i^x }
\end{equation}
where $\sigma^x_i$ and $\sigma^z_i$ are the usual Pauli matrices for spin $S= \frac{1}{2} $.
We recognize as basic bonds the operators $\{\sigma^x_i\}$ and $\{\sigma^z_i \sigma^z_{i+1}\}$ and their relations can be summarized as follows: (i) each bonds square to the identity operator; (ii) the bonds $\sigma_i^x$ anticommutes only with $\sigma^z_i \sigma^z_{i+1}$ and $\sigma^z_{i-1} \sigma^z_i$ and commutes with every other bond; (iii) the bonds $\sigma^z_i \sigma^z_{i+1}$ anticommutes only with $\sigma^x_i$ and $\sigma^x_{i+1}$.
Given the symmetric roles that the basic bonds $\sigma^x_i$ and $\sigma^z_i \sigma^z_{i+1}$ play with each other, one can set up a mapping $\Phi$ as follows:
\begin{equation}
    \Phi(\sigma^z_i \sigma^z_{i+1}) = \sigma^x_i, \qquad
    \Phi(\sigma^x_i) = \sigma^z_{i-1} \sigma^z_i.
    \label{eq:duality_ising}
\end{equation}
This transformation defined on the bond generators alone extends to the full bond-algebra $\mathcal{A}_{\text{Ising}}$ and it is clear that preserves all the important algebraic relationship and is one-to-one, hence it is an \emph{isomorphism} of $\mathcal{A}_{\text{Ising}}$ onto itself.
The Hamiltonian $\HIsing$ is just an element of $\mathcal{A}_{\text{Ising}}$ and through $\Phi$ gets transformed as
\begin{equation}
    \Phi( \HIsing(\lambda) )
    = \sum_{i} \pqty{ \sigma_i^x + \lambda \sigma^z_{i} \sigma^z_{i+1}} \\
    = \lambda \HIsing(\lambda^{-1}),
\end{equation}
which is just what we stated in \todo{manca eqref}
%\eqref{eq:ising_hamiltonian_duality},
i.e.~it maps into itself with inverted couplings.

An isomorphism like $\Phi$ is physically sound if it is \emph{unitarily implementable} \cite{cobanera2011bond}, which means that there is a unitary matrix $\mathcal{U}$ such that the duality isomorphism reads
\begin{equation}
    \Phi(\mathcal{O}) =
    \mathcal{U} \mathcal{O} \mathcal{U}^{\dagger}, \quad
    \forall \mathcal{O} \in \mathcal{A},
\end{equation}
where $\mathcal{A}$ is the operator algebra of the model under investigation.

%
% SUBSECTION: Gauge-reducing dualities
%
\subsection{Gauge-reducing dualities}%
\label{sub:gauge_reducing_dualities}

In this section we will review the notion of \emph{gauge-reducing dualities}.
Gauge symmetries are \emph{local symmetries} of the model that signal the presence of \emph{redundant degrees of freedom}, in fact gauge invariance can be thought as a set of \emph{local constraints} on the elementary degrees of freedom of the model.
This means that the state space of the model is larger than set of physical states and these are exactly the states that are invariant under the action of the gauge symmetries, which would mean that they satisfies the local constraints of the model.
The same can be applied to the Hermitian operators and the observables of the model.
An Hermitian operator represent a physical observable only if it commutes with the gauge symmetries, which makes them gauge invariant.

When dealing with a gauge model, it would be natural to assume that, in order to establish a duality with any gauge symmetries, these have to be eliminated from the former model.
In other terms, that it would be necessary to project out the operator content on the subspace of physical states first or proceed with gauge-fixing.
Although this is a common and traditional approach to dualities, with bond algebras this is not strictly necessary.
As stated in \cite{cobanera2011bond}, with the bond-algebraic approach one can find mappings to models without any gauge symmetry that preserve all the important algebraic properties.

The procedure goes as follows: consider a gauge model and let $H_{\text{G}}$ be its Hamiltonian and $G_{\Gamma}$ its gauge symmetries.
Naturally, an operator is said to be gauge-invariant only if it commutes with all the $G_{\Gamma}$ and
clearly the Hamiltonian has to be gauge-invariant, hence $[H, G_{\Gamma}] = 0$.
Now let $H_{\text{GR}}$ be the dual Hamiltonian of a non-gauge model.
A \emph{gauge-reducing duality} $\Phi_{\text{GR}}$ maps $H_{\text{G}}$ onto $H_{\text{GR}}$ while making all the gauge symmetries of the former model trivial, which means:
\begin{equation}
    \Phi_{GR}(H_{\text{G}}) = H_{\text{GR}}, \qquad
    \Phi_{GR}(G_{\Gamma}) = \identity, \quad \forall \Gamma.
\end{equation}

Unlike the dualities in Sec.~\ref{sub:the_bond_algebraic_approach}, a gauge-reducing duality like $\Phi_{\text{GR}}$ has to be implementable as a \emph{projective unitary operator} $\mathcal{U}$.
Formally, this can be written as
\begin{equation}
    \Phi_{\text{GR}} ( \mathcal{O} ) = \mathcal{U} \mathcal{O} \mathcal{U}^\dagger, \quad
    \mathcal{U} \mathcal{U}^\dagger = \identity, \quad
    \mathcal{U}^\dagger \mathcal{U} = P_{\text{GI}}
\end{equation}
where $P_{\text{GI}}$ is the projector of the subspace of gauge-invariant states, i.e.~$G_{\Gamma} \ket{\psi} = \ket{\psi}$ for all $\Gamma$.
Roughly speaking, this projective unitary operator can be represented as rectangular matrix that preserves the norm of gauge-invariant states while projecting out all the other states.

A clear example of a gauge-reducing duality is provided by the $\Z_N$, $d=2$ gauge model

\begin{equation}
	H_{\text{G}} = \sum_r \left( V_{(r,\hat{1})} + V_{(r,\hat{2})} + \lambda U_r \right).
\end{equation}

Its group of gauge symmetries is generated by \eqref{eq:gauss_operator}. In the simplest case where $N=2$, the $V$'s can be represented by the Pauli operators $\sigma^z$ and the $U$'s by $\sigma^x$. In so doing, the Gauss operator becomes
\begin{equation}
	G_r =
	\sigma^z_{(r, \hat{1})}
	\sigma^z_{(r, \hat{2})}
	\sigma^z_{(r, -\hat{1})}
	\sigma^z_{(r, -\hat{2})},
	\label{eq:gauss_operator_Z2}
\end{equation}
and commutes with $H_{\text{G}}$ and with the bonds $\left\{\sigma^z_{(r,\hat{1})},\ \sigma^z_{(r,\hat{2})},\ U_r\right\} $.
In other words, the bond algebra they generate is gauge-invariant, and satisfy three simple relations: (i) all the bonds square to the identity, (ii) each spin $\sigma^z$ anti-commutes with two adjacent plaquettes operators $U$, and (iii) each plaquette operator $U$ anti-commutes with four spins $\sigma^z$.
This set of relations are identical to those satisfied by the bonds of the $d=2$ quantum Ising model, and the mapping is the following:

\begin{equation}
    \begin{split}
        \Phi\left( \sigma^z_{(r,\hat{1})} \right) & = \sigma^x_{(r-\hat{2})} \sigma^x_r, \\
        \Phi\left( \sigma^z_{(r,\hat{2})} \right) & = \sigma^x_{(r-\hat{1})} \sigma^x_r, \\
        \Phi(U_r) & = \sigma^z_r.
    \end{split}
    \label{eq:duality_2d}
\end{equation}

Thus, $\Phi$ maps $H_{\text{G}}$ to $H_{\text{Ising}}$, if we identify the constants $\lambda \leftrightarrow h$ and $1 \leftrightarrow J$.
Moreover, $\Phi$ is a gauge-reducing duality homomorphism, since $\Phi(G_r) = \identity$.
Therefore, $H_{\text{Ising}}$ represents all the physics contained in $H_{\text{gauge}}$, but without all the gauge redundancies.
In the general case, i.e.~for a generic $\Z_N$ symmetry, the duality leads to an $N$-clock model \cite{radicevic2019spin}.

% vim: spelllang=en


% SECTION: Dualities of ladder LGTs
\section{Dualities of the ladder}%
\label{sec:dualities_of_the_ladder}

%--------------------------------------------------
% SUBSECTION: Clock models
%--------------------------------------------------
\subsection{Quantum clock models}%
\label{sub:clock_models}

\Acp{clock} are a class of models that generalizes the \ac{ising} \cite{fendley2014parafermions, baxter1989clock}.
They show a resemblance to the $\Z_N$ \ac{lgt} models we introduced previously, in Sec.~\ref{sec:generalization_to_zn}.
In fact, this similarity will later be exploited in order to obtain a gauge-reducing duality of the $\Z_N$ \ac{lgt} ladder models.

% For a discussion about clock models we start from the Hamiltonian of the \ac{ising} with transverse field, which can simply be written as
% \begin{equation}
%     H = - \sum_{i} \sigma^z_i \sigma^z_{i+1} - h \sum_{i} \sigma^x_i,
%     \label{eq:ising_hamiltonian_duality}
% \end{equation}
% where $\sigma^{x,z}_i$ are the usual $2 \times 2$ Pauli matrices for each site $i$:
% \begin{equation}
%     \sigma^x_i = \pmqty{ 0 & 1 \\ 1 &  0 }, \quad
%     \sigma^z_i = \pmqty{ 1 & 0 \\ 0 & -1 }.
% \end{equation}
The Hamiltonian \eqref{eq:hamiltonian_ising}  of \ac{ising}, with trasverse field, uses Pauli matrices $\sigma^z$ and $\sigma^x$ as basic operators and they have the fundamental property that they \emph{anticommutes} on the same site, $\acomm{\sigma^z_i}{\sigma^x_i} = 0$
This relation rewritten as
\begin{equation}
    \sigma^z_i \sigma^x_i = - \sigma^x_i \sigma^z_i,
    \label{eq:anticommutation_Pauli_matrices}
\end{equation}
which be read as follows:\emph{if the two operators are exchanged, then a phase $-1$ is acquired}.
Another important fact about Pauli matrices we want highlight is that they \emph{square to the identity}:
\begin{equation}
    (\sigma^x_i)^2 = (\sigma^z_i)^2 = \identity.
    \label{eq:identity_Pauli_matrices}
\end{equation}
% In a \ac{clock}, these Pauli matrices
% They are a set of unitary matrices that commute on different sites, while on the same site they anticommute $\sigma^x \sigma^z = - \sigma^z \sigma^x$.
% Another way to put it is to say that the exchange of $\sigma_x$ and $\sigma_z$ on the same site produces a phase $e^{i \pi} = -1$.

\Ac{clock} are generalizations of the \ac{ising}, but not to higher spins.
A $p$-state \ac{clock} (or simply a $p$-clock model) utilizes a set of unitary operators that generalizes \eqref{eq:anticommutation_Pauli_matrices} and \eqref{eq:identity_Pauli_matrices} in the following sense:
the operators $\sigma^x$ and $\sigma^z$ are promoted to the \emph{clock operators} $X$ and $Z$, respectively;
they are $p \times p$ unitary matrices whose exchange produces a phase $\omega = e^{i 2 \pi / p}$ and their $p$-th power is equal to the identity.
The algebraic properties of these clock operators $X$ and $Z$ can be summarized as follows:
\begin{equation}
    \begin{aligned}
        X Z & = \omega Z X, &
        X^p & =  Z^p = \identity_p, \\
        X^\dagger & = X^{-1} = X^{p-1}, &
        Z^\dagger & = Z^{-1} = Z^{p-1}
    \end{aligned}
    \label{eq:clock_operator_algebra}
\end{equation}

We see that the Schwinger-Weyl algebra in \eqref{eq:schwinger_weyl_algebra} and the clock operator algebra in \eqref{eq:clock_operator_algebra} are basically the same, but there are some key differences to point out betweens a $\Z_N$ \ac{lgt} and a $p$-clock model.

The \ac{dof} of a $\Z_N$ \ac{lgt} live on the links of the lattice while in a $p$-clock model they live on the sites.
But the most important aspect is that we don't have any gauge symmetry in a $p$-clock model, hence we do not have to impose any local constraints or physical conditions.
These models can be derived as the quantum Hamiltonians of the classical 2D vector Potts model, which is a discretization of the 2D planar XY model \cite{ortiz2012dualities}.

A typical $p$-clock model Hamiltonian with transverse field has the form
\begin{equation}
    \HamilClock(\lambda) = - \sum_{i} Z_i Z_{i+1} - \lambda \sum_{i} X_i + \hc
    \label{eq:clock_hamiltonian}
\end{equation}
which is, as expected, very similar to the quantum Ising Hamiltonian in \eqref{eq:hamiltonian_ising}.
Furthermore, just like the latter, $p$-clock models with only transverse field are \emph{self-dual}:
the clocks can be mapped into the kinks (or domain walls) and one would obtain the same exact Hamiltonian description but with inverted transverse field \cite{ortiz2012dualities}.
For $p < 5$, the clock models presents a self dual point in $\lambda = 1$, that separates an ordered phase from a disordered one.
On the other hand, for $p \geq 5$ we have an intermediate continuous critical phase between the ordered and disordered phase with two BKT transition points, which are related to each other through the self-duality \cite{sun2019phase}.

These models have been thoroughly studied, even with the addition of a longitudinal field $\propto Z_i$ \cite{baxter1982exactlysm} or chiral interactions.
In particular, in the case of chiral interactions, it was shown \cite{fendley2012parafermions} that the Hamiltonian \eqref{eq:clock_hamiltonian} can be mapped to a parafermionic chain through a Fradkin-Kadanoff transformation, and in presence of a $\mathbb{Z}_3$ symmetry, it shows three different phases \cite{zhuang2015clock}, if open boundaries are implemented: a trivial, a topological and an incommensurate (IC) phase.
The case which presents a real longitudinal field term was considered in \cite{huang2019clock},  where some of the critical exponents have been estimated.
The general case, where chiral interactions are included in a $\mathbb{Z}_N$ model, has been studied in \cite{fendley2012parafermions}.
Here, the author considered the model as an extension of the Ising/Majorana chain and found the edge modes of the theory.
He also calculated the points, in the parameter space, where the model is integrable or `superintegrable'.
All these studies are motivated by theoretical interest and recent experiments, which can be analysed by the above models \cite{bernien2017probing}.


% vim: spelllang=en



\begin{figure}[t]
    \centering
    \begin{tikzpicture}
    % ladder
    \draw[ladder] (-0.5,0) grid (7.5,1);

    % Plaquette operator
    \draw[U] (1,0) -- (2,0);
    \draw[U] (2,0) -- (2,1);
    \draw[U] (2,1) -- (1,1);
    \draw[U] (1,1) -- (1,0);
    \draw[box] (0.75,-0.25) rectangle (2.25, 1.25) node [above left, Blue] {$U_x$};

    % horizontal electric operators
    \draw[V] (4,0) -- (5,0);
    \draw[V] (4,1) -- (3,1);
    \draw[box] (3.75,-0.25) rectangle (5.25, 0.25) +(-0.25,0) node [above left, Red] {$\Vdown_x$};;
    \draw[box] (2.75, 0.75) rectangle (4.25, 1.25) node [above left, Red] {$\Vup_x$};

    % vertical electric operator
    \draw[V] (6,0) -- (6,1);
    \draw[box] (5.75,-0.25) rectangle (6.25, 1.25) node [above, Red] {$V^0_x$};

    % ladder sites
    \DrawSites{0,...,7}{0,1};

    % chain
    \draw[ladder] (-0,-2) -- (7,-2);

    % clock operators
    \draw (1.5,-2) node [X site] {} node [text=Blue, below=5pt] {$X_i$};
    \draw (3.5,-2) node [Z site] {} node [text=Red, below=5pt] {$Z^\dagger_i$};
    \draw (4.5,-2) node [Z site] {} node [text=Red, below=5pt] {$\omega^k Z_i$};
    \draw [Z] (5.5,-2) -- (6.5,-2) node [pos=0.5, below=5pt] {$Z^\dagger_i Z_{i-1}$};

    % chain sites
    \foreach \x in {0,2,5,6} \draw (\x+0.5,-2) node [dual site] {};

    % arrows
    \draw[freccia] (1.5,-0.25) -- (1.5, -2);
    \draw[freccia] (3.5, 0.75) -- (3.5, -2);
    \draw[freccia] (4.5,-0.25) -- (4.5, -2);
    \draw[freccia] (6.0,-0.25) -- (6.0, -2);

    % labels
    \draw (-0.75,0.5) node [left, align=right] {$\Z_N$ ladder \\ LGT};
    \draw (-0.75, -2) node [left, align=right] {$N$-clock \\ chain};
\end{tikzpicture}

    \caption{Visual representation of the duality transformation from the $\Z_N$ ladder LGT to the $N$-clock model.
    The plaquette operator $U_x$ and the electric operators $\Vup$ and $\Vdown$ map to one-site operators in the clock model, while
    the remaining electric operator $V^0$ maps to a hopping term between nearest neighbouring sites.}
    \label{fig:ladder_duality}
\end{figure}

%--------------------------------------------------
% SUBSECTION: Duality on the clock models
%--------------------------------------------------
\subsection{Duality onto clock models}
\label{sub:duality_onto_clock_models}

In this section we will show how to construct a mapping of the $\Z_N$ ladder LGT onto a $N$-clock model on a chain with a transversal field and a longitudinal field, the latter depending on the topological sector of the ladder LGT.

The first step is the decomposition of the set of bonds in \eqref{eq:hamiltonian_base}.
Obviously, the magnetic terms $U_{\square}$ have to be separated from the electric terms $V_\ell$, but the latter cannot be all treated the same.
It is clear from the geometry of the ladder, that the links $\runglink$ have a different role when compared with the links $\toplink$ and $\botlink$, because the former are \emph{domain walls} while the latter are not.
Therefore, the duality transformation has to distinguish between the vertical links and horizontal links.
Furthermore, also the top links $\toplink$ and bottom links $\botlink$ have to be treated separately because the electric operator on them have different commutation relations with the plaquette operators.
In fact, using the notation introduced in Sec.~\ref{sec:abelian_models_on_the_ladder}, we have
\begin{equation}
    U_x \Vdown_x = \omega \Vdown_x U_x, \qquad
    U_x \Vup_x = \omega^{-1} \Vup_x U_x.
    \label{eq:comm_rel_ladder}
\end{equation}
and indeed they acquire different phases.

The plan is to associate to each plaquette a clock degree of freedom, hence we identify a plaquette $\square_x$ with a site $i$ of a clock chain and the magnetic flux of a plaquette becomes the ``fundamental gauge invariant degree of freedom'' of the LGT ladder model.
Given the fact that we are working in the electric basis, we chose for convenience to map the $\Z_N$ magnetic operator $U_x$ to the ``momentum'' operator $X_i$ of the $N$-clock chain.
The electric field on a vertical link $\runglink$ is the result of the flux difference between the two plaquettes that it separates, which suggests that the operator $V_{\runglink}$ have to be mapped to a kinetic-type term like $Z_i^\dagger Z_{i-1}$.
This can be readily verified.
From \eqref{eq:plaq_op_ladder} we get
\begin{equation*}
    V^0_x U_x = \omega^{-1} U_x V_x^0, \qquad
    V^0_x U_{x-1} = \omega U_{x-1} V_x^0,
\end{equation*}
therefore the maps
\begin{equation*}
    U_x \mapsto X_i, \qquad
    V^0_x \mapsto Z_i^\dagger Z_{i-1},
    \label{eq:elec_h_and_plaq_op_map}
\end{equation*}
clearly conserves the commutation relations of $U_x$ and $V^0_x$.

For now we are left with task of finding a suitable mapping of $\Vup$ and $\Vdown$.
With respect to the other bonds of the theory, both of them commute with $V^0$ while for \eqref{eq:comm_rel_ladder} holds for $U_x$.
Hence, a suitable and general mapping of $\Vup$ and $\Vdown$ can be:
\begin{equation}
    \Vdown_x \mapsto \coeffdown_i Z_i, \qquad
    \Vup_x \mapsto \coeffup_i Z_i^\dagger,
    \label{eq:elec_op_horiz_ladder_map}
\end{equation}
where $\coeffdown_i$ and $\coeffup_i$ are complex numbers.
Although, they cannot be any complex number.
Both $\Vdown_x$ and $\Vup_x$ have to be mapped onto unitary operators, which limits the numbers $\coeffdown_i$ and $\coeffup_i$ to be \emph{complex phases}.

To further constraint the value of these coefficients, we can use the Gauss law.
In particular, given the fact that we are looking for a gauge-reducing duality, the aim is to make the Gauss law trivial.
Using the mappings \eqref{eq:elec_h_and_plaq_op_map} and \eqref{eq:elec_op_horiz_ladder_map} in \eqref{eq:gauss_law_ladder} yields
\begin{equation}
    \begin{split}
        G^\uparrow_x & \mapsto
        (\coeffup_i Z_i^\dagger) (\coeffup_{i-1} Z_{i-1}^\dagger) (Z_i^\dagger Z_{i-1})^\dagger
        = \coeffup_i (\coeffup_{i-1})^*, \\
        G^\downarrow_x & \mapsto
        (\coeffdown_i Z_i^\dagger) (Z_i^\dagger Z_{i-1}) (\coeffdown_{i-1} Z_{i-1}^\dagger)
        = \coeffdown_i (\coeffdown_{i-1})^*
    \end{split}
    \label{eq:gauss_law_map_ladder}
\end{equation}

Gauss law have to be satisfied in a pure gauge theory, which mean that we have to impose $G^\uparrow_x = \identity$ and $G^\downarrow_x = \identity$ for all $x$.
This is only possible if
\begin{equation}
    \coeffdown_i = \coeffdown, \qquad
    \coeffup_i = \coeffup, \qquad
    \forall i.
\end{equation}

Furthermore, thanks to \eqref{eq:gauss_law_map_ladder}  we also know how to introduce static matter into this duality, because it can be thought as a violation of the Gauss law.
We just have to change the phases $\coeffup_i$ and $\coeffdown_i$.

The last factor to consider is how the $\coeffup$ and $\coeffdown$ are related on the same site $i$.
In this regard, the topological sectors of the theory come to the rescue.
As established in Sec.~\ref{sec:abelian_models_on_the_ladder}, the topological sectors are identified by the eigenvalue of $S_2$ in \eqref{eq:nonlocal_op_ZN}, which in the ladder geometry becomes
\begin{equation}
    S_2 = \Vup_x \Vdown_x
    \label{eq:top_string_op_ladder}
\end{equation}
for any fixed $x$.
Its eigenvalue are simply $\omega^k$, for $k = 0, \dots, N-1$.

Given a topological sector $\omega^k$, using the mapping \eqref{eq:elec_op_horiz_ladder_map} on \eqref{eq:top_string_op_ladder} yields
\begin{equation}
    S_2 \; \longmapsto \; ( \coeffup Z^\dagger_i ) ( \coeffdown Z_i ) = \coeffup \coeffdown = \omega^k.
\end{equation}
From here, in order to solve for the coefficients $\coeffup$ and $\coeffdown$, one needs only to fix one of the to $1$ and the other to $\omega^k$.
We choose to fix these coefficients as follows:
\begin{equation}
    \coeffup = 1, \qquad
    \coeffdown = \omega^k.
\end{equation}

In conclusion, we summarize the duality mapping for the topological sector $\omega^k$ of the $\Z_N$ LGT on a ladder:
\begin{equation}
    \begin{aligned}
        U_x      & \; \longmapsto \; X_i, \quad &
        V^0_x    & \; \longmapsto \; Z^\dagger_i Z_{i-1}, \\
        \Vup_x   & \; \longmapsto \; Z_i^\dagger, \quad &
        \Vdown_x & \; \longmapsto \; \omega^k Z_i.
    \end{aligned}
    \label{eq:ladder_duality}
\end{equation}

With the duality \eqref{eq:ladder_duality}, from \eqref{eq:ladder_hamiltonian} in the sector $(\omega^k, 1)$ we obtain
\begin{equation}
    H_{\text{ladder}}(\lambda) \; \longmapsto \; \lambda H_{\text{clock}}(\lambda^{-1})
\end{equation}
where
\begin{equation}
    H_{\text{clock}}(\lambda^{-1}) =
    - \sum_{i} Z_i^\dagger Z_{i-1}
    - \lambda^{-1} \sum_{i} X_i
    - (1 + \omega^k) \sum_{i} Z_i
    + \text{h.c.}
    \label{eq:dual_ladder_hamiltonian}
\end{equation}

We see that \eqref{eq:dual_ladder_hamiltonian} is a clock model with both \emph{transversal} and \emph{longitudinal} fields.
In particular, the longitudinal field carries the information of the topological sector of the ladder model.

Interestingly, for $N$ even the sector $k = N/2$ has a special role.
Within this sector $\omega^k = -1$, for which the \emph{longitudinal field disappears} and $H_{\text{clock}}$ reduces to self-dual quantum clock models with a known quantum phase transition.
This phase transitions for $k = N/2$ can be put in correspondence with a \emph{confined-deconfined} transition, which will be discussed in much more detail in the next section.

Let us remark that the complex coupling $(1 + \omega^n)$ does not make the Hamiltonian  (\ref{eq:dual_ladder_hamiltonian}) necessarily chiral \cite{fendley2012parafermions, whitsitt2018clock}.
In fact, one can get the real Hamiltonian
\begin{equation}
    H_N = H_p(1/\lambda) - 2 \cos \pqty{\frac{\pi n}{N}} \sum_{i} \pqty{Z_i + Z_i^{\dagger}}.
    \label{eq:dual_ladder_hamiltonian_real}
\end{equation}
by absorbing the complex phase in the $Z_i$-operators, with the transformation $Z_i \mapsto w^{-n/2} Z_i$. This transformation globally rotates the eigenvalues of the $Z_i$-operators, while preserving the algebra relations.
For $n$ even, this is just a permutation of the eigenvalues, meaning that it does not affect the Hamiltonian spectrum. Instead, for $n$ odd, up to a reorder, the eigenvalues are shifted by an angle $\pi/N$, i.e.~half the phase of $\omega$.
The energy contribution of the extra term in \eqref{eq:dual_ladder_hamiltonian_real}  depends on the real part of these eigenvalues and for $n$ odd we obtain that the lowest energy state is no longer unique, in fact it is doubly degenerate.
This means that for $\lambda \to \infty$, where the extra term becomes dominant, we expect an ordered phase with a doubly degenerate ground state.
Finally, one can easily prove that the sectors $n$ and $N-n$ are equivalent
\footnote{For the sector $N-n$ we have that the overall factor $\cos(\pi(N-n)/N)$ is just $-\cos(\pi n/N)$.
The minus sign can then be again absorbed into the $Z$'s operators.
This overall operation is equivalent to the mapping $Z \mapsto \omega^{-n/2} Z$ for the sector $N-n$.}.


% SECTION: A case study: Z2, Z3 and Z4
\section{A case study: \texorpdfstring{$N=2, 3$}{N=2, 3} and \texorpdfstring{$4$}{4}}
\label{sec:a_case_study_N_2_3_4}


In this section we present the results of the numerical investigations of \cite{pradhan2022ladder}.
But first, we present the reasoning for the choice of order parameters used for investigating the phase diagram, and second we show how the duality have been used for resolving the Gauss law in numerics.


\subsection{Investigating the phase diagram}%
\label{sub:investigating_the_phase_diagram}

We wish to study the phase diagram of the $\Z_N$ \ac{lgt} phase diagram, in particular we are interested in any \emph{confined} or \emph{deconfined} phase.
In a pure gauge theory, these phases are investigated non-local order parameters like the \emph{\ac{wl}} (not be confused with the non-contractible \ac{wl}s in \eqref{eq:nonlocal_op_ZN}) or \emph{string tension}.
This is because we expect the deconfined phase to be a topological phase, which can be investigated only with non-local order parameters.

Given a closed region $\mathcal{R}$, a \ac{wl} operator $W_{\mathcal{R}}$ is defined as
\begin{equation}
    W_{\mathcal{R}} = \prod_{\square \in \mathcal{R}} U_{\square}.
    \label{eq:closed_wilson_loop}
\end{equation}
Alternatively, considering the oriented boundary $\partial \mathcal{R}$ one can write
\begin{equation}
    W_{\mathcal{R}} = \prod_{\ell \in \partial \mathcal{R}} U_{\ell},
\end{equation}
where the Hermitian conjugate is implied everytime we move in the negative directions.
It is also implied that the curve $\partial \mathcal{R}$ is a contractible loop.
Wilson showed in \cite{wilson1974confinement} that quark confinement is related to the expectation value $\ev{W_{\mathcal{R}}}$ of a \ac{wl}, which can be thought as a gauge field average on a region.
In particular, in the presence of quark confinement the gauge field average follows an \emph{area law}, where it decays exponentially with the area enclosed by $\mathcal{R}$.
On the other hand, in the deconfined phase we have a \emph{perimeter law}, where the gauge field average decays exponentially with the perimeter of $\mathcal{R}$.

Unfortunately on a ladder geometry there is not much difference between the area and the perimeter of a \ac{wl}.
In fact, in units of the lattice spacing, the area of a \ac{wl} over $n$ plaquettes is $n$ while its perimeter is just $2n+2$.
They both grow linearly.
Nonetheless, we can still look at the behaviour of the \ac{wl}, for a fixed length, at different couplings $\lambda$.

When the coupling $\lambda$ in \eqref{eq:hamiltonian_base} is equal to zero, the \ac{tc} is recovered and in any of its topological sector the ground state is the equal superposition of all the states with any number of closed electrical loops, in a similar fashion to coherent states.
This makes the \ac{tc} a \emph{quantum loop gas}, which is a \emph{deconfined phase}.
Furthermore, the operator $W_{\mathcal{R}}$ in \eqref{eq:closed_wilson_loop} creates an electrical loop around the region $\mathcal{R}$.
From the constraints
% TODO aggiustare ref
% \eqref{eq:constraints_gs_toric_code},
it can easily be proved that $W_{\mathcal{R}}$ leaves the \ac{tc} ground states unchanged, showing in fact that they behaves as coherent states, which leads to $\ev{W_{\mathcal{R}}} = 1$.

Therefore, $\ev{W_{\mathcal{R}}} \approx 1$ signals a deconfined phase and on the other hand a vanishing $\ev{W_{\mathcal{R}}} \approx 0$ corresponds to confined phase.
For this reason, even tough we lack an area/perimeter law on the ladder geometry it is still sensible to look at the behaviour of the \ac{wl}.


Another possible approach for investigating the phase diagram is to use the \emph{string tension}.
In two dimensions, given an \emph{open} curve $\tilde{\mathcal{C}}$ on the dual lattice $\tilde{\mathbb{L}}$ we can construct an open \ac{ths} operator $S_{\tilde{\mathcal{C}}}$ as
\begin{equation}
    S_{\tilde{\mathcal{C}}} = \prod_{\ell \in \tilde{\mathcal{C}}} V_{\ell}
\end{equation}
with the usual caveat: we have to take the Hermitian conjugate everytime the path goes in the negative direction.
Then the string tension is just the expectation value $\ev{S_{\tilde{\mathcal{C}}}}$ it is called in this way because it related to the potential energy (tension) between two magnetic fluxes created at the ends of the curve $\tilde{\mathcal{C}}$.
Henceforth, in a deconfined phase $\ev{S_{\tilde{\mathcal{C}}}} \approx 0$, which means that the magnetic fluxes can be moved freely with no cost in energy, like in the \ac{tc}.





% \section{Numerical analysis}%
% \label{sec:numerical_analysis}
%
% In this work we studied numerically the different topological sectors (and their phase diagrams) of the $\Z_N$ \ac{lgt} on ladder for $N=2,3,4$ through \emph{exact diagonalization} (ED).
% We chose ED instead of other variational methods like DMRG because we were able to construct exactly the Hilbert space of the different topological sectors of the models, exploiting the duality in Sec.~\ref{sec:dualities_of_the_ladder}.
%

\subsection{Implementing the Gauss law}%
\label{sub:implementing_the_gauss_law}

\begin{figure}[t]
    \centering
    \begin{tikzpicture}[
    font=\small,
    scale=0.75,
    % site/.style = {circle, inner sep=0 pt, minimum size=4pt, draw=black, fill=white},
    % up/.style = {ultra thick, green!70!black},
    legend/.style = {text=black, inner sep=5pt}
]

%%% Sector n=0 vacuum
\begin{scope}[local bounding box=trivial]
    \node at (3,1) [above=5pt, legend]  {vacuum $\ket{\Omega_0}$ of the sector $n=0$};
    % ladder
    \draw[ladder] (-0.5,0) grid (6.5,1);
    % sites
    \DrawSites{0,1,...,6}{0,1}
    \useasboundingbox (-1,0) -- (7,0) -- +(0,-0.5);
\end{scope}

%%% Sector n=1 vacuum
\begin{scope}[yshift=-3.5cm, local bounding box=topol]
    \node at (3,1) [above=5pt, legend]  {vacuum $\ket{\Omega_1}$ of the sector $n=1$};
    % ladder
    \draw[ladder] (-0.5,0) grid (6.5,1);
    % Wilson loop
    \draw[up] (-0.5, 0) -- (6.5, 0);
    % sites
    \DrawSites{0,1,...,6}{0,1}
    \useasboundingbox (-1,0) -- (7,0) -- +(0,-0.5);
\end{scope}

% legend
\begin{scope}[xshift=8cm, yshift=1cm, local bounding box=legend]
    \draw [Gray, thin] (0,0.75) -- +(0.5,0) node [right, legend] {$\ket{0}$};
    \draw [up] (0,0)    -- +(0.5,0) node [right, legend] {$\ket{1}$};
    \useasboundingbox (-0.25,0);
\end{scope}


\draw[thin, Gray] (trivial.south west) rectangle (trivial.north east);
\draw[thin, Gray] (topol.south west) rectangle (topol.north east);
\draw [shorten >= 3pt] (trivial.east)
    edge [-{Latex}, Gray, very thick, out=0, in=0]
    node [font=\normalsize, right, text=black] {$\Wilson_1$}
    (topol.east);
\end{tikzpicture}

    \caption[Vacuum states of the super-selection sectors of the $\Z_2$ ladder \ac{lgt}]{The different ``Fock vacua'' $\ket{\Omega_{(0,0)}}$ and $\ket{\Omega_{(1,0)}}$ of the $\Z_2$ ladder \ac{lgt}.
        The latter can be obtained from the former by applying the \ac{wl} operator $W_1$.
        The states $\ket{0}$ and $\ket{1}$ refers to the eigenstates of the electric field operator $V$, which is just $\sigma_{z}$ in the $\Z_2$ model.
    }
    \label{fig:z2_vacua}
\end{figure}

\begin{figure}
    \centering
    \begin{tikzpicture}[scale=0.6]
    % Clock chain
    \begin{scope}[xshift=6.5cm, yshift=4cm, local bounding box=chain]
        \draw[ladder] (-0.5, 0) -- (5.5,0) node [pos=0.5, above=10pt, inner sep=5pt, black] {dual 2--clock chain};
        \foreach \x/\Arrow in {0/\UpArrow, 1/\DownArrow, 2/\DownArrow, 3/\UpArrow, 4/\UpArrow, 5/\DownArrow} {
            \Arrow{\x}{0};
            \draw (\x, 0) node [site] {};
        }
        \useasboundingbox (-1.5, 0) -- (6.5,0) -- +(0,-1);
    \end{scope}

    % LGT sector (0,0)
    \begin{scope}[local bounding box=trivial]
        \node at (3,1) [above, inner sep=5pt] {$\Z_2$ LGT, sector $(0,0)$};
        \draw[ladder] (-0.5,0) grid (6.5,1);
        \draw[up, flux] (0,0) rectangle (1,1);
        \draw[up, flux] (3,0) rectangle (5,1);
        % sites
        \foreach \y in {0,1} \foreach \x in {0,1,...,6} \draw (\x,\y) node [site] {};
        \useasboundingbox (-1.5, 0) -- (7.5,0) -- +(0,-0.75);
    \end{scope}

    % LGT sector (1,0)
    \begin{scope}[xshift=12cm, local bounding box=topological]
        \node at (3,1) [above, inner sep=5pt] {$\Z_2$ LGT, sector $(1,0)$};
        \draw[ladder] (-0.5,0) grid (6.5,1);
        \draw[up] (-0.5, 0) -- (0,0) -- (0,1) -- (1,1) -- (1,0) -- (3,0) -- (3,1) -- (5,1) -- (5,0) -- (6.5,0);
        \fill[flux] (0,0) rectangle (1,1);
        \fill[flux] (3,0) rectangle (5,1);
        % sites
        \foreach \y in {0,1} \foreach \x in {0,1,...,6} \draw (\x,\y) node [site] {};
        \useasboundingbox (-1.5, 0) -- (7.5,0) -- +(0,-0.75);
    \end{scope}

    % Bounding boxes
    \draw [thin, Gray] (chain.north west) rectangle (chain.south east);
    \draw [thin, Gray] (trivial.north west) rectangle (trivial.south east);
    \draw [thin, Gray] (topological.north west) rectangle (topological.south east);

    % Arrows between the bounding boxes
    \draw [thick, Gray, shorten >= 3pt] (chain.west)
        edge [bend right, -{Latex}, Gray, very thick] node [above left, text=black] {$\ket{\Omega_{(0,0)}}$}
        (trivial.north);
    \draw [thick, Gray, shorten >= 3pt] (chain.east)
        edge [bend left,  -{Latex}, Gray, very thick] node [above right, text=black] {$\ket{\Omega_{(1,0)}}$}
        (topological.north);
\end{tikzpicture}

    \caption[Duality between clock states and ladder states]{Duality between the states of a $2$--chain and the states of a $\Z_2$ ladder \ac{lgt} in the different sectors $(0,0)$ (no non-contractible electric loop) and $(1,0)$ (one non-contractible loop around the ladder).
        In the sector $(0,0)$ it is evident that all the physical states contains closed electric loops.
        On the other hand, in the sector $(1,0)$ the physical states are all the possible deformation of the electric string that goes around the ladder.}
    \label{fig:z2_states}
\end{figure}

In order to proceed with ED one has to provide two things: (i) the basic operators of the theory ($U_{\ell}$ and $V$) and (ii) the physical (gauge-invariant) Hilbert space, given a lattice with specified size and boundary conditions.
The former was fairly standard while the latter was the most challenging and interesting part to implement.

If one has to work with only physical states, then one has to check the Gauss law for every site.
With the brute-force method one has to generate all the possible states and then filter out all the states that violate Gauss law.
This method, like any brute-force method, is not very efficient.
To better exemplify this, consider a $\Z_2$ theory on a $L \times L$ periodic lattice, which have $L^2$ sites and $2L^2$ links.
There are therefore $2^{2 L^2}$ possible states and for each one up to $L^2$ checks (one per site) has to be performed.
Moreover, it can be showed that there are only $2^{L^2}$ \emph{physical} states.
As a result, the construction of the physical Hilbert space involves $O(L^2 2^{2 L ^2})$ operations in a search space of $2^{2 L^2}$ objects for finding only $2^{L^2}$ elements.
All of this makes the inefficiency of this brute-force method very clear, even for moderately small lattices.


The approach adopted in this work exploits the duality in Sec.~\ref{sec:dualities_of_the_ladder} and represents an \emph{exponential speedup} with respect to the brute-force method.
It is not a search or pattern-matching algorithm, each physical configuration is procedurally generated from the states of the dual clock model.

Given a $\Z_N$ \ac{lgt} on a lattice of size $L \times L$, we consider the dual $N$-clock model on a similar lattice with $A = L^2$ sites,
In its Hilbert space $\mathcal{H}_{N\text{-clock}}$ there is no gauge constraint or physical condition to apply,
hence the basis is the set of states $\ket{ \{s_i\} } \equiv \ket{s_0 s_1 \cdots s_{A-1}}$ with each $s_i = 0, \dots, N-1$.
From a state $\ket{ \{s_i\} }$ we can obtain the dual state for the \ac{lgt} model in the $(m,n)$ sector:
\begin{equation}
    \ket{\{s_i\} } \; \longmapsto \;
    \prod_{i=0}^{A-1} U_i^{s_i} \ket{\Omega_{(n,m)}},
\end{equation}
where $U_i$ is the plaquette operator on the $i$-th plaquette and $\ket{\Omega_{(m,n)}}$ is the ``Fock vacuum'' of the $(m,n)$ sector.
As one can deduce, the information about the topological sector of the \ac{lgt} model is carried in the Hamiltonian $H_{N\text{-clock}}$ of the dual clock model and not in the structure of $\mathcal{H}_{N \text{-clock}}$.
This means that is possible to build each sector $\Hphys^{(n,m)}$ in \eqref{eq:decomposizione_Hphys} from $\mathcal{H}_{N \text{-clock}}$, with the appropriate $\ket{\Omega_{(n,m)}}$.


Moreover, also the ``Fock vacuums'' $\ket{\Omega_{(n,m)}}$ can be obtained easily, thanks to \eqref{eq:azione_wilson_loop}:
\begin{equation}
    \ket{\Omega_{(n,m)}} = (W_1)^n (W_2)^m \ket{\Omega_{(0,0)}},
\end{equation}
where $\ket{\Omega_{(0,0)}}$ is just the state $\ket{000 \cdots 0}$ (in the electric basis) for all the links.


If we want to quantify the obtained speedup with this method, in the case of a $\Z_2$ theory on a square lattice $L \times L$ there are $2^{L^2}$ possible clock configurations.
For each configuration, there are at most $L^2$ magnetic fluxes to apply.
This translates into $O(L^2 2^{L^2})$ operations, which is an exponential speedup with respect to the brute-force (notice the lack of a factor 2 in the exponent) and is easily generalizable for any $\Z_N$.
Although, it remains an open question whether a similar method can be applied for gauge theories with non-Abelian finite groups.



% \subsection{Non-local order parameters}%
% \label{sub:non_local_order_parameters}

% In Sec.~\ref{sub:investigating_the_phase_diagram} we talked about how a \ac{wl} $W_{\mathcal{R}}$ or an \ac{ths} $S_{\tilde{\mathcal{C}}}$ work as a non-local order parameters and can be used to investigated the phase diagram of a $\Z_N$ \ac{lgt} model.
% In fact, we analyzed these exact observables on the ladder geometry for $N = 2,3$ and $4$.
% Given a ladder of length $L$, the \ac{wl} $W$ have been calculated over a region that covers the first $L/2$ plaquettes, while the \ac{ths} cuts through the first $L/2$ plaquettes (see Fig.~\ref{fig:nlop_ladder}).

% \begin{figure}[t]
%     \centering
%     \input{assets/figures/nonlocal_order_parameters.tex}
%     \caption[Non-local order parameters on the ladder]{The non-local order parameters that have been used for investigating the phase diagram of $\Z_N$ ladder \ac{lgt}.
%     \emph{Top}: half-ladder \ac{wl}.
%     \emph{Bottom}: half-ladder \ac{ths} operator.}%
%     \label{fig:nlop_ladder}
% \end{figure}



\subsection{Numerical results}
\label{sub:numerical_results}

In the following, we present the results with $N=2,3$ and $4$, for different lengths.

\subsubsection*{Results for \texorpdfstring{$N=2$}{N=2}}

As a warm up, we consider the $\Z_2$ ladder \ac{lgt}, with lengths $L=10,12,\dots,18$.
This model is equivalent to a $p=2$ clock model, which is just the quantum Ising chain, with only two super-selection sectors for $n=0$ and $n=1$.
The dual Hamiltonian \eqref{eq:dual_ladder_hamiltonian_real} for $\Z_2$ and sector $n=0$ is
\begin{equation}
    \HamilDual[2]_{n=0}(\lambda^{-1}) = \HamilClock[2](\lambda^{-1}) - 2 \sum_{i} \qty( Z_i + Z_i^{\dagger} ),
\end{equation}
while for $n=1$ we just have
\begin{equation}
    \HamilDual[2]_{n=1}(\lambda^{-1}) = \HamilClock[2](\lambda^{-1}).
\end{equation}

When $n=1$ the Hamiltonian $\HamilDual[2]$ contains only the transverse field, hence it is integrable \cite{baxter1982exactlysm}.
Thus, we expect a critical point for $\lambda \simeq 1$, which will be a DCPT in the gauge model language.
This is clearly seen in the behaviour of the half-ladder \ac{wl}, as shown in the lower panel of Fig.~\ref{fig:z2_wilson}.
For $n=0$, both the transverse and longitudinal fields  are present, hence the model is no longer integrable \cite{banuls2011thermalization, kormos2017confinement, pomponio2022bloch} and we expect to always see a confined phase, except for $\lambda = 0$.
This is indeed confirmed by the behaviour of the half-ladder \ac{wl} shown in the upper panel of Fig.~\ref{fig:z2_wilson}.

Notice that for $n=0$ in Fig.~\ref{fig:z2_wilson}, there is a lack of line crossings between values of \ac{wl} for different $L$.
This suggests that in the thermodynamic limit, we will have a single point $\ev{W} \neq 0$ for $\lambda = 0$, and a flat line $\ev*{W} = 0$ for $\lambda \neq 0$, confirming the prediction of a always confining phase (excluded $\lambda = 0$).

We can further characterize the phases of the two sectors by looking at the structure of the ground state, for $\lambda<1$ and $\lambda>1$, which is possible thanks to the exact diagonalization.
In particular, in the deconfined phase of the sector $n=1$, the ground state is a superposition of the deformations of the non-contractible electric string that makes the $n=1$ vacuum $\ket{\Omega_1}$.
For this reason, this phase can be thought as a \emph{kink condensate} \cite{fradkin1978order} (which is equivalent to a paramagnetic phase), where each kink corresponds to a deformation of the string.
Instead, for $\lambda > 1$, where we have confinement (as in the $n=0$ sector), the ground state is essentially a product state, akin to a ferromagnetic state.
This analysis has been performed at the end of Sec.~\ref{sub:numerical_results}.

\begin{figure}[t]
    \centering
    \begin{tikzpicture}[
        % scale=0.8,
        font=\small,
        notes/.style={gray!20!black, font=\scriptsize}
    ]
    \begin{groupplot}[
        group style={
            group name=wilson2,
            group size=1 by 2,
            vertical sep=3pt,
            x descriptions at=edge bottom%,
            % every plot/.style={thick}
        },
        width=7.5cm,
        height=5cm,
        no marks,
        table/col sep=comma,
        table/x=coupling,
        xtick pos=left,
        ytick pos=left,
        ylabel={$W$},
        xlabel={$\lambda$},
        try min ticks=5,
        cycle list name=exotic,
        legend style={font=\tiny, draw=gray!40},
        legend image post style={scale=0.5}
        ]
        % Sector n=0
        \ZTwoWilsonGraph{00}
        \legend{
            {$L = 10$},
            {$L = 12$},
            {$L = 14$},
            {$L = 16$},
            {$L = 18$}
        }
        % notes on the graphs
        \draw (axis cs:0,1) node (deconf) [circle, fill=gray!40!black, inner sep=0pt, minimum size=4pt] {};
        \draw[<-, shorten <=2pt, gray!40!black] (deconf) -- +(0.3,0) node [right, notes] {deconfined point};
        \draw[black] (axis cs:1,0) node [above, notes] {confined};

        % Sector n=1
        \ZTwoWilsonGraph{10}
        % notes on the graphs
        \draw (axis cs:0.3,0) node [above=10pt, notes] {deconfined};
        \draw (axis cs:1.7,0) node [above=10pt, notes] {confined};
        \draw[dashed, gray] (axis cs:1,0) -- (axis cs:1,1);
    \end{groupplot}
    \draw (wilson2 c1r1.east) node[above, rotate=-90] {sector $n=0$};
    \draw (wilson2 c1r2.east) node[above, rotate=-90] {sector $n=1$};
    \draw (wilson2 c1r1.north) node[above, font=\normalsize] {$\Z_2$ Wilson loop};
\end{tikzpicture}

    \vspace*{-10pt}
    \caption[\ac{wl}s for the $\Z_2$ ladder \ac{lgt}]{$\Z_2$ \ac{wl} in the sectors $n=0$ (\emph{top}) and $n=1$ (\emph{bottom}), for sizes $L=10,12, \dots,18$.
    The sector $n=0$ presents only a deconfined point at $\lambda=0$ and then decays rapidly into a confined phase, while the sector $n=1$ has a phase transition for $\lambda \simeq 1$.
    }
    \label{fig:z2_wilson}
\end{figure}



\subsubsection*{Results for \texorpdfstring{$N=3$}{N=3}}%


The $\Z_3$ \ac{lgt} is studied for lengths $L=7,9,11$ and $13$.
This model can be mapped to a $3$-clock model, which is equivalent to a $3$-state quantum Potts model with a longitudinal field, which is present in all sectors, as one can see from \eqref{eq:dual_ladder_hamiltonian_real}.
The dual Hamiltonian $\HamilDual[3](\lambda^{-1})$ for the sector $n=0$ is
\begin{equation}
    \HamilDual[3]_{n=0}(\lambda^{-1}) = \HamilClock[3](\lambda^{-1}) - 2 \sum_{i} \qty(Z_i + Z_i^{\dagger}),
\end{equation}
while for $n=1$ and $2$ (which are symmetric to each other) we have
\begin{equation}
    \HamilDual[3]_{n=1,2}(\lambda^{-1}) = \HamilClock[3](\lambda^{-1}) - 2 \cos( \frac{\pi}{3} ) \sum_{i} \qty(\tilde{Z}_i + \tilde{Z}_i^{\dagger}).
    \label{eq:dual_clock_3_1}
\end{equation}
Remember that $\tilde{Z}_i$ stands for the $Z_i$ operator with eigenvalues shifted by $\omega^{1/2} = e^{i \pi /N}$.
In all three cases we have a longitudinal field, which is expected to disrupt any paramagnetic state.
Thus, we do not expect to observe a phase transition, and this is confirmed by the behaviour observed in Fig.~\ref{fig:z3_wilson}.
Meanwhile, all the sectors present a deconfined point at $\lambda = 0$, as expected.

In the case $n=0$, for $\lambda > 0$ we recognize a quick transition to a confined phase, similar to what happens in \cite{burrello2021ladder}.
This behaviour is similar to what has been observed for the $\Z_2$ and $n=0$ case in Fig.~\ref{fig:z2_wilson}, hence the same reasoning apply.
While for $n=1$ and $2$ (which are equivalent), the model exhibits a smoother \emph{crossover} to an ordered phase characterized by a doubly-degenerate ground state, for $\lambda > 1$.
Notice that, as discussed above,  the presence of the ``skew'' longitudinal field breaks the three-fold degeneracy expected in the ordered phase of the $3$-clock model into a two-fold degeneracy only.
Additionally, for $L=13$ we notice that a slight bump start to appear.
If some speculation is allowed, this fact, united with the crossover region, may suggest that is some intermediate phase between the deconfined point and the confined region.
For this kind of analysis, higher lattice sizes are necessary which means that \ac{ed} is no longer adequate.
Thankfully, now that we are confident in the duality between ladder \acp{lgt} and \acp{clock}, we can directly study this region in the \acp{clock} setup, by simulating \eqref{eq:dual_clock_3_1} with for example \ac{dmrg}.


\begin{figure}[t]
    \centering
    \newcommand{\ZThreeWilsonGraph}[1]{
        \nextgroupplot
        \addplot+[thick] table [y=7x2]  {assets/graphs/data/Z3_wilson_#1.csv};
        \addplot+[thick] table [y=9x2]  {assets/graphs/data/Z3_wilson_#1.csv};
        \addplot+[thick] table [y=11x2] {assets/graphs/data/Z3_wilson_#1.csv};
        \addplot+[thick] table [y=13x2] {assets/graphs/data/Z3_wilson_#1.csv};
}

\begin{tikzpicture}[
        notes/.style={gray!20!black, font=\scriptsize},
        font=\small
    ]
    \begin{groupplot}[
        group style={
            group name=wilson3,
            group size=1 by 2,
            vertical sep=3pt,
            x descriptions at=edge bottom%,
            % every plot/.style={thick}
        },
        width=7.5cm,
        height=5cm,
        no marks,
        table/col sep=comma,
        table/x=coupling,
        xtick pos=left,
        ytick pos=left,
        ylabel={$W$},
        xlabel={$\lambda$},
        try min ticks=5,
        cycle list name=exotic,
        legend style={font=\tiny, draw=gray!40},
        legend image post style={scale=0.5}
        ]

        % Sector n=0
        \ZThreeWilsonGraph{00}
        \legend{
            {$L = 7$},
            {$L = 9$},
            {$L = 11$},
            {$L = 13$}
        }
        \draw (axis cs:0,1) node (deconf) [circle, fill=gray!40!black, inner sep=0pt, minimum size=4pt] {};
        \draw[<-, shorten <=2pt, gray!40!black] (deconf) -- +(0.3,-0.07) node [right, notes, align=left, anchor=west] {deconfined\\point};
        \draw[black] (axis cs:1.2,0) node [above, notes] {confined};

        % Sector n=1
        \ZThreeWilsonGraph{10}
        \draw (axis cs:0,1) node (deconf) [circle, fill=gray!40!black, inner sep=0pt, minimum size=4pt] {};
        \draw[<-, shorten <=2pt, gray!40!black] (deconf) -- +(0.15,-0.4) node [below, notes,align=center] {deconfined\\point};
        \draw (axis cs:1.65,0) node [above=3pt, notes] {double degeneracy};
        \draw (axis cs:0.65,0.5) node [above, rotate=-50, notes] {crossover};

    \end{groupplot}

    \begin{axis}[
            no marks,
            width=4cm,
            height=2.75cm,
            table/col sep=comma,
            table/x=coupling,
            xtick={0.5,0.75,1.0,1.25},
            xtick align=center,
            ytick align=center,
            xtick pos=left,
            ytick pos=left,
            ymax=3.5,
            at={(wilson3 c1r2.north east)},
            anchor={north east},
            font=\tiny
        ]
        \addplot[thick, blue] table [y=DeltaE1] {assets/graphs/data/Z3_gap_sec_1.csv}
            node [pos=0.8, black, above] {$\Delta E_1$};
        \addplot[thick, red]  table [y=DeltaE2] {assets/graphs/data/Z3_gap_sec_1.csv}
            node [pos=0.6, black, left=2pt] {$\Delta E_2$};
    \end{axis}

    \draw (wilson3 c1r1.east) node[above, rotate=-90] {sector $n=0$};
    \draw (wilson3 c1r2.east) node[above, rotate=-90] {sector $n=1, 2$};
    \draw (wilson3 c1r1.north) node[above, font=\normalsize] {$\Z_3$ Wilson loop};

\end{tikzpicture}

    \vspace*{-10pt}
    \caption[\ac{wl}s for the $\Z_3$ ladder \ac{lgt}]{$\Z_3$ \ac{wl} for the sectors $n=0$ (\emph{top}) and $n=1,2$ (\emph{bottom}, which are equivalent), for sizes $L = 7,9,11$ and $13$.
       Inset: energy differences $\Delta E_i = E_i - E_0$ for $i=1,2$, as a function of the coupling $\lambda$, in the sectors $n=1,2$, showing the emergence of a double-degenerate ground state for $\lambda > 1$.
}
    \label{fig:z3_wilson}
\end{figure}



\subsubsection*{Results for \texorpdfstring{$N=4$}{N=4}}%


The $\Z_4$ ladder \ac{lgt} have four super-selection sectors.
The behaviour of half-ladder \ac{wl}s as function of $\lambda$ is shown in Fig.~\ref{fig:z4_wilson}.
The Hamiltonian in the first sector, $n=0$, is
\begin{equation}
    \HamilDual[4]_{n=0}(\lambda^{-1}) = \HamilClock[4](\lambda^{-1}) - 2 \sum_{i} \qty(Z_i + Z_i^{\dagger}),
\end{equation}
As in the previous models, for $n=0$ we see a deconfined point at $\lambda = 0$, followed by a sharp transition to a confined phase.
Likewise, the lack of line crossings of the \ac{wl} at different $L$ suggests that in the limit $L \to \infty$ we will only have $\ev*{W} \neq 0$ for $\lambda = 0$.

The dual Hamiltonian of the sector $n=2$,
\begin{equation}
    \HamilDual[4]_{n=2}(\lambda^{-1}) = \HamilClock[4](\lambda^{-1}),
\end{equation}
has no longitudinal field, it is the only one to present a clear DCPT for $\lambda \approx 1$, as it is expected from the fact that the $4$-clock model is equivalent to two decoupled Ising chains \cite{ortiz2012dualities}.

In the two equivalent sectors $n=1$ and $3$, where the dual Hamiltonian is
\begin{equation}
    \HamilDual[4]_{n=1,3}(\lambda^{-1}) =
    \HamilClock[4](\lambda^{-1}) - 2 \cos( \frac{\pi}{4} ) \sum_{i} \qty(\tilde{Z}_i + \tilde{Z}_i^{\dagger}),
\end{equation}
the longitudinal field is non-zero and the \ac{wl} shows a peculiar behaviour, at least for the largest size ($L=10$) of the chain: it decreases fast as soon $\lambda > 0$, to stabilize to a finite value in the region $0.5 \lesssim \lambda \lesssim 1$, before tending to zero.
It is comparable to $\Z_3$ and $n=1,2$ situation, where a slight bump appear when the size $L$ is increased.
The characteristics of this phase (with the crossover region for $\Z_3$ and $n=1,2$) would deserve a deeper analysis, that we plan to do in a future work.
For $\lambda \gtrsim 1$, the system enters a deconfined phase with a double degenerate ground state, as for the $\Z_3$ model.


\begin{figure}[t]
    \centering
    \newcommand{\ZFourWilsonGraph}[1]{
    \nextgroupplot
    \addplot+ [thick] table [y=6x2]  {assets/graphs/data/Z4_wilson_#1.csv};
    \addplot+ [thick] table [y=8x2]  {assets/graphs/data/Z4_wilson_#1.csv};
    \addplot+ [thick] table [y=10x2] {assets/graphs/data/Z4_wilson_#1.csv};
}

\begin{tikzpicture}[
        font=\small,
        notes/.style={gray!20!black, font=\scriptsize}
    ]
    \begin{groupplot}[
            group style={
                group name=wilson4,
                group size=1 by 3,
                vertical sep=3pt,
                horizontal sep=3pt,
                x descriptions at=edge bottom,
                y descriptions at=edge left% ,
                % every plot/.style={thick}
            },
            width=7.5cm,
            height=5cm,
            no marks,
            table/col sep=comma,
            table/x=coupling,
            xtick pos=left,
            ytick pos=left,
            ylabel={$W$},
            xlabel={$\lambda$},
            try min ticks=5,
            cycle list name=exotic,
            legend style={font=\tiny, draw=gray!40},
            legend image post style={scale=0.5}
        ]
        % Sector n=0
        \ZFourWilsonGraph{00}
        \legend{{$L = 6$}, {$L = 8$}, {$L = 10$}}
        \draw (axis cs:0,1) node (deconf) [circle, fill=gray!40!black, inner sep=0pt, minimum size=4pt] {};
        \draw[<-, shorten <=2pt, gray!40!black] (deconf) -- +(0.35,-0.075) node [right, notes, align=left] {deconfined\\point};
        \draw[black] (axis cs:1.4,0) node [above, notes] {confined};

        % Sector n=1
        \ZFourWilsonGraph{10}
        \draw (axis cs:0,1) node (deconf) [circle, fill=gray!40!black, inner sep=0pt, minimum size=4pt] {};
        \draw[<-, shorten <=2pt, gray!40!black] (deconf) -- +(0.35,-0.075) node [right, notes, align=left] {deconfined\\point};
        \draw (axis cs:1.7,0) node [above=5pt, notes] {double degen.};

        % Sector n=2
        \ZFourWilsonGraph{20}
        \draw (axis cs:0.3,1) node [below=12pt, notes] {deconfined};
        \draw (axis cs:1.7,0) node [above=15pt, notes] {confined};
        \draw[dashed, gray] (axis cs:1.0,0) -- (axis cs:1.0,1);
    \end{groupplot}

    %
    % Energy gap
    %
    \begin{axis}[
            no marks,
            width=4cm,
            height=2.75cm,
            table/col sep=comma,
            table/x=coupling,
            xtick={0.5,0.75,1.0,1.25},
            xtick align=center,
            ytick align=center,
            xtick pos=left,
            ytick pos=left,
            ymax=2.5,
            at={(wilson4 c1r2.north east)},
            anchor={north east},
            font=\tiny
        ]
        \addplot[thick, blue] table [y=DeltaE1] {assets/graphs/data/Z4_gap_sec_1.csv}
        node [pos=0.8, black, above] {$\Delta E_1$};
        \addplot[thick, red]  table [y=DeltaE2] {assets/graphs/data/Z4_gap_sec_1.csv}
        node [pos=0.4, black, above] {$\Delta E_2$};
    \end{axis}

    \draw (wilson4 c1r1.east) node[above, rotate=-90] {sector $n=0$};
    \draw (wilson4 c1r2.east) node[above, rotate=-90] {sector $n=1, 3$};
    \draw (wilson4 c1r3.east) node[above, rotate=-90] {sector $n=2$};

    % title
    \draw (wilson4 c1r1.north) node[above, font=\normalsize] {$\Z_4$ Wilson loop};
\end{tikzpicture}

    \vspace*{-10pt}
    \caption[\ac{wl}s for the $\Z_4$ ladder \ac{lgt}]{$\Z_4$ \ac{wl} for sectors $n=0, \dots, 3$ and sizes $L=6, \dots, 10$.
        Only the sector $n = 2$ has a clear deconfined-confined phase transition, as expected from the duality with the $4$-clock model.
    }
    \label{fig:z4_wilson}
\end{figure}

\subsection{Distribution of the amplitudes in the ground state}
\label{sub:amplitudes_distribution}

In the $N=2$ case, we further differentiate the phase diagrams of the two sectors by looking at the ground state amplitudes distribution, for $\lambda<1$ and $\lambda>1$.
Obviously, the ground state can be written as a superposition of the gauge invariant states of $\Hphys$ in the given sector
\begin{equation}
    \ket{\Psi_{\text{g.s.} }}= \sum_n c_n \ket{n},
    \label{eq:gs_amplitudes}
\end{equation}
The basis $\ket{n}$ and the amplitudes $c_n$ are sorted in a decreasing order with respect to the modulus of the latter.
The first state of the list, with amplitude $c_1$, is always the Fock vacua $\ket{\Omega}$ of the sector, hence we consider the distribution of the ratios $\abs{c_n / c_1}$, which are plotted in Fig.~\ref{fig:gs_ampl_distr_0.1_Z2}--\ref{fig:gs_ampl_distr_1.5_Z2} for $\lambda=0.1$ and $\lambda=1.5$, respectively.
The most interesting one is at $\lambda = 0.1$, where the difference between the deconfined phase in the sector $(1,0)$ and the confined one in the sector $(0,0)$ can be seen.
In particular, in the deconfined phase the ground state is a superposition of deformations of the Fock vacuum, i.e~the non-contractible electric string, which can be thought as a \emph{kink condensate} \cite{fradkin1978order} (or as a paramagnetic phase), where each kink corresponds to a deformation of the string.
Meanwhile, for $\lambda > 1$, where we have confinement in both sectors, the ground state is essentially a product state, akin to a ferromagnetic state.
This is explained in Fig.~\ref{fig:gs_ampl_distr_0.1_Z2} and Fig.~\ref{fig:gs_ampl_distr_1.5_Z2}.


\begin{figure}[h]
    \centering
    \hspace{3em}$\Z_2$ g.s.~amplitudes distribution, $\lambda=1.5$\\[5pt]
    \begin{tikzpicture}[
    lattice/.style = {Gray, thin, solid},
    on/.style = {Green, very thick, solid},
    lab/.style = {scale=0.35},
    box/.style = {draw=black, dotted, inner sep=4pt},
    arr/.style = {<-, black},
    flux/.style = {fill=Green, fill opacity=0.1},
    font=\small,
    ]

\begin{axis}[
        height=5cm,
        width=8cm,
        tick align=outside,
        tick pos=left,
        xmajorticks=false,
        ylabel={$|c_n/c_1|$},
        ymin=0, ymax=1.1,
        ytick style={color=black},
        ytick={0,0.2,0.4,0.6,0.8,1,1.2},
        yticklabels={$0.0$, $0.2$, $0.4$, $0.6$, $0.8$, $1.0$, $1.2$}
]
\addplot [very thick, blue!70]
    table {%
    0 1
    1 0.0831366777420044
    12 0.0831366777420044
    13 0.00918400287628174
    24 0.00918400287628174
    25 0.00691556930541992
    78 0.0069117546081543
    79 0.0010453462600708
    186 0.000763535499572754
    189 0.00057530403137207
    199 0.000574946403503418
    };

% Title
\node [anchor=north east, draw=gray] at (axis description cs: 0.99, 0.99) (title) {sector $(0,0)$};

\node at (0, 1) (vacuum) {};
\end{axis}

% Vacuum
\draw[arr] (vacuum) node  [circle, fill=black, inner sep=0pt, minimum size=3pt] {}
    -- +(1,0) node (vacuum-label) {};
\draw (vacuum-label) node [right, box] {
    \begin{tikzpicture}[lab]
        \draw[lattice]  (-0.5, 0) grid (5.5, 1);
    \end{tikzpicture}
};

\end{tikzpicture}
\\[-2pt]\hspace{0.4pt}
    \begin{tikzpicture}[
    lattice/.style = {Gray, thin, solid},
    on/.style = {Green, very thick, solid},
    lab/.style = {scale=0.35},
    box/.style = {draw=black, dotted, inner sep=4pt},
    arr/.style = {<-, black},
    flux/.style = {fill=Green, fill opacity=0.1}
    ]

    \begin{axis}[
        height=7cm,
        width=9cm,
        tick align=outside,
        tick pos=left,
        % title={sector $(0, 0)$},
        xlabel={$n$},
        ylabel={$|c_n/c_1|$},
        ymin=0, ymax=1.1,
        ytick style={color=black},
        ytick={0,0.2,0.4,0.6,0.8,1,1.2},
        yticklabels={$0.0$, $0.2$, $0.4$, $0.6$, $0.8$, $1.0$, $1.2$}
        ]
        \addplot [very thick, blue!70]
        table {%
            0 1
            1 1
            2 0.172052025794983
            25 0.172052025794983
            26 0.0594711303710938
            49 0.0594711303710938
            50 0.0305156707763672
            73 0.0305156707763672
            74 0.0297093391418457
            157 0.0296221971511841
            158 0.0258673429489136
            181 0.0258673429489136
            182 0.0130573511123657
            199 0.0130573511123657
        };

        % Title
        \node [anchor=north east, draw=gray] at (axis description cs: 0.99, 0.99) (title) {sector $n=1$};

        \node at (0, 1) (vacuum) {};

    \end{axis}

    % Vacuum
    \draw[arr] (vacuum) node  [circle, fill=black, inner sep=0pt, minimum size=3pt] {}
    -- +(1,0) node (vacuum-label) {};
    \draw (vacuum-label) node [right, box] {
        \begin{tikzpicture}[lab]
            \draw[lattice]  (-0.5, 0) grid (5.5, 1);
            \draw[on] (-0.5,0) -- (5.5, 0);
        \end{tikzpicture}
    };

\end{tikzpicture}

    \caption[$\Z_2$ ground state amplitude distribution for $\lambda = 1.5$]{
    $\Z_2$ ground state amplitude distribution for $\lambda=1.5$ of the first 200 states and with lattice size $12 \times 2$.
    For both sectors $(0,0)$ (\emph{top}) and $(1,0)$ (\emph{bottom}) we are in a confined phase, which corresponds to a ferromagnetic phase in the Ising chain.
    Here we see a polarized state where the domain walls are suppressed and the ground state is essentially a product state.
    }
    \label{fig:gs_ampl_distr_1.5_Z2}
\end{figure}


\begin{figure}[h]
    \centering
    \hspace{3em}$\Z_2$ g.s.~amplitudes distribution, $\lambda=0.1$ \\[5pt]
    \begin{tikzpicture}[
    lattice/.style = {Gray, thin, solid},
    on/.style = {Green, very thick, solid},
    lab/.style = {scale=0.35},
    box/.style = {draw=black, dotted, inner sep=4pt},
    arr/.style = {<-, black},
    flux/.style = {fill=Green, fill opacity=0.1}
    ]

    \begin{axis}[
        height=7cm,
        width=9cm,
        tick align=outside,
        tick pos=left,
        xmajorticks=false,
        ylabel={$|c_n/c_1|$},
        ymin=0, ymax=1.1,
        ytick style={color=black},
        ytick={0,0.2,0.4,0.6,0.8,1,1.2},
        yticklabels={$0.0$, $0.2$, $0.4$, $0.6$, $0.8$, $1.0$, $1.2$}
        ]
        \addplot [very thick, blue!70]
        table {%
            0 1
            1 0.72670304775238
            12 0.72670304775238
            13 0.581751346588135
            24 0.581751346588135
            25 0.529128313064575
            36 0.529128313064575
            37 0.52813982963562
            78 0.52809739112854
            79 0.466958403587341
            90 0.466958403587341
            91 0.423632383346558
            114 0.423632383346558
            115 0.422796964645386
            186 0.422760725021362
            187 0.385271906852722
            198 0.385271906852722
            199 0.384550213813782
        };

        % Title
        \node [anchor=north east, draw=gray] at (axis description cs: 0.99, 0.99) (title) {sector $n=0$};

        \node at (0, 1) (vacuum) {};
        \node at (6, 0.72670304775238)  (1-1-loop) {};
        \node at (16, 0.581751346588135) (1-2-loop) {};
        \node at (50, 0.529128313064575) (2-1-loop) {};
        \node at (83, 0.466958403587341) (1-3-loop) {};
        \node at (130, 0.423632383346558) (1-2-1-1-loop) {};
        \node at (192, 0.385271906852722) (3-1-loop) {};

    \end{axis}

    % Vacuum
    \draw[arr] (vacuum) node  [circle, fill=black, inner sep=0pt, minimum size=3pt] {}
    -- +(1,0) node (vacuum-label) {};
    \draw (vacuum-label) node [right, box] {
        \begin{tikzpicture}[lab]
            \draw[lattice]  (-0.5, 0) grid (5.5, 1);
        \end{tikzpicture}
    };

    % 1 single plaquette loop
    \draw[arr] (1-1-loop) -- +(1,0.7) node (1-1-loop-label) {};
    \draw (1-1-loop-label) node [right, box] {
        \begin{tikzpicture}[lab]
            \draw[lattice]  (-0.5, 0) grid (5.5, 1);
            \fill[flux] (2,0) rectangle (3,1);
            \draw[on] (2,0) rectangle (3,1);
        \end{tikzpicture}
    };

    % 1 double plaquette loop
    \draw[arr] (1-2-loop) -- +(0.6,0.6) node (1-2-loop-label) {};
    \draw (1-2-loop-label) node [right, box] {
        \begin{tikzpicture}[lab]
            \draw[lattice]  (-0.5, 0) grid (5.5, 1);
            \fill[flux] (1,0) rectangle (3,1);
            \draw[on] (1,0) rectangle (3,1);
        \end{tikzpicture}
    };


    % 2 single plaquette loops
    \draw[arr] (2-1-loop) -- +(-0.2, -0.8) node (2-1-loop-label) {};
    \draw (2-1-loop-label) node [below, box] {
        \begin{tikzpicture}[lab]
            \draw[lattice]  (-0.5, 0) grid (5.5, 1);
            \fill[flux] (0,0) rectangle (1,1);
            \fill[flux] (3,0) rectangle (4,1);
            \draw[on] (0,0) rectangle (1,1);
            \draw[on] (3,0) rectangle (4,1);
        \end{tikzpicture}
    };

    % 1 triple plaquette loops
    \draw[arr] (1-3-loop) -- +(1,0.6) node (1-3-loop-label) {};
    \draw (1-3-loop-label) node [right, box] {
        \begin{tikzpicture}[lab]
            \draw[lattice]  (-0.5, 0) grid (5.5, 1);
            \fill[flux] (1,0) rectangle (4,1);
            \draw[on] (1,0) rectangle (4,1);
        \end{tikzpicture}
    };

    % 1 double 1 single plaquette loops
    \draw[arr] (1-2-1-1-loop) -- +(-0.1,-0.6) node (1-2-1-1-loop-label) {};
    \draw (1-2-1-1-loop-label) node [below, box] {
        \begin{tikzpicture}[lab]
            \draw[lattice]  (-0.5, 0) grid (5.5, 1);
            \fill[flux] (1,0) rectangle (2,1);
            \fill[flux] (3,0) rectangle (5,1);
            \draw[on] (1,0) rectangle (2,1);
            \draw[on] (3,0) rectangle (5,1);
        \end{tikzpicture}
    };

    % 3 single plaquette loops
    \draw[arr] (3-1-loop) -- +(-0.1,-1.2) node (3-1-loop-label) {};
    \draw (3-1-loop-label) node [below left, box] {
        \begin{tikzpicture}[lab]
            \draw[lattice]  (-0.5, 0) grid (5.5, 1);
            \fill[flux] (0,0) rectangle +(1,1);
            \fill[flux] (2,0) rectangle +(1,1);
            \fill[flux] (4,0) rectangle +(1,1);
            \draw[on] (0,0) rectangle +(1,1);
            \draw[on] (2,0) rectangle +(1,1);
            \draw[on] (4,0) rectangle +(1,1);
        \end{tikzpicture}
    };



\end{tikzpicture}
\\[-2pt]\hspace{0.4pt}
    \begin{tikzpicture}[
    lattice/.style = {Gray, thin, solid},
    on/.style = {Green, very thick, solid},
    lab/.style = {scale=0.35},
    box/.style = {draw=black, dotted, inner sep=4pt},
    arr/.style = {<-, black},
    flux/.style = {fill=Green, fill opacity=0.1}
    ]

    \begin{axis}[
        height=7cm,
        width=9cm,
        tick align=outside,
        tick pos=left,
        % title={sector $(0, 0)$},
        xlabel={$n$},
        ylabel={$|c_n/c_1|$},
        ymin=0, ymax=1.1,
        ytick style={color=black},
        ytick={0,0.2,0.4,0.6,0.8,1,1.2},
        yticklabels={$0.0$, $0.2$, $0.4$, $0.6$, $0.8$, $1.0$, $1.2$}
        ]
        \addplot [very thick, blue!70]
        table {%
            0,1
            1,1
            2,0.902501583099365
            25,0.902501583099365
            26,0.90012514591217
            133,0.900000095367432
            134,0.816542267799377
            157,0.816542267799377
            158,0.814608097076416
            199,0.814515113830566
        };

        % Title
        \node [anchor=north east, draw=gray] at (axis description cs: 0.99, 0.99) (title) {sector $n=1$};

        \node at (0.5, 1) (vacuum) {};
        \node at (10, 0.902501583099365) (1-1-loop) {};
        \node at (70, 0.900000095367432) (1-big-loop) {};
        \node at (160, 0.816542267799377) (2-1-loop) {};
    \end{axis}

    % Vacuum
    \draw[arr] (vacuum) node  [circle, fill=black, inner sep=0pt, minimum size=3pt] {}
    -- +(1,0) node (vacuum-label) {};
    \draw (vacuum-label) node [right, box] {
        \begin{tikzpicture}[lab]
            \draw[lattice]  (-0.5, 0) grid (5.5, 1);
            \draw[on] (-0.5,0) -- (5.5, 0);
        \end{tikzpicture}
    };

    % 1 single plaquette loop
    \draw[arr] (1-1-loop) -- +(0.5,-0.5) node (1-1-loop-label) {};
    \draw (1-1-loop-label) node [below, box] {
        \begin{tikzpicture}[lab]
            \draw[lattice]  (-0.5, 0) grid (5.5, 1);
            \fill[flux] (2,0) rectangle (3,1);
            \draw[on] (-0.5, 0) -- (2, 0) -- (2, 1) -- (3, 1) -- (3, 0) -- (5.5, 0);
        \end{tikzpicture}
    };



    % 1 big loop
    \draw[arr] (1-big-loop) -- +(0.0,-1.5) node (1-big-loop-label) {};
    \draw (1-big-loop-label) node [below, box] {
        \begin{tikzpicture}[lab]
            \draw[lattice]  (-0.5, 0) grid (5.5, 1);
            \draw[on] (-0.5, 0) -- (2, 0) -- (2, 1) -- (4, 1) -- (4, 0) -- (5.5, 0);
            \fill[flux] (2,0) rectangle (4,1);
            \begin{scope}[yshift=-1.5cm]
                \draw[lattice]  (-0.5, 0) grid (5.5, 1);
                \draw[on] (-0.5, 0) -- (1, 0) -- (1, 1) -- (4, 1) -- (4, 0) -- (5.5, 0);
                \fill[flux] (1,0) rectangle (4,1);
            \end{scope}
            \begin{scope}[yshift=-3.0cm]
                \draw[lattice]  (-0.5, 0) grid (5.5, 1);
                \draw[on] (-0.5, 0) -- (1, 0) -- (1, 1) -- (5, 1) -- (5, 0) -- (5.5, 0);
                \fill[flux] (1,0) rectangle (5,1);
            \end{scope}
            \node at (2.5, -3) [font=\footnotesize, inner sep=0pt] {$\vdots$};
        \end{tikzpicture}
    };


    % 2 single plaquette loops
    \draw[arr] (2-1-loop) -- +(-0.2,-1.0) node (2-1-loop-label) {};
    \draw (2-1-loop-label) node [below, box] {
        \begin{tikzpicture}[lab]
            \draw[lattice]  (-0.5, 0) grid (5.5, 1);
            \fill[flux] (1,0) rectangle (2,1);
            \fill[flux] (3,0) rectangle (4,1);
            \draw[on] (-0.5, 0) -- (1, 0) -- (1, 1) -- (2, 1) -- (2, 0) -- (3,0) -- (3,1) -- (4,1) -- (4,0) -- (5.5, 0);
        \end{tikzpicture}
    };


\end{tikzpicture}

    \caption[$\Z_2$ ground state amplitude distribution for $\lambda = 0.1$]{
        $\Z_2$ ground state amplitude distribution for $\lambda=0.1$ of the first 200 states and with lattice size $12 \times 2$.
        \emph{Top}: distribution of the ratios $|{c_n/c_1}|$ for the sector $(0,0)$ (see \eqref{eq:gs_amplitudes}).
        We see that the heaviest states that enters the ground state, apart from the vacuum that sets the scale, are made of small electric loops, typical of a confined phase.
        \emph{Bottom}: the same distribution of ratios for the sector $(1,0)$.
        We see that the heaviest states are made of bigger and bigger deformations of the electric string that goes around the ladder.
        This happens because the energy contributions depends only on the domain walls between two plaquettes with different flux content.
        This behaviour is similar to the so-called \emph{kink condensation} in spin chains \cite{fradkin1978order}, where the disordered state can be expressed as a superposition of all possible configuration of kinks (i.e.~domain walls between two differently ordered regions).
        In the language of the duality, this deconfined phase then maps to the paramagnetic phase of the quantum Ising model with \emph{only} transverse field.
    }
    \label{fig:gs_ampl_distr_0.1_Z2}
\end{figure}







\section{Concluding remarks}
\label{sec:concluding_remarks}

In this work, we proposed an exact gauge preserving duality transformation that maps the  $\mathbb{Z}_N$ lattice gauge theory on a ladder onto a 1D $N-$clock model in a transversal field, coupled to a possibly complex longitudinal field which depends on the super-selection sector.

This map allowed us to perform numerical simulations with an \ac{ed} algorithm with sizes up to $L=18, 13, 10$ for $N=2,3,4$ respectively.
To study the phases of the model and a possible \ac{dcpt} transition, we calculated the Wilson loops in the different topological sectors, finding an unusual behaviour in the sectors with $n$ odd (mod $N$), possibly suggesting the emergence of a new phase, such as for example the incommensurate phase appearing in chiral clock models \cite{huse1983chiral, whitsitt2018clock, zhuang2015clock}, whose characterization requires however to consider longer sizes of the chain in order to evaluate the asymptotic behaviour of correlators.

This will be the subject of future work, in which we can also consider the possibility to include static and dynamical matter in the lattice gauge model.
Another possible direction would be the extension of these duality transformations to non-Abelian gauge theories.
