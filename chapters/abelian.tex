%======================================================================
% CHAPTER: DUALITIES IN LATTICE GAUGE THEORIES
%======================================================================
\chapter{Dualities in Abelian Models}
\label{chap:dualities_in_abelian_models}

In this chapter we present the findings of \cite{pradhan2022ladder}, where a duality transformation from the gauge-invariant subspace of a $\Z_N$ \ac{lgt} on a ladder geometry to an $N$-clock model on a single chain.
The main feature of this mapping is the emergence of a longitudinal field in the clock model, whose value depends on the super-selection sector of the gauge model.
% , implying that the different sectors of the gauge theory can show quite different phase diagrams.
In order to investigate this and see if confined phases might emerge, we perform a numerical analysis for $N = 2, 3,$ and $4$, using \acl{ed}.


% SECTION: Toric Code and its features
%----------------------------------------
% SECTION: Toric Code and its features
%----------------------------------------
\section{Toric Code and its features}
\label{sec:toric_code_and_its_features}

The \ac{tc} is two-dimensional model of spin-\onehalf{} \ac{dof}, which can be regarded as an example of a pure $\Z_2$ \ac{lgt}.
We consider the model on a $L \times L$ square lattice $\Lattice$ with periodic boundary conditions.
The \ac{dof} are defined on the links $\link$ of the lattice and the link Hilbert space is $\C^2$.
The main operators used for this model are the Pauli matrices
\begin{equation}
    X_{\link} = \begin{pmatrix}
        0 & 1 \\ 1 & 0
    \end{pmatrix}
    \qand
    Z_{\link} = \begin{pmatrix}
        1 & 0 \\ 0 & -1
    \end{pmatrix},
    \label{eq:matrices_X_Z}
\end{equation}
which here have been written in the computational basis $\{\ket{0}, \ket{1}\}$, where the $Z$-matrix is diagonal.
It is important to note that these matrices $X$ and $Z$ \emph{anticommutes} on the same link.
\begin{equation}
    \acomm{X_{\link}}{Z_{\link}} = 0,
\end{equation}
while they commute with operators of other links.

The main \emph{local} operators that enters the Hamiltonian are the \emph{star operator} and \emph{plaquette operator} of the lattice.
The term \emph{star} refers to the links attached to a common vertex $v$, while by \emph{plaquette} $p$ we mean the links around a face of the lattice.
A \emph{star operator} and a \emph{plaquette operator} are respectively defined as
\begin{equation}
    A_v = \prod_{\link \in v} Z_\link, \qquad
    B_p = \prod_{\link \in p} X_\link.
    \label{eq:star_plaq_op_def}
\end{equation}
where $v$ is a vertex and $p$ a plaquette (see Fig.~\ref{fig:toric_code_operators}).
One can easily see that
\begin{equation}
    \comm{A_v}{A_{v^{\prime}}} = 0 \qand
    \comm{B_p}{B_{p^{\prime}}} = 0
    \label{eq:star_plaq_op_comm_1}
\end{equation}
for all vertices $v$ and $v^{\prime} $, and all plaquettes $p$ and $p^{\prime} $.
But it is also true that
\begin{equation}
    \comm{A_v}{B_p} = 0
    \label{eq:star_plaq_op_comm_2}
\end{equation}
for all $v$ and $p$.
This is because a star and a plaquette share zero or two links, so the signs factors from the anticommutation of $X$ and $Z$ cancel out.
The eigenvalues of the Pauli matrices are just $\pm 1$, so the same holds true for $A_s$ and $B_p$.
Moreover, like the Pauli matrices, also $A_s^2 = \identity$ and $B_p^2 = \identity$.


Now, given the operators in \eqref{eq:star_plaq_op_def}, we can write down the Hamiltonian of the \ac{tc}:
\begin{equation}
    \HamilTC = - \sum_{v} A_v - \sum_{p} B_p
    \label{eq:toric_code_hamiltonian}
\end{equation}
which is \emph{exactly solvable}, due to \eqref{eq:star_plaq_op_comm_1} and \eqref{eq:star_plaq_op_comm_2}.


\begin{figure}[t]
    \SideFigure[label={fig:toric_code_operators}, desc={Vertex and plaquette operators of the \ac{tc}}]{ %
        \begin{tikzpicture}[scale=1.2]
    % Lattice
    \draw[lattice, step=1] (-0.5,-0.5) grid (4.5,2.5);

    % Plaquette operator
    \draw[X] (3,1) -- (4,1) node [pos=0.5, below] {$X$};
    \draw[X] (4,1) -- (4,2) node [pos=0.5, right] {$X$};
    \draw[X] (4,2) -- (3,2) node [pos=0.5, above] {$X$};
    \draw[X] (3,2) -- (3,1) node [pos=0.5, left]  {$X$};
    \draw[blue, ultra thick, pattern=north east lines, pattern color=blue] (3,1) rectangle (4,2);
    \draw (3.5,1.5) node [fill=white, rounded corners] {$B_p$};

    % Gauss operator
    \draw[Z] (1, 1) -- (2, 1) node [pos=0.5, below right] {$Z$};
    \draw[Z] (1, 1) -- (1, 2) node [pos=0.5, above left] {$Z$};
    \draw[Z] (1, 1) -- (0, 1) node [pos=0.5, below left] {$Z$};
    \draw[Z] (1, 1) -- (1, 0) node [pos=0.5, below right] {$Z$};
    \draw (1,1) node [above right] {$A_s$};

    % Sites
    \DrawSites{0,1,...,4}{0,1,2}

\end{tikzpicture}

    }{%
        Graphical representation of the vertex operator $A_v$ and plaquette operator $B_p$, defined in \eqref{eq:star_plaq_op_def}.
    }
\end{figure}



%
% SUBSECTION: Topological ground states
%
\subsection{Topological Ground states}
\label{sub:topological_ground_states}

Given the commutation relations of the $A_v$ and $B_p$ operators in \eqref{eq:star_plaq_op_comm_1} and \eqref{eq:star_plaq_op_comm_2}, one can find the ground state $\ket{\Omega}$ by simply imposing the constraints
\begin{equation}
    A_v \ket{\Omega} = \ket{\Omega} \qand
    B_p \ket{\Omega} = \ket{\Omega}, \qquad \forall v,\, p.
    \label{eq:ground_state_constraints}
\end{equation}
From these constraints one can explicitly construct a ground state for the \ac{tc} in the following way.
Working in the $Z$-basis, we can start from
\begin{equation}
    \ket{0}_{\text{TC}} = \bigotimes_{\link \in \Lattice} \ket{0}_{\link},
\end{equation}
which is the state where every link is in the $\ket{0}$, where $Z \ket{0} = \ket{0}$.
This state obviously satisfy the first condition in \eqref{eq:ground_state_constraints}.

Now, regarding the $B_p$'s operators, consider a single plaquette in the state $\ket{0}_p$ where every link is in the $\ket{0}$ state.
The action of $B_p$ flips the state of every link, from $\ket{0}$ to $\ket{1}$, obtaining $\ket{1}_p$.
Therefore, a plaquette is in an eigenstate of $B_p$ if is in an equal superposition of $\ket{0}_p$ and $\ket{1}_p$.
Knowing this, it is straightforward to see that the operator $(\identity + B_p)/\sqrt{2}$ generates an eigenstate of $B_p$ from $\ket{0}_p$.
In fact, a simple calculation
\begin{equation}
    B_p \frac{\identity + B_p}{\sqrt{2}} \ket{0}_p = \frac{B_p + \identity}{\sqrt{2}} \ket{0}_p
\end{equation}
shows that we obtain an eigenstate of $B_p$ with eigenvalue $+1$, due to $B_p^2 = \identity$.

Therefore, we can obtain a ground state for the \ac{tc} Hamiltonian
\begin{equation}
    \ket{\Omega} = \prod_{p} \frac{\identity + B_p}{\sqrt{2}} \ket{0}_{\text{TC}}.
\end{equation}
More generally, we can define the space of ground states
\begin{equation}
    \mathcal{G} = \qty{ \ket{\Omega} : A_s \ket{\Omega} = \ket{\Omega}, \quad B_p \ket{\Omega} = \ket{\Omega} \quad \forall s, p },
    \label{eq:TC_protected_subspace}
\end{equation}
whose content \emph{depends on the topology of the lattice}.
For example, we will show that with \emph{periodic boundary conditions} then there are
\emph{four degenerate ground states}.


Consider a lattice $\Lattice$ of size $L \times L$ with periodic boundary conditions in both directions, i.e.~a torus.
From \eqref{eq:ground_state_constraints}, we have $2L^2$ constraints.
These are not all independent because if we multiply them all, we obtain
\begin{equation}
    \prod_{v} A_v = \identity \qand
    \prod_{p} B_p = \identity,
\end{equation}
which actually means that there are $2L^2 - 2$ independent conditions.
The total Hilbert space has dimension $2^{2L^2}$.
Combined with $2L^2 - 2$ independent conditions we obtain $2^{2L^2 - 2L^2 + 2} = 4$ independent states.
Therefore, $\dim \mathcal{G} = 4$ because we have four degenerate distinct ground states.
These are eigenstates of all $A_v$ and $B_p$, with all the same eigenvalues.
Any other that commutes with the Hamiltonian is given by a product of $A_v$ and $B_p$, so it cannot distinguish the different ground states.


The only way to distinguish these ground states is through \emph{non-local operators} that commute with the Hamiltonian in \eqref{eq:toric_code_hamiltonian}.
Non-local in this instance means not expressible as a product or sum of vertex and plaquette operators.
But first let look more closely at \emph{local operators}.


Consider any region $\mathcal{R}$ on the lattice $\Lattice$.
Without loss of generality, let $\mathcal{R}$ be a connected region, which means it is just a set of jointed plaquettes.
On this region $\mathcal{R}$ we can define a local operator $W$ as a product of $B_p$ operators:
\begin{equation}
    W = \prod_{p \in \mathcal{R}} B_p.
\end{equation}
This operator commutes will the terms of the Hamiltonian \eqref{eq:toric_code_hamiltonian}.
Due to $X^2_{\link} = \identity$, the previous equation can be rewritten as
\begin{equation}
    W = \prod_{\link \in \partial \mathcal{R}} X_{\link}.
\end{equation}
In other words, $W$ is equivalent to the product of $X$'s along the closed curve given by the boundary $\partial \mathcal{R}$ of $\mathcal{R}$.
In fact, the $B_p$ themselves are defined as product of $X$'s along a closed curve, the plaquette.
In a sense, they are all \emph{string operators} on closed curves.

The same argument can be repeated for $A_v$ with the minor caveat that the dual lattice have to considered.
In the dual lattice $\DualLattice$, to each plaquette $p$ of $\Lattice$ corresponds a vertex $v^{\ast}$ on the dual lattice.
Then, to each link $\link$ in $\Lattice$ corresponds a link $\link^{\ast}$ in $\DualLattice$ in the perpendicular direction.
In this way, a star becomes a plaquette in the dual lattice and we can repeat the same argument.
% From here, we can repeat the same argument.
Consider a region $\mathcal{R}^{\ast}$ a local operator $S$ such that
\begin{equation}
    S = \prod_{v \in \mathcal{R}^{\ast}} A_v,
\end{equation}
and, due to $Z^2_{\link} = \identity$, this is equal to
\begin{equation}
    S = \prod_{\link \in \partial \mathcal{R}^{\ast}} Z_{\link}.
\end{equation}
The local operator $S$ is a string of $Z$'s operators along the closed curve given by the boundary $\partial \mathcal{R}$ in $\DualLattice$.
The same can be said for $A_v$, it is a string operator around the smallest possible curve in $\DualLattice$.
% Indeed, a star on the direct lattice becomes a plaquette on the dual lattice.
% So a product of $A_v$ is equivalent to a string of $Z$ operators along closed curves on the dual lattice.
% All these curve have a common property, they are \emph{contractible}, meaning that they can be continuously deformed to a single point.
% The operators we have called local so far are string operators over contractible curves, but if we are on a lattice with non-trivial topology, like a torus, then we can also have \emph{non-contractible} curves.
We can conclude that all the local operators that commutes with Hamiltonian are just string operators over closed curve in either $\Lattice$ or $\DualLattice$.
But, these operators have all a common feature, they are defined on \emph{contractible} curves.
Meaning that they can be ``continuously'' deformed to a single point.

\begin{figure}[t]
    \SideFigure[label=fig:nonlocal_operators_TC, desc={Non-local operators in the \ac{tc}}]{%
        \begin{tikzpicture}
    % lattice grid
    \draw[lattice, step=1] (-0.5,-0.5) grid (4.5,4.5);

    % Wilson loops
    \draw[X]
        (-0.5, 1) -- (4.5, 1)
        node [pos=1, right] {$\mathcal{L}_1$}
        node [pos=0.85, above, black] {$W_1$}
        ;
    \draw[X]
        (3, -0.5) -- (3, 4.5)
        node [pos=0, below] {$\mathcal{L}_2$}
        node [pos=0.15, left, black] {$W_2$}
        ;

    % 't Hooft strings
    \draw[Z, dashed]
        (-0.5,3.5) -- (4.5,3.5)
        node[pos=0, left] {$\mathcal{C}_1$}
        ;
    \foreach \x in {0,...,4} { \draw[Z] (\x, 3) -- +(0, 1); }
    \draw (2,3) node [below right] {$S_1$};
    \draw[Z, dashed]
        (0.5,-0.5) -- (0.5,4.5)
        node[pos=1, above] {$\mathcal{C}_2$}
        ;
    \foreach \y in {0,...,4} { \draw[Z] (0, \y) -- +(1, 0); }
    \draw (1,2) node [below right] {$S_2$};
    \foreach \y in {0,...,4} \foreach \x in {0,...,4} \draw (\x,\y) node [site] {};
\end{tikzpicture}

    }{%
        Graphical representation of the different types of non-local operators.
        On the non-contractible loops $\mathcal{L}_1$ and $\mathcal{L}_2$ (in the direct lattice) we have defined $\Wilson_1$ and $\Wilson_2$ (see \eqref{eq:nonlocal_W_toric} ).
        While on the non-contractible cuts $\mathcal{C}_1$ and $\mathcal{C}_2$ (in the dual lattice) we have the operators $\overline{S}_1$ and $\overline{S}_2$.
    }
\end{figure}


String operators on non-contractible curves, either on the direct or dual lattice, are the non-local operators we have been looking for distinguish the states in $\mathcal{G}$.
Consider two non-contractible loops $\mathcal{L}_{1}$ and $\mathcal{L}_{2}$ on $\Lattice$ along the $\hat{1}$ and $\hat{2}$ direction respectively, like in Fig.~\ref{fig:nonlocal_operators_TC}.
On these paths we can define the string operators $\Wilson_1$ and $\Wilson_2$ as
\begin{equation}
    \Wilson_1 = \prod_{\link \in \mathcal{L}_1} X_\link, \qquad
    \Wilson_2 = \prod_{\link \in \mathcal{L}_2} X_\link.
    \label{eq:nonlocal_W_toric}
\end{equation}
It can be proved that they commute with all the terms of the Hamiltonian, even though they cannot be expressed as a product of them.
The same can be repeated on the dual lattice $\DualLattice$, by considering two non-contractible cuts $\mathcal{C}_1$ and $\mathcal{C}_2$ and defining $\tHooft_1$ and $\tHooft_2$ as
\begin{equation}
    \tHooft_1 = \prod_{\link \in \mathcal{C}_1} Z_\link, \qquad
    \tHooft_2 = \prod_{\link \in \mathcal{C}_2} Z_\link.
    \label{eq:nonlocal_S_toric}
\end{equation}
Likewise, the operators in \eqref{eq:nonlocal_S_toric} commutes with all the vertex and plaquettes operators but they do not commute with the operators in \eqref{eq:nonlocal_W_toric}.


In fact, \eqref{eq:nonlocal_W_toric} and \eqref{eq:nonlocal_S_toric} anticommutes,
\begin{equation}
    \acomm*{\Wilson_1}{\tHooft_2} = 0 \qand
    \acomm*{\Wilson_2}{\tHooft_1} = 0,
    \label{eq:anticommutation_W_S_toric}
\end{equation}
while
\begin{equation}
    \comm{\Wilson_1}{\Wilson_2} = 0 \qand
    \comm{\tHooft_1}{\tHooft_2} = 0.
\end{equation}
These relations can be thought as the same commutation relations of the $X$ and $Z$ gates of two qubits.

Therefore, the \ac{tc} (on a torus) has a protected subspace $\mathcal{G}$, see \eqref{eq:TC_protected_subspace}, that behaves like the Hilbert space of two qubits and the operators \eqref{eq:nonlocal_W_toric} and \eqref{eq:nonlocal_S_toric} acts like unitary gates on this space.
Unfortunately, we cannot do quantum computation with these topological qubits because there is no entangling gates.
Nonetheless they can be used for storing information in a fault-tolerant way, because in order to flip a topological qubit you would need to act with a non-local operator that involves a large amount of links.


% \subsection{Particle excitations}%
% \label{sub:particle_excitations}
%
% Until now we have only discussed the ground states of the \ac{tc}, without touching the rest of the low energy sectors.
% In other words, how do we describe the excitations of this model?
% As we said, \eqref{eq:ground_state_constraints} are the set of constraints that defines the ground states.
% Therefore, every time a given state $\ket{\Psi}$ violates these equations, we will say that it contains \emph{particles}, which can be of different types.
% If $A_v \ket{\Psi} = - \ket{\Psi}$, then we will say that the vertex $v$ contains a $z$-type particle.
% Likewise, if $B_p \ket{\Psi} = - \ket{\Psi}$, then the plaquette $p$ contains a $x$-type particle.
%
% Now the question: starting from a ground state $\ket{\Omega}$, how can we introduce some particles?
% The answer is \emph{string operators}.
% We are not considering closed strings, like we did in Sec.~\ref{sub:topological_ground_states}, but any open string.
% The shortest open string that we can consider is a single link.
% So a $Z$-string on a single link is just $Z_j$, where $j$ is a label of a generic link.
% Consider now the state
% \begin{equation}
%     \ket*{\Psi^Z} = Z_j \ket{\Omega},
% \end{equation}
% This state hosts particles at the ``boundaries'' of the $j$-th link, i.e.~the vertices touching $j$ which we call $v_0$ and $v_1$.
% This can be proved by simply showing that $[ A_v, Z_j ] = 0$ for $v \neq v_0$ and $v \neq v_1$ and $\{ A_{v_0}, Z_j \} = \{ A_{v_1}, Z_j \} = 0$.
% Which immediately implies that
% \begin{equation}
%     A_{v_0} \ket*{\Psi^Z} = A_{v_1} \ket*{\Psi^Z} = - \ket*{\Psi^Z}
% \end{equation}
%


%
% SUBSECTION: Z2 \ac{lgt}
%
\subsection{\acl{tc} as a \texorpdfstring{$\Z_2$}{Z2} \acs{lgt}}
\label{sub:toric_code_as_a_z2_lattice_gauge_theory}

The \ac{tc} was formulated as a type of error-correcting code for quantum computing, but it can be reinterpreted as a pure $\Z_2$ \ac{lgt}.
This is a type of \ac{lgt} where we allow only two possible states for the gauge field.

On a single link $\link$, we consider the $X_{\link}$ as the gauge field operator, while $Z_{\link}$ the electric field operator.
In this way, we can automatically see that the term $B_p$ is the magnetic energy because it has the same form of single-plaquette \ac{wl}.
Furthermore, the vertex operator $A_v$ can be read as a gauge transformation on the vertex $v$, because the $Z$'s operators flips the states in the $X$-basis, which would corresponds to gauge field configurations.

Now that we know the form of gauge transformations, we call a state \emph{physical} or \emph{gauge-invariant} if
\begin{equation}
    A_v \ket{\phi} = \ket{\phi} \quad \forall v \in \Lattice,
    \label{eq:toric_code_physical_state}
\end{equation}
which leads to the definition of the \emph{physical Hilbert space}:
\begin{equation}
    \Hphys = \{\ket{\phi} \; \text{s.t.} \; A_v \ket{\phi} = \ket{\phi} \quad \forall v \in \Lattice\}.
    \label{eq:phys_Hilbert_space_toric}
\end{equation}
For greater clarity, lets work in the \emph{electric basis}, which is just the $Z$-basis where the electric field is diagonal.
The electric field operator $Z$ has eigenvalue $+1$ and $-1$, corresponding respectively to the states $\ket{0}$ and $\ket{1}$.
In order to meet the condition in \eqref{eq:toric_code_physical_state}, a vertex configuration must have an even number of links in the $\ket{1}$ state (examples can be seen in Fig.~\ref{fig:gauge_inv_vertices_z2}).

\begin{figure}[t]
    \SideFigure[label=fig:gauge_inv_vertices_z2, desc={Gauge-invariant vertex states for the $\Z_2$ \ac{lgt}}]{%
        \begin{tikzpicture}[scale=0.35]

    % First row
    \draw[lattice] (-1.75, -1.75) grid (1.75, 1.75);
    \node[site] at (0, 0) {};

    \begin{scope}[xshift=4cm]
        \draw[lattice] (-1.75, -1.75) grid (1.75, 1.75);
        \draw[up] (-1.75, 0) -- (1.75, 0);
        \node[site] at (0, 0) {};
    \end{scope}

    \begin{scope}[xshift=8cm]
        \draw[lattice] (-1.75, -1.75) grid (1.75, 1.75);
        \draw[up] (0, -1.75) -- (0, 0) -- (1.75, 0);
        \node[site] at (0, 0) {};
    \end{scope}

    \begin{scope}[xshift=12cm]
        \draw[lattice] (-1.75, -1.75) grid (1.75, 1.75);
        \draw[up] (0, -1.75) -- (0, 0) -- (-1.75, 0);
        \node[site] at (0, 0) {};
    \end{scope}


    % Second row
    \begin{scope}[yshift=-4cm]
        \draw[lattice, up] (-1.75, -1.75) grid (1.75, 1.75);
        \node[site] at (0, 0) {};
    \end{scope}

    \begin{scope}[yshift=-4cm, xshift=4cm]
        \draw[lattice] (-1.75, -1.75) grid (1.75, 1.75);
        \draw[up] (0, -1.75) -- (0, 1.75);
        \node[site] at (0, 0) {};
    \end{scope}

    \begin{scope}[yshift=-4cm, xshift=8cm]
        \draw[lattice] (-1.75, -1.75) grid (1.75, 1.75);
        \draw[up] (0, 1.75) -- (0, 0) -- (1.75, 0);
        \node[site] at (0, 0) {};
    \end{scope}

    \begin{scope}[yshift=-4cm, xshift=12cm]
        \draw[lattice] (-1.75, -1.75) grid (1.75, 1.75);
        \draw[up] (-1.75, 0) -- (0, 0) -- (0, 1.75);
        \node[site] at (0, 0) {};
    \end{scope}


    % \node[font=\normalsize, right] at (14,0) {$\dots$};
\end{tikzpicture}

    }{%
        Gauge-invariant vertex states for the $\Z_2$ \ac{lgt}.
        Green lines represent the $\ket{1}$ link state.
    }
\end{figure}

We have already argued that the $B_p$'s give the magnetic energy, and obviously the $Z$'s give the electric energy.
Hence, the pure gauge theory Hamiltonian is just
\begin{equation}
    H^{\Z_2} = - \sum_{p} B_p - \lambda \sum_{\link} Z_{\link},
    \label{eq:z2_lgt_hamiltonian}
\end{equation}
where $\lambda$ is a generic coupling that tunes the strength of the electric field with respect to the magnetic field.
Notice that we no longer have a dynamical vertex term in \eqref{eq:z2_lgt_hamiltonian} because we have imposed the condition \eqref{eq:toric_code_physical_state} on the physical states.

In order to better explain the different phases we can have by varying the coupling $\lambda$ in \eqref{eq:z2_lgt_hamiltonian}, we want to have a closer look at the physical states.
We have already seen that the condition \eqref{eq:toric_code_physical_state} constraints the types of vertex configurations.
From the allowed configuration, we can see that the only possible lattice states (in the electric basis) are states made of \emph{closed electric loops}.
An example of such state can be seen in Fig.~\ref{fig:physical_state_z2}.


For $\lambda = 0$ we recover the \ac{tc} and its ground state can be reinterpreted as an equal superposition of all the possible configuration of closed electric loops.
This kind of phase is also called a \emph{loop condensate}.
For large $\lambda$ the electric term dominates over the magnetic term, hence all the links will favor the state $\ket{0}$.
So in the regime of strong coupling we expect to be in a \emph{polarized phase}, where the presence of electric loops is suppressed.
Therefore, there is a critical coupling $\lambda_c$ for which we have a \emph{phase transition}.
In the language of gauge theories, the loop condensate corresponds to a \emph{deconfined phase} while the polarized one is a \emph{confined phase}.
Hence, for $\lambda_c$ we have a deconfined-confined phase transition.
% \todo{inserire diagramma di fase}

\begin{figure}[t]
    \SideFigure[label=fig:physical_state_z2, desc={Physical states in the $\Z_2$ \ac{lgt}}]{%
        \begin{tikzpicture}[scale=0.5]
    \draw[lattice] (-1, -1) grid (11, 11);

    \draw[up] (0, 0) rectangle (2, 2);
    \draw[up] (6, 0) rectangle (10, 2);
    \draw[xshift=-2cm, up] (4, 4) -- (4, 10) -- (8, 10) -- (8, 6) -- (6, 6) -- (6, 4) -- cycle;
    \draw[up] (-1, 6) -- (0, 6) -- (0, 11);
    \draw[up] (11, 4) -- (8, 4) -- (8, 8) -- (11, 8);

    % sites
    \DrawSites{0,2,...,10}{0,2,...,10}
\end{tikzpicture}

    }{%
        Physical states in the $\Z_2$ \ac{lgt} are made of closed electric loops.
    }
\end{figure}



\subsection{Super-selection sectors}
\label{sub:super_selection_sectors}

We have already seen in Sec.~\ref{sub:topological_ground_states} the non-local operators $\Wilson_{1,2}$ and $\overline{S}_{1,2}$ that can classify the topological ground states.
They can be treated on equal footing in the pure \ac{tc}, because they both commutes with all the terms of the Hamiltonian \eqref{eq:toric_code_hamiltonian}.
This is no longer true in \eqref{eq:z2_lgt_hamiltonian}, when the electric term is present.
Both kind of operators are gauge-invariant, in the sense that they commute with the gauge transformations $A_v$, i.e.
\begin{equation}
    \comm{\Wilson_{1,2}}{A_v} = 0
    \qand
    \comm{\overline{S}_{1,2}}{A_v} = 0,
    \quad \text{for all $v \in \Lattice$}
\end{equation}
but only the $\overline{S}_{1,2}$ string operator commute with the electric field $Z_{\link}$.

This means, that we can classify all the state of $\Hphys$ (see \eqref{eq:phys_Hilbert_space_toric}) through their $\overline{S}_1$ and $\overline{S}_2$ eigenvalues, because they commute with the Hamiltonian.
Therefore, we obtain a decomposition of the physical Hilbert space in super-selection sectors
\begin{equation}
    \Hphys = \Hphys^{(0,0)} \oplus \Hphys^{(0,1)} \oplus \Hphys^{(1,0)} \oplus \Hphys^{(1,1)},
    \label{eq:Hilbert_space_decomposition_Z2}
\end{equation}
where for each $\ket{\phi} \in \Hphys^{(n,m)}$ we have
\begin{equation}
    S_1 \ket{\phi} = (-1)^{m} \ket{\phi}
    \qand
    S_2 \ket{\phi} = (-1)^{n} \ket{\phi},
\end{equation}
where $n, m = 0, 1$.

The string operators $\Wilson_{1,2}$ do not commute with the Hamiltonian \eqref{eq:z2_lgt_hamiltonian}, hence they cannot be used to classify the states in $\Hphys$.
On the other hand, given the algebraic relations \eqref{eq:anticommutation_W_S_toric}, they are able to modify the effect of $\overline{S}_{1,2}$.
In fact, $\Wilson_{1,2}$ can change the super-selection sectors:
\begin{equation}
    W_1 : \Hphys^{(n,m)} \mapsto \Hphys^{(n+1,m)}
    \qand
    W_2 : \Hphys^{(n,m)} \mapsto \Hphys^{(n,m+1)},
    \label{eq:action_W_on_sectors}
\end{equation}
where the $n,m = 0, 1$ and the addition is taken modulus $2$.


\begin{figure}[t]
    \centering
    \begin{tikzpicture}[scale=0.4]
    \draw[lattice] (-1, -1) grid (9, 9);
    \draw[up] (2, 6) rectangle +(2,2);
    \draw[up] (0, 0) rectangle +(2,2);
    \draw[up] (4,0) -- (8,0) -- (8,6) -- (6,6) -- (6,4) -- (4,4) -- cycle;

    \draw[Z, dashed] (-1.5, 5) -- (9.5, 5)
        node [pos=0, below left, black] {$\overline{S}_1 = +1$};


    % sites
    \DrawSites{0,2,...,8}{0,2,...,8}

    % starting point for W2 arrow
    \node (start) at (2, 9) {};

    \begin{scope}[xshift=14cm]
        \draw[lattice] (-1, -1) grid (9, 9);
        \draw[up] (2, 9) -- (2, 8) -- (4, 8) -- (4, 6) -- (2, 6) -- (2, 2) -- (0, 2) -- (0, 0) -- (2, 0) -- (2, -1);
        \draw[up] (4,0) -- (8,0) -- (8,6) -- (6,6) -- (6,4) -- (4,4) -- cycle;

        \draw[Z, dashed] (-1.5, 5) -- (9.5, 5)
            node [pos=1, below right, black] {$\overline{S}_1 = -1$};

        % sites
        \DrawSites{0,2,...,8}{0,2,...,8}

        % stopping point for W2 arrow
        \node (stop) at (2, 9) {};
    \end{scope}

    \draw[] (start)
        edge[bend left, X, ->]
        node [pos=0.5, above, font=\small, black] {$\overline{W}_2$} (stop) ;
\end{tikzpicture}

    \vspace*{-0.5cm}
    \caption[States in different super-selection sectors]{%
        Pictorial representation of states of different super-selection sector and the action of the string operator $W_2$.
        Notice that the action $\Wilson_2$ introduces a non-contractible electric loop in the state, which modifies the value of $\overline{S}_1$, which, in a sense, measure the presence of non-contractible electric loops in the orthogonal direction $\hat{2}$.%
        }
    \label{fig:sector_mapping}
\end{figure}

From a more physical point of view, the operator $\overline{S}_i$ (with $i=1,2$) measures the presence non-contractible electric loops in the state in the direction orthogonal to $\hat{i}$.
Therefore, the decomposition in \eqref{eq:Hilbert_space_decomposition_Z2} divides the physical Hilbert space by the number of non-contractible electric loops in each direction.
On the other hand, the operators $\Wilson_i$ introduces a non-contractible electric loop in the $\hat{i}$ direction, which explains \eqref{eq:action_W_on_sectors} .
This can be seen in Fig.~\ref{fig:sector_mapping}.
Notice that in the case of $\Z_2$ \ac{lgt} we can have at most one non-trivial electric loops.
For examples, two parallels electric loops can be obtained by a strip of $B_p$ operators, without requiring the $\Wilson$ string operators.


% SECTION: Generalization to ZN
%----------------------------------------
% SECTION:
%----------------------------------------
\section{Generalization to \texorpdfstring{$\Z_N$}{Z\_N}}%
\label{sec:generalization_to_zn}
In this section we are going to generalize the $\Z_2$ \ac{lgt} to a class of Abelian \ac{lgt} with discrete symmetry $\Z_N$.
This class is of particular interest because they approximate a $U(1)$ gauge theory in the limit $N to \infty$.



%
% SUBSECTION: Schwinger-Weyl algebra
%
\subsection{Schwinger-Weyl algebra}%
\label{sub:schwinger_weyl_algebra}

According to Wilson's Hamiltonian approach to \ac{lgt}s \cite{wilson1974confinement}, $U(1)$ gauge fields are defined on the links of a lattice $\lattice$ either in a pair of conjugate variables, the electric field  $E_\link$ and either the vector potential $A_\link$, satisfying
\begin{equation}
    \comm{E_\link}{A_{\link^{\prime}}}  = i \delta_{\link , \link^{\prime} },
\end{equation}
or equivalently the \emph{magnetic operator}, also called \emph{comparator},
$U_\link = e^{-i A_{\link} }$, such that
\begin{equation}
    \comm{E_\link}{U_{\link^{\prime}}} =  \delta_{\link , \link^{\prime} } \, U_{\link},
\end{equation}
all acting on an infinite dimensional Hilbert space defined on each link $\link \in \lattice$.
This form of the canonical commutation relations represents the infinitesimal version of the relations:
\begin{equation}
     e^{i\xi E} e^{-i\eta A } e^{-i\xi E} = e^{i\xi \eta} e^ {-i\eta A },
\end{equation}
for any $\xi, \eta \in \mathbb{R}$,
that define the Schwinger-Weyl group \cite{notarnicola2015discrete, ercolessi2018znmodels, schwinger1960unitary}.

For a discrete group like $\Z_N$, the notion of infinitesimal generators loses any meaning and we are led to directly consider, for each link $\link \in \lattice$, two unitary operators
$V_\link, \, U_\link$, such that \cite{schwinger1960unitary, schwinger2001symbolism}
\begin{equation}
    V_\link U_\link V_\link^{\dagger}=e^{2\pi i/N}U_\link, \qquad
    U_\link^N=\identity_N, \qquad
    V_\link^N=\identity_N.
    \label{eq:schwinger_weyl_algebra}
\end{equation}
while they commute on different links.
Thus, by representing $\Z_N$  with the set of the $N$ roots of unity $e^{i 2 \pi k/N}$\, ($k=1, \cdots, N$), commonly referred to as the discretized circle,
we see that $V$ plays the role of a ``position operator'' on the discretized circle, while $U$ that of a ``momentum operator''.


\begin{figure}[t]
    \SideFigure[label=eq:link_operator_ZN, desc={Link operators in the $\Z_5$ case}]{%
        \begin{tikzpicture}[scale=1.1]
    \draw (1,0) arc (0:360:1);
    \foreach \angle in {72, 144, ..., 360}
    \node[site, fill=Yellow, minimum size=7pt] (state \angle) at (\angle:1) {};
    \draw (state 360.east) node[right] {$1$};
    \draw (state 72.north)  node[above] {$\omega$};
    \draw (state 144.west) node[above left] {$\omega^2$};
    \draw (state 216.west) node[left] {$\omega^{3}$};
    \draw (state 288.south) node[below] {$\omega^{4}$};

    \draw[Z, -stealth, shorten >=5pt] (0,0) -- (72:1) node [pos=0.5, left, black] {$V$};

    \draw[X, -stealth] (1.5, 0) arc (0:40:1.5) node [pos=0.5, right, black] {$U$};

    \draw (0,0) node[site, fill=black, draw=black] {};

\end{tikzpicture}

    }{%
        The operators $U$ and $V$ of a single link, in the $\Z_5$ case. The $V$ plays the role of a position operator, while $U$ that of a shift operator.
    }
\end{figure}


These algebraic relations admit a faithful finite-dimensional representation of dimension $N$ \cite{weyl1950theory}, for any integer $N$, which is obtained as follows.
To each link $\link$, we can associate an $N$-dimensional Hilbert space $\HilbertSpace_\link \simeq \C^N$.
As an orthonormal basis for $\HilbertSpace_\link$ we choose the \emph{electric basis} $\{\ket{v_{k,\link}}, k=1, \dots, N\}$, that diagonalizes the operator $V_\link$.
With this choice, we can promptly write the actions of $U_\link$ and $V_\link$:
\begin{equation}
    \begin{split}
        U_\link \ket{v_{k,\link}} & = \ket{v_{k+1,\link}}, \qquad \text{(mod $N$)} \\
        % U\ket{v_{N,\link}} = \ket{v_{1,\link}}\\
        V_\link \ket{v_{k,\link}} & = \omega^k \ket{v_{k,\link}},
    \end{split}
    \label{eq:elect_basis_op_action}
\end{equation}
where $\omega = e^{2 \pi i / N}$ and $k = 0, \dots, N-1$.
It is immediate to find the action for the Hermitian conjugates $U^{\dagger}_\link$ and $V^{\dagger}_\link$:
\begin{equation}
    \begin{split}
        U^\dagger_\link \ket{v_{k,\link}} & = \ket{v_{k-1,\link}}, \qquad \text{(mod $N$)} \\
        V^\dagger_\link \ket{v_{k,\link}} & = \omega^{-k} \ket{v_{k,\link}}.
    \end{split}
    \label{eq:elect_basis_op_action_hc}
\end{equation}
With this choice, $U_\link$ and $V_\link$ in matrix form are written as
\begin{equation}
    U_\link =
    \setlength\arraycolsep{5pt}
    \begin{pmatrix}
        0      & 0      & \cdots & \cdots & 1      \\
        1      & 0      & \cdots & \cdots & 0      \\[-5pt]
        0      & 1      & \ddots &        & . \\[-5pt]
        \vdots & \vdots & \ddots & \ddots & \vdots \\
        0      & 0      & \cdots & 1      & 0
    \end{pmatrix}
    \qand
    \setlength\arraycolsep{2pt}
    V_\link =
    \begin{pmatrix}
        1 \\
        & \omega \\
        &         & \omega^2 \\[-7pt]
        &         &           & \ddots \\[-2pt]
        &         &           &         & \omega^{N-1}
    \end{pmatrix}.
\end{equation}
We choose to work in this particular basis and the various $k$ can be interpreted as the quantized values of the electric field on the links.

In the $\Z_N$ case it is important to fix the orientation of the lattice $\lattice$, because for $N \geq 3$ we have $U^{\dagger} \neq U$ and $V^{\dagger} \neq V$.
We choose the orientation shown in Fig.~\ref{fig:link_labels}.
On a two-dimensional square lattice of size $L \times L$, the links $\link$ of the lattice can also be labeled with $(x, \pm\hat{i})$, where $x \in \lattice$ is a site and
$\hat{i}=\hat{1}, \hat{2}$ the two independent unit vectors.
In this way, $(x, \pm\hat{i})$ will refer to the link that start in $x$ and goes in the positive (negative) direction~$\hat{i}$.
As we will see later, the choice of the orientation affects the definition of any string operator.
The general rule for when defining a string operator as a product of $\mathcal{O}$ operators, where $\mathcal{O}$ is either $U$ or $V$ for example, is to use $\mathcal{O}$ when going in the positive direction or $\mathcal{O}^{\dagger}$ otherwise.
%This notation will be simplified when we reduce to the ladder case.


\begin{figure}
    \SideFigure[label=fig:link_labels, desc={Sites and links in a two-dimensional lattice}]{%
        \begin{tikzpicture}[
        font=\footnotesize,
        scale=1.3,
        site/.style = {circle, inner sep=0 pt, minimum size=5pt, draw=black, fill=white},
        decoration={
            markings,
            mark=at position 0.35 with {\arrow{>}},
            mark=at position 0.85 with {\arrow{>}}
        }
    ]
    % Lattice
    \draw[Gray, thin] (-1.75, -1.5) grid (1.75, 1.5);
    % links
    \draw[very thick, postaction={decorate}]
        (-1, 0) -- (1, 0)
        node [pos=0.75, below] {$(x, \hat{1})$}
        % node [pos=0.25, below] {$(x, -\hat{1})$}
        ;
    \draw[very thick, postaction={decorate}]
        (0, -1) -- (0, 1)
        node [pos=0.8, right] {$(x, \hat{2})$}
        % node [pos=0.15, right] {$(x, -\hat{2})$}
        ;
    % sites
    \foreach \x in {-1,...,1}
        \foreach \y in {-1,...,1}
            \draw (\x, \y) node [site] {};
    \draw (0, 0) node [Gray, above right] {$x$};
    \draw (1, 0) node [Gray, above right] {$x + \hat{1}$};
    \draw (0, 1) node [Gray, above right] {$x + \hat{2}$};
\end{tikzpicture}

    }{%
        Labelling of the sites and the links in the two dimensional lattice.
        A site is labeled simply with $x = (x_1, x_2)$, while $\hat{1} = (1,0)$ and $\hat{2} = (0,1)$ stand for the unit vectors of the lattice.
        A link $\link$ is denoted with a pair $(x, \pm\hat{i})$, with $\hat{i} = \hat{1}, \hat{2}$.
    }
\end{figure}


%
% SUBSECTION: Schwinger-Weyl algebra
%
\subsection{Gauge invariance and Hamiltonian}%
\label{sub:gauge_invariance_and_hamiltonian}

Gauge transformations transforms the vector potential while preserving the electric field.
For a $U(1)$ gauge theory, a local phase transformation is induced by a real function $\alpha_x$
defined on the vertices $x\in \mathbb L$, such that  $A_{\link} \mapsto A_{\link} + (\alpha_{x_2} - \alpha_{x_1})$ or equivalently
\begin{equation}
    U_{\link} \mapsto
    e^{i(\alpha_{x_2} - \alpha_{x_1}) E_{\link}}  U_{\link}   e^{-i(\alpha_{x_2} - \alpha_{x_1} )E_{\link}},
\end{equation}
where $x_1, x_2$ are the initial and final vertices of the (directed) link $\link$.
In the case of a discrete symmetry, a gauge transformation at a site $x \in \lattice$ is a product of $V$'s (and $V^\dagger$'s) defined on the links which comes out (and enters) the vertex.
More specifically, for a two dimensional lattice,
if the link $\link$ at site $x$ is oriented in the positive direction, i.e.~either $(x, +\hat{1})$ or $(x, +\hat{2})$, then $V$ is used, otherwise $V^\dagger$.
Thus, the single local gauge transformation at the site $x$ is enforced by the operator:
\begin{equation}
    G_x =
    V_{(x, \hat{1})}^{\phantom{\dagger}}
    V_{(x, \hat{2})}^{\phantom{\dagger}}
    V^\dagger_{(x, -\hat{1})}
    V^\dagger_{(x, -\hat{2})},
    \label{eq:gauss_operator}
\end{equation}
as shown in the left part of in Fig. \ref{fig:star_plaq_operators}.

The operators that enters the Hamiltonian have to be gauge invariant, i.e.~commute with all the operators $G_x$.
Using \eqref{eq:gauss_operator} and recalling \eqref{eq:schwinger_weyl_algebra}, it is possible to see that the $V_\link$'s commute with $G_x$ (as expected), while the $U_\link$'s do not.
In spite of that, we can build gauge-invariant operators out of the comparators $U_\link$.
Generalizing directly from \ac{tc} case, one another gauge-invariant operator is the \emph{plaquette operator}, which we will denote with $\PlaqOp$, that will play the role of the magnetic operator.
A plaquette now has an orientation.
Given a plaquette $\plaquette$ with vertices $\{x, x+\hat{1}, x+\hat{1}+\hat{2}, x+\hat{2}\}$, we consider the path that start from $x$ and goes in the counterclockwise direction.
On this plaquette, the operator $\PlaqOp$ is defined as follows:
\begin{equation}
    \PlaqOp =
    U_{(x, \hat{1})}
    U_{(x + \hat{1}, \hat{2})}
    U_{(x + \hat{1} + \hat{2}, -\hat{1})}^\dagger
    U_{(x + \hat{2}, -\hat{2})}^\dagger,
    \label{eq:plaq_operator}
\end{equation}
which can be seen on the right in Fig.~\ref{fig:star_plaq_operators}.

\begin{figure}[t]
    \SideFigure[label=fig:star_plaq_operators, desc={Gauss and plaquette operator in a $\Z_N$ \ac{lgt}}]{
        \begin{tikzpicture}[
        scale=1.2,
        site/.style = {circle, inner sep=0 pt, minimum size=3pt, draw=black, fill=white},
        decoration={
            markings,
            mark=at position 0.65 with {\arrow{>}}
        },
        plaq/.style={Blue, very thick, postaction={decorate}},
        gauss/.style={Red, very thick, postaction={decorate}}
        ]
    % Lattice
    \draw[Gray,thin] (-0.5,-0.5) grid (4.5,2.5);

    % Plaquette operator
    \draw[plaq] (3,1) -- (4,1) node [pos=0.5, below] {$U$};
    \draw[plaq] (4,1) -- (4,2) node [pos=0.5, right] {$U$};
    \draw[plaq] (4,2) -- (3,2) node [pos=0.5, above] {$U^\dagger$};
    \draw[plaq] (3,2) -- (3,1) node [pos=0.5, left]  {$U^\dagger$};
    \draw[Blue, ultra thick, pattern=north east lines, pattern color=Blue] (3,1) rectangle (4,2);
    \draw (3.5,1.5) node [fill=white, rounded corners] {$U_{\square}$};

    % Gauss operator
    \draw[gauss] (1, 1) -- (2, 1) node [pos=0.5, below right] {$V$};
    \draw[gauss] (1, 1) -- (1, 2) node [pos=0.5, above left] {$V$};
    \draw[gauss] (0, 1) -- (1, 1) node [pos=0.5, below left] {$V^\dagger$};
    \draw[gauss] (1, 0) -- (1, 1) node [pos=0.5, below right] {$V^\dagger$};

    \foreach \y in {0,1,2} \foreach \x in {0,1,...,4} \draw (\x,\y) node [site] {};

    \draw (1,1) node [above right, outer sep=5pt, inner sep=3pt, draw=Red, rounded corners=3pt] {$G_x$};
\end{tikzpicture}

    }{
        Pictorial representation of the Gauss operators $G_x$ in \eqref{eq:gauss_operator} (\emph{left}) and plaquette operator $\PlaqOp$ in \eqref{eq:plaq_operator} (\emph{right}).
    }
\end{figure}


The whole operator algebra $\algebra$ of the theory is generated by the set of all $U_\link$ and $V_\link$ (and their Hermitian conjugates) of all the links of the lattice $\lattice$, while the  \emph{gauge-invariant subalgebra} $\algebra_{\gi}$ consists of operators that commutes with all the $G_x$:
\begin{equation}
    \mathcal{A}_\gi = \{ O_{\gi} \in \mathcal{A} \;:\; [O_{\gi}, G_x] = 0 \quad \forall x \in \lattice \}.
\end{equation}
Guided by the \ac{tc}, we already know that the set $\{\PlaqOp, V_\link\}$ (for all plaquettes $\plaquette$ and all links $\link$) does not generate the whole algebra $\mathcal{A}_\gi$, in the case of periodic boundary conditions.
Indeed, we have yet to add string operators on non-contractible loops.

In Sec.~\ref{sub:topological_ground_states} we have already introduced the non-local operators $\Wilson_{i}$ and $\tHooft_{i}$, with $i=1,2$.
These can readily be generalized to the $\Z_N$ case, by replacing $X_{\link}$ and $Z_{\link}$ with $U_{\link}$ and $V_{\link}$ respectively.
More precisely, consider direct non-contractible loops $\mathcal{L}_i$ and cuts $\mathcal{C}_i$ (in the $i$-th direction).
Then $\Wilson_i$ and $\tHooft_i$ operators are defined as
\begin{equation}
    \Wilson_i = \prod_{\link \in \mathcal{L}_i} U_\link
    \qand
    \tHooft_i = \prod_{\link \in \mathcal{C}_i} V_\link,
    \label{eq:nonlocal_op_ZN}
\end{equation}
with the caveat that when going in the negative direction, $U^{\dagger}$ and $V^{\dagger}$ have to be used.
Operators $\Wilson_i$ will also be called \emph{\ac{wl}s}, while the $\tHooft_i$ will be called \emph{\ac{ths}s}.
These operators are pictured in Fig.~\ref{fig:nonlocal_operators}.

Both sets of operators, $\Wilson_i$ and $\tHooft$, are gauge invariant but only the \ac{wl}s cannot be expressed as product of neither $\PlaqOp$ and $V_{\link}$.
Therefore, they have to be added explicitly to the set of generators of $\mathcal{A}_\gi$ in order to obtain the whole algebra.
Similar to the \ac{tc}, these non-local operators will play a fundamental role in defining the super-selection sectors of the theory.

The class of models we consider are described by the Hamiltonian \cite{tagliacozzo2011entanglement, hamma2008adiabatic, trebst2007topological}:
\begin{equation}
    H_{\Z_N}(\lambda) = - \sum_{\plaquette} \PlaqOp - \lambda \sum_{\link} V_{\link} + \text{h.c.},
    \label{eq:hamiltonian_base}
\end{equation}
where the first sum is over the plaquettes $\plaquette$ of the lattice while the second sum is over the links $\link$.
% One can easily see that this Hamiltonian is local and gauge-invariant, hence the dynamics it describes it is fully contained in $\Hphys$.
As stated before, the operators $\PlaqOp$ plays the role of a \emph{magnetic} term, to be more precise it is the magnetic flux inside the plaquette $\plaquette$, while $V$ is the \emph{electric} term.
The coupling $\lambda$ tunes the relative strength of the electric and magnetic energy contribution.

\begin{figure}[t]
    \SideFigure[label=fig:nonlocal_operators, desc={Non-local operators in the $\Z_N$ \ac{lgt}}]{%
        \begin{tikzpicture}[
        site/.style = {circle, inner sep=0 pt, minimum size=3pt, draw=black, fill=white},
    ]
    % lattice grid
    \draw[Gray,thin] (-0.5,-0.5) grid (4.5,4.5);

    % Wilson loops
    \draw[Blue, ultra thick]
        (-0.5, 1) -- (4.5, 1)
        node [pos=1, right] {$\mathcal{C}_1$}
        node [pos=0.85, above, black] {$\overline{W}_1$}
        ;
    \draw[Blue, ultra thick]
        (3, -0.5) -- (3, 4.5)
        node [pos=0, below] {$\mathcal{C}_2$}
        node [pos=0.15, left, black] {$\overline{W}_2$}
        ;

    % 't Hooft strings
    \draw[Red, very thick, dashed]
        (-0.5,3.5) -- (4.5,3.5)
        node[pos=0, left] {$\tilde{\mathcal{C}}_1$}
        ;
    \foreach \x in {0,...,4} { \draw[Red, ultra thick] (\x, 3) -- +(0, 1); }
    \draw (2,3) node [below right] {$\overline{S}_1$};
    \draw[Red, very thick, dashed]
        (0.5,-0.5) -- (0.5,4.5)
        node[pos=1, above] {$\tilde{\mathcal{C}}_2$}
        ;
    \foreach \y in {0,...,4} { \draw[Red, ultra thick] (0, \y) -- +(1, 0); }
    \draw (1,2) node [below right] {$\overline{S}_2$};
    \foreach \y in {0,...,4} \foreach \x in {0,...,4} \draw (\x,\y) node [site] {};
\end{tikzpicture}

    }{Graphical representation of the non-local operators $\Wilson_{1,2}$ (in blue) and $\tHooft_{1,2}$ (in red) and their respective paths $\mathcal{L}_{1,2}$ and $\mathcal{C}_{1,2}$.}
\end{figure}


% These models are akin to the \ac{tc} \cite{kitaev2003fault}, which can be thought as a prime example of a $\Z_2$ \ac{lgt}.
% More precisely, $H_{\Z_2}$ in \eqref{eq:hamiltonian_base} can be thought as a \emph{deformation} of the former, where an external ``transverse'' field is added to it.
% Indeed, using the notation used so far, the \ac{tc} can be formulated as:
% \begin{equation}
%     H_{\text{TC}} = - J_m \sum_{\plaquette} \PlaqOp - J_e \sum_{x} G_x.
%     \label{eq:hamiltoniana_toric_code}
% \end{equation}
% whose ground states $\ket{\Psi}$ satisfies the constraints
% \begin{equation}
%     \PlaqOp \ket{\Psi} = \ket{\Psi} \;\; \forall \; \plaquette, \quad
%     G_x \ket{\Psi} = \ket{\Psi} \;\; \forall x.
%     \label{eq:constraints_gs_toric_code}
% \end{equation}
% Only elementary excitations above the ground state can violate these constraints and they can be of two type: a \emph{magnetic vortex} (which violates the plaquette constraint) or a \emph{electric charge} (which violates the Gauss law).
% If one imposes $J_e \gg J_m$ to enforce Gauss law, in the low-energy sector there are no electric charges and one recovers the pure gauge $\Z_2$ model of \eqref{eq:hamiltonian_base} for $\lambda = 0$.
% Therefore, in general the $\Z_N$ models described in \eqref{eq:hamiltonian_base} can be considered as generalization of the \ac{tc}, from the point of view of \ac{lgt}s.


%
% SUBSECTION: Physical Hilbert space and super-selection sectors
%
\subsection{Physical Hilbert space and super-selection sectors}
\label{sub:physical_hilbert_space_and_super_selection_sectors}


% \todo{Sottosezione da riscrivere in luce della sezione sul toric code}
The total Hilbert space $\mathcal{H}_{\text{tot}}$ is given by
\begin{equation}
    \HilbertSpace_{\text{tot}} = \bigotimes_{\link \in \lattice} \HilbertSpace_{\link},
\end{equation}
where $\HilbertSpace_\link \simeq \C^N$ in the case of $\Z_N$ theory.
A state of the whole lattice $\ket{\phi_{\phys}} \in \HilbertSpace_{\text{tot}}$ is said to be \emph{physical} if it is a \emph{gauge-invariant state}:
\begin{equation}
    % G_x \ket{\Psi_{\text{ph}}} = \ket{\Psi_{\text{ph}}}, \qquad \forall x \in \lattice
    G_x \ket{\phi_{\phys}} = \ket{\phi_\phys}, \qquad \forall \, x \in \lattice.
    \label{eq:gauss_law}
\end{equation}
This condition can be translated into a constraint on the eigenvalues $v_{\link}$ of the electric operators $V_{\link}$.
Given that a link $\link$ can be referred to as $(x, \hat{i})$,  then the constraint \eqref{eq:gauss_law} can be translated to
% $v_{(x, \pm \hat{i})}= \omega^{k_{(x, \pm \hat{i})} }$  of the operators $V_\link$ on the links $\link = (x, \pm \hat{i})$ of the vertex $x$:
\begin{equation}
    v_{(x, \hat{1})}^{\phantom{\ast}}
    v_{(x, \hat{2})}^{\phantom{\ast}}
    v_{(x, -\hat{1})}^\ast
    v_{(x, -\hat{2})}^\ast = 1.
\end{equation}
For a $\Z_N$ theory we have $v_\link = \omega^{k_{\link}}$, where $\omega = e^{i2 \pi / N}$, which leads to
% or, because of (\ref{eq:elect_basis_op_action}):
\begin{equation}
    \sum_{i=1,2} \pqty{ k_{(x, \hat{i})} - k_{(x, -\hat{i})} } = 0 \quad \text{mod $N$}.
    \label{eq:gauss_law_elec_eigvals}
\end{equation}
for \eqref{eq:gauss_law}.
Given the fact that the $k$ in \eqref{eq:schwinger_weyl_algebra} represent the values of the electric field, one can see that \eqref{eq:gauss_law_elec_eigvals} can be interpreted as a discretized version of the Gauss law $\nabla \cdot \vec{E} = 0$ in two dimensions, for a pure gauge theory.

% Let us consider the \ac{tc}.
% One of its main features is the presence of topologically protected degenerate ground states \cite{kitaev2003fault}.
Consider now the \emph{physical Hilbert space} for a $\Z_N$ theory:
\begin{equation}
    \Hphys = \{ \ket{\phi_\phys} : G_x \ket{\phi_\phys} = \ket{\phi_\phys} \quad \forall \, x \in \lattice \}.
    \label{eq:decomposizione_Hphys}
\end{equation}
This space can be decomposed into super-selection sectors, like it has been done for the $\Z_2$ theory in Sec.~\ref{sub:super_selection_sectors}.
In fact, it can be generalized in a straightforward way, using the string operators in \eqref{eq:nonlocal_op_ZN} (showed in Fig.~\ref{fig:nonlocal_operators}).
The physical Hilbert space $\Hphys$ decomposes as
\begin{equation}
    \Hphys = \bigoplus_{n,m=0}^{N-1} \Hphys^{(n,m)},
\end{equation}
where each sector $(n,m)$ satisfy
\begin{equation}
    S_1 \ket{\phi} = \omega^{m} \ket{\phi}
    \qand
    S_2 \ket{\phi} = \omega^{n} \ket{\phi}
\end{equation}
for $\ket{\phi} \in \Hphys^{(n,m)}$.
This is possible because the \ac{ths}s $\tHooft$ commutes with all the terms of the Hamiltonian.

On the other hand, the \ac{wl}s $\Wilson_i$ do not commute with all the Hamiltonian \eqref{eq:hamiltonian_base}, in particular with the electric operators $V_{\link}$, but are still gauge-invariant.
Nonetheless, we are interested in the commutation relation between the \ac{wl}s $\Wilson_i$ and 't Hoof string $\tHooft_i$:
\begin{equation}
    W_1 S_2 = \omega S_2 W_1
    \qand
    W_2 S_1 = \omega S_1 W_2.
    \label{eq:relation_W_S_ZN}
\end{equation}
It is a direct generalization of the relations \eqref{eq:anticommutation_W_S_toric} of the \ac{tc}, where the phase $-1$ is substituted with a characteristic phase $\omega$.
Given \eqref{eq:relation_W_S_ZN}, it is easy to see that the \ac{wl}s have the ability to change the super-selection sectors:
\begin{equation}
    W_1 : \Hphys^{(n,m)} \to \Hphys^{(n+1,m)}
    \qand
    W_2 : \Hphys^{(n,m)} \to \Hphys^{(n,m+1)},
    \label{eq:azione_wilson_loop}
\end{equation}
where the addition is taken modulus $N$.

% In order to illustrate this, besides $\Wilson_1$ and $\Wilson_2$, defined in \eqref{eq:nonlocal_op_ZN}, another type of non-local operators have to be introduced.
% They are defined on \emph{cuts} of the lattice $\lattice$, i.e.~paths on the dual lattice $\tilde{\lattice}$.
% Consider \emph{non-contractible} cuts $\tilde{\mathcal{C}}_1$ and $\tilde{\mathcal{C}}_2$ along the directions $\hat{1}$ and $\hat{2}$, respectively.
% On this cuts, the (\ac{ths}) operators $\tHooft_1$ and $\tHooft_2$ are constructed as

% in a similar fashion to \eqref{eq:nonlocal_op_ZN}.
% This is shown in red in Fig.~\ref{fig:nonlocal_operators}.
% The operators $\Wilson_i$ and $\tHooft_i$ ($i=1,2$) commutes with all the operators $\PlaqOp$ and $G_x$ in the \ac{tc} Hamiltonian $H_{\text{TC}}$ of
% TODO aggiustare ref
%\eqref{eq:hamiltoniana_toric_code},
% but do not commute with each other.
% In fact, we have $\Wilson_i \tHooft_j = - \tHooft_j \Wilson_i$ if $i \neq j$.
% This means that $H_{\text{TC}}$ can be block-diagonalized with respect to the eigenvalues of $\tHooft_i$ (or $\Wilson_i$), while $\Wilson_j$ (or $\tHooft_j$) connects one block to the other.
% Furthermore, since in the case of the  $\Z_2$ symmetry,
% $\tHooft_i$ (or $\Wilson_i$) has only two eigenvalues (equal to $\pm 1$), there are a total of $2 \times 2 = 4$ degenerates ground states, which are topologically protected, thanks to the fact
% that $\Wilson_j$ (or $\tHooft_j$) cannot be expressed in terms of the local operators $\PlaqOp$ and $G_x$.
% Notice that, as it can be easily seen, in the \ac{tc} the role of $\Wilson_i$ and $\tHooft_i$ can be interchanged.

% Let us now turn to $\Z_N$ \ac{lgt} models.
% The operators $\Wilson_i$ no longer commute with the Hamiltonian \eqref{eq:hamiltonian_base} which now contains an electric field term.
% Thus, $\lambda \neq 0$, we have no degenerate ground states.
% But we can still use the $\tHooft_i$ operators to decompose the Hilbert space $\Hphys$, since they still commute with all the \emph{local operators} $\PlaqOp$ and $V_{\link}$ (thus also with $H_{\Z_N}$).
% Now one can see that the operator $\tHooft_i$ ($i=1,2$) of \eqref{eq:nonlocal_op_ZN} has $N$ eigenvalues $\omega^n$, with $n=1, \dots, N-1$.
% Hence, one can decompose $\Hphys$ as sum of super-selection sectors
% \begin{equation}
%     \Hphys = \bigoplus_{n, m=0}^{N-1} \Hphys^{(n, m)},
%     \label{eq:decomposizione_Hphys}
% \end{equation}
% where for each $\ket{\phi} \in \Hphys^{(n, m)}$ we have:
% \begin{equation}
%     \tHooft_1 \ket{\phi} = \omega^{m}\ket{\phi}, \quad
%     \tHooft_2 \ket{\phi} = \omega^{n}\ket{\phi}.
% \end{equation}
% Let us consider now the role of the \ac{wl}s $\Wilson_i$.
% One can easily see that:
% \begin{equation}
%     \Wilson_2 \tHooft_1 = \omega \tHooft_1 \Wilson_2, \qquad
%     \Wilson_1 \tHooft_2 = \omega \tHooft_2 \Wilson_1.
% \end{equation}
% It follows that $\Wilson_{1,2}$ acts a shift operators for the eigenspaces of $\tHooft_{2,1}$:
% \begin{equation}
%     \Wilson_1 : \Hphys^{(n, m)} \to \Hphys^{(n + 1, m)}, \quad
%     \Wilson_2 : \Hphys^{(n, m)} \to \Hphys^{(n, m + 1)},
% \end{equation}
% where the integers $n + 1$ and $m + 1$ have to be taken $\mathrm{mod}\; N$.

From a physical point of view, the \ac{wl}s operators $\Wilson_1$ and $\Wilson_2$ create non-contractible electric loops around the lattice, while  the \ac{ths}s $\tHooft_2$ and $\tHooft_1$ detect the presence and the strength of these electric loops.
Exactly like in the case of the $\Z_2$ \ac{lgt}, but the the difference that now the non-contractible electric strings can have different ``strengths'', given by the different eigenvalues of $\tHooft_i$.
Therefore, it is clear that the Hilbert subspace $\Hphys^{(n, m)}$ is the subspace of all the states that contains an electric loop of strength $\omega^n$ and $\omega^{m}$ along the $\hat{1}$ and $\hat{2}$ direction, respectively.

Furthermore, the evolution of a state in $\Hphys^{(n,m)}$ with the Hamiltonian in \eqref{eq:hamiltonian_base} is confined in $\Hphys^{(n,m)}$.
This is because none of the local terms in the Hamiltonian can change the super-selection sector, only the non-local \ac{wl}s.
In this chapter we will see how this fact can have major consequences when considering $\Z_N$ models on particular lattice geometries, in particular on the \emph{ladder}.


% SECTION: Abelian models on the ladder
%--------------------------------------------------
% SECTION: Abelian models on the ladder
%--------------------------------------------------
\section{Abelian models on the ladder}
\label{sec:abelian_models_on_the_ladder}

In this short chapter we will introduce $\Z_N$ \ac{lgt} on a \emph{ladder geometry}.
This type of lattice can be considered as a strip of a two-dimensional square lattice.
The peculiarity of this geometry is that it allows the existence of magnetic terms in a quasi one-dimensional lattice, which usually are not possible in a pure one-dimensional systems.
Moreover, since the Hilbert space is highly constrained, it allows the possibility to study systems of moderate size through exact diagonalization.
The latter will be analyzed in the last section.

A \emph{ladder} is a lattice $\mathbb{L}$ made of two parallels chains, the \emph{legs}, coupled to each other by the \emph{rungs} to form square plaquettes.
On the ladder, each rung is identified by a coordinate $i=1,\dots,L$, where $L$ is the length of the ladder, and the two vertices on the rung are denoted with $i^{\uparrow}$ and $i^{\downarrow}$ in the upper and lower leg, respectively (see Fig.~\ref{fig:ladder_operators}).
Links, as usual, will be denoted by $\link$.
On the legs they are labelled as $\toplink_i$ (upper leg) or $\botlink_i$ (lower leg), while on the rungs they are labelled $\runglink_i$.

\begin{figure}[t]
    \centering
    \begin{tikzpicture}[
    scale=0.5,
    ]

    %
    % Link operators
    %

    % ladder
    \draw[lattice] (1,0) grid (9,2);

    % label
    \node [font=\small] at (5, 3.6) {Link operators};

    % Magnetic field operator field operators
    \node[below] at (2,0) {$i^{\downarrow}$};
    \node[above] at (2,2) {$i^{\uparrow}$};
    \draw[U] (2,0) -- (2,2) node [pos=0.5, right] {$\Urung_i$};
    \draw[U] (2,2) -- (4,2) node [pos=0.5, above] {$\Uup_i$};
    \draw[U] (2,0) -- (4,0) node [pos=0.5, below] {$\Udown_i$};

    % Electric field operators
    \node[below] at (6,0) {$j^{\downarrow}$};
    \node[above] at (6,2) {$j^{\uparrow}$};
    \draw[V] (6,0) -- (6,2) node [pos=0.5, right] {$\Vrung_j$};
    \draw[V] (6,2) -- (8,2) node [pos=0.5, above] {$\Vup_j$};
    \draw[V] (6,0) -- (8,0) node [pos=0.5, below] {$\Vdown_j$};

    % Sites
    \DrawSites{2,4,...,8}{0,2}

    %
    % Local operators
    %
    \begin{scope}[xshift=12cm]
        % ladder
        \draw[lattice] (-1,0) grid (13,2);

        % labels
        \node[font=\small] at (8.3, 3.6) {Gauss op.};
        \node[font=\small] at (1, 3.6) {Plaq.~op.};

        % Plaquette operator
        \draw[Grey80] (0,0) node [below] {$i^{\downarrow}$};

        % \draw[Grey80] (0,2) node [above] {$i^{\uparrow}$};
        \draw[Blue80, ultra thick, pattern=north east lines, pattern color=Blue80] (0,0) rectangle +(2,2);
        \draw[U] (0,0) -- (2,0);
        \draw[U] (2,0) -- (2,2);
        \draw[U] (2,2) -- (0,2);
        \draw[U] (0,2) -- (0,0);
        \node at (1,1) [fill=white, rounded corners, text=Blue80] {$U_i$};

        % Gauss law Up
        \draw[V] (6,0) -- (6,2);
        \draw[V] (4,2) -- (6,2);
        \draw[V] (6,2) -- (8,2);
        \node[Red, above] at (6,2) {$\GaussUp_j$};
        \node[below left] at (6, 2) {$j^{\uparrow}$};

        % Gauss law down
        \draw[V] (8,0) -- (10,0);
        \draw[V] (10,0) -- (10,2);
        \draw[V] (10,0) -- (12,0);
        \node[Red][below] at (10,0) {$\GaussDown_k$};
        \node[above left] at (10, 0) {$k^{\downarrow}$};

        % Sites
        \DrawSites{0,2,...,12}{0,2}
    \end{scope}

\end{tikzpicture}

    \caption[Operators of a $\Z_N$ ladder \ac{lgt}]{%
        Picture of the different ladder operators.
        \emph{Left}: the magnetic and electric link operators.
        \emph{Right}: plaquette operator $U_i$ and the Gauss operators $\GaussUp_j$ and $\GaussDown_k$.
        Notice that operators and sites on the upper leg are indicated with an up arrow, on the lower leg with a down arrow and on the rungs with a superscript $0$.
    }
    \label{fig:ladder_operators}
\end{figure}


We preserve the same formulation of $\Z_N$ \ac{lgt} but in order to lighten our notation,
we use the symbols $V^0_i, \; U^0_i$ for the operators defined on the rung $i$, and  $V^{\rho}_i, \; U^{\rho}_i$ with $\rho = \uparrow, \downarrow$ for the operators on the horizontal links of the upper and lower leg, respectively, to the right of the rung.
In synthesis:
\begin{equation}
    \begin{split}
        U_{\runglink_i} & \equiv \Urung_i, \quad
        U_{\botlink_i}   \equiv \Udown_i, \quad
        U_{\toplink_i}   \equiv \Uup_i \\
        V_{\runglink_i} & \equiv \Vrung_i, \quad
        V_{\botlink_i}   \equiv \Vdown_i, \quad
        V_{\toplink_i}   \equiv \Vup_i.
    \end{split}
\end{equation}
% and see Fig.~\ref{fig:ladder_operators}.
The plaquette operator on the right of the rung $i$ will be labeled as $U_i$:
\begin{equation}
    U_i = \Udown_i \Urung_{i+1} (\Uup_i)^{\dagger} (\Urung_i)^{\dagger}.
    \label{eq:plaq_op_ladder}
\end{equation}
Moreover, on a ladder the vertices are three-legged, so the Gauss operators are slightly modified:
\begin{equation}
    \GaussUp_i
    = \Vup_i ( \Vup_{i-1} )^\dagger ( \Vrung_i )^\dagger \text{~and~}
    \GaussDown_i
    = \Vdown_i \Vrung_i ( \Vdown_{i-1} )^\dagger,
    \label{eq:gauss_law_ladder}
\end{equation}
where $\GaussUp_i$ and $\GaussDown_i$ refers, respectively, to the Gauss operators on the vertices $i^{\uparrow}$ and $i^{\downarrow}$.
As a reference see Fig.~\ref{fig:ladder_operators}.

Finally, we write explicitly the Hamiltonian for a $\Z_N$ \ac{lgt} on a ladder:
\begin{equation}
    \HamilLadder(\lambda) =
    - \sum_{i} \bqty{ U_i + \lambda \pqty{ \Vup_i + \Vdown_i + V^0_i } + \text{h.c.} }.
    \label{eq:ladder_hamiltonian}
\end{equation}


For what concerns the super-selection sectors of the theory,
non-contractible loops are possible now only in the $\hat{2}$ direction.
Therefore, out of the \ac{wl} operators in \eqref{eq:nonlocal_op_ZN} only $\overline{W}_1$ is well defined, meaning that we can create non-contractible electric loops along the $\hat{1}$.
Hence, only $\overline{S}_2$ in \eqref{eq:nonlocal_op_ZN} (the \ac{ths} conjugate to $W_1$) can be used as a mean for distinguishing these different sectors.
Explicitly, the \ac{wl} $\Wilson_1$ and $\tHooft_2$ can be written as
\begin{equation}
    \Wilson_1 = \prod_{i} \Udown_i \qand
    \tHooft_2 = \Vup_{i_0} \Vdown_{i_0},
\end{equation}
where $i_0$ is any chosen rung (see Fig.~\ref{fig:nonlocal_operators_ladder}).
Furthermore, it does not make sense to consider the \ac{ths} $S_1$ because it is equal to the product of all the Gauss operators on either one of the legs,
\begin{equation}
    \tHooft_1 = \prod_{i} \GaussDown_i = \prod_{i} \GaussUp_i,
\end{equation}
so it always equal to the identity on physical states, signaling the obvious fact that we do not have non-contractible electric loops around the $\hat{2}$ direction.
We can conclude that the physical Hilbert space can be decomposed in only $N$ sectors as
\begin{equation}
    \Hphys = \Hphys^{(0)} \oplus \Hphys^{(1)} \oplus \dots \oplus \Hphys^{(N-1)},
\end{equation}
and in each sector we have that
\begin{equation}
    S \ket{\phi} = \omega^n \ket{\phi} \quad \text{if} \quad \ket{\phi} \in \Hphys^{(n)}.
\end{equation}

\begin{figure}[t]
    \SideFigure[label=fig:nonlocal_operators_ladder, desc={Non-local operators on the ladder}]{%
        \begin{tikzpicture}[
        scale=0.5,
        font=\small
    ]
    % ladder
    \draw[lattice] (1,0) grid (11,2);

    % S string
    \draw[Z, dashed] (5, -0.8) -- (5, 2.8)
        node [above, black] {$\tHooft_2 = \Vup_{i_0} \Vdown_{i_0}$};
    \draw[Z, ->-=0.5] (4, 2) -- (6, 2);

    % W string
    \draw[X] (1, 0) -- (11, 0);
    \node [below, black] at (9, 0) {$\Wilson_1 = \prod_{i} U^{\downarrow}_i $};
    \foreach \x in {2, 4, ..., 8} \draw[U] (\x, 0) -- +(2,0);

    % sites
    \DrawSites{2,4,...,10}{0,2}
\end{tikzpicture}

    }{%
        Picture of the non-local string operators $\Wilson_1$ and $\tHooft_2$ on the ladder.%
    }
\end{figure}


Due to the fact that the ladder is quasi one-dimensional, the presence of non-contractible electric loops can highly affects the physical states.
Take the case of a $\Z_2$ theory, which is pictured in Fig.~\ref{fig:states_different_sectors}.
It has just two sectors: $n=0$ and $n=1$.
In the former all the physical configuration are made of closed loop, distributed along the $\hat{1}$ direction.
While in the latter, the physical configurations are just deformations of of one single electric loop that goes around the ladder.
This can make us reasonably believe that the two sectors might have completely different physical content.


\begin{figure}[t]
    \SideFigure[label=fig:states_different_sectors, desc={Physical states in different sectors in the $\Z_2$ ladder \ac{lgt}}]{%
        \begin{tikzpicture}[scale=0.4]
    %
    % Sector n=0
    %

    % ladder
    \draw[lattice] (-1,0) grid (13,2);
    \node[font=\small, above] at (2,2.5) {$\Z_2$ sector $n=0$};

    % loops
    \draw[up] (2, 0) rectangle (4,2);
    \draw[up] (6, 0) rectangle (10,2);
    \draw[up] (13, 0) -- (12, 0) -- (12, 2) -- (13, 2);

    % sites
    \DrawSites{0,2,...,12}{0,2}

    %
    % Sector n=1
    %
    \begin{scope}[yshift=-5cm]
        % ladder
        \draw[lattice] (-1,0) grid (13,2);
        \node[font=\small, above] at (2,2.5) {$\Z_2$ sector $n=1$};

        % big loop
        \draw[up]
        (-1, 0) -- (2, 0) -- (2, 2) -- (4, 2) -- (4, 0) --
        (6, 0) -- (6, 2) -- (10, 2) -- (10, 0) --
        (12, 0) -- (12, 2) -- (13, 2);

        % sites
        \DrawSites{0,2,...,12}{0,2}
    \end{scope}
\end{tikzpicture}

    }{%
        Example of two physical configurations (in the electric basis) in a $\Z_2$ theory in the two different super-selection sectors.
        This shows that states belonging to two different sectors can be quite different.%
    }
\end{figure}


Like in the two-dimensional case, the Hamiltonian can be reduced to a single super-selection sector.
One of the main features of this is that once the sector is fixed, it is possible to write a duality transformation of the  Hamiltonian to a pure one-dimensional \emph{quantum clock model}, resolving entirely the \emph{gauge symmetries}.
Thanks to this duality map, we will see how that the different sectors have very different behaviour and each can have its own unique phase diagram.
The latter is the object of discussion of the second part of this chapter, but before doing so we need to introduce the notion of \emph{dualities} and, in particular, \emph{the bond-algebraic approach to dualities}.


% SECTION: Bond-algebra approach to dualities
%----------------------------------------
% SECTION: Dualities in physics
%----------------------------------------
\section{Dualities in physics}%
\label{sec:dualities_in_physics}


Duality is a simple yet powerful idea in physics.
They can be intended as specific mathematical transformations connecting seemingly unrelated physical phenomena.
They have been know for a long time, indeed a first example would be the duality of the electromagnetic field in the absence of sources, noticed by Heaviside in 1884.
Generally in physics, the concept of duality is connected to ideas, like symmetries, mappings between different coupling regimes, perturbative expansions for strongly correlated systems, and the wave-particle duality of quantum mechanics \cite{savit1980duality, cobanera2011bond}.

They play a major role in statistical physics and condensed matter.
In statistical mechanics, dualities were introduced for the first time by Kramers and Wannier \cite{kramers1941statistics}, who found a relation between the high temperature and low temperature regimes of the two-dimensional Ising mode.
In this way, they were able to find the critical temperature years before Onsager solution \cite{onsager1944ising}.
In this case we speak of self-dualities, where the same model is mapped onto itself but in a different coupling regime.
The essential legacy of Kramers and Wannier is the fact that self-dualities can put constraints on the phase boundaries and the exact location of critical points.

Not all dualities are self-dualities.
In fact, it also possible to relate two apparently different physical models with a duality transformation.
A known example is the Jordan-Wigner transformation \cite{schultz1964ising, jordan1928pauli}, where spin \ac{dof} (which are bosonic in nature) are mapped onto fermionic \ac{dof} in one-dimension. %\todo{elaborare}.
This duality shows that, in fact, there is not much difference between bosonic and fermionic \ac{dof} in one dimension.

% \todo{forse aggiungere qualcosa in più}

%
% SUBSECTION: The bond-algebraic approach
%
\subsection{The bond-algebraic approach}
\label{sub:the_bond_algebraic_approach}

In the following section we will quickly review the bond-algebraic approach to dualities \cite{cobanera2011bond}, because it offers a powerful and convenient way for dealing with duality transformations, in particular when gauge symmetries are involved.
The concept of \emph{bond-algebra} was first introduced in \cite{nussinov2009bond} and it exploits the fact that most \emph{Hamiltonian are a sum of simple and (quasi)local terms}:
\begin{equation}
    H = \sum_{\Gamma} \lambda_{\Gamma} h_{\Gamma},
\end{equation}
where $\Gamma$ is a set of indices (e.g.~the lattice sites but can be completely general) and $\lambda_{\Gamma}$ are numbers (usually the couplings).
Roughly speaking, by quasi-local we mean that a operator $h_{\Gamma}$ involves a small number of \ac{dof} that are spatially near each other (for example nearest neighbour).
The terms $h_{\Gamma}$ are called \emph{bond operators} (or simply \emph{bonds}).
From the bonds $h_{\Gamma}$ we obtain a \emph{bond algebra} $\algebra\{h_{\Gamma}\}$, which is the algebra of all the operators generated by all the possible products and sums of the bonds $h_{\Gamma}$ and their Hermitian conjugates.
In practical terms, given a set of bonds $\{h_{\Gamma}\}$, the bond-algebra $\algebra\{h_{\Gamma}\}$ is the algebra spanned by
\begin{equation*}
    \{
        \identity, h_{\Gamma}, \,
        h_{\Gamma}^{\dagger}, \,
        h_{\Gamma} h_{\Gamma^{\prime}}, \,
        h_{\Gamma}^{\dagger}  h_{\Gamma^{\prime}}, \,
        h_{\Gamma} h_{\Gamma^{\prime}}^{\dagger}, \,
        h_{\Gamma}^{\dagger}  h_{\Gamma^{\prime}}^{\dagger}, \,
        h_{\Gamma} h_{\Gamma^{\prime}} h_{\Gamma^{\prime\prime}},
        \dots
    \}
\end{equation*}
By construction, $\algebra\{h_{\Gamma}\}$ is closed under the operation Hermitian conjugation, but since an Hamiltonian $H$ is Hermitian then $h_{\Gamma}^{\dagger} = h_{\Gamma^{\prime}}$ for some $\Gamma^{\prime}$.
Therefore, $\algebra \{h_{\Gamma}\}$ is simply spanned by
\begin{equation*}
    \{
        \identity,
        h_{\Gamma}, \,
        h_{\Gamma} h_{\Gamma^{\prime}}, \,
        h_{\Gamma} h_{\Gamma^{\prime}} h_{\Gamma^{\prime\prime}}, \,
        \dots
    \}
\end{equation*}
Notice that the bonds $h_{\Gamma}$ that generate $\algebra\{ h_{\Gamma} \}$ do not need to be independent.


It is important to point out that a single Hamiltonian $H$ can have different bond algebras associated to it.
In fact, a bond algebra is determined by the partitioning of the bonds in $H$.
In principle, given any two decomposition of the same Hamiltonian,
\begin{equation*}
    H
    = \sum_{\Gamma} \lambda_{\Gamma} h_{\Gamma}
    = \sum_{\Sigma} \lambda^\prime_{\Sigma} h^\prime_{\Sigma},
\end{equation*}
one should expect $\algebra\{h_{\Gamma}\} \neq \algebra\{ h^{\prime}_{\Sigma} \}$ in general (see \cite{cobanera2011bond}).
To make an example, consider the Hamiltonian
\begin{equation*}
    H = \sum_{i} \qty( h_x \sigma^x_i + h_y \sigma^z_i ).
\end{equation*}
We can either partition the bonds by taking $\sigma^x_i$ and $\sigma^z_i$ as generators separately or by taking $h_x \sigma^x_i + h_z \sigma^z_i$ as a single bond.
In the former case we would obtain $\algebra\{\sigma^x, \sigma^z\} $, while in the latter we would have $\algebra\{h_x \sigma^x_i + h_z \sigma^z_i\}$.
These two algebras are clearly different,
\begin{equation*}
    \algebra\{\sigma^x, \sigma^z\}
    \neq
    \algebra\{h_x \sigma^x_i + h_z \sigma^z_i\},
\end{equation*}
because $\algebra\{h_x \sigma^x_i + h_z \sigma^z_i\}$ is commutative, while $\algebra\{\sigma^x, \sigma^z\} $ is not.

In the framework of bond-algebra, quantum dualities can be formulated as \emph{homomorphisms of bonds-algebras}.
By homomorphism we intend a map $\Phi$ between two algebras $\algebra_1$ and $\algebra_2$ that preserves the linear and multiplicative structure of the algebras.
In mathematical terms, given any $u,v \in \algebra_1$ and any complex number $\lambda$ we have
\begin{equation*}
    \Phi(u + \lambda v) = \Phi(u) + \lambda \Phi(v)
    \qand
    \Phi(u v) = \Phi(u) \Phi(v).
\end{equation*}

To be more precise with our definition of quantum duality,
consider two Hamiltonians $H_1$ and $H_2$ that act on Hilbert spaces of the same dimensions.
They are said to be \emph{dual} if there is some bond-algebra $\algebra_{H_1}$ of $H_1$ that is homomorphic to some bond-algebra $\algebra_{H_2}$ of $H_2$ and if the homomorphism $\Phi : \algebra_{H_1} \to \algebra_{H_2}$ maps $H_1$ onto $H_2$, $\Phi(H_1) = H_2$.
These mappings do not need to be isomorphisms (i.e. invertible), especially when gauge symmetries are involved, and we will explain why later.

In a traditional approach to quantum dualities, one tries to map each degree of freedom of $H_1$ onto a degree of freedom of $H_2$.
This can be rather cumbersome, because in this way most duality transformations appear to be non-local.
In other words, one degree of freedom on one side may correspond to a large number of \ac{dof} on the other side.
This is apparent, for example, with the Jordan-Wigner transformation, where a single spin is dual to a whole chain of fermions.

Quantum dualities in the bond-algebraic approach are instead \emph{local}, meaning that each single \emph{bond} $h_{\Gamma_1}$ of $H_1$ is mapped onto a single bond $h_{\Gamma_2}$ of $H_2$.
This may translates in non-locality when treating elementary \ac{dof} and this is due to the fact that the generators of a bond algebra are usually two- (or more) body operators and expressing the elementary \ac{dof} with these operators may require large (if not infinite) products.

An isomorphism like $\Phi$ is physically sound if it is \emph{unitarily implementable} \cite{cobanera2011bond}, which means that there is a unitary matrix $\mathcal{U}$ such that the duality isomorphism reads
\begin{equation}
    \Phi(\mathcal{O}) =
    \mathcal{U} \mathcal{O} \mathcal{U}^{\dagger}, \quad
    \forall \mathcal{O} \in \algebra,
\end{equation}
where $\algebra$ is the operator algebra of the model under investigation \cite{cobanera2011bond}.
% \todo{elaborate}


To make the bond-algebraic approach more clear we will consider one example: the \emph{\acf{ising}}.
In this model we will see an example of self-duality through the use of disorder variables.
Our intent is not to shine new physics but to show how the use of bond-algebras offers a clear \emph{formalism} for treating dualities.

% \todo{parlare anche della Jordan-Wigner?}
% approach clearer we now apply it to the 1D quantum Ising model with transverse field.


\subsection{The quantum Ising model}
\label{sub:the_quantum_ising_model}

The \acl{ising} with a transverse field is a chain of spin-\onehalf described by the Hamiltonian
\begin{equation}
    \HamilIsing(h) = \sum_{i} \qty( \sigma^z_i \sigma^z_{i+1} + h \sigma^x_i ),
    \label{eq:hamiltonian_ising}
\end{equation}
where the sums runs over the sites of the chain and $h$ is the transverse field strength.
Notice that the Hamiltonian $\HamilIsing$ is indeed a sum of quasi-local terms.
In particular we have two types of terms: the interaction term $\sigma^z_i \sigma^z_{i+1}$ and the transverse field $\sigma^x_i$.
They are local or quasi-local because they involve at most two neighbouring sites.
These two sets of terms are the bonds of the Hamiltonian $\HamilIsing$, therefore bond-algebra $\algebra \{\sigma^z_i \sigma^z_{i+1}, \sigma^x_i\}$ is spanned by:
\begin{equation*}
    \{
        \identity, \,
        \sigma^z_i \sigma^z_{i+1}, \,
        \sigma^z_i \sigma^z_{i+1} \sigma^z_j \sigma^z_{j+1}, \,
        \dots, \,
        \sigma^x_i, \,
        \sigma^x_i \sigma^x_j, \,
        \dots, \,
        \sigma^z_i \sigma^z_{i+1} \sigma^x_i, \,
        \dots
    \}.
\end{equation*}
We consider an infinite chain in order to avoid subtleties with the boundaries conditions, which can have major effects on a duality transformation.

The algebraic relations that defines the generators of $\algebra^{\ising}$ can be summarized as follows:
\begin{enumerate}
    \item each bonds square to the identity operator,
        \begin{equation*}
            (\sigma^z_i \sigma^z_{i+1})^2 = (\sigma^x_i)^2 = \identity.
        \end{equation*}

    \item the bonds $\sigma_i^x$ anticommutes with $\sigma^z_i \sigma^z_{i+1}$ and $\sigma^z_{i-1} \sigma^z_i$ while commuting with the others,
        \begin{equation*}
            \acomm{\sigma_i^x}{\sigma^z_i \sigma^z_{i+1}} =
            \acomm{\sigma_i^x}{\sigma^z_{i-1} \sigma^z_i} = 0.
        \end{equation*}


    \item the bonds $\sigma^z_i \sigma^z_{i+1}$ anticommutes with $\sigma^x_i$ and $\sigma^x_{i+1}$ while commuting with the others,
        \begin{equation*}
            \acomm{\sigma^z_i \sigma^z_{i+1}}{\sigma^x_i} =
            \acomm{\sigma^z_i \sigma^z_{i+1}}{\sigma^x_{i+1}} = 0.
        \end{equation*}
\end{enumerate}

Given the symmetric roles that the basic bonds $\sigma^x_i$ and $\sigma^z_i \sigma^z_{i+1}$ play with each other, we can set up a mapping $\Phi^{\ising}$ that exchange their roles:
\begin{equation}
    \Phi^{\ising}(\sigma^z_i \sigma^z_{i+1}) = \sigma^x_i, \qquad
    \Phi^{\ising}(\sigma^x_i) = \sigma^z_{i-1} \sigma^z_{i}.
    \label{eq:duality_ising}
\end{equation}
This transformation can be extended to the whole $\algebra^{\ising}$ through the homomorphic property of $\Phi^{\ising}$.
It preserves all the important algebraic relationship and is one-to-one, hence it is an \emph{isomorphism} of $\algebra^{\ising}$ onto itself.
The Hamiltonian $\HamilIsing$ is just an element of $\algebra^{\ising}$.
We can apply $\Phi^{\ising}$ to $\HamilIsing$ and use its homomorphic property, which yields
\begin{equation}
    \begin{split}
        \Phi^{\ising}( \HamilIsing(h) )
        & = \sum_{i} \pqty{ \Phi^{\ising} (\sigma^z_i \sigma^z_{i+1}) + h \Phi^{\ising}(\sigma^x_i)} \\
        & = \sum_{i} \pqty{ \sigma^x_i + h \sigma^z_{i-1} \sigma^z_{i}} \\
        & = h \sum_{i} \pqty{ \sigma^z_i \sigma^z_{i+1} + h^{-1} \sigma^x_i}.
    \end{split}
\end{equation}
Notice that the indices in the sum can be freely shifted because we are working with an infinite number of sites.
We have thus obtained
\begin{equation}
    \Phi^{\ising} (\HamilIsing(h)) = h \HamilIsing(h^{-1}),
\end{equation}
henceforth $\Phi^{\ising}$ is a \emph{self-duality} of \eqref{eq:hamiltonian_ising}.
Notice that $\HamilIsing(h)$ is mapped onto itself but with the inverted coupling, $h \mapsto h^{-1}$, meaning that we can map the strongly coupled phase $h \gg 1$ into the weakly coupled phase $h \ll 1$, and vice versa.
This is basically the quantum version the \emph{Kramers-Wannier duality} \cite{kramers1941statistics, fradkin1978order}.

If we think of the term $\sigma^x_i$ as living on the site $i$ and of $\sigma^z_i \sigma^z_{i+1}$ as of living on the \emph{link} between the site $i$ and $i+1$, then we can think of $\Phi^{\ising}$ as mapping \eqref{eq:hamiltonian_ising} onto the \emph{dual lattice}.
In fact, the dual lattice of a chain is still a chain and the site term $\sigma^x_i$ is mapped onto a link term $\sigma^z_i \sigma^z_{i+1}$, and vice versa.


\begin{figure}[t]
    \SideFigure[label=fig:kramers_wannier, desc={Self-duality map of the \ac{ising}}]{
        \begin{tikzpicture}[scale=0.6]
    \draw[lattice] (-1,0) grid (9,0);
    \draw[lattice, xshift=-1cm] (0,-2) grid (10,-2);

    % spin operators
    \draw[X] (2, 0) -- (4, 0) node [midway, black, above] {$\sigma^z_{i} \sigma^z_{i+1}$};
    \foreach \x in {0,2,...,8} \draw (\x,0) node [site] {};
    \draw (6,0) node [Z site] {} node [black, above] {$\sigma^x_j$};

    % disorder operators
    \draw[X] (5, -2) -- (7, -2) node [midway, black, below] {$\sigma^z_{j} \sigma^z_{j+1}$};
    \foreach \x in {-1,1,...,9} \draw (\x,-2) node [site] {};
    \draw (3, -2) node [Z site] {} node [black, below] {$\sigma^x_i$};

    \draw[freccia] (3,0) -- (3,-2) node [pos=0.5, right] {$\Phi^\ising$};
    \draw[freccia] (6,0) -- (6,-2) node [pos=0.5, right] {$\Phi^\ising$};
\end{tikzpicture}

    }{
        Pictorial representation of the duality map $\Phi^{\ising}$, that maps the \ac{ising} on the same model on the dual lattice
    }
\end{figure}


We want to have a clearer physical picture of the duality map $\Phi^{\ising}$ and build a bridge with the traditional approach to dualities for the \ac{ising}.
For this reason we want to find the \emph{elementary \ac{dof} of the dual model}.
The \ac{dof} of the dual model lives on the sites of the dual lattice, which corresponds to the links of the original lattice.
On these dual sites we again have spin-\onehalf \ac{dof} and, for more clarity, we use $\mu^x$ and $\mu^z$ for referring to the Pauli matrices acting on these new spins.
The dual site $i$ corresponds to the link $(i, i+1)$, while the dual link $(i-1, i)$  corresponds to the site $i$.

From \eqref{eq:duality_ising}, we already know that
\begin{equation}
    \sigma^z_i \sigma^z_{i+1} = \mu^x_i
    \qand
    \sigma^x_i = \mu^z_{i-1} \mu^z_i.
\end{equation}
The role of $\mu^x_i$ is evident, it measure the alignment of two neighbouring spins on the sites $i$ and $i+1$, while the meaning of $\mu^z_i$ is still opaque.
We can arrive at the definition of $\mu^z_i$ by exploiting the map $\Phi^{\ising}$.
The bond $\mu^z_{i-1} \mu^z_i$ corresponds to the image of $\sigma^x_i$ through $\Phi^{\ising}$, so we know how they are mapped.
If we isolate $\mu^z_i$ with an \emph{infinite product}, we then obtain
\begin{equation}
    \mu^z_i
    = \prod_{j = -\infty}^{i} \mu^z_{j-1} \mu^z_j
    = \prod_{j = -\infty}^{i} \sigma^x_j.
    \label{eq:def_mu_z}
\end{equation}
We see that $\mu^z_i$ flips all the spins before the $i$-th site.
From \eqref{eq:def_mu_z}, we can see the \emph{non-local} origin of the dual \ac{dof} in traditional dualities.
When working with two, or more, body terms, in order to isolate a single body term the use of large (or even infinite) product is necessary.

To understand the role of $\mu^x_i$ and $\mu^z_i$, consider now the ferromagnetic ground states $\ket{\Omega_{\rho}}$ of \eqref{eq:hamiltonian_ising}, where $\rho=\uparrow, \downarrow$.
Say we start from $\ket{\Omega_{\uparrow}}$, without loss of generality.
The action of $\mu^z_i$ on $\ket{\Omega_{\rho}}$ is to create a \emph{kink}, which is a domain wall between two ordered regions.
From this point of view, a single spin-flip $\sigma^x_i \ket{\Omega_{\uparrow}}$ creates a \emph{kink-antikink pair}.

\begin{figure}[t]
    \SideFigure[label=fig:kink_states, desc={ferromagnetic ground states and kink states in the \ac{ising}}]{%
        \begin{tikzpicture}[scale=0.5]
    \draw[ladder] (-0.5, 0) -- (6.5, 0);
    \DrawSites{0,1,...,6}{0,0}
    \foreach \x in {0,1,...,6} \Spin{up}{\x}{0};
    \node[right] at (7, 0) { $\ket{\Omega_{\uparrow}}$};
    \node[left] at (-1, 0) {(a)};

    \begin{scope}[yshift=-2.5cm]
        \draw[ladder] (-0.5, 0) -- (6.5, 0);
        \DrawSites{0,1,...,6}{0,0}
        \foreach \x in {0,1,...,6} \Spin{down}{\x}{0};
        \node[right] at (7, 0) { $\ket{\Omega_{\downarrow}}$};
        \node[left] at (-1, 0) {(b)};
    \end{scope}

    \begin{scope}[yshift=-5cm]
        \draw[ladder] (-0.5, 0) -- (6.5, 0);
        \DrawSites{0,1,...,6}{0,0}
        \foreach \x in {0,1,2,3} \Spin{down}{\x}{0};
        \foreach \x in {4,5,6} \Spin{up}{\x}{0};
        \draw[very thick, dashed] (3.5, -0.9) -- +(0, 1.8);
        \node[right] at (7, 0) { $\mu^z_i \ket{\Omega_{\uparrow}}$ };
        \node[left] at (-1, 0) {(c)};
    \end{scope}

    \begin{scope}[yshift=-7.5cm]
        \draw[ladder] (-0.5, 0) -- (6.5, 0);
        \DrawSites{0,1,...,6}{0,0}
        \foreach \x in {0,1,2} \Spin{up}{\x}{0};
        \Spin{down}{3}{0}
        \foreach \x in {4,5,6} \Spin{up}{\x}{0};
        \draw[very thick, dashed] (3.5, -0.9) -- +(0, 1.8);
        \draw[very thick, dashed] (2.5, -0.9) -- +(0, 1.8);
        \node[right] at (7, 0) { $\sigma^x_i \ket{\Omega_{\uparrow}}$ };
        \node[left] at (-1, 0) {(d)};
    \end{scope}
\end{tikzpicture}

    }{%
        \emph{(a)} and \emph{(b)} ferromagnetic ground states $\ket{\Omega_{\uparrow}}$ and $\ket{\ket{\Omega_{\downarrow}}}$.
        \emph{(c)} kink created on the link between site $i$ and $i+1$ by the operator $\mu^z_i$.
        \emph{(d)} kink-antikink pairs created around the site $i$ by the spin flip $\sigma^x_i$.
    }
\end{figure}


%
% SUBSECTION: Gauge-reducing dualities
%
\subsection{Gauge-reducing dualities}%
\label{sub:gauge_reducing_dualities}

In this section we will review the notion of \emph{gauge-reducing dualities},
In order to do so we start by highlighting the difference between ordinary symmetries and quantum symmetries.

Following the statement of Wigner's theorem\citneeded, a quantum symmetry is a unitary or anti-unitary mapping that commute with the Hamiltonian.
This does not mean that all symmetries have the same physical meaning or mathematical consequences.
By the term ``\emph{ordinary symmetries}'' we refer to the most common types of symmetry that we encounter in physical systems that usually correspond to \emph{global transformation} of the physical apparatus or system, like for example rotational invariance.
These symmetries have a direct physical impact, since they can influence the level degeneracy of an Hamiltonian and force strict selection rules.

On the other hand, gauge symmetries are \emph{local symmetries} of the model that signal the presence of \emph{redundant \ac{dof}}.
In fact, it is better to think of gauge symmetries as \emph{local constraints} on the elementary \ac{dof} of the gauge model.
As a result, the state space of a gauge model is larger than set of physical states, which are exactly the states that are invariant under the action of the gauge symmetries.
The same reasoning applies to the observables of the gauge model.
An observables is represented by an Hermitian operator and a \emph{physical observable} is represented by an Hermitian operator that commutes with gauge symmetries.

So, if physical states and physical observables already satisfies the local constraints of the gauge symmetries, this means that the physical impact of the latter is already encoded in the former.
It is clear that the ordinary symmetries and gauge symmetries are very different and is better them conceptually far apart as possible \cite{cobanera2011bond}.

When dealing with a gauge model, it would be natural to assume that, in order to establish a duality with any gauge symmetries, these have to be eliminated from the former model first.
In other terms, that it would be necessary to project the operator content on the subspace of physical states first or, alternatively, proceed with gauge-fixing.
Although this is common in traditional approach to dualities, with bond algebras this is not strictly necessary.
As stated in \cite{cobanera2011bond}, with the bond-algebraic approach one can find mappings to models without any gauge symmetry that preserve all the important algebraic properties, without the need to projection or gauge-fixing \todo{argomentare?}.

The procedure goes as follows: consider a gauge model and let $H^{\text{G}}$ be its Hamiltonian and $G_{\Gamma}$ its gauge symmetries.
An operator $\mathcal{O}$ is gauge-invariant if and only if it commutes with all the $G_{\Gamma}$:\begin{equation*}
    \mathcal{O} \text{~physical}
    \quad \Longleftrightarrow \quad
    \comm{\mathcal{O}}{G_{\Gamma}} = 0 \quad \forall \Gamma.
\end{equation*}
Clearly, the Hamiltonian has to be gauge-invariant, hence $[H^{\text{GR}}, G_{\Gamma}] = 0$.
Now let $H^{\text{GR}}$ be the dual Hamiltonian of a non-gauge, or gauge-reduced, model.
Furthermore, let $\algebra^{\text{G}}$ and $\algebra^{\text{GR}}$ be the bond-algebra of the gauge and gauge-reduced models, respectively.
A \emph{gauge-reducing duality} is a map
\begin{equation*}
    \Phi^{\text{GR}} : \algebra^{\text{G}} \to \algebra^{\text{GR}}
\end{equation*}
such that $H^{\text{G}}$ is mapped onto $H^{\text{GR}}$ while making all the gauge symmetries of the gauge model trivial:
\begin{equation}
    \Phi^{\text{GR}}(H^{\text{G}}) = H^{\text{GR}}
    \qand
    \Phi^{\text{GR}}(G_{\Gamma}) = \identity \quad \forall \Gamma.
\end{equation}

Unlike the dualities in Sec.~\ref{sub:the_bond_algebraic_approach}, a gauge-reducing duality like $\Phi_{\text{GR}}$ has to be implementable as a \emph{projective unitary operator} $\mathcal{U}$.
Formally, this can be written as
\begin{equation}
    \Phi_{\text{GR}} ( \mathcal{O} ) = \mathcal{U} \mathcal{O} \mathcal{U}^\dagger, \quad
    \mathcal{U} \mathcal{U}^\dagger = \identity, \quad
    \mathcal{U}^\dagger \mathcal{U} = P_{\text{GI}}
\end{equation}
where $P_{\text{GI}}$ is the projector of the subspace of gauge-invariant states, i.e.~$G_{\Gamma} \ket{\psi} = \ket{\psi}$ for all $\Gamma$.
Roughly speaking, this projective unitary operator can be represented as rectangular matrix that preserves the norm of gauge-invariant states while projecting out all the other states.

In the next section we will use an example of gauge-reducing duality, which will be instrumental for the rest of the chapter.


\subsection{Dualities in two dimensions}
\label{sub:dualities_in_two_dimensions}


As an example of gauge-reducing duality, we will apply the technology introduced in Sec.~\ref{sub:gauge_reducing_dualities} to the $\Z_2$ \ac{lgt} in two-dimensions.
We resume the Hamiltonian \eqref{eq:z2_lgt_hamiltonian}
\begin{equation*}
    H^{\Z_2}
    = - \sum_{p} B_p - \lambda \sum_{\link} Z_{\link}
    = - \sum_{p} B_p - \lambda \sum_{x} \qty(Z_{(x, +\hat{1})} + Z_{(x, +\hat{2})}),
\end{equation*}
and its Gauss (or vertex) operators
\begin{equation}
    A_v = \prod_{\link \in v}  Z_{\link},
\end{equation}
which generate the gauge symmetries and commute with the Hamiltonian
\begin{equation}
    \comm{H^{\Z_2}}{A_v} = 0 \quad \forall v \in \Lattice.
\end{equation}

In particular, each term of the Hamiltonian commutes with the Gauss operators, which means that the bond algebra they generate is gauge-invariant,
This bond-algebra satisfy three simple relations:
\begin{enumerate}[label=(\roman*)]
    \item all the bonds square to the identity,
    \item each spin $Z$ anti-commutes with two adjacent plaquettes operators $U$
    \item each plaquette operator $U$ anti-commutes with four spins $Z$.
\end{enumerate}

The model $H^{\Z_2}$ is dual to the $d=2$ \ac{ising}.
The Hamiltonian of the latter in two-dimensions is
\begin{equation}
    \HamilIsing =
    - \sum_{i} \qty(
        \sigma^z_i \sigma^z_{i+\hat{1}} +
        \sigma^z_i \sigma^z_{i+\hat{2}} +
        h \sigma^x_i
    ),
\end{equation}
where the index $i$ runs over the sites.
One recognizes as separate bonds the terms $\sigma^z_i \sigma^z_{i+\hat{1}}$, $\sigma^z_i \sigma^z_{i+\hat{2}}$, and $\sigma^x_i$.
It is immediate to see that these bonds satisfy the same relations of the bonds of $H^{\Z_2}$.

The dual model of $H^{\Z_2}$ lives on the dual lattice.
Therefore we identify a plaquette $p$ in the gauge model with a site $i$ of the \ac{ising}, while $x$ will refer to the lower left site of the plaquette $p$.
With this notation, we can now build the duality mapping $\Phi^{\text{2d}}$ as follows:
\begin{equation}
    \Phi^{\text{2d}}(Z_{(x, \hat{1})} )  = \sigma^z_{(i-\hat{2})} \sigma^z_i, \quad
    \Phi^{\text{2d}}(Z_{(x, \hat{2})} )  = \sigma^z_{(i-\hat{1})} \sigma^z_i, \quad
    \Phi^{\text{2d}}(U_p) = \sigma^x_i.
    \label{eq:duality_2d}
\end{equation}
Applying to $\Phi^{\text{2d}}$ to $H^{\Z_2}$ we obtain
\begin{equation}
    \Phi^{2d} (H^{\Z_2}) =
    - \sum_{i} \sigma^x_i - \lambda \sum_{i} \qty(
        \sigma^z_{(i-\hat{2})} \sigma^z_i +
        \sigma^z_{(i-\hat{1})} \sigma^z_i
    ) =
    \lambda \HamilIsing(\lambda^{-1})
\end{equation}
Thus, $\Phi^{\text{2d}}$ maps $H^{\Z_2}$ to $\HamilIsing$, up to a multiplicative constant, if we identify the constants $\lambda \leftrightarrow h^{-1}$.

From \eqref{eq:duality_2d}, one readily obtains
\begin{equation*}
    \Phi^{\text{2d}}(G_x) = \identity,
\end{equation*}
which means that $\Phi^{\text{2d}}$ is in fact a \emph{gauge-reducing duality}.
Therefore, $\HamilIsing$ represents all the physics contained in $H_{\text{gauge}}$, but without all the redundant \ac{dof}.
In the general case, i.e.~for a generic $\Z_N$ symmetry, the duality leads to an $N$-clock model \cite{radicevic2019spin}.

The reason why it is possible to encode the physical content of the gauge model in a simpler \ac{ising} is the following.
The physical states of a pure gauge model is made of closed electric loops and each electric loop can be thought as containing magnetic flux.
So, each physical state can be fully described by indicating which plaquettes contains magnetic flux and which do not.
The electric lines naturally arises as domain walls between plaquettes with different flux.

Basically, the duality mapping $\Phi^{\text{2d}}$ assigns to each plaquette a spin-\onehalf{} \ac{dof}, indicating the flux state.
Everything else readily follows.
The plaquette operator $U_p$ flips the state of the plaquette, therefore it should be mapped to an operator that flips the spin in $p$, thus $\sigma^x$.
The electric fields $V_{(x, \hat{1})}$ and $V_{(x, \hat{2})}$ are just domain walls between plaquettes, therefore they should be mapped to interaction terms like $\sigma^z_{i-\hat{1}} \sigma^z_{i}$ and $\sigma^z_{i-\hat{2}} \sigma^z_i$.


\todo{Parlare del caso ZN}

% vim: spelllang=en


% SECTION: Dualities of ladder LGTs
\section{Dualities of the ladder}%
\label{sec:dualities_of_the_ladder}

%--------------------------------------------------
% SECTION: Clock models
%--------------------------------------------------
\subsection{Clock models}%
\label{sub:clock_models}

In this section we will deal with a class of generalizations of the quantum Ising model known as \emph{clock models} \cite{fendley2014parafermions, baxter1989clock}, which shows a resemblance to the $\Z_N$ LGT models we introduced previously.
This similarity will later be exploited in order to obtain a complete description of the LGT models without any redundant gauge-symmetry.

For a discussion about clock models we start from the Hamiltonian of the quantum Ising model with a transverse field, which can simply be written as
\begin{equation}
    H = - \sum_{i} \sigma^z_i \sigma^z_{i+1} - h \sum_{i} \sigma^x_i,
    \label{eq:ising_hamiltonian_duality}
\end{equation}
where $\sigma^{x,z}_i$ are the usual $2 \times 2$ Pauli matrices for each site $i$:
\begin{equation}
    \sigma^x_i = \pmqty{ 0 & 1 \\ 1 &  0 }, \quad
    \sigma^z_i = \pmqty{ 1 & 0 \\ 0 & -1 }.
\end{equation}
They are a set of unitary matrices that commute on different sites, while on the same site they anticommute $\sigma^x \sigma^z = - \sigma^z \sigma^x$.
Another way to put it is to say that the exchange of $\sigma_x$ and $\sigma_z$ on the same site produces a phase $e^{i \pi} = -1$.

Clock models can be thought as generalizations of the quantum Ising model, but not to higher spins.
A $p$-state clock model (or simply a $p$-clock model) utilizes a set of unitary operators that generalize the algebra of Pauli matrices in the following sense:
the operators $\sigma_x$ and $\sigma_z$ get promoted to the \emph{clock operators} $X$ and $Z$, respectively, which are $p \times p$ unitary matrices whose exchange produces a phase $\omega = e^{i 2 \pi / p}$, instead of $-1$.
The algebraic properties of these clock operators $X$ and $Z$ can be summarized as follows:
\begin{equation}
    \begin{aligned}
        X Z & = \omega Z X, &
        X^p & =  Z^p = \One_p, \\
        X^\dagger & = X^{-1} = X^{p-1}, &
        Z^\dagger & = Z^{-1} = Z^{p-1}
    \end{aligned}
    \label{eq:clock_operator_algebra}
\end{equation}

We see that the Schwinger-Weyl algebra in \eqref{eq:schwinger_weyl_algebra} and the clock operator algebra in \eqref{eq:clock_operator_algebra} are basically the same, but there are some key differences to point out betweens a $\Z_N$ LGT and a $p$-clock model.

The degrees of freedom of a $\Z_N$ LGT live on the links of the lattice while in a $p$-clock model they live on the sites.
But the most important aspect is that we don't have any gauge symmetry in a $p$-clock model, hence we do not have to impose any local constraints or physical conditions.
These models can be derived as the quantum Hamiltonians of the classical 2D vector Potts model, which is a discretization of the 2D planar XY model \cite{ortiz2012dualities}.

A typical $p$-clock model Hamiltonian with transverse field has the form
\begin{equation}
    \Hclock(\lambda) = - \sum_{i} Z_i Z_{i+1} - \lambda \sum_{i} X_i + \hc
    \label{eq:clock_hamiltonian}
\end{equation}
which is, as expected, very similar to the quantum Ising Hamiltonian in \eqref{eq:ising_hamiltonian_duality}.
Furthermore, just like the latter, $p$-clock models with only transverse field are \emph{self-dual}:
the clocks can be mapped into the kinks (or domain walls) and one would obtain the same exact Hamiltonian description but with inverted transverse field \cite{ortiz2012dualities}.
For $p < 5$, the clock models presents a self dual point in $\lambda = 1$, that separates an ordered phase from a disordered one.
On the other hand, for $p \geq 5$ we have an intermediate continuous critical phase between the ordered and disordered phase with two BKT transition points, which are related to each other through the self-duality \cite{sun2019phase}.

These models have been thoroughly studied, even with the addition of a longitudinal field $\propto Z_i$ \cite{baxter1982exactlysm} or chiral interactions.
In particular, in the case of chiral interactions, it was shown \cite{fendley2012parafermions} that the Hamiltonian \eqref{eq:clock_hamiltonian} can be mapped to a parafermionic chain through a Fradkin-Kadanoff transformation, and in presence of a $\mathbb{Z}_3$ symmetry, it shows three different phases \cite{zhuang2015clock}, if open boundaries are implemented: a trivial, a topological and an incommensurate (IC) phase.
The case which presents a real longitudinal field term was considered in \cite{huang2019clock},  where some of the critical exponents have been estimated.
The general case, where chiral interactions are included in a $\mathbb{Z}_N$ model, has been studied in \cite{fendley2012parafermions}.
Here, the author considered the model as an extension of the Ising/Majorana chain and found the edge modes of the theory.
He also calculated the points, in the parameter space, where the model is integrable or `superintegrable'.
All these studies are motivated by theoretical interest and recent experiments, which can be analysed by the above models \cite{bernien2017probing}.


% vim: spelllang=en



\begin{figure}[t]
    \centering
    \begin{tikzpicture}

    % ladder
    \draw[ladder] (-0.5,0) grid (7.5,1);

    % Plaquette operator
    \draw[U] (1,0) -- (2,0);
    \draw[U] (2,0) -- (2,1);
    \draw[U] (2,1) -- (1,1);
    \draw[U] (1,1) -- (1,0);
    \draw[box] (0.75,-0.25) rectangle (2.25, 1.25) node [above left, Blue] {$U_x$};

    % horizontal electric operators
    \draw[V] (4,0) -- (5,0);
    \draw[V] (4,1) -- (3,1);
    \draw[box] (3.75,-0.25) rectangle (5.25, 0.25) +(-0.25,0) node [above left, Red] {$\Vdown_x$};;
    \draw[box] (2.75, 0.75) rectangle (4.25, 1.25) node [above left, Red] {$\Vup_x$};

    % vertical electric operator
    \draw[V] (6,0) -- (6,1);
    \draw[box] (5.75,-0.25) rectangle (6.25, 1.25) node [above, Red] {$V^0_x$};

    % ladder sites
    \DrawSites{0,...,7}{0,1};

    % chain
    \draw[ladder] (-0,-2) -- (7,-2);

    % clock operators
    \draw (1.5,-2) node [X site] {} node [text=Blue, below=5pt] {$X_i$};
    \draw (3.5,-2) node [Z site] {} node [text=Red, below=5pt] {$Z^\dagger_i$};
    \draw (4.5,-2) node [Z site] {} node [text=Red, below=5pt] {$\omega^k Z_i$};
    \draw [Z] (5.5,-2) -- (6.5,-2) node [pos=0.5, below=5pt] {$Z^\dagger_i Z_{i-1}$};

    % chain sites
    \foreach \x in {0,2,5,6} \draw (\x+0.5,-2) node [dual site] {};

    % arrows
    \draw[freccia] (1.5,-0.25) -- (1.5, -2);
    \draw[freccia] (3.5, 0.75) -- (3.5, -2);
    \draw[freccia] (4.5,-0.25) -- (4.5, -2);
    \draw[freccia] (6.0,-0.25) -- (6.0, -2);

    % labels
    \draw (-0.75,0.5) node [left, align=right] {$\Z_N$ ladder \\ LGT};
    \draw (-0.75, -2) node [left, align=right] {$N$-clock \\ chain};
\end{tikzpicture}

    \caption[Duality map of $\Z_N$ ladder \ac{lgt}]{Visual representation of the duality transformation from the $\Z_N$ ladder \ac{lgt} to the $N$-clock model.
    The plaquette operator $U_x$ and the electric operators $\Vup$ and $\Vdown$ map to one-site operators in the clock model, while
    the remaining electric operator $V^0$ maps to a hopping term between nearest neighbouring sites.}
    \label{fig:ladder_duality}
\end{figure}

%--------------------------------------------------
% SUBSECTION: Duality on the clock models
%--------------------------------------------------
\subsection{Duality onto clock models}
\label{sub:duality_onto_clock_models}

In this section we will show how to construct a mapping of the $\Z_N$ ladder \ac{lgt} onto a $N$-clock model on a chain with a transversal field and a longitudinal field, the latter depending on the topological sector of the ladder \ac{lgt}.

The first step is the decomposition of the set of bonds in \eqref{eq:hamiltonian_base}.
Obviously, the magnetic terms $U_{\square}$ have to be separated from the electric terms $V_\ell$, but the latter cannot be all treated the same.
It is clear from the geometry of the ladder, that the links $\runglink$ have a different role when compared with the links $\toplink$ and $\botlink$, because the former are \emph{domain walls} while the latter are not.
Therefore, the duality transformation has to distinguish between the vertical links and horizontal links.
Furthermore, also the top links $\toplink$ and bottom links $\botlink$ have to be treated separately because the electric operator on them have different commutation relations with the plaquette operators.
In fact, using the notation introduced in Sec.~\ref{sec:abelian_models_on_the_ladder}, we have
\begin{equation}
    U_x \Vdown_x = \omega \Vdown_x U_x, \qquad
    U_x \Vup_x = \omega^{-1} \Vup_x U_x.
    \label{eq:comm_rel_ladder}
\end{equation}
and indeed they acquire different phases.

The plan is to associate to each plaquette a clock degree of freedom, hence we identify a plaquette $\square_x$ with a site $i$ of a clock chain and the magnetic flux of a plaquette becomes the ``fundamental gauge invariant degree of freedom'' of the \ac{lgt} ladder model.
Given the fact that we are working in the electric basis, we chose for convenience to map the $\Z_N$ magnetic operator $U_x$ to the ``momentum'' operator $X_i$ of the $N$-clock chain.
The electric field on a vertical link $\runglink$ is the result of the flux difference between the two plaquettes that it separates, which suggests that the operator $V_{\runglink}$ have to be mapped to a kinetic-type term like $Z_i^\dagger Z_{i-1}$.
This can be readily verified.
From \eqref{eq:plaq_op_ladder} we get
\begin{equation*}
    V^0_x U_x = \omega^{-1} U_x V_x^0, \qquad
    V^0_x U_{x-1} = \omega U_{x-1} V_x^0,
\end{equation*}
therefore the maps
\begin{equation*}
    U_x \mapsto X_i, \qquad
    V^0_x \mapsto Z_i^\dagger Z_{i-1},
    \label{eq:elec_h_and_plaq_op_map}
\end{equation*}
clearly conserves the commutation relations of $U_x$ and $V^0_x$.

For now we are left with task of finding a suitable mapping of $\Vup$ and $\Vdown$.
With respect to the other bonds of the theory, both of them commute with $V^0$ while for \eqref{eq:comm_rel_ladder} holds for $U_x$.
Hence, a suitable and general mapping of $\Vup$ and $\Vdown$ can be:
\begin{equation}
    \Vdown_x \mapsto \coeffdown_i Z_i, \qquad
    \Vup_x \mapsto \coeffup_i Z_i^\dagger,
    \label{eq:elec_op_horiz_ladder_map}
\end{equation}
where $\coeffdown_i$ and $\coeffup_i$ are complex numbers.
Although, they cannot be any complex number.
Both $\Vdown_x$ and $\Vup_x$ have to be mapped onto unitary operators, which limits the numbers $\coeffdown_i$ and $\coeffup_i$ to be \emph{complex phases}.

To further constraint the value of these coefficients, we can use the Gauss law.
In particular, given the fact that we are looking for a gauge-reducing duality, the aim is to make the Gauss law trivial.
Using the mappings \eqref{eq:elec_h_and_plaq_op_map} and \eqref{eq:elec_op_horiz_ladder_map} in \eqref{eq:gauss_law_ladder} yields
\begin{equation}
    \begin{split}
        G^\uparrow_x & \mapsto
        (\coeffup_i Z_i^\dagger) (\coeffup_{i-1} Z_{i-1}^\dagger) (Z_i^\dagger Z_{i-1})^\dagger
        = \coeffup_i (\coeffup_{i-1})^*, \\
        G^\downarrow_x & \mapsto
        (\coeffdown_i Z_i^\dagger) (Z_i^\dagger Z_{i-1}) (\coeffdown_{i-1} Z_{i-1}^\dagger)
        = \coeffdown_i (\coeffdown_{i-1})^*
    \end{split}
    \label{eq:gauss_law_map_ladder}
\end{equation}

Gauss law have to be satisfied in a pure gauge theory, which mean that we have to impose $G^\uparrow_x = \identity$ and $G^\downarrow_x = \identity$ for all $x$.
This is only possible if
\begin{equation}
    \coeffdown_i = \coeffdown, \qquad
    \coeffup_i = \coeffup, \qquad
    \forall i.
\end{equation}

Furthermore, thanks to \eqref{eq:gauss_law_map_ladder}  we also know how to introduce static matter into this duality, because it can be thought as a violation of the Gauss law.
We just have to change the phases $\coeffup_i$ and $\coeffdown_i$.

The last factor to consider is how the $\coeffup$ and $\coeffdown$ are related on the same site $i$.
In this regard, the topological sectors of the theory come to the rescue.
As established in Sec.~\ref{sec:abelian_models_on_the_ladder}, the topological sectors are identified by the eigenvalue of $S_2$ in \eqref{eq:nonlocal_op_ZN}, which in the ladder geometry becomes
\begin{equation}
    S_2 = \Vup_x \Vdown_x
    \label{eq:top_string_op_ladder}
\end{equation}
for any fixed $x$.
Its eigenvalue are simply $\omega^k$, for $k = 0, \dots, N-1$.

Given a topological sector $\omega^k$, using the mapping \eqref{eq:elec_op_horiz_ladder_map} on \eqref{eq:top_string_op_ladder} yields
\begin{equation}
    S_2 \; \longmapsto \; ( \coeffup Z^\dagger_i ) ( \coeffdown Z_i ) = \coeffup \coeffdown = \omega^k.
\end{equation}
From here, in order to solve for the coefficients $\coeffup$ and $\coeffdown$, one needs only to fix one of the to $1$ and the other to $\omega^k$.
We choose to fix these coefficients as follows:
\begin{equation}
    \coeffup = 1, \qquad
    \coeffdown = \omega^k.
\end{equation}

In conclusion, we summarize the duality mapping for the topological sector $\omega^k$ of the $\Z_N$ \ac{lgt} on a ladder:
\begin{equation}
    \begin{aligned}
        U_x      & \; \longmapsto \; X_i, \quad &
        V^0_x    & \; \longmapsto \; Z^\dagger_i Z_{i-1}, \\
        \Vup_x   & \; \longmapsto \; Z_i^\dagger, \quad &
        \Vdown_x & \; \longmapsto \; \omega^k Z_i.
    \end{aligned}
    \label{eq:ladder_duality}
\end{equation}

With the duality \eqref{eq:ladder_duality}, from \eqref{eq:ladder_hamiltonian} in the sector $(\omega^k, 1)$ we obtain
\begin{equation}
    H_{\text{ladder}}(\lambda) \; \longmapsto \; \lambda H_{\text{clock}}(\lambda^{-1})
\end{equation}
where
\begin{equation}
    H_{\text{clock}}(\lambda^{-1}) =
    - \sum_{i} Z_i^\dagger Z_{i-1}
    - \lambda^{-1} \sum_{i} X_i
    - (1 + \omega^k) \sum_{i} Z_i
    + \text{h.c.}
    \label{eq:dual_ladder_hamiltonian}
\end{equation}

We see that \eqref{eq:dual_ladder_hamiltonian} is a clock model with both \emph{transversal} and \emph{longitudinal} fields.
In particular, the longitudinal field carries the information of the topological sector of the ladder model.

Interestingly, for $N$ even the sector $k = N/2$ has a special role.
Within this sector $\omega^k = -1$, for which the \emph{longitudinal field disappears} and $H_{\text{clock}}$ reduces to self-dual quantum clock models with a known quantum phase transition.
This phase transitions for $k = N/2$ can be put in correspondence with a \emph{confined-deconfined} transition, which will be discussed in much more detail in the next section.

Let us remark that the complex coupling $(1 + \omega^n)$ does not make the Hamiltonian  (\ref{eq:dual_ladder_hamiltonian}) necessarily chiral \cite{fendley2012parafermions, whitsitt2018clock}.
In fact, one can get the real Hamiltonian
\begin{equation}
    H_N = H_p(1/\lambda) - 2 \cos \pqty{\frac{\pi n}{N}} \sum_{i} \pqty{Z_i + Z_i^{\dagger}}.
    \label{eq:dual_ladder_hamiltonian_real}
\end{equation}
by absorbing the complex phase in the $Z_i$-operators, with the transformation $Z_i \mapsto w^{-n/2} Z_i$. This transformation globally rotates the eigenvalues of the $Z_i$-operators, while preserving the algebra relations.
For $n$ even, this is just a permutation of the eigenvalues, meaning that it does not affect the Hamiltonian spectrum. Instead, for $n$ odd, up to a reorder, the eigenvalues are shifted by an angle $\pi/N$, i.e.~half the phase of $\omega$.
The energy contribution of the extra term in \eqref{eq:dual_ladder_hamiltonian_real}  depends on the real part of these eigenvalues and for $n$ odd we obtain that the lowest energy state is no longer unique, in fact it is doubly degenerate.
This means that for $\lambda \to \infty$, where the extra term becomes dominant, we expect an ordered phase with a doubly degenerate ground state.
Finally, one can easily prove that the sectors $n$ and $N-n$ are equivalent
\footnote{For the sector $N-n$ we have that the overall factor $\cos(\pi(N-n)/N)$ is just $-\cos(\pi n/N)$.
The minus sign can then be again absorbed into the $Z$'s operators.
This overall operation is equivalent to the mapping $Z \mapsto \omega^{-n/2} Z$ for the sector $N-n$.}.


% SECTION: A case study: Z2, Z3 and Z4
\section{A case study: \texorpdfstring{$N=2, 3$}{N=2, 3} and \texorpdfstring{$4$}{4}}
\label{sec:a_case_study_N_2_3_4}

\newcommand{\ZTwoWilsonGraph}[1]{
        \nextgroupplot
        \addplot+[thick] table [y=10x2] {assets/graphs/data/Z2_wilson_#1.csv};
        \addplot+[thick] table [y=12x2] {assets/graphs/data/Z2_wilson_#1.csv};
        \addplot+[thick] table [y=14x2] {assets/graphs/data/Z2_wilson_#1.csv};
        \addplot+[thick] table [y=16x2] {assets/graphs/data/Z2_wilson_#1.csv};
        \addplot+[thick] table [y=18x2] {assets/graphs/data/Z2_wilson_#1.csv};
}
\newcommand{\ZThreeWilsonGraph}[1]{
        \nextgroupplot
        \addplot+[thick] table [y=7x2]  {assets/graphs/data/Z3_wilson_#1.csv};
        \addplot+[thick] table [y=9x2]  {assets/graphs/data/Z3_wilson_#1.csv};
        \addplot+[thick] table [y=11x2] {assets/graphs/data/Z3_wilson_#1.csv};
        \addplot+[thick] table [y=13x2] {assets/graphs/data/Z3_wilson_#1.csv};
}
\newcommand{\ZFourWilsonGraph}[1]{
    \nextgroupplot
    \addplot+ [thick] table [y=6x2]  {assets/graphs/data/Z4_wilson_#1.csv};
    \addplot+ [thick] table [y=8x2]  {assets/graphs/data/Z4_wilson_#1.csv};
    \addplot+ [thick] table [y=10x2] {assets/graphs/data/Z4_wilson_#1.csv};
}



\subsection{Investigating the phase diagram}%
\label{sub:investigating_the_phase_diagram}

We wish to study the phase diagram of the $\Z_N$ \ac{lgt} phase diagram, in particular we are interested in any \emph{confined} or \emph{deconfined} phase.
In a pure gauge theory, these phases are investigated non-local order parameters like the \emph{\ac{wl}} (not be confused with the non-contractible \ac{wl}s in \eqref{eq:nonlocal_op_ZN}) or \emph{string tension}.
This is because we expect the deconfined phase to be a topological phase, which can be investigated only with non-local order parameters.

Given a closed region $\mathcal{R}$, a \ac{wl} operator $W_{\mathcal{R}}$ is defined as
\begin{equation}
    W_{\mathcal{R}} = \prod_{\square \in \mathcal{R}} U_{\square}.
    \label{eq:closed_wilson_loop}
\end{equation}
Alternatively, considering the oriented boundary $\partial \mathcal{R}$ one can write
\begin{equation}
    W_{\mathcal{R}} = \prod_{\ell \in \partial \mathcal{R}} U_{\ell},
\end{equation}
where the Hermitian conjugate is implied everytime we move in the negative directions.
It is also implied that the curve $\partial \mathcal{R}$ is a contractible loop.
Wilson showed in \cite{wilson1974confinement} that quark confinement is related to the expectation value $\ev{W_{\mathcal{R}}}$ of a \ac{wl}, which can be thought as a gauge field average on a region.
In particular, in the presence of quark confinement the gauge field average follows an \emph{area law}, where it decays exponentially with the area enclosed by $\mathcal{R}$.
On the other hand, in the deconfined phase we have a \emph{perimeter law}, where the gauge field average decays exponentially with the perimeter of $\mathcal{R}$.

Unfortunately on a ladder geometry there is not much difference between the area and the perimeter of a \ac{wl}.
In fact, in units of the lattice spacing, the area of a \ac{wl} over $n$ plaquettes is $n$ while its perimeter is just $2n+2$.
They both grow linearly.
Nonetheless, we can still look at the behaviour of the \ac{wl}, for a fixed length, at different couplings $\lambda$.

When the coupling $\lambda$ in \eqref{eq:hamiltonian_base} is equal to zero, the \ac{tc} is recovered and in any of its topological sector the ground state is the equal superposition of all the states with any number of closed electrical loops, in a similar fashion to coherent states.
This makes the \ac{tc} a \emph{quantum loop gas}, which is a \emph{deconfined phase}.
Furthermore, the operator $W_{\mathcal{R}}$ in \eqref{eq:closed_wilson_loop} creates an electrical loop around the region $\mathcal{R}$.
From the constraints
% TODO aggiustare ref
% \eqref{eq:constraints_gs_toric_code},
it can easily be proved that $W_{\mathcal{R}}$ leaves the \ac{tc} ground states unchanged, showing in fact that they behaves as coherent states, which leads to $\ev{W_{\mathcal{R}}} = 1$.

Therefore, $\ev{W_{\mathcal{R}}} \approx 1$ signals a deconfined phase and on the other hand a vanishing $\ev{W_{\mathcal{R}}} \approx 0$ corresponds to confined phase.
For this reason, even tough we lack an area/perimeter law on the ladder geometry it is still sensible to look at the behaviour of the \ac{wl}.


Another possible approach for investigating the phase diagram is to use the \emph{string tension}.
In two dimensions, given an \emph{open} curve $\tilde{\mathcal{C}}$ on the dual lattice $\tilde{\mathbb{L}}$ we can construct an open \ac{ths} operator $S_{\tilde{\mathcal{C}}}$ as
\begin{equation}
    S_{\tilde{\mathcal{C}}} = \prod_{\ell \in \tilde{\mathcal{C}}} V_{\ell}
\end{equation}
with the usual caveat: we have to take the Hermitian conjugate everytime the path goes in the negative direction.
Then the string tension is just the expectation value $\ev{S_{\tilde{\mathcal{C}}}}$ it is called in this way because it related to the potential energy (tension) between two magnetic fluxes created at the ends of the curve $\tilde{\mathcal{C}}$.
Henceforth, in a deconfined phase $\ev{S_{\tilde{\mathcal{C}}}} \approx 0$, which means that the magnetic fluxes can be moved freely with no cost in energy, like in the \ac{tc}.




% \section{Numerical analysis}%
% \label{sec:numerical_analysis}
%
% In this work we studied numerically the different topological sectors (and their phase diagrams) of the $\Z_N$ LGT on ladder for $N=2,3,4$ through \emph{exact diagonalization} (ED).
% We chose ED instead of other variational methods like DMRG because we were able to construct exactly the Hilbert space of the different topological sectors of the models, exploiting the duality in Sec.~\ref{sec:dualities_of_the_ladder}.
%

\subsection{Implementing the Gauss law}%
\label{sub:implementing_the_gauss_law}

\begin{figure}[t]
    \centering
    \begin{tikzpicture}[
    font=\small,
    scale=0.75,
    % site/.style = {circle, inner sep=0 pt, minimum size=4pt, draw=black, fill=white},
    % up/.style = {ultra thick, green!70!black},
    legend/.style = {text=black, inner sep=5pt}
]

%%% Sector n=0 vacuum
\begin{scope}[local bounding box=trivial]
    \node at (3,1) [above=5pt, legend]  {vacuum $\ket{\Omega_0}$ of the sector $n=0$};
    % ladder
    \draw[ladder] (-0.5,0) grid (6.5,1);
    % sites
    \DrawSites{0,1,...,6}{0,1}
    \useasboundingbox (-1,0) -- (7,0) -- +(0,-0.5);
\end{scope}

%%% Sector n=1 vacuum
\begin{scope}[yshift=-3.5cm, local bounding box=topol]
    \node at (3,1) [above=5pt, legend]  {vacuum $\ket{\Omega_1}$ of the sector $n=1$};
    % ladder
    \draw[ladder] (-0.5,0) grid (6.5,1);
    % Wilson loop
    \draw[up] (-0.5, 0) -- (6.5, 0);
    % sites
    \DrawSites{0,1,...,6}{0,1}
    \useasboundingbox (-1,0) -- (7,0) -- +(0,-0.5);
\end{scope}

% legend
\begin{scope}[xshift=8cm, yshift=1cm, local bounding box=legend]
    \draw [Gray, thin] (0,0.75) -- +(0.5,0) node [right, legend] {$\ket{0}$};
    \draw [up] (0,0)    -- +(0.5,0) node [right, legend] {$\ket{1}$};
    \useasboundingbox (-0.25,0);
\end{scope}


\draw[thin, Gray] (trivial.south west) rectangle (trivial.north east);
\draw[thin, Gray] (topol.south west) rectangle (topol.north east);
\draw [shorten >= 3pt] (trivial.east)
    edge [-{Latex}, Gray, very thick, out=0, in=0]
    node [font=\normalsize, right, text=black] {$\Wilson_1$}
    (topol.east);
\end{tikzpicture}

    \caption{The different ``Fock vacua'' $\ket{\Omega_{(0,0)}}$ and $\ket{\Omega_{(1,0)}}$ of the $\Z_2$ ladder LGT.
        The latter can be obtained from the former by applying the Wilson loop operator $W_1$.
        The states $\ket{0}$ and $\ket{1}$ refers to the eigenstates of the electric field operator $V$, which is just $\sigma_{z}$ in the $\Z_2$ model.
    }
    \label{fig:z2_vacua}
\end{figure}

\begin{figure*}
    \centering
    \begin{tikzpicture}[scale=0.6]
    % Clock chain
    \begin{scope}[xshift=6.5cm, yshift=4cm, local bounding box=chain]
        \draw[ladder] (-0.5, 0) -- (5.5,0) node [pos=0.5, above=10pt, inner sep=5pt, black] {dual 2--clock chain};
        \foreach \x/\Arrow in {0/\UpArrow, 1/\DownArrow, 2/\DownArrow, 3/\UpArrow, 4/\UpArrow, 5/\DownArrow} {
            \Arrow{\x}{0};
            \draw (\x, 0) node [site] {};
        }
        \useasboundingbox (-1.5, 0) -- (6.5,0) -- +(0,-1);
    \end{scope}

    % LGT sector (0,0)
    \begin{scope}[local bounding box=trivial]
        \node at (3,1) [above, inner sep=5pt] {$\Z_2$ LGT, sector $(0,0)$};
        \draw[ladder] (-0.5,0) grid (6.5,1);
        \draw[up, flux] (0,0) rectangle (1,1);
        \draw[up, flux] (3,0) rectangle (5,1);
        % sites
        \foreach \y in {0,1} \foreach \x in {0,1,...,6} \draw (\x,\y) node [site] {};
        \useasboundingbox (-1.5, 0) -- (7.5,0) -- +(0,-0.75);
    \end{scope}

    % LGT sector (1,0)
    \begin{scope}[xshift=12cm, local bounding box=topological]
        \node at (3,1) [above, inner sep=5pt] {$\Z_2$ LGT, sector $(1,0)$};
        \draw[ladder] (-0.5,0) grid (6.5,1);
        \draw[up] (-0.5, 0) -- (0,0) -- (0,1) -- (1,1) -- (1,0) -- (3,0) -- (3,1) -- (5,1) -- (5,0) -- (6.5,0);
        \fill[flux] (0,0) rectangle (1,1);
        \fill[flux] (3,0) rectangle (5,1);
        % sites
        \foreach \y in {0,1} \foreach \x in {0,1,...,6} \draw (\x,\y) node [site] {};
        \useasboundingbox (-1.5, 0) -- (7.5,0) -- +(0,-0.75);
    \end{scope}

    % Bounding boxes
    \draw [thin, Gray] (chain.north west) rectangle (chain.south east);
    \draw [thin, Gray] (trivial.north west) rectangle (trivial.south east);
    \draw [thin, Gray] (topological.north west) rectangle (topological.south east);

    % Arrows between the bounding boxes
    \draw [thick, Gray, shorten >= 3pt] (chain.west)
        edge [bend right, -{Latex}, Gray, very thick] node [above left, text=black] {$\ket{\Omega_{(0,0)}}$}
        (trivial.north);
    \draw [thick, Gray, shorten >= 3pt] (chain.east)
        edge [bend left,  -{Latex}, Gray, very thick] node [above right, text=black] {$\ket{\Omega_{(1,0)}}$}
        (topological.north);
\end{tikzpicture}

    \caption{Duality between the states of a $2$--chain and the states of a $\Z_2$ ladder LGT in the different sectors $(0,0)$ (no non-contractible electric loop) and $(1,0)$ (one non-contractible loop around the ladder).
        In the sector $(0,0)$ it is evident that all the physical states contains closed electric loops.
        On the other hand, in the sector $(1,0)$ the physical states are all the possible deformation of the electric string that goes around the ladder.}
    \label{fig:z2_states}
\end{figure*}

In order to proceed with ED one has to provide two things: (i) the basic operators of the theory ($U_{\ell}$ and $V$) and (ii) the physical (gauge-invariant) Hilbert space, given a lattice with specified size and boundary conditions.
The former was fairly standard while the latter was the most challenging and interesting part to implement.

If one has to work with only physical states, then one has to check the Gauss law for every site.
With the brute-force method one has to generate all the possible states and then filter out all the states that violate Gauss law.
This method, like any brute-force method, is not very efficient.
To better exemplify this, consider a $\Z_2$ theory on a $L \times L$ periodic lattice, which have $L^2$ sites and $2L^2$ links.
There are therefore $2^{2 L^2}$ possible states and for each one up to $L^2$ checks (one per site) has to be performed.
Moreover, it can be showed that there are only $2^{L^2}$ \emph{physical} states.
As a result, the construction of the physical Hilbert space involves $O(L^2 2^{2 L ^2})$ operations in a search space of $2^{2 L^2}$ objects for finding only $2^{L^2}$ elements.
All of this makes the inefficiency of this brute-force method very clear, even for moderately small lattices.


The approach adopted in this work exploits the duality in Sec.~\ref{sec:dualities_of_the_ladder} and represents an \emph{exponential speedup} with respect to the brute-force method.
It is not a search or pattern-matching algorithm, each physical configuration is procedurally generated from the states of the dual clock model.

Given a $\Z_N$ LGT on a lattice of size $L \times L$, we consider the dual $N$-clock model on a similar lattice with $A = L^2$ sites,
In its Hilbert space $\mathcal{H}_{N\text{-clock}}$ there is no gauge constraint or physical condition to apply,
hence the basis is the set of states $\ket{ \{s_i\} } \equiv \ket{s_0 s_1 \cdots s_{A-1}}$ with each $s_i = 0, \dots, N-1$.
From a state $\ket{ \{s_i\} }$ we can obtain the dual state for the LGT model in the $(m,n)$ sector:
\begin{equation}
    \ket{\{s_i\} } \; \longmapsto \;
    \prod_{i=0}^{A-1} U_i^{s_i} \ket{\Omega_{(n,m)}},
\end{equation}
where $U_i$ is the plaquette operator on the $i$-th plaquette and $\ket{\Omega_{(m,n)}}$ is the ``Fock vacuum'' of the $(m,n)$ sector.
As one can deduce, the information about the topological sector of the LGT model is carried in the Hamiltonian $H_{N\text{-clock}}$ of the dual clock model and not in the structure of $\mathcal{H}_{N \text{-clock}}$.
This means that is possible to build each sector $\Hphys^{(n,m)}$ in \eqref{eq:decomposizione_Hphys} from $\mathcal{H}_{N \text{-clock}}$, with the appropriate $\ket{\Omega_{(n,m)}}$.


Moreover, also the ``Fock vacuums'' $\ket{\Omega_{(n,m)}}$ can be obtained easily, thanks to \eqref{eq:azione_wilson_loop}:
\begin{equation}
    \ket{\Omega_{(n,m)}} = (W_1)^n (W_2)^m \ket{\Omega_{(0,0)}},
\end{equation}
where $\ket{\Omega_{(0,0)}}$ is just the state $\ket{000 \cdots 0}$ (in the electric basis) for all the links.


If we want to quantify the obtained speedup with this method, in the case of a $\Z_2$ theory on a square lattice $L \times L$ there are $2^{L^2}$ possible clock configurations.
For each configuration, there are at most $L^2$ magnetic fluxes to apply.
This translates into $O(L^2 2^{L^2})$ operations, which is an exponential speedup with respect to the brute-force (notice the lack of a factor 2 in the exponent) and is easily generalizable for any $\Z_N$.
Although, it remains an open question whether a similar method can be applied for gauge theories with non-Abelian finite groups.



\subsection{Non-local order parameters}%
\label{sub:non_local_order_parameters}

In Sec.~\ref{sub:investigating_the_phase_diagram} we talked about how a Wilson loop $W_{\mathcal{R}}$ or an 't Hooft string $S_{\tilde{\mathcal{C}}}$ work as a non-local order parameters and can be used to investigated the phase diagram of a $\Z_N$ LGT model.
In fact, we analyzed these exact observables on the ladder geometry for $N = 2,3$ and $4$.
Given a ladder of length $L$, the Wilson loop $W$ have been calculated over a region that covers the first $L/2$ plaquettes, while the 't Hooft string cuts through the first $L/2$ plaquettes (see Fig.~\ref{fig:nlop_ladder}).

\begin{figure}[h]
    \centering
    \begin{tikzpicture}[
    scale=0.9,
    font=\small,
    site/.style = {circle, inner sep=0 pt, minimum size=3.5pt, draw=black, fill=white},
    string/.style={{Circle[length=4pt, width=4pt]}-{Circle[length=4pt, width=4pt]}, very thick, dashed, Red}
    ]
    %%% Wilson loop

    % Lattice
    \draw[Gray, thin] (-0.5,0) grid (8.5,1);
    % Loop interior
    \draw[Blue, ultra thick, pattern=north east lines, pattern color=Blue] (0,0) rectangle (4,1);
    % Labels
    \draw (2,0.5) node [fill=white, rounded corners] {$W$};
    \draw (0,0) node [below] {$0$};
    \draw (4,0) node [below] {$L/2$};
    \draw (8,0) node [below] {$L$};
    \draw (4,1.5) node {Wilson loop};
    \foreach \y in {0,1} \foreach \x in {0,...,8} \draw (\x,\y) node [site] {};

    %%% 't Hooft string
    \begin{scope}[yshift=-3cm]
        % Lattice
        \draw[Gray, thin] (-0.5,0) grid (8.5,1);
        % String
        \draw [string] (0.5,0.5) -- (3.5,0.5) node [black, above, pos=0.65] {$S$};
        \foreach \x in {1,2,3} \draw [Red, ultra thick] (\x,0) -- (\x,1);
        % Labels
        \draw (0,0) node [below] {$0$};
        \draw (4,0) node [below] {$L/2$};
        \draw (8,0) node [below] {$L$};
        \draw (4,1.5) node {'t Hooft string};
        \foreach \y in {0,1} \foreach \x in {0,...,8} \draw (\x,\y) node [site] {};
    \end{scope}
\end{tikzpicture}

    \caption{The non-local order parameters that have been used for investigating the phase diagram of $\Z_N$ ladder LGT.
    \emph{Top}: half-ladder Wilson loop.
    \emph{Bottom}: half-ladder 't Hooft string operator.}%
    \label{fig:nlop_ladder}
\end{figure}




We wish to investigate the presence of a \emph{deconfined-confined phase transition} (DCPT) for a given $\Z_N$ ladder LGT.
In a pure gauge theory, these phases can be detected
with the perimeter/area law for Wilson loops \cite{wilson1974confinement},
which can be expressed as the products of magnetic operators over a given region.
Unfortunately, in a  ladder geometry there is not much difference between the area and the perimeter of a loop, since they both grow linearly in the size system $L$.

Nonetheless, we expect a phase transition by varying $\lambda$ \cite{trebst2007topological, hamma2008adiabatic, tagliacozzo2011entanglement} that can still be captured by an operator like $W_{\mathcal{R}}= \prod_{i \in \mathcal{R}} U_{i}$, the product of magnetic operators $U$'s over a (connected) region $\mathcal{R}$.
Indeed, when $\lambda=0$, the Hamiltonian \eqref{eq:ladder_hamiltonian} is analogous to a Toric Code \cite{kitaev2003fault} which is known to be in a deconfined phase, where the (topologically distinct) ground states are obtained as uniform superpositions of the gauge-invariant states, i.e.~closed electric loops.
On these ground states $\ev{W_{\mathcal{R}}} = 1$, hence
 a value $\ev{W_{\mathcal{R}}} \approx 1$ signals a deconfined phase.
On the other hand, when $\lambda \rightarrow \infty$, the electric loops are suppressed, hence
$\ev{W_{\mathcal{R}}} \approx 0$, signalling a confined phase.

In the dual clock model picture, the Wilson loop translates to a disorder operator \cite{fradkin1978order}, which means that a deconfined phase can be thought of as a paramagnetic (or disordered) phase, while the confined phase is like a ferromagnetic (or ordered) phase.
Moreover, the longitudinal field breaks the $N$-fold symmetry of the ferromagnetic phase into a one-fold or two-fold degeneracy, depending on the parity ($n$ even/odd) of the super-selection sector.

We study the $\Z_N$ LGT on a ladder numerically through \emph{exact diagonalization}, by evaluating the half-ladder Wilson loop, i.e.~
\begin{equation}
    W = U_1 U_2 \cdots U_{L/2},
\end{equation}
and working in the restricted physical Hilbert space $\Hphys^{(n)}$ ($n=0,\dots,N-1$), which has dimension $N^L$, much smaller than $N^{3L}$ (the dimension of the total Hilbert space).

The naive and brute-force method for building $\Hphys$ would require checking the Gauss law at every site (which are $O(3L)$ operations) for all the possible $N^{3L}$ candidate states.
On the other hand, the gauge-reducing duality to clock models provides a faithful and efficient method for building the $N^{L+1}$ basis states of $\Hphys$, yielding a major speedup with respect to the naive method.
%(see \cite{Note1}).
The procedure is quite simple and it consists in treating a clock state as a plaquette flux state in the following way.
Let $\ket{\Omega_0}$ be the vacuum state where all the links are in the $\ket{0}$ state.
For each sector $n$ we can build a ``vacuum'' state $\ket{\Omega_n}$ by applying $\overline{W}$ in \eqref{eq:nonlocal_op_ZN} $n$ times on the true vacuum, i.e.~$\ket{\Omega_n} = \overline{W}^{n} \ket{\Omega_0}$.
Then, let $\ket{s_1 s_2 \cdots}$ be a configuration of the dual $N$-clock model, where $s_i = 0, \dots, N-1$.
Now, the equivalent ladder state in the $n$-th sector can be obtained with $\prod_{i} U_i^{s_i} \ket{\Omega_n}$.

In the following, we present the results with $N=2,3$ and $4$, for different lengths.



\begin{figure}[t]
    \centering
    \begin{tikzpicture}[
        % scale=0.8,
        font=\small,
        notes/.style={gray!20!black, font=\scriptsize}
    ]
    \begin{groupplot}[
        group style={
            group name=wilson2,
            group size=1 by 2,
            vertical sep=3pt,
            x descriptions at=edge bottom%,
            % every plot/.style={thick}
        },
        width=7.5cm,
        height=5cm,
        no marks,
        table/col sep=comma,
        table/x=coupling,
        xtick pos=left,
        ytick pos=left,
        ylabel={$W$},
        xlabel={$\lambda$},
        try min ticks=5,
        cycle list name=exotic,
        legend style={font=\tiny, draw=gray!40},
        legend image post style={scale=0.5}
        ]
        % Sector n=0
        \ZTwoWilsonGraph{00}
        \legend{
            {$L = 10$},
            {$L = 12$},
            {$L = 14$},
            {$L = 16$},
            {$L = 18$}
        }
        % notes on the graphs
        \draw (axis cs:0,1) node (deconf) [circle, fill=gray!40!black, inner sep=0pt, minimum size=4pt] {};
        \draw[<-, shorten <=2pt, gray!40!black] (deconf) -- +(0.3,0) node [right, notes] {deconfined point};
        \draw[black] (axis cs:1,0) node [above, notes] {confined};

        % Sector n=1
        \ZTwoWilsonGraph{10}
        % notes on the graphs
        \draw (axis cs:0.3,0) node [above=10pt, notes] {deconfined};
        \draw (axis cs:1.7,0) node [above=10pt, notes] {confined};
        \draw[dashed, gray] (axis cs:1,0) -- (axis cs:1,1);
    \end{groupplot}
    \draw (wilson2 c1r1.east) node[above, rotate=-90] {sector $n=0$};
    \draw (wilson2 c1r2.east) node[above, rotate=-90] {sector $n=1$};
    \draw (wilson2 c1r1.north) node[above, font=\normalsize] {$\Z_2$ Wilson loop};
\end{tikzpicture}

    \vspace*{-10pt}
    \caption{$\Z_2$ Wilson loop in the sectors $n=0$ (\emph{top}) and $n=1$ (\emph{bottom}), for sizes $L=10,12, \dots,18$.
    The sector $n=0$ presents only a deconfined point at $\lambda=0$ and then decays rapidly into a confined phase, while the sector $n=1$ has a phase transition for $\lambda \simeq 1$.
    }
    \label{fig:z2_wilson}
\end{figure}

\smallskip

% \begin{figure*}[t]
%     \centering
%     \begin{tikzpicture}[
        scale=0.8,
        font=\footnotesize,
        notes/.style = {gray!20!black, font=\scriptsize, align=left}
        ]
    %============================================================
    % Z2 Wilson loop
    %============================================================
    \begin{groupplot}[
        group style={
            group name=wilson2,
            group size=1 by 2,
            vertical sep=3pt,
            x descriptions at=edge bottom%,
            % every plot/.style={thick}
        },
        width=6.5cm,
        height=4.5cm,
        no marks,
        table/col sep=comma,
        table/x=coupling,
        xtick pos=left,
        ytick pos=left,
        ylabel={$W$},
        xlabel={$\lambda$},
        try min ticks=5,
        cycle list name=exotic,
        legend style={font=\tiny, draw=gray!40},
        legend image post style={scale=0.5}
        ]
        % Sector n=0
        \ZTwoWilsonGraph{00}
        \legend{
            {$L = 10$},
            {$L = 12$},
            {$L = 14$},
            {$L = 16$},
            {$L = 18$}
        }
        % notes on the graphs
        \draw (axis cs:0,1) node (deconf) [circle, fill=gray!40!black, inner sep=0pt, minimum size=4pt] {};
        \draw[<-, shorten <=2pt, gray!40!black] (deconf) -- +(0.3,-0.1) node [right, notes, anchor=west] {deconfined\\point};
        \draw[black] (axis cs:1,0) node [above, notes] {confined};

        % Sector n=1
        \ZTwoWilsonGraph{10}
        % notes on the graphs
        \draw (axis cs:0.3,0) node [above=10pt, notes] {deconfined};
        \draw (axis cs:1.7,0) node [above=10pt, notes] {confined};
        \draw[dashed, gray] (axis cs:1,0) -- (axis cs:1,1);
    \end{groupplot}
    % Sector labels
    \draw (wilson2 c1r1.east) node[above, rotate=-90] {$n=0$};
    \draw (wilson2 c1r2.east) node[above, rotate=-90] {$n=1$};
    \draw (wilson2 c1r1.north) node[above, font=\normalsize] {\textbf{(a)} $\Z_2$ Wilson loop};


    %============================================================
    % Z3 Wilson loop
    %============================================================
    \begin{scope}[xshift=6.5cm]
        \begin{groupplot}[
                group style={
                    group name=wilson3,
                    group size=1 by 2,
                    vertical sep=3pt,
                    x descriptions at=edge bottom%,
                    % every plot/.style={thick}
                },
                width=6.5cm,
                height=4.5cm,
                no marks,
                table/col sep=comma,
                table/x=coupling,
                xtick pos=left,
                ytick pos=left,
                ylabel={$W$},
                xlabel={$\lambda$},
                try min ticks=5,
                cycle list name=exotic,
                legend style={font=\tiny, draw=gray!40},
                legend image post style={scale=0.5}
            ]

            % Sector n=0
            \ZThreeWilsonGraph{00}
            \legend{
                {$L = 7$},
                {$L = 9$},
                {$L = 11$},
                {$L = 13$}
            }
            \draw (axis cs:0,1) node (deconf) [circle, fill=gray!40!black, inner sep=0pt, minimum size=4pt] {};
            \draw[<-, shorten <=2pt, gray!40!black] (deconf) -- +(0.3,-0.07) node [right, notes, align=left, anchor=west] {deconfined\\point};
            \draw[black] (axis cs:1.2,0) node [above, notes] {confined};

            % Sector n=1
            \ZThreeWilsonGraph{10}
            \draw (axis cs:0,1) node (deconf) [circle, fill=gray!40!black, inner sep=0pt, minimum size=4pt] {};
            \draw[<-, shorten <=2pt, gray!40!black] (deconf) -- +(0.15,-0.4) node [below, notes,align=center] {deconfined\\point};
            \draw (axis cs:1.65,0) node [above=3pt, notes] {double degeneracy};
            \draw (axis cs:0.65,0.5) node [above, rotate=-45, notes] {crossover};

        \end{groupplot}

        \begin{axis}[
                no marks,
                width=3.5cm,
                height=2.75cm,
                table/col sep=comma,
                table/x=coupling,
                xtick={0.5,0.75,1.0,1.25},
                xtick align=center,
                ytick align=center,
                xtick pos=left,
                ytick pos=left,
                ymax=3.5,
                at={(wilson3 c1r2.north east)},
                anchor={north east},
                font=\tiny
            ]
            \addplot[thick, blue] table [y=DeltaE1] {graphs/data/Z3_gap_sec_1.csv}
            node [pos=0.8, black, above] {$\Delta E_1$};
            \addplot[thick, red]  table [y=DeltaE2] {graphs/data/Z3_gap_sec_1.csv}
            node [pos=0.6, black, left=2pt] {$\Delta E_2$};
        \end{axis}
        \draw (wilson3 c1r1.east) node[above, rotate=-90] {$n=0$};
        \draw (wilson3 c1r2.east) node[above, rotate=-90] {$n=1, 2$};
        \draw (wilson3 c1r1.north) node[above, font=\normalsize] {\textbf{(b)} $\Z_3$ Wilson loop};
    \end{scope}



    %============================================================
    % Z4 Wilson loop
    %============================================================
    \begin{scope}[xshift=13cm, yshift=1cm]
        \begin{groupplot}[
                group style={
                    group name=wilson4,
                    group size=1 by 3,
                    vertical sep=3pt,
                    horizontal sep=3pt,
                    x descriptions at=edge bottom,
                    y descriptions at=edge left% ,
                    % every plot/.style={thick}
                },
                width=6.5cm,
                height=3.5cm,
                no marks,
                table/col sep=comma,
                table/x=coupling,
                xtick pos=left,
                ytick pos=left,
                ylabel={$W$},
                xlabel={$\lambda$},
                try min ticks=5,
                cycle list name=exotic,
                legend style={font=\tiny, draw=gray!40},
                legend image post style={scale=0.5}
            ]
            % Sector n=0
            \ZFourWilsonGraph{00}
            \legend{{$L = 6$}, {$L = 8$}, {$L = 10$}}
            \draw (axis cs:0,1) node (deconf) [circle, fill=gray!40!black, inner sep=0pt, minimum size=4pt] {};
            \draw[<-, shorten <=2pt, gray!40!black] (deconf) -- +(0.35,-0.075) node [right, notes, align=left] {deconfined\\point};
            \draw[black] (axis cs:1.4,0) node [above, notes] {confined};

            % Sector n=1
            \ZFourWilsonGraph{10}
            \draw (axis cs:0,1) node (deconf) [circle, fill=gray!40!black, inner sep=0pt, minimum size=4pt] {};
            \draw[<-, shorten <=2pt, gray!40!black] (deconf) -- +(0.35,-0.075) node [right, notes, align=left] {deconf.\\point};
            \draw (axis cs:1.7,0) node [above=3pt, notes] {double degen.};

            % Sector n=2
            \ZFourWilsonGraph{20}
            \draw (axis cs:0.3,1) node [below=12pt, notes] {deconfined};
            \draw (axis cs:1.7,0) node [above=15pt, notes] {confined};
            \draw[dashed, gray] (axis cs:1.0,0) -- (axis cs:1.0,1);
    \end{groupplot}


    \begin{axis}[
            no marks,
            width=4cm,
            height=2.5cm,
            table/col sep=comma,
            table/x=coupling,
            xtick={0.5,0.75,1.0,1.25},
            xtick align=center,
            ytick align=center,
            xtick pos=left,
            ytick pos=left,
            ymax=2.5,
            at={(wilson4 c1r2.north east)},
            anchor={north east},
            font=\tiny
        ]
        \addplot[thick, blue] table [y=DeltaE1] {graphs/data/Z4_gap_sec_1.csv}
        node [pos=0.8, black, above] {$\Delta E_1$};
        \addplot[thick, red]  table [y=DeltaE2] {graphs/data/Z4_gap_sec_1.csv}
        node [pos=0.4, black, above] {$\Delta E_2$};
    \end{axis}

    \draw (wilson4 c1r1.east) node[above, rotate=-90] {$n=0$};
    \draw (wilson4 c1r2.east) node[above, rotate=-90] {$n=1, 3$};
    \draw (wilson4 c1r3.east) node[above, rotate=-90] {$n=2$};

    % title
    \draw (wilson4 c1r1.north) node[above, font=\normalsize] {\textbf{(c)} $\Z_4$ Wilson loop};
    \end{scope}
\end{tikzpicture}

%     \vspace*{-10pt}
%     \caption{
%         \textbf{(a)} $\Z_2$ Wilson loop in the sectors $n=0$ (\emph{top}) and $n=1$ (\emph{bottom}), for sizes $L=10,12, \dots,18$.
%         The sector $n=0$ presents only a deconfined point at $\lambda=0$ and then decays rapidly into a confined phase, while the sector $n=1$ has a phase transition for $\lambda \simeq 1$.
%         \textbf{(b)} $\Z_3$ Wilson loop for the sectors $n=0$ (\emph{top}) and $n=1,2$ (\emph{bottom}, which are equivalent), for sizes $L = 7,9,11$ and $13$.
%         Inset: energy differences $\Delta E_i = E_i - E_0$ for $i=1,2$, as a function of the coupling $\lambda$, in the sectors $n=1,2$, showing the emergence of a double-degenerate ground state for $\lambda > 1$.
%         \textbf{(c)} $\Z_4$ Wilson loop for sectors $n=0, \dots, 3$ and sizes $L=6, \dots, 10$.
%         Only the sector $n = 2$ has a clear deconfined-confined phase transition, as expected from the duality with the $4$-clock model.
%     }
%     \label{fig:wilson_loops}
% \end{figure*}



\paragraph{Results for \texorpdfstring{$N=2$}{N=2}}
As a warm up, we consider the $\Z_2$ ladder LGT, with lengths $L=10,12,\dots,18$.
This model is equivalent to a $p=2$ clock model, which is just the quantum Ising chain, with only two super-selection sectors for $n=0$ and $n=1$.
When $n=1$, the Hamiltonian  contains only the transverse filed and is integrable \cite{baxter1982exactlysm}.
Thus, we expect a critical point for $\lambda \simeq 1$, which will be a DCPT in the gauge model language.
This is clearly seen in the behaviour of the half-ladder Wilson loop, as shown in the lower panel of Fig.~\ref{fig:z2_wilson}.
For $n=0$, both the transverse and longitudinal fields  are present, the model is no longer integrable  \cite{banuls2011thermalization, kormos2017confinement, pomponio2022bloch} and we expect to always see a confined phase, except for $\lambda = 0$.
This is indeed confirmed by the behaviour of the half-ladder Wilson loop shown in the upper panel of Fig.~\ref{fig:z2_wilson}.

We can further characterize the phases of the two sectors by looking at the structure of the ground state, for $\lambda<1$ and $\lambda>1$, which is possible thanks to the exact diagonalization.
In particular, in the deconfined phase of the sector $n=1$, the ground state is a superposition of the deformations of the non-contractible electric string that makes the $n=1$ vacuum $\ket{\Omega_1}$.
For this reason, this phase can be thought as a \emph{kink condensate} \cite{fradkin1978order} (which is equivalent to a paramagnetic phase), where each kink corresponds to a deformation of the string.
Instead, for $\lambda > 1$, where we have confinement (as in the $n=0$ sector), the ground state is essentially a product state, akin to a ferromagnetic state. %(see \cite{Note1})


\begin{figure}[t]
    \centering
    \newcommand{\ZThreeWilsonGraph}[1]{
        \nextgroupplot
        \addplot+[thick] table [y=7x2]  {assets/graphs/data/Z3_wilson_#1.csv};
        \addplot+[thick] table [y=9x2]  {assets/graphs/data/Z3_wilson_#1.csv};
        \addplot+[thick] table [y=11x2] {assets/graphs/data/Z3_wilson_#1.csv};
        \addplot+[thick] table [y=13x2] {assets/graphs/data/Z3_wilson_#1.csv};
}

\begin{tikzpicture}[
        notes/.style={gray!20!black, font=\scriptsize},
        font=\small
    ]
    \begin{groupplot}[
        group style={
            group name=wilson3,
            group size=1 by 2,
            vertical sep=3pt,
            x descriptions at=edge bottom%,
            % every plot/.style={thick}
        },
        width=7.5cm,
        height=5cm,
        no marks,
        table/col sep=comma,
        table/x=coupling,
        xtick pos=left,
        ytick pos=left,
        ylabel={$W$},
        xlabel={$\lambda$},
        try min ticks=5,
        cycle list name=exotic,
        legend style={font=\tiny, draw=gray!40},
        legend image post style={scale=0.5}
        ]

        % Sector n=0
        \ZThreeWilsonGraph{00}
        \legend{
            {$L = 7$},
            {$L = 9$},
            {$L = 11$},
            {$L = 13$}
        }
        \draw (axis cs:0,1) node (deconf) [circle, fill=gray!40!black, inner sep=0pt, minimum size=4pt] {};
        \draw[<-, shorten <=2pt, gray!40!black] (deconf) -- +(0.3,-0.07) node [right, notes, align=left, anchor=west] {deconfined\\point};
        \draw[black] (axis cs:1.2,0) node [above, notes] {confined};

        % Sector n=1
        \ZThreeWilsonGraph{10}
        \draw (axis cs:0,1) node (deconf) [circle, fill=gray!40!black, inner sep=0pt, minimum size=4pt] {};
        \draw[<-, shorten <=2pt, gray!40!black] (deconf) -- +(0.15,-0.4) node [below, notes,align=center] {deconfined\\point};
        \draw (axis cs:1.65,0) node [above=3pt, notes] {double degeneracy};
        \draw (axis cs:0.65,0.5) node [above, rotate=-50, notes] {crossover};

    \end{groupplot}

    \begin{axis}[
            no marks,
            width=4cm,
            height=2.75cm,
            table/col sep=comma,
            table/x=coupling,
            xtick={0.5,0.75,1.0,1.25},
            xtick align=center,
            ytick align=center,
            xtick pos=left,
            ytick pos=left,
            ymax=3.5,
            at={(wilson3 c1r2.north east)},
            anchor={north east},
            font=\tiny
        ]
        \addplot[thick, blue] table [y=DeltaE1] {assets/graphs/data/Z3_gap_sec_1.csv}
            node [pos=0.8, black, above] {$\Delta E_1$};
        \addplot[thick, red]  table [y=DeltaE2] {assets/graphs/data/Z3_gap_sec_1.csv}
            node [pos=0.6, black, left=2pt] {$\Delta E_2$};
    \end{axis}

    \draw (wilson3 c1r1.east) node[above, rotate=-90] {sector $n=0$};
    \draw (wilson3 c1r2.east) node[above, rotate=-90] {sector $n=1, 2$};
    \draw (wilson3 c1r1.north) node[above, font=\normalsize] {$\Z_3$ Wilson loop};

\end{tikzpicture}

    \vspace*{-10pt}
    \caption{$\Z_3$ Wilson loop for the sectors $n=0$ (\emph{top}) and $n=1,2$ (\emph{bottom}, which are equivalent), for sizes $L = 7,9,11$ and $13$.
       Inset: energy differences $\Delta E_i = E_i - E_0$ for $i=1,2$, as a function of the coupling $\lambda$, in the sectors $n=1,2$, showing the emergence of a double-degenerate ground state for $\lambda > 1$.
}
    \label{fig:z3_wilson}
\end{figure}


\smallskip

\paragraph{Results for \texorpdfstring{$N=3$}{N=3}}%
The $\Z_3$ LGT is studied for lengths $L=7,9,11$ and $13$.
This model can be mapped to a $3$-clock model, which is equivalent to a $3$-state quantum Potts model with a longitudinal field, which is present in all sectors, as one can see from \eqref{eq:dual_ladder_hamiltonian_real}.
This field is expected to disrupt any ordered state and thus it is not possible to observe a phase transition, as it is confirmed  by the behaviour of the half-ladder Wilson loops  shown in Fig.~\ref{fig:z3_wilson}.
In addition, all the sectors present a deconfined point at $\lambda = 0$.
In the case $n=0$, for $\lambda > 0$ we recognize a quick transition to a confined phase, similar to what happens in \cite{burrello2021ladder}.
While for $n=1$ and $2$ (which are equivalent), the model exhibits a smoother \emph{crossover} to an ordered phase characterized by a doubly-degenerate ground state, for $\lambda > 1$.
Notice that, as discussed above,  the presence of the ``skew'' longitudinal field breaks the three-fold degeneracy expected in the ordered phase of the $3$-clock model into a two-fold degeneracy only.

\smallskip

\paragraph{Results for \texorpdfstring{$N=4$}{N=4}}%
The $\Z_4$ ladder LGT have four super-selection sectors.
The behaviour of half-ladder Wilson loops as function of $\lambda$ is shown in Fig.~\ref{fig:z4_wilson}.
As in the previous models, for $n=0$ we see a deconfined point at $\lambda = 0$, followed by a sharp transition to a confined phase.
The sector $n=2$, which has no longitudinal field, is the only one to present a clear DCPT for $\lambda \approx 1$, as it is expected from the fact that the $4$-clock model is equivalent to two decoupled Ising chains \cite{ortiz2012dualities}.
In the two equivalent sectors $n=1$ and $3$, where the longitudinal field is complex, the Wilson loop shows a peculiar behaviour, at least for the largest size ($L=10$) of the chain: it decreases fast as soon $\lambda > 0$, to stabilize to a finite value in the region $0.5 \lesssim \lambda \lesssim 1$, before tending to zero.
The characteristics of this phase would deserve a deeper analysis, that we plan to do in a future work.
For $\lambda \gtrsim 1$, the system enters a deconfined phase with a double degenerate ground state, as for the $\Z_3$ model.


\begin{figure}[t]
    \centering
    \begin{tikzpicture}[
        font=\small,
        notes/.style={gray!20!black, font=\scriptsize}
    ]
    \begin{groupplot}[
            group style={
                group name=wilson4,
                group size=1 by 3,
                vertical sep=3pt,
                horizontal sep=3pt,
                x descriptions at=edge bottom,
                y descriptions at=edge left% ,
                % every plot/.style={thick}
            },
            width=7.5cm,
            height=5cm,
            no marks,
            table/col sep=comma,
            table/x=coupling,
            xtick pos=left,
            ytick pos=left,
            ylabel={$W$},
            xlabel={$\lambda$},
            try min ticks=5,
            cycle list name=exotic,
            legend style={font=\tiny, draw=gray!40},
            legend image post style={scale=0.5}
        ]
        % Sector n=0
        \ZFourWilsonGraph{00}
        \legend{{$L = 6$}, {$L = 8$}, {$L = 10$}}
        \draw (axis cs:0,1) node (deconf) [circle, fill=gray!40!black, inner sep=0pt, minimum size=4pt] {};
        \draw[<-, shorten <=2pt, gray!40!black] (deconf) -- +(0.35,-0.075) node [right, notes, align=left] {deconfined\\point};
        \draw[black] (axis cs:1.4,0) node [above, notes] {confined};

        % Sector n=1
        \ZFourWilsonGraph{10}
        \draw (axis cs:0,1) node (deconf) [circle, fill=gray!40!black, inner sep=0pt, minimum size=4pt] {};
        \draw[<-, shorten <=2pt, gray!40!black] (deconf) -- +(0.35,-0.075) node [right, notes, align=left] {deconfined\\point};
        \draw (axis cs:1.7,0) node [above=5pt, notes] {double degen.};

        % Sector n=2
        \ZFourWilsonGraph{20}
        \draw (axis cs:0.3,1) node [below=12pt, notes] {deconfined};
        \draw (axis cs:1.7,0) node [above=15pt, notes] {confined};
        \draw[dashed, gray] (axis cs:1.0,0) -- (axis cs:1.0,1);
    \end{groupplot}

    %
    % Energy gap
    %
    \begin{axis}[
            no marks,
            width=4cm,
            height=2.75cm,
            table/col sep=comma,
            table/x=coupling,
            xtick={0.5,0.75,1.0,1.25},
            xtick align=center,
            ytick align=center,
            xtick pos=left,
            ytick pos=left,
            ymax=2.5,
            at={(wilson4 c1r2.north east)},
            anchor={north east},
            font=\tiny
        ]
        \addplot[thick, blue] table [y=DeltaE1] {assets/graphs/data/Z4_gap_sec_1.csv}
        node [pos=0.8, black, above] {$\Delta E_1$};
        \addplot[thick, red]  table [y=DeltaE2] {assets/graphs/data/Z4_gap_sec_1.csv}
        node [pos=0.4, black, above] {$\Delta E_2$};
    \end{axis}

    \draw (wilson4 c1r1.east) node[above, rotate=-90] {sector $n=0$};
    \draw (wilson4 c1r2.east) node[above, rotate=-90] {sector $n=1, 3$};
    \draw (wilson4 c1r3.east) node[above, rotate=-90] {sector $n=2$};

    % title
    \draw (wilson4 c1r1.north) node[above, font=\normalsize] {$\Z_4$ Wilson loop};
\end{tikzpicture}

    \vspace*{-10pt}
    \caption{$\Z_4$ Wilson loop for sectors $n=0, \dots, 3$ and sizes $L=6, \dots, 10$.
        Only the sector $n = 2$ has a clear deconfined-confined phase transition, as expected from the duality with the $4$-clock model.
    }
    \label{fig:z4_wilson}
\end{figure}

\smallskip

% \paragraph{Conclusions and outlooks}
% In this work, we proposed an exact gauge preserving duality transformation that maps the  $\mathbb{Z}_N$ lattice gauge theory on a ladder onto a 1D $N-$clock model in a transversal field, coupled to a possibly complex longitudinal field which depends on the super-selection sector.
%
% This map allowed us to perform numerical simulations with an exact diagonalization algorithm with sizes up to $L=18, 13, 10$ for $N=2,3,4$ respectively.
% To study the phases of the model and a possible DCPT transition, we calculated the Wilson loops in the different topological sectors, finding an unusual behaviour in the sectors with $n$ odd (mod $N$), possibly suggesting the emergence of a new phase, such as for example the incommensurate phase appearing in chiral clock models \cite{huse1983chiral, whitsitt2018clock, zhuang2015clock}, whose characterization requires however to consider longer sizes of the chain in order to evaluate the asymptotic behaviour of correlators.
% This will be the subject of future work, in which we can also consider the possibility to include static and dynamical matter in the lattice gauge model.

% \medskip

% \begin{acknowledgments}
%     Numerical simulations have been performed with the QuSpin library for exact diagonalization \cite{weinberg2017quspin, weinberg2019quspin}.
%     We thank M.~Burrello and O.~Pomponio for useful discussions.
%     This research is partially supported by INFN through the project “QUANTUM”, the project “SFT" and the project “QuantHEP" of the QuantERA ERA-NET Co-fund in Quantum Technologies (GA No. 731473).
% \end{acknowledgments}



\section{Concluding remarks}
\label{sec:concluding_remarks}

In this work, we proposed an exact gauge preserving duality transformation that maps the  $\mathbb{Z}_N$ lattice gauge theory on a ladder onto a 1D $N-$clock model in a transversal field, coupled to a possibly complex longitudinal field which depends on the super-selection sector.

This map allowed us to perform numerical simulations with an \ac{ed} algorithm with sizes up to $L=18, 13, 10$ for $N=2,3,4$ respectively.
To study the phases of the model and a possible \ac{dcpt} transition, we calculated the Wilson loops in the different topological sectors, finding an unusual behaviour in the sectors with $n$ odd (mod $N$), possibly suggesting the emergence of a new phase, such as for example the incommensurate phase appearing in chiral clock models \cite{huse1983chiral, whitsitt2018clock, zhuang2015clock}, whose characterization requires however to consider longer sizes of the chain in order to evaluate the asymptotic behaviour of correlators.

This will be the subject of future work, in which we can also consider the possibility to include static and dynamical matter in the lattice gauge model.
Another possible direction would be the extension of these duality transformations to non-Abelian gauge theories.
