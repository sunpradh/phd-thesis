\section{A case study: \texorpdfstring{$N=2, 3$}{N=2, 3} and \texorpdfstring{$4$}{4}}
\label{sec:a_case_study_N_2_3_4}


In this section we present the results of the numerical investigations of \cite{pradhan2022ladder}.
But first, we present the reasoning for the choice of order parameters used for investigating the phase diagram, and second we show how the duality have been used for resolving the Gauss law in numerics.


\subsection{Investigating the phase diagram}%
\label{sub:investigating_the_phase_diagram}

We wish to study the phase diagram of the $\Z_N$ \ac{lgt} phase diagram, in particular we are interested in any \emph{confined} or \emph{deconfined} phase.
In a pure gauge theory, these phases are investigated non-local order parameters like the \emph{\ac{wl}} (not be confused with the non-contractible \ac{wl}s in \eqref{eq:nonlocal_op_ZN}) or \emph{string tension}.
This is because we expect the deconfined phase to be a topological phase, which can be investigated only with non-local order parameters.

Given a closed region $\mathcal{R}$, a \ac{wl} operator $W_{\mathcal{R}}$ is defined as
\begin{equation}
    W_{\mathcal{R}} = \prod_{\square \in \mathcal{R}} U_{\square}.
    \label{eq:closed_wilson_loop}
\end{equation}
Alternatively, considering the oriented boundary $\partial \mathcal{R}$ one can write
\begin{equation}
    W_{\mathcal{R}} = \prod_{\ell \in \partial \mathcal{R}} U_{\ell},
\end{equation}
where the Hermitian conjugate is implied everytime we move in the negative directions.
It is also implied that the curve $\partial \mathcal{R}$ is a contractible loop.
Wilson showed in \cite{wilson1974confinement} that quark confinement is related to the expectation value $\ev{W_{\mathcal{R}}}$ of a \ac{wl}, which can be thought as a gauge field average on a region.
In particular, in the presence of quark confinement the gauge field average follows an \emph{area law}, where it decays exponentially with the area enclosed by $\mathcal{R}$.
On the other hand, in the deconfined phase we have a \emph{perimeter law}, where the gauge field average decays exponentially with the perimeter of $\mathcal{R}$.

Unfortunately on a ladder geometry there is not much difference between the area and the perimeter of a \ac{wl}.
In fact, in units of the lattice spacing, the area of a \ac{wl} over $n$ plaquettes is $n$ while its perimeter is just $2n+2$.
They both grow linearly.
Nonetheless, we can still look at the behaviour of the \ac{wl}, for a fixed length, at different couplings $\lambda$.

When the coupling $\lambda$ in \eqref{eq:hamiltonian_base} is equal to zero, the \ac{tc} is recovered and in any of its topological sector the ground state is the equal superposition of all the states with any number of closed electrical loops, in a similar fashion to coherent states.
This makes the \ac{tc} a \emph{quantum loop gas}, which is a \emph{deconfined phase}.
Furthermore, the operator $W_{\mathcal{R}}$ in \eqref{eq:closed_wilson_loop} creates an electrical loop around the region $\mathcal{R}$.
From the constraints
% TODO aggiustare ref
% \eqref{eq:constraints_gs_toric_code},
it can easily be proved that $W_{\mathcal{R}}$ leaves the \ac{tc} ground states unchanged, showing in fact that they behaves as coherent states, which leads to $\ev{W_{\mathcal{R}}} = 1$.

Therefore, $\ev{W_{\mathcal{R}}} \approx 1$ signals a deconfined phase and on the other hand a vanishing $\ev{W_{\mathcal{R}}} \approx 0$ corresponds to confined phase.
For this reason, even tough we lack an area/perimeter law on the ladder geometry it is still sensible to look at the behaviour of the \ac{wl}.


Another possible approach for investigating the phase diagram is to use the \emph{string tension}.
In two dimensions, given an \emph{open} curve $\tilde{\mathcal{C}}$ on the dual lattice $\tilde{\mathbb{L}}$ we can construct an open \ac{ths} operator $S_{\tilde{\mathcal{C}}}$ as
\begin{equation}
    S_{\tilde{\mathcal{C}}} = \prod_{\ell \in \tilde{\mathcal{C}}} V_{\ell}
\end{equation}
with the usual caveat: we have to take the Hermitian conjugate everytime the path goes in the negative direction.
Then the string tension is just the expectation value $\ev{S_{\tilde{\mathcal{C}}}}$ it is called in this way because it related to the potential energy (tension) between two magnetic fluxes created at the ends of the curve $\tilde{\mathcal{C}}$.
Henceforth, in a deconfined phase $\ev{S_{\tilde{\mathcal{C}}}} \approx 0$, which means that the magnetic fluxes can be moved freely with no cost in energy, like in the \ac{tc}.





% \section{Numerical analysis}%
% \label{sec:numerical_analysis}
%
% In this work we studied numerically the different topological sectors (and their phase diagrams) of the $\Z_N$ \ac{lgt} on ladder for $N=2,3,4$ through \emph{exact diagonalization} (ED).
% We chose ED instead of other variational methods like DMRG because we were able to construct exactly the Hilbert space of the different topological sectors of the models, exploiting the duality in Sec.~\ref{sec:dualities_of_the_ladder}.
%

\subsection{Implementing the Gauss law}%
\label{sub:implementing_the_gauss_law}

\begin{figure}[t]
    \centering
    \begin{tikzpicture}[
    font=\small,
    scale=0.75,
    % site/.style = {circle, inner sep=0 pt, minimum size=4pt, draw=black, fill=white},
    % up/.style = {ultra thick, green!70!black},
    legend/.style = {text=black, inner sep=5pt}
]

%%% Sector n=0 vacuum
\begin{scope}[local bounding box=trivial]
    \node at (3,1) [above=5pt, legend]  {vacuum $\ket{\Omega_0}$ of the sector $n=0$};
    % ladder
    \draw[ladder] (-0.5,0) grid (6.5,1);
    % sites
    \DrawSites{0,1,...,6}{0,1}
    \useasboundingbox (-1,0) -- (7,0) -- +(0,-0.5);
\end{scope}

%%% Sector n=1 vacuum
\begin{scope}[yshift=-3.5cm, local bounding box=topol]
    \node at (3,1) [above=5pt, legend]  {vacuum $\ket{\Omega_1}$ of the sector $n=1$};
    % ladder
    \draw[ladder] (-0.5,0) grid (6.5,1);
    % Wilson loop
    \draw[up] (-0.5, 0) -- (6.5, 0);
    % sites
    \DrawSites{0,1,...,6}{0,1}
    \useasboundingbox (-1,0) -- (7,0) -- +(0,-0.5);
\end{scope}

% legend
\begin{scope}[xshift=8cm, yshift=1cm, local bounding box=legend]
    \draw [Gray, thin] (0,0.75) -- +(0.5,0) node [right, legend] {$\ket{0}$};
    \draw [up] (0,0)    -- +(0.5,0) node [right, legend] {$\ket{1}$};
    \useasboundingbox (-0.25,0);
\end{scope}


\draw[thin, Gray] (trivial.south west) rectangle (trivial.north east);
\draw[thin, Gray] (topol.south west) rectangle (topol.north east);
\draw [shorten >= 3pt] (trivial.east)
    edge [-{Latex}, Gray, very thick, out=0, in=0]
    node [font=\normalsize, right, text=black] {$\Wilson_1$}
    (topol.east);
\end{tikzpicture}

    \caption[Vacuum states of the super-selection sectors of the $\Z_2$ ladder \ac{lgt}]{The different ``Fock vacua'' $\ket{\Omega_{(0,0)}}$ and $\ket{\Omega_{(1,0)}}$ of the $\Z_2$ ladder \ac{lgt}.
        The latter can be obtained from the former by applying the \ac{wl} operator $W_1$.
        The states $\ket{0}$ and $\ket{1}$ refers to the eigenstates of the electric field operator $V$, which is just $\sigma_{z}$ in the $\Z_2$ model.
    }
    \label{fig:z2_vacua}
\end{figure}

\begin{figure}
    \centering
    \begin{tikzpicture}[scale=0.6]
    % Clock chain
    \begin{scope}[xshift=6.5cm, yshift=4cm, local bounding box=chain]
        \draw[ladder] (-0.5, 0) -- (5.5,0) node [pos=0.5, above=10pt, inner sep=5pt, black] {dual 2--clock chain};
        \foreach \x/\Arrow in {0/\UpArrow, 1/\DownArrow, 2/\DownArrow, 3/\UpArrow, 4/\UpArrow, 5/\DownArrow} {
            \Arrow{\x}{0};
            \draw (\x, 0) node [site] {};
        }
        \useasboundingbox (-1.5, 0) -- (6.5,0) -- +(0,-1);
    \end{scope}

    % LGT sector (0,0)
    \begin{scope}[local bounding box=trivial]
        \node at (3,1) [above, inner sep=5pt] {$\Z_2$ LGT, sector $(0,0)$};
        \draw[ladder] (-0.5,0) grid (6.5,1);
        \draw[up, flux] (0,0) rectangle (1,1);
        \draw[up, flux] (3,0) rectangle (5,1);
        % sites
        \foreach \y in {0,1} \foreach \x in {0,1,...,6} \draw (\x,\y) node [site] {};
        \useasboundingbox (-1.5, 0) -- (7.5,0) -- +(0,-0.75);
    \end{scope}

    % LGT sector (1,0)
    \begin{scope}[xshift=12cm, local bounding box=topological]
        \node at (3,1) [above, inner sep=5pt] {$\Z_2$ LGT, sector $(1,0)$};
        \draw[ladder] (-0.5,0) grid (6.5,1);
        \draw[up] (-0.5, 0) -- (0,0) -- (0,1) -- (1,1) -- (1,0) -- (3,0) -- (3,1) -- (5,1) -- (5,0) -- (6.5,0);
        \fill[flux] (0,0) rectangle (1,1);
        \fill[flux] (3,0) rectangle (5,1);
        % sites
        \foreach \y in {0,1} \foreach \x in {0,1,...,6} \draw (\x,\y) node [site] {};
        \useasboundingbox (-1.5, 0) -- (7.5,0) -- +(0,-0.75);
    \end{scope}

    % Bounding boxes
    \draw [thin, Gray] (chain.north west) rectangle (chain.south east);
    \draw [thin, Gray] (trivial.north west) rectangle (trivial.south east);
    \draw [thin, Gray] (topological.north west) rectangle (topological.south east);

    % Arrows between the bounding boxes
    \draw [thick, Gray, shorten >= 3pt] (chain.west)
        edge [bend right, -{Latex}, Gray, very thick] node [above left, text=black] {$\ket{\Omega_{(0,0)}}$}
        (trivial.north);
    \draw [thick, Gray, shorten >= 3pt] (chain.east)
        edge [bend left,  -{Latex}, Gray, very thick] node [above right, text=black] {$\ket{\Omega_{(1,0)}}$}
        (topological.north);
\end{tikzpicture}

    \caption[Duality between clock states and ladder states]{Duality between the states of a $2$--chain and the states of a $\Z_2$ ladder \ac{lgt} in the different sectors $(0,0)$ (no non-contractible electric loop) and $(1,0)$ (one non-contractible loop around the ladder).
        In the sector $(0,0)$ it is evident that all the physical states contains closed electric loops.
        On the other hand, in the sector $(1,0)$ the physical states are all the possible deformation of the electric string that goes around the ladder.}
    \label{fig:z2_states}
\end{figure}

In order to proceed with ED one has to provide two things: (i) the basic operators of the theory ($U_{\ell}$ and $V$) and (ii) the physical (gauge-invariant) Hilbert space, given a lattice with specified size and boundary conditions.
The former was fairly standard while the latter was the most challenging and interesting part to implement.

If one has to work with only physical states, then one has to check the Gauss law for every site.
With the brute-force method one has to generate all the possible states and then filter out all the states that violate Gauss law.
This method, like any brute-force method, is not very efficient.
To better exemplify this, consider a $\Z_2$ theory on a $L \times L$ periodic lattice, which have $L^2$ sites and $2L^2$ links.
There are therefore $2^{2 L^2}$ possible states and for each one up to $L^2$ checks (one per site) has to be performed.
Moreover, it can be showed that there are only $2^{L^2}$ \emph{physical} states.
As a result, the construction of the physical Hilbert space involves $O(L^2 2^{2 L ^2})$ operations in a search space of $2^{2 L^2}$ objects for finding only $2^{L^2}$ elements.
All of this makes the inefficiency of this brute-force method very clear, even for moderately small lattices.


The approach adopted in this work exploits the duality in Sec.~\ref{sec:dualities_of_the_ladder} and represents an \emph{exponential speedup} with respect to the brute-force method.
It is not a search or pattern-matching algorithm, each physical configuration is procedurally generated from the states of the dual clock model.

Given a $\Z_N$ \ac{lgt} on a lattice of size $L \times L$, we consider the dual $N$-clock model on a similar lattice with $A = L^2$ sites,
In its Hilbert space $\mathcal{H}_{N\text{-clock}}$ there is no gauge constraint or physical condition to apply,
hence the basis is the set of states $\ket{ \{s_i\} } \equiv \ket{s_0 s_1 \cdots s_{A-1}}$ with each $s_i = 0, \dots, N-1$.
From a state $\ket{ \{s_i\} }$ we can obtain the dual state for the \ac{lgt} model in the $(m,n)$ sector:
\begin{equation}
    \ket{\{s_i\} } \; \longmapsto \;
    \prod_{i=0}^{A-1} U_i^{s_i} \ket{\Omega_{(n,m)}},
\end{equation}
where $U_i$ is the plaquette operator on the $i$-th plaquette and $\ket{\Omega_{(m,n)}}$ is the ``Fock vacuum'' of the $(m,n)$ sector.
As one can deduce, the information about the topological sector of the \ac{lgt} model is carried in the Hamiltonian $H_{N\text{-clock}}$ of the dual clock model and not in the structure of $\mathcal{H}_{N \text{-clock}}$.
This means that is possible to build each sector $\Hphys^{(n,m)}$ in \eqref{eq:decomposizione_Hphys} from $\mathcal{H}_{N \text{-clock}}$, with the appropriate $\ket{\Omega_{(n,m)}}$.


Moreover, also the ``Fock vacuums'' $\ket{\Omega_{(n,m)}}$ can be obtained easily, thanks to \eqref{eq:azione_wilson_loop}:
\begin{equation}
    \ket{\Omega_{(n,m)}} = (W_1)^n (W_2)^m \ket{\Omega_{(0,0)}},
\end{equation}
where $\ket{\Omega_{(0,0)}}$ is just the state $\ket{000 \cdots 0}$ (in the electric basis) for all the links.


If we want to quantify the obtained speedup with this method, in the case of a $\Z_2$ theory on a square lattice $L \times L$ there are $2^{L^2}$ possible clock configurations.
For each configuration, there are at most $L^2$ magnetic fluxes to apply.
This translates into $O(L^2 2^{L^2})$ operations, which is an exponential speedup with respect to the brute-force (notice the lack of a factor 2 in the exponent) and is easily generalizable for any $\Z_N$.
Although, it remains an open question whether a similar method can be applied for gauge theories with non-Abelian finite groups.



% \subsection{Non-local order parameters}%
% \label{sub:non_local_order_parameters}

% In Sec.~\ref{sub:investigating_the_phase_diagram} we talked about how a \ac{wl} $W_{\mathcal{R}}$ or an \ac{ths} $S_{\tilde{\mathcal{C}}}$ work as a non-local order parameters and can be used to investigated the phase diagram of a $\Z_N$ \ac{lgt} model.
% In fact, we analyzed these exact observables on the ladder geometry for $N = 2,3$ and $4$.
% Given a ladder of length $L$, the \ac{wl} $W$ have been calculated over a region that covers the first $L/2$ plaquettes, while the \ac{ths} cuts through the first $L/2$ plaquettes (see Fig.~\ref{fig:nlop_ladder}).

% \begin{figure}[t]
%     \centering
%     \input{assets/figures/nonlocal_order_parameters.tex}
%     \caption[Non-local order parameters on the ladder]{The non-local order parameters that have been used for investigating the phase diagram of $\Z_N$ ladder \ac{lgt}.
%     \emph{Top}: half-ladder \ac{wl}.
%     \emph{Bottom}: half-ladder \ac{ths} operator.}%
%     \label{fig:nlop_ladder}
% \end{figure}



\subsection{Numerical results}
\label{sub:numerical_results}

In the following, we present the results with $N=2,3$ and $4$, for different lengths.

\subsubsection*{Results for \texorpdfstring{$N=2$}{N=2}}

As a warm up, we consider the $\Z_2$ ladder \ac{lgt}, with lengths $L=10,12,\dots,18$.
This model is equivalent to a $p=2$ clock model, which is just the quantum Ising chain, with only two super-selection sectors for $n=0$ and $n=1$.
The dual Hamiltonian \eqref{eq:dual_ladder_hamiltonian_real} for $\Z_2$ and sector $n=0$ is
\begin{equation}
    \HamilDual[2]_{n=0}(\lambda^{-1}) = \HamilClock[2](\lambda^{-1}) - 2 \sum_{i} \qty( Z_i + Z_i^{\dagger} ),
\end{equation}
while for $n=1$ we just have
\begin{equation}
    \HamilDual[2]_{n=1}(\lambda^{-1}) = \HamilClock[2](\lambda^{-1}).
\end{equation}

When $n=1$ the Hamiltonian $\HamilDual[2]$ contains only the transverse field, hence it is integrable \cite{baxter1982exactlysm}.
Thus, we expect a critical point for $\lambda \simeq 1$, which will be a DCPT in the gauge model language.
This is clearly seen in the behaviour of the half-ladder \ac{wl}, as shown in the lower panel of Fig.~\ref{fig:z2_wilson}.
For $n=0$, both the transverse and longitudinal fields  are present, hence the model is no longer integrable \cite{banuls2011thermalization, kormos2017confinement, pomponio2022bloch} and we expect to always see a confined phase, except for $\lambda = 0$.
This is indeed confirmed by the behaviour of the half-ladder \ac{wl} shown in the upper panel of Fig.~\ref{fig:z2_wilson}.

Notice that for $n=0$ in Fig.~\ref{fig:z2_wilson}, there is a lack of line crossings between values of \ac{wl} for different $L$.
This suggests that in the thermodynamic limit, we will have a single point $\ev{W} \neq 0$ for $\lambda = 0$, and a flat line $\ev*{W} = 0$ for $\lambda \neq 0$, confirming the prediction of a always confining phase (excluded $\lambda = 0$).

We can further characterize the phases of the two sectors by looking at the structure of the ground state, for $\lambda<1$ and $\lambda>1$, which is possible thanks to the exact diagonalization.
In particular, in the deconfined phase of the sector $n=1$, the ground state is a superposition of the deformations of the non-contractible electric string that makes the $n=1$ vacuum $\ket{\Omega_1}$.
For this reason, this phase can be thought as a \emph{kink condensate} \cite{fradkin1978order} (which is equivalent to a paramagnetic phase), where each kink corresponds to a deformation of the string.
Instead, for $\lambda > 1$, where we have confinement (as in the $n=0$ sector), the ground state is essentially a product state, akin to a ferromagnetic state.
This analysis has been performed at the end of Sec.~\ref{sub:numerical_results}.

\begin{figure}[t]
    \centering
    \begin{tikzpicture}[
        % scale=0.8,
        font=\small,
        notes/.style={gray!20!black, font=\scriptsize}
    ]
    \begin{groupplot}[
        group style={
            group name=wilson2,
            group size=1 by 2,
            vertical sep=3pt,
            x descriptions at=edge bottom%,
            % every plot/.style={thick}
        },
        width=7.5cm,
        height=5cm,
        no marks,
        table/col sep=comma,
        table/x=coupling,
        xtick pos=left,
        ytick pos=left,
        ylabel={$W$},
        xlabel={$\lambda$},
        try min ticks=5,
        cycle list name=exotic,
        legend style={font=\tiny, draw=gray!40},
        legend image post style={scale=0.5}
        ]
        % Sector n=0
        \ZTwoWilsonGraph{00}
        \legend{
            {$L = 10$},
            {$L = 12$},
            {$L = 14$},
            {$L = 16$},
            {$L = 18$}
        }
        % notes on the graphs
        \draw (axis cs:0,1) node (deconf) [circle, fill=gray!40!black, inner sep=0pt, minimum size=4pt] {};
        \draw[<-, shorten <=2pt, gray!40!black] (deconf) -- +(0.3,0) node [right, notes] {deconfined point};
        \draw[black] (axis cs:1,0) node [above, notes] {confined};

        % Sector n=1
        \ZTwoWilsonGraph{10}
        % notes on the graphs
        \draw (axis cs:0.3,0) node [above=10pt, notes] {deconfined};
        \draw (axis cs:1.7,0) node [above=10pt, notes] {confined};
        \draw[dashed, gray] (axis cs:1,0) -- (axis cs:1,1);
    \end{groupplot}
    \draw (wilson2 c1r1.east) node[above, rotate=-90] {sector $n=0$};
    \draw (wilson2 c1r2.east) node[above, rotate=-90] {sector $n=1$};
    \draw (wilson2 c1r1.north) node[above, font=\normalsize] {$\Z_2$ Wilson loop};
\end{tikzpicture}

    \vspace*{-10pt}
    \caption[\ac{wl}s for the $\Z_2$ ladder \ac{lgt}]{$\Z_2$ \ac{wl} in the sectors $n=0$ (\emph{top}) and $n=1$ (\emph{bottom}), for sizes $L=10,12, \dots,18$.
    The sector $n=0$ presents only a deconfined point at $\lambda=0$ and then decays rapidly into a confined phase, while the sector $n=1$ has a phase transition for $\lambda \simeq 1$.
    }
    \label{fig:z2_wilson}
\end{figure}



\subsubsection*{Results for \texorpdfstring{$N=3$}{N=3}}%


The $\Z_3$ \ac{lgt} is studied for lengths $L=7,9,11$ and $13$.
This model can be mapped to a $3$-clock model, which is equivalent to a $3$-state quantum Potts model with a longitudinal field, which is present in all sectors, as one can see from \eqref{eq:dual_ladder_hamiltonian_real}.
The dual Hamiltonian $\HamilDual[3](\lambda^{-1})$ for the sector $n=0$ is
\begin{equation}
    \HamilDual[3]_{n=0}(\lambda^{-1}) = \HamilClock[3](\lambda^{-1}) - 2 \sum_{i} \qty(Z_i + Z_i^{\dagger}),
\end{equation}
while for $n=1$ and $2$ (which are symmetric to each other) we have
\begin{equation}
    \HamilDual[3]_{n=1,2}(\lambda^{-1}) = \HamilClock[3](\lambda^{-1}) - 2 \cos( \frac{\pi}{3} ) \sum_{i} \qty(\tilde{Z}_i + \tilde{Z}_i^{\dagger}).
    \label{eq:dual_clock_3_1}
\end{equation}
Remember that $\tilde{Z}_i$ stands for the $Z_i$ operator with eigenvalues shifted by $\omega^{1/2} = e^{i \pi /N}$.
In all three cases we have a longitudinal field, which is expected to disrupt any paramagnetic state.
Thus, we do not expect to observe a phase transition, and this is confirmed by the behaviour observed in Fig.~\ref{fig:z3_wilson}.
Meanwhile, all the sectors present a deconfined point at $\lambda = 0$, as expected.

In the case $n=0$, for $\lambda > 0$ we recognize a quick transition to a confined phase, similar to what happens in \cite{burrello2021ladder}.
This behaviour is similar to what has been observed for the $\Z_2$ and $n=0$ case in Fig.~\ref{fig:z2_wilson}, hence the same reasoning apply.
While for $n=1$ and $2$ (which are equivalent), the model exhibits a smoother \emph{crossover} to an ordered phase characterized by a doubly-degenerate ground state, for $\lambda > 1$.
Notice that, as discussed above,  the presence of the ``skew'' longitudinal field breaks the three-fold degeneracy expected in the ordered phase of the $3$-clock model into a two-fold degeneracy only.
Additionally, for $L=13$ we notice that a slight bump start to appear.
If some speculation is allowed, this fact, united with the crossover region, may suggest that is some intermediate phase between the deconfined point and the confined region.
For this kind of analysis, higher lattice sizes are necessary which means that \ac{ed} is no longer adequate.
Thankfully, now that we are confident in the duality between ladder \acp{lgt} and \acp{clock}, we can directly study this region in the \acp{clock} setup, by simulating \eqref{eq:dual_clock_3_1} with for example \ac{dmrg}.


\begin{figure}[t]
    \centering
    \newcommand{\ZThreeWilsonGraph}[1]{
        \nextgroupplot
        \addplot+[thick] table [y=7x2]  {assets/graphs/data/Z3_wilson_#1.csv};
        \addplot+[thick] table [y=9x2]  {assets/graphs/data/Z3_wilson_#1.csv};
        \addplot+[thick] table [y=11x2] {assets/graphs/data/Z3_wilson_#1.csv};
        \addplot+[thick] table [y=13x2] {assets/graphs/data/Z3_wilson_#1.csv};
}

\begin{tikzpicture}[
        notes/.style={gray!20!black, font=\scriptsize},
        font=\small
    ]
    \begin{groupplot}[
        group style={
            group name=wilson3,
            group size=1 by 2,
            vertical sep=3pt,
            x descriptions at=edge bottom%,
            % every plot/.style={thick}
        },
        width=7.5cm,
        height=5cm,
        no marks,
        table/col sep=comma,
        table/x=coupling,
        xtick pos=left,
        ytick pos=left,
        ylabel={$W$},
        xlabel={$\lambda$},
        try min ticks=5,
        cycle list name=exotic,
        legend style={font=\tiny, draw=gray!40},
        legend image post style={scale=0.5}
        ]

        % Sector n=0
        \ZThreeWilsonGraph{00}
        \legend{
            {$L = 7$},
            {$L = 9$},
            {$L = 11$},
            {$L = 13$}
        }
        \draw (axis cs:0,1) node (deconf) [circle, fill=gray!40!black, inner sep=0pt, minimum size=4pt] {};
        \draw[<-, shorten <=2pt, gray!40!black] (deconf) -- +(0.3,-0.07) node [right, notes, align=left, anchor=west] {deconfined\\point};
        \draw[black] (axis cs:1.2,0) node [above, notes] {confined};

        % Sector n=1
        \ZThreeWilsonGraph{10}
        \draw (axis cs:0,1) node (deconf) [circle, fill=gray!40!black, inner sep=0pt, minimum size=4pt] {};
        \draw[<-, shorten <=2pt, gray!40!black] (deconf) -- +(0.15,-0.4) node [below, notes,align=center] {deconfined\\point};
        \draw (axis cs:1.65,0) node [above=3pt, notes] {double degeneracy};
        \draw (axis cs:0.65,0.5) node [above, rotate=-50, notes] {crossover};

    \end{groupplot}

    \begin{axis}[
            no marks,
            width=4cm,
            height=2.75cm,
            table/col sep=comma,
            table/x=coupling,
            xtick={0.5,0.75,1.0,1.25},
            xtick align=center,
            ytick align=center,
            xtick pos=left,
            ytick pos=left,
            ymax=3.5,
            at={(wilson3 c1r2.north east)},
            anchor={north east},
            font=\tiny
        ]
        \addplot[thick, blue] table [y=DeltaE1] {assets/graphs/data/Z3_gap_sec_1.csv}
            node [pos=0.8, black, above] {$\Delta E_1$};
        \addplot[thick, red]  table [y=DeltaE2] {assets/graphs/data/Z3_gap_sec_1.csv}
            node [pos=0.6, black, left=2pt] {$\Delta E_2$};
    \end{axis}

    \draw (wilson3 c1r1.east) node[above, rotate=-90] {sector $n=0$};
    \draw (wilson3 c1r2.east) node[above, rotate=-90] {sector $n=1, 2$};
    \draw (wilson3 c1r1.north) node[above, font=\normalsize] {$\Z_3$ Wilson loop};

\end{tikzpicture}

    \vspace*{-10pt}
    \caption[\ac{wl}s for the $\Z_3$ ladder \ac{lgt}]{$\Z_3$ \ac{wl} for the sectors $n=0$ (\emph{top}) and $n=1,2$ (\emph{bottom}, which are equivalent), for sizes $L = 7,9,11$ and $13$.
       Inset: energy differences $\Delta E_i = E_i - E_0$ for $i=1,2$, as a function of the coupling $\lambda$, in the sectors $n=1,2$, showing the emergence of a double-degenerate ground state for $\lambda > 1$.
}
    \label{fig:z3_wilson}
\end{figure}



\subsubsection*{Results for \texorpdfstring{$N=4$}{N=4}}%


The $\Z_4$ ladder \ac{lgt} have four super-selection sectors.
The behaviour of half-ladder \ac{wl}s as function of $\lambda$ is shown in Fig.~\ref{fig:z4_wilson}.
The Hamiltonian in the first sector, $n=0$, is
\begin{equation}
    \HamilDual[4]_{n=0}(\lambda^{-1}) = \HamilClock[4](\lambda^{-1}) - 2 \sum_{i} \qty(Z_i + Z_i^{\dagger}),
\end{equation}
As in the previous models, for $n=0$ we see a deconfined point at $\lambda = 0$, followed by a sharp transition to a confined phase.
Likewise, the lack of line crossings of the \ac{wl} at different $L$ suggests that in the limit $L \to \infty$ we will only have $\ev*{W} \neq 0$ for $\lambda = 0$.

The dual Hamiltonian of the sector $n=2$,
\begin{equation}
    \HamilDual[4]_{n=2}(\lambda^{-1}) = \HamilClock[4](\lambda^{-1}),
\end{equation}
has no longitudinal field, it is the only one to present a clear DCPT for $\lambda \approx 1$, as it is expected from the fact that the $4$-clock model is equivalent to two decoupled Ising chains \cite{ortiz2012dualities}.

In the two equivalent sectors $n=1$ and $3$, where the dual Hamiltonian is
\begin{equation}
    \HamilDual[4]_{n=1,3}(\lambda^{-1}) =
    \HamilClock[4](\lambda^{-1}) - 2 \cos( \frac{\pi}{4} ) \sum_{i} \qty(\tilde{Z}_i + \tilde{Z}_i^{\dagger}),
\end{equation}
the longitudinal field is non-zero and the \ac{wl} shows a peculiar behaviour, at least for the largest size ($L=10$) of the chain: it decreases fast as soon $\lambda > 0$, to stabilize to a finite value in the region $0.5 \lesssim \lambda \lesssim 1$, before tending to zero.
It is comparable to $\Z_3$ and $n=1,2$ situation, where a slight bump appear when the size $L$ is increased.
The characteristics of this phase (with the crossover region for $\Z_3$ and $n=1,2$) would deserve a deeper analysis, that we plan to do in a future work.
For $\lambda \gtrsim 1$, the system enters a deconfined phase with a double degenerate ground state, as for the $\Z_3$ model.


\begin{figure}[t]
    \centering
    \newcommand{\ZFourWilsonGraph}[1]{
    \nextgroupplot
    \addplot+ [thick] table [y=6x2]  {assets/graphs/data/Z4_wilson_#1.csv};
    \addplot+ [thick] table [y=8x2]  {assets/graphs/data/Z4_wilson_#1.csv};
    \addplot+ [thick] table [y=10x2] {assets/graphs/data/Z4_wilson_#1.csv};
}

\begin{tikzpicture}[
        font=\small,
        notes/.style={gray!20!black, font=\scriptsize}
    ]
    \begin{groupplot}[
            group style={
                group name=wilson4,
                group size=1 by 3,
                vertical sep=3pt,
                horizontal sep=3pt,
                x descriptions at=edge bottom,
                y descriptions at=edge left% ,
                % every plot/.style={thick}
            },
            width=7.5cm,
            height=5cm,
            no marks,
            table/col sep=comma,
            table/x=coupling,
            xtick pos=left,
            ytick pos=left,
            ylabel={$W$},
            xlabel={$\lambda$},
            try min ticks=5,
            cycle list name=exotic,
            legend style={font=\tiny, draw=gray!40},
            legend image post style={scale=0.5}
        ]
        % Sector n=0
        \ZFourWilsonGraph{00}
        \legend{{$L = 6$}, {$L = 8$}, {$L = 10$}}
        \draw (axis cs:0,1) node (deconf) [circle, fill=gray!40!black, inner sep=0pt, minimum size=4pt] {};
        \draw[<-, shorten <=2pt, gray!40!black] (deconf) -- +(0.35,-0.075) node [right, notes, align=left] {deconfined\\point};
        \draw[black] (axis cs:1.4,0) node [above, notes] {confined};

        % Sector n=1
        \ZFourWilsonGraph{10}
        \draw (axis cs:0,1) node (deconf) [circle, fill=gray!40!black, inner sep=0pt, minimum size=4pt] {};
        \draw[<-, shorten <=2pt, gray!40!black] (deconf) -- +(0.35,-0.075) node [right, notes, align=left] {deconfined\\point};
        \draw (axis cs:1.7,0) node [above=5pt, notes] {double degen.};

        % Sector n=2
        \ZFourWilsonGraph{20}
        \draw (axis cs:0.3,1) node [below=12pt, notes] {deconfined};
        \draw (axis cs:1.7,0) node [above=15pt, notes] {confined};
        \draw[dashed, gray] (axis cs:1.0,0) -- (axis cs:1.0,1);
    \end{groupplot}

    %
    % Energy gap
    %
    \begin{axis}[
            no marks,
            width=4cm,
            height=2.75cm,
            table/col sep=comma,
            table/x=coupling,
            xtick={0.5,0.75,1.0,1.25},
            xtick align=center,
            ytick align=center,
            xtick pos=left,
            ytick pos=left,
            ymax=2.5,
            at={(wilson4 c1r2.north east)},
            anchor={north east},
            font=\tiny
        ]
        \addplot[thick, blue] table [y=DeltaE1] {assets/graphs/data/Z4_gap_sec_1.csv}
        node [pos=0.8, black, above] {$\Delta E_1$};
        \addplot[thick, red]  table [y=DeltaE2] {assets/graphs/data/Z4_gap_sec_1.csv}
        node [pos=0.4, black, above] {$\Delta E_2$};
    \end{axis}

    \draw (wilson4 c1r1.east) node[above, rotate=-90] {sector $n=0$};
    \draw (wilson4 c1r2.east) node[above, rotate=-90] {sector $n=1, 3$};
    \draw (wilson4 c1r3.east) node[above, rotate=-90] {sector $n=2$};

    % title
    \draw (wilson4 c1r1.north) node[above, font=\normalsize] {$\Z_4$ Wilson loop};
\end{tikzpicture}

    \vspace*{-10pt}
    \caption[\ac{wl}s for the $\Z_4$ ladder \ac{lgt}]{$\Z_4$ \ac{wl} for sectors $n=0, \dots, 3$ and sizes $L=6, \dots, 10$.
        Only the sector $n = 2$ has a clear deconfined-confined phase transition, as expected from the duality with the $4$-clock model.
    }
    \label{fig:z4_wilson}
\end{figure}

\subsection{Distribution of the amplitudes in the ground state}
\label{sub:amplitudes_distribution}

In the $N=2$ case, we further differentiate the phase diagrams of the two sectors by looking at the ground state amplitudes distribution, for $\lambda<1$ and $\lambda>1$.
Obviously, the ground state can be written as a superposition of the gauge invariant states of $\Hphys$ in the given sector
\begin{equation}
    \ket{\Psi_{\text{g.s.} }}= \sum_n c_n \ket{n},
    \label{eq:gs_amplitudes}
\end{equation}
The basis $\ket{n}$ and the amplitudes $c_n$ are sorted in a decreasing order with respect to the modulus of the latter.
The first state of the list, with amplitude $c_1$, is always the Fock vacua $\ket{\Omega}$ of the sector, hence we consider the distribution of the ratios $\abs{c_n / c_1}$, which are plotted in Fig.~\ref{fig:gs_ampl_distr_0.1_Z2}--\ref{fig:gs_ampl_distr_1.5_Z2} for $\lambda=0.1$ and $\lambda=1.5$, respectively.
The most interesting one is at $\lambda = 0.1$, where the difference between the deconfined phase in the sector $(1,0)$ and the confined one in the sector $(0,0)$ can be seen.
In particular, in the deconfined phase the ground state is a superposition of deformations of the Fock vacuum, i.e~the non-contractible electric string, which can be thought as a \emph{kink condensate} \cite{fradkin1978order} (or as a paramagnetic phase), where each kink corresponds to a deformation of the string.
Meanwhile, for $\lambda > 1$, where we have confinement in both sectors, the ground state is essentially a product state, akin to a ferromagnetic state.
This is explained in Fig.~\ref{fig:gs_ampl_distr_0.1_Z2} and Fig.~\ref{fig:gs_ampl_distr_1.5_Z2}.


\begin{figure}[h]
    \centering
    \hspace{3em}$\Z_2$ g.s.~amplitudes distribution, $\lambda=1.5$\\[5pt]
    \begin{tikzpicture}[
    lattice/.style = {Gray, thin, solid},
    on/.style = {Green, very thick, solid},
    lab/.style = {scale=0.35},
    box/.style = {draw=black, dotted, inner sep=4pt},
    arr/.style = {<-, black},
    flux/.style = {fill=Green, fill opacity=0.1},
    font=\small,
    ]

\begin{axis}[
        height=5cm,
        width=8cm,
        tick align=outside,
        tick pos=left,
        xmajorticks=false,
        ylabel={$|c_n/c_1|$},
        ymin=0, ymax=1.1,
        ytick style={color=black},
        ytick={0,0.2,0.4,0.6,0.8,1,1.2},
        yticklabels={$0.0$, $0.2$, $0.4$, $0.6$, $0.8$, $1.0$, $1.2$}
]
\addplot [very thick, blue!70]
    table {%
    0 1
    1 0.0831366777420044
    12 0.0831366777420044
    13 0.00918400287628174
    24 0.00918400287628174
    25 0.00691556930541992
    78 0.0069117546081543
    79 0.0010453462600708
    186 0.000763535499572754
    189 0.00057530403137207
    199 0.000574946403503418
    };

% Title
\node [anchor=north east, draw=gray] at (axis description cs: 0.99, 0.99) (title) {sector $(0,0)$};

\node at (0, 1) (vacuum) {};
\end{axis}

% Vacuum
\draw[arr] (vacuum) node  [circle, fill=black, inner sep=0pt, minimum size=3pt] {}
    -- +(1,0) node (vacuum-label) {};
\draw (vacuum-label) node [right, box] {
    \begin{tikzpicture}[lab]
        \draw[lattice]  (-0.5, 0) grid (5.5, 1);
    \end{tikzpicture}
};

\end{tikzpicture}
\\[-2pt]\hspace{0.4pt}
    \begin{tikzpicture}[
    lattice/.style = {Gray, thin, solid},
    on/.style = {Green, very thick, solid},
    lab/.style = {scale=0.35},
    box/.style = {draw=black, dotted, inner sep=4pt},
    arr/.style = {<-, black},
    flux/.style = {fill=Green, fill opacity=0.1}
    ]

    \begin{axis}[
        height=7cm,
        width=9cm,
        tick align=outside,
        tick pos=left,
        % title={sector $(0, 0)$},
        xlabel={$n$},
        ylabel={$|c_n/c_1|$},
        ymin=0, ymax=1.1,
        ytick style={color=black},
        ytick={0,0.2,0.4,0.6,0.8,1,1.2},
        yticklabels={$0.0$, $0.2$, $0.4$, $0.6$, $0.8$, $1.0$, $1.2$}
        ]
        \addplot [very thick, blue!70]
        table {%
            0 1
            1 1
            2 0.172052025794983
            25 0.172052025794983
            26 0.0594711303710938
            49 0.0594711303710938
            50 0.0305156707763672
            73 0.0305156707763672
            74 0.0297093391418457
            157 0.0296221971511841
            158 0.0258673429489136
            181 0.0258673429489136
            182 0.0130573511123657
            199 0.0130573511123657
        };

        % Title
        \node [anchor=north east, draw=gray] at (axis description cs: 0.99, 0.99) (title) {sector $n=1$};

        \node at (0, 1) (vacuum) {};

    \end{axis}

    % Vacuum
    \draw[arr] (vacuum) node  [circle, fill=black, inner sep=0pt, minimum size=3pt] {}
    -- +(1,0) node (vacuum-label) {};
    \draw (vacuum-label) node [right, box] {
        \begin{tikzpicture}[lab]
            \draw[lattice]  (-0.5, 0) grid (5.5, 1);
            \draw[on] (-0.5,0) -- (5.5, 0);
        \end{tikzpicture}
    };

\end{tikzpicture}

    \caption[$\Z_2$ ground state amplitude distribution for $\lambda = 1.5$]{
    $\Z_2$ ground state amplitude distribution for $\lambda=1.5$ of the first 200 states and with lattice size $12 \times 2$.
    For both sectors $(0,0)$ (\emph{top}) and $(1,0)$ (\emph{bottom}) we are in a confined phase, which corresponds to a ferromagnetic phase in the Ising chain.
    Here we see a polarized state where the domain walls are suppressed and the ground state is essentially a product state.
    }
    \label{fig:gs_ampl_distr_1.5_Z2}
\end{figure}


\begin{figure}[h]
    \centering
    \hspace{3em}$\Z_2$ g.s.~amplitudes distribution, $\lambda=0.1$ \\[5pt]
    \begin{tikzpicture}[
    lattice/.style = {Gray, thin, solid},
    on/.style = {Green, very thick, solid},
    lab/.style = {scale=0.35},
    box/.style = {draw=black, dotted, inner sep=4pt},
    arr/.style = {<-, black},
    flux/.style = {fill=Green, fill opacity=0.1}
    ]

    \begin{axis}[
        height=7cm,
        width=9cm,
        tick align=outside,
        tick pos=left,
        xmajorticks=false,
        ylabel={$|c_n/c_1|$},
        ymin=0, ymax=1.1,
        ytick style={color=black},
        ytick={0,0.2,0.4,0.6,0.8,1,1.2},
        yticklabels={$0.0$, $0.2$, $0.4$, $0.6$, $0.8$, $1.0$, $1.2$}
        ]
        \addplot [very thick, blue!70]
        table {%
            0 1
            1 0.72670304775238
            12 0.72670304775238
            13 0.581751346588135
            24 0.581751346588135
            25 0.529128313064575
            36 0.529128313064575
            37 0.52813982963562
            78 0.52809739112854
            79 0.466958403587341
            90 0.466958403587341
            91 0.423632383346558
            114 0.423632383346558
            115 0.422796964645386
            186 0.422760725021362
            187 0.385271906852722
            198 0.385271906852722
            199 0.384550213813782
        };

        % Title
        \node [anchor=north east, draw=gray] at (axis description cs: 0.99, 0.99) (title) {sector $n=0$};

        \node at (0, 1) (vacuum) {};
        \node at (6, 0.72670304775238)  (1-1-loop) {};
        \node at (16, 0.581751346588135) (1-2-loop) {};
        \node at (50, 0.529128313064575) (2-1-loop) {};
        \node at (83, 0.466958403587341) (1-3-loop) {};
        \node at (130, 0.423632383346558) (1-2-1-1-loop) {};
        \node at (192, 0.385271906852722) (3-1-loop) {};

    \end{axis}

    % Vacuum
    \draw[arr] (vacuum) node  [circle, fill=black, inner sep=0pt, minimum size=3pt] {}
    -- +(1,0) node (vacuum-label) {};
    \draw (vacuum-label) node [right, box] {
        \begin{tikzpicture}[lab]
            \draw[lattice]  (-0.5, 0) grid (5.5, 1);
        \end{tikzpicture}
    };

    % 1 single plaquette loop
    \draw[arr] (1-1-loop) -- +(1,0.7) node (1-1-loop-label) {};
    \draw (1-1-loop-label) node [right, box] {
        \begin{tikzpicture}[lab]
            \draw[lattice]  (-0.5, 0) grid (5.5, 1);
            \fill[flux] (2,0) rectangle (3,1);
            \draw[on] (2,0) rectangle (3,1);
        \end{tikzpicture}
    };

    % 1 double plaquette loop
    \draw[arr] (1-2-loop) -- +(0.6,0.6) node (1-2-loop-label) {};
    \draw (1-2-loop-label) node [right, box] {
        \begin{tikzpicture}[lab]
            \draw[lattice]  (-0.5, 0) grid (5.5, 1);
            \fill[flux] (1,0) rectangle (3,1);
            \draw[on] (1,0) rectangle (3,1);
        \end{tikzpicture}
    };


    % 2 single plaquette loops
    \draw[arr] (2-1-loop) -- +(-0.2, -0.8) node (2-1-loop-label) {};
    \draw (2-1-loop-label) node [below, box] {
        \begin{tikzpicture}[lab]
            \draw[lattice]  (-0.5, 0) grid (5.5, 1);
            \fill[flux] (0,0) rectangle (1,1);
            \fill[flux] (3,0) rectangle (4,1);
            \draw[on] (0,0) rectangle (1,1);
            \draw[on] (3,0) rectangle (4,1);
        \end{tikzpicture}
    };

    % 1 triple plaquette loops
    \draw[arr] (1-3-loop) -- +(1,0.6) node (1-3-loop-label) {};
    \draw (1-3-loop-label) node [right, box] {
        \begin{tikzpicture}[lab]
            \draw[lattice]  (-0.5, 0) grid (5.5, 1);
            \fill[flux] (1,0) rectangle (4,1);
            \draw[on] (1,0) rectangle (4,1);
        \end{tikzpicture}
    };

    % 1 double 1 single plaquette loops
    \draw[arr] (1-2-1-1-loop) -- +(-0.1,-0.6) node (1-2-1-1-loop-label) {};
    \draw (1-2-1-1-loop-label) node [below, box] {
        \begin{tikzpicture}[lab]
            \draw[lattice]  (-0.5, 0) grid (5.5, 1);
            \fill[flux] (1,0) rectangle (2,1);
            \fill[flux] (3,0) rectangle (5,1);
            \draw[on] (1,0) rectangle (2,1);
            \draw[on] (3,0) rectangle (5,1);
        \end{tikzpicture}
    };

    % 3 single plaquette loops
    \draw[arr] (3-1-loop) -- +(-0.1,-1.2) node (3-1-loop-label) {};
    \draw (3-1-loop-label) node [below left, box] {
        \begin{tikzpicture}[lab]
            \draw[lattice]  (-0.5, 0) grid (5.5, 1);
            \fill[flux] (0,0) rectangle +(1,1);
            \fill[flux] (2,0) rectangle +(1,1);
            \fill[flux] (4,0) rectangle +(1,1);
            \draw[on] (0,0) rectangle +(1,1);
            \draw[on] (2,0) rectangle +(1,1);
            \draw[on] (4,0) rectangle +(1,1);
        \end{tikzpicture}
    };



\end{tikzpicture}
\\[-2pt]\hspace{0.4pt}
    \begin{tikzpicture}[
    lattice/.style = {Gray, thin, solid},
    on/.style = {Green, very thick, solid},
    lab/.style = {scale=0.35},
    box/.style = {draw=black, dotted, inner sep=4pt},
    arr/.style = {<-, black},
    flux/.style = {fill=Green, fill opacity=0.1}
    ]

    \begin{axis}[
        height=7cm,
        width=9cm,
        tick align=outside,
        tick pos=left,
        % title={sector $(0, 0)$},
        xlabel={$n$},
        ylabel={$|c_n/c_1|$},
        ymin=0, ymax=1.1,
        ytick style={color=black},
        ytick={0,0.2,0.4,0.6,0.8,1,1.2},
        yticklabels={$0.0$, $0.2$, $0.4$, $0.6$, $0.8$, $1.0$, $1.2$}
        ]
        \addplot [very thick, blue!70]
        table {%
            0,1
            1,1
            2,0.902501583099365
            25,0.902501583099365
            26,0.90012514591217
            133,0.900000095367432
            134,0.816542267799377
            157,0.816542267799377
            158,0.814608097076416
            199,0.814515113830566
        };

        % Title
        \node [anchor=north east, draw=gray] at (axis description cs: 0.99, 0.99) (title) {sector $n=1$};

        \node at (0.5, 1) (vacuum) {};
        \node at (10, 0.902501583099365) (1-1-loop) {};
        \node at (70, 0.900000095367432) (1-big-loop) {};
        \node at (160, 0.816542267799377) (2-1-loop) {};
    \end{axis}

    % Vacuum
    \draw[arr] (vacuum) node  [circle, fill=black, inner sep=0pt, minimum size=3pt] {}
    -- +(1,0) node (vacuum-label) {};
    \draw (vacuum-label) node [right, box] {
        \begin{tikzpicture}[lab]
            \draw[lattice]  (-0.5, 0) grid (5.5, 1);
            \draw[on] (-0.5,0) -- (5.5, 0);
        \end{tikzpicture}
    };

    % 1 single plaquette loop
    \draw[arr] (1-1-loop) -- +(0.5,-0.5) node (1-1-loop-label) {};
    \draw (1-1-loop-label) node [below, box] {
        \begin{tikzpicture}[lab]
            \draw[lattice]  (-0.5, 0) grid (5.5, 1);
            \fill[flux] (2,0) rectangle (3,1);
            \draw[on] (-0.5, 0) -- (2, 0) -- (2, 1) -- (3, 1) -- (3, 0) -- (5.5, 0);
        \end{tikzpicture}
    };



    % 1 big loop
    \draw[arr] (1-big-loop) -- +(0.0,-1.5) node (1-big-loop-label) {};
    \draw (1-big-loop-label) node [below, box] {
        \begin{tikzpicture}[lab]
            \draw[lattice]  (-0.5, 0) grid (5.5, 1);
            \draw[on] (-0.5, 0) -- (2, 0) -- (2, 1) -- (4, 1) -- (4, 0) -- (5.5, 0);
            \fill[flux] (2,0) rectangle (4,1);
            \begin{scope}[yshift=-1.5cm]
                \draw[lattice]  (-0.5, 0) grid (5.5, 1);
                \draw[on] (-0.5, 0) -- (1, 0) -- (1, 1) -- (4, 1) -- (4, 0) -- (5.5, 0);
                \fill[flux] (1,0) rectangle (4,1);
            \end{scope}
            \begin{scope}[yshift=-3.0cm]
                \draw[lattice]  (-0.5, 0) grid (5.5, 1);
                \draw[on] (-0.5, 0) -- (1, 0) -- (1, 1) -- (5, 1) -- (5, 0) -- (5.5, 0);
                \fill[flux] (1,0) rectangle (5,1);
            \end{scope}
            \node at (2.5, -3) [font=\footnotesize, inner sep=0pt] {$\vdots$};
        \end{tikzpicture}
    };


    % 2 single plaquette loops
    \draw[arr] (2-1-loop) -- +(-0.2,-1.0) node (2-1-loop-label) {};
    \draw (2-1-loop-label) node [below, box] {
        \begin{tikzpicture}[lab]
            \draw[lattice]  (-0.5, 0) grid (5.5, 1);
            \fill[flux] (1,0) rectangle (2,1);
            \fill[flux] (3,0) rectangle (4,1);
            \draw[on] (-0.5, 0) -- (1, 0) -- (1, 1) -- (2, 1) -- (2, 0) -- (3,0) -- (3,1) -- (4,1) -- (4,0) -- (5.5, 0);
        \end{tikzpicture}
    };


\end{tikzpicture}

    \caption[$\Z_2$ ground state amplitude distribution for $\lambda = 0.1$]{
        $\Z_2$ ground state amplitude distribution for $\lambda=0.1$ of the first 200 states and with lattice size $12 \times 2$.
        \emph{Top}: distribution of the ratios $|{c_n/c_1}|$ for the sector $(0,0)$ (see \eqref{eq:gs_amplitudes}).
        We see that the heaviest states that enters the ground state, apart from the vacuum that sets the scale, are made of small electric loops, typical of a confined phase.
        \emph{Bottom}: the same distribution of ratios for the sector $(1,0)$.
        We see that the heaviest states are made of bigger and bigger deformations of the electric string that goes around the ladder.
        This happens because the energy contributions depends only on the domain walls between two plaquettes with different flux content.
        This behaviour is similar to the so-called \emph{kink condensation} in spin chains \cite{fradkin1978order}, where the disordered state can be expressed as a superposition of all possible configuration of kinks (i.e.~domain walls between two differently ordered regions).
        In the language of the duality, this deconfined phase then maps to the paramagnetic phase of the quantum Ising model with \emph{only} transverse field.
    }
    \label{fig:gs_ampl_distr_0.1_Z2}
\end{figure}




