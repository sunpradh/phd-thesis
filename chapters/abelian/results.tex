\section{A case study: \texorpdfstring{$N=2, 3$}{N=2, 3} and \texorpdfstring{$4$}{4}}
\label{sec:a_case_study_N_2_3_4}

\newcommand{\ZTwoWilsonGraph}[1]{
        \nextgroupplot
        \addplot+[thick] table [y=10x2] {assets/graphs/data/Z2_wilson_#1.csv};
        \addplot+[thick] table [y=12x2] {assets/graphs/data/Z2_wilson_#1.csv};
        \addplot+[thick] table [y=14x2] {assets/graphs/data/Z2_wilson_#1.csv};
        \addplot+[thick] table [y=16x2] {assets/graphs/data/Z2_wilson_#1.csv};
        \addplot+[thick] table [y=18x2] {assets/graphs/data/Z2_wilson_#1.csv};
}
\newcommand{\ZThreeWilsonGraph}[1]{
        \nextgroupplot
        \addplot+[thick] table [y=7x2]  {assets/graphs/data/Z3_wilson_#1.csv};
        \addplot+[thick] table [y=9x2]  {assets/graphs/data/Z3_wilson_#1.csv};
        \addplot+[thick] table [y=11x2] {assets/graphs/data/Z3_wilson_#1.csv};
        \addplot+[thick] table [y=13x2] {assets/graphs/data/Z3_wilson_#1.csv};
}
\newcommand{\ZFourWilsonGraph}[1]{
    \nextgroupplot
    \addplot+ [thick] table [y=6x2]  {assets/graphs/data/Z4_wilson_#1.csv};
    \addplot+ [thick] table [y=8x2]  {assets/graphs/data/Z4_wilson_#1.csv};
    \addplot+ [thick] table [y=10x2] {assets/graphs/data/Z4_wilson_#1.csv};
}



\subsection{Investigating the phase diagram}%
\label{sub:investigating_the_phase_diagram}

We wish to study the phase diagram of the $\Z_N$ \ac{lgt} phase diagram, in particular we are interested in any \emph{confined} or \emph{deconfined} phase.
In a pure gauge theory, these phases are investigated non-local order parameters like the \emph{\ac{wl}} (not be confused with the non-contractible \ac{wl}s in \eqref{eq:nonlocal_op_ZN}) or \emph{string tension}.
This is because we expect the deconfined phase to be a topological phase, which can be investigated only with non-local order parameters.

Given a closed region $\mathcal{R}$, a \ac{wl} operator $W_{\mathcal{R}}$ is defined as
\begin{equation}
    W_{\mathcal{R}} = \prod_{\square \in \mathcal{R}} U_{\square}.
    \label{eq:closed_wilson_loop}
\end{equation}
Alternatively, considering the oriented boundary $\partial \mathcal{R}$ one can write
\begin{equation}
    W_{\mathcal{R}} = \prod_{\ell \in \partial \mathcal{R}} U_{\ell},
\end{equation}
where the Hermitian conjugate is implied everytime we move in the negative directions.
It is also implied that the curve $\partial \mathcal{R}$ is a contractible loop.
Wilson showed in \cite{wilson1974confinement} that quark confinement is related to the expectation value $\ev{W_{\mathcal{R}}}$ of a \ac{wl}, which can be thought as a gauge field average on a region.
In particular, in the presence of quark confinement the gauge field average follows an \emph{area law}, where it decays exponentially with the area enclosed by $\mathcal{R}$.
On the other hand, in the deconfined phase we have a \emph{perimeter law}, where the gauge field average decays exponentially with the perimeter of $\mathcal{R}$.

Unfortunately on a ladder geometry there is not much difference between the area and the perimeter of a \ac{wl}.
In fact, in units of the lattice spacing, the area of a \ac{wl} over $n$ plaquettes is $n$ while its perimeter is just $2n+2$.
They both grow linearly.
Nonetheless, we can still look at the behaviour of the \ac{wl}, for a fixed length, at different couplings $\lambda$.

When the coupling $\lambda$ in \eqref{eq:hamiltonian_base} is equal to zero, the \ac{tc} is recovered and in any of its topological sector the ground state is the equal superposition of all the states with any number of closed electrical loops, in a similar fashion to coherent states.
This makes the \ac{tc} a \emph{quantum loop gas}, which is a \emph{deconfined phase}.
Furthermore, the operator $W_{\mathcal{R}}$ in \eqref{eq:closed_wilson_loop} creates an electrical loop around the region $\mathcal{R}$.
From the constraints
% TODO aggiustare ref
% \eqref{eq:constraints_gs_toric_code},
it can easily be proved that $W_{\mathcal{R}}$ leaves the \ac{tc} ground states unchanged, showing in fact that they behaves as coherent states, which leads to $\ev{W_{\mathcal{R}}} = 1$.

Therefore, $\ev{W_{\mathcal{R}}} \approx 1$ signals a deconfined phase and on the other hand a vanishing $\ev{W_{\mathcal{R}}} \approx 0$ corresponds to confined phase.
For this reason, even tough we lack an area/perimeter law on the ladder geometry it is still sensible to look at the behaviour of the \ac{wl}.


Another possible approach for investigating the phase diagram is to use the \emph{string tension}.
In two dimensions, given an \emph{open} curve $\tilde{\mathcal{C}}$ on the dual lattice $\tilde{\mathbb{L}}$ we can construct an open \ac{ths} operator $S_{\tilde{\mathcal{C}}}$ as
\begin{equation}
    S_{\tilde{\mathcal{C}}} = \prod_{\ell \in \tilde{\mathcal{C}}} V_{\ell}
\end{equation}
with the usual caveat: we have to take the Hermitian conjugate everytime the path goes in the negative direction.
Then the string tension is just the expectation value $\ev{S_{\tilde{\mathcal{C}}}}$ it is called in this way because it related to the potential energy (tension) between two magnetic fluxes created at the ends of the curve $\tilde{\mathcal{C}}$.
Henceforth, in a deconfined phase $\ev{S_{\tilde{\mathcal{C}}}} \approx 0$, which means that the magnetic fluxes can be moved freely with no cost in energy, like in the \ac{tc}.




\input{chapters/abelian/numerical.tex}



We wish to investigate the presence of a \emph{deconfined-confined phase transition} (DCPT) for a given $\Z_N$ ladder LGT.
In a pure gauge theory, these phases can be detected
with the perimeter/area law for Wilson loops \cite{wilson1974confinement},
which can be expressed as the products of magnetic operators over a given region.
Unfortunately, in a  ladder geometry there is not much difference between the area and the perimeter of a loop, since they both grow linearly in the size system $L$.

Nonetheless, we expect a phase transition by varying $\lambda$ \cite{trebst2007topological, hamma2008adiabatic, tagliacozzo2011entanglement} that can still be captured by an operator like $W_{\mathcal{R}}= \prod_{i \in \mathcal{R}} U_{i}$, the product of magnetic operators $U$'s over a (connected) region $\mathcal{R}$.
Indeed, when $\lambda=0$, the Hamiltonian \eqref{eq:ladder_hamiltonian} is analogous to a Toric Code \cite{kitaev2003fault} which is known to be in a deconfined phase, where the (topologically distinct) ground states are obtained as uniform superpositions of the gauge-invariant states, i.e.~closed electric loops.
On these ground states $\ev{W_{\mathcal{R}}} = 1$, hence
 a value $\ev{W_{\mathcal{R}}} \approx 1$ signals a deconfined phase.
On the other hand, when $\lambda \rightarrow \infty$, the electric loops are suppressed, hence
$\ev{W_{\mathcal{R}}} \approx 0$, signalling a confined phase.

In the dual clock model picture, the Wilson loop translates to a disorder operator \cite{fradkin1978order}, which means that a deconfined phase can be thought of as a paramagnetic (or disordered) phase, while the confined phase is like a ferromagnetic (or ordered) phase.
Moreover, the longitudinal field breaks the $N$-fold symmetry of the ferromagnetic phase into a one-fold or two-fold degeneracy, depending on the parity ($n$ even/odd) of the superselection sector.

We study the $\Z_N$ LGT on a ladder numerically through \emph{exact diagonalization}, by evaluating the half-ladder Wilson loop, i.e.~
\begin{equation}
    W = U_1 U_2 \cdots U_{L/2},
\end{equation}
and working in the restricted physical Hilbert space $\Hphys^{(n)}$ ($n=0,\dots,N-1$), which has dimension $N^L$, much smaller than $N^{3L}$ (the dimension of the total Hilbert space).

The naive and brute-force method for building $\Hphys$ would require checking the Gauss law at every site (which are $O(3L)$ operations) for all the possible $N^{3L}$ candidate states.
On the other hand, the gauge-reducing duality to clock models provides a faithful and efficient method for building the $N^{L+1}$ basis states of $\Hphys$, yielding a major speedup with respect to the naive method.
%(see \cite{Note1}).
The procedure is quite simple and it consists in treating a clock state as a plaquette flux state in the following way.
Let $\ket{\Omega_0}$ be the vacuum state where all the links are in the $\ket{0}$ state.
For each sector $n$ we can build a ``vacuum'' state $\ket{\Omega_n}$ by applying $\overline{W}$ in \eqref{eq:nonlocal_op_ZN} $n$ times on the true vacuum, i.e.~$\ket{\Omega_n} = \overline{W}^{n} \ket{\Omega_0}$.
Then, let $\ket{s_1 s_2 \cdots}$ be a configuration of the dual $N$-clock model, where $s_i = 0, \dots, N-1$.
Now, the equivalent ladder state in the $n$-th sector can be obtained with $\prod_{i} U_i^{s_i} \ket{\Omega_n}$.

In the following, we present the results with $N=2,3$ and $4$, for different lengths.



\begin{figure}[t]
    \centering
    \begin{tikzpicture}[
        % scale=0.8,
        font=\small,
        notes/.style={gray!20!black, font=\scriptsize}
    ]
    \begin{groupplot}[
        group style={
            group name=wilson2,
            group size=1 by 2,
            vertical sep=3pt,
            x descriptions at=edge bottom%,
            % every plot/.style={thick}
        },
        width=7.5cm,
        height=5cm,
        no marks,
        table/col sep=comma,
        table/x=coupling,
        xtick pos=left,
        ytick pos=left,
        ylabel={$W$},
        xlabel={$\lambda$},
        try min ticks=5,
        cycle list name=exotic,
        legend style={font=\tiny, draw=gray!40},
        legend image post style={scale=0.5}
        ]
        % Sector n=0
        \ZTwoWilsonGraph{00}
        \legend{
            {$L = 10$},
            {$L = 12$},
            {$L = 14$},
            {$L = 16$},
            {$L = 18$}
        }
        % notes on the graphs
        \draw (axis cs:0,1) node (deconf) [circle, fill=gray!40!black, inner sep=0pt, minimum size=4pt] {};
        \draw[<-, shorten <=2pt, gray!40!black] (deconf) -- +(0.3,0) node [right, notes] {deconfined point};
        \draw[black] (axis cs:1,0) node [above, notes] {confined};

        % Sector n=1
        \ZTwoWilsonGraph{10}
        % notes on the graphs
        \draw (axis cs:0.3,0) node [above=10pt, notes] {deconfined};
        \draw (axis cs:1.7,0) node [above=10pt, notes] {confined};
        \draw[dashed, gray] (axis cs:1,0) -- (axis cs:1,1);
    \end{groupplot}
    \draw (wilson2 c1r1.east) node[above, rotate=-90] {sector $n=0$};
    \draw (wilson2 c1r2.east) node[above, rotate=-90] {sector $n=1$};
    \draw (wilson2 c1r1.north) node[above, font=\normalsize] {$\Z_2$ Wilson loop};
\end{tikzpicture}

    \vspace*{-10pt}
    \caption{$\Z_2$ Wilson loop in the sectors $n=0$ (\emph{top}) and $n=1$ (\emph{bottom}), for sizes $L=10,12, \dots,18$.
    The sector $n=0$ presents only a deconfined point at $\lambda=0$ and then decays rapidly into a confined phase, while the sector $n=1$ has a phase transition for $\lambda \simeq 1$.
    }
    \label{fig:z2_wilson}
\end{figure}

\smallskip

% \begin{figure*}[t]
%     \centering
%     \begin{tikzpicture}[
        scale=0.8,
        font=\footnotesize,
        notes/.style = {gray!20!black, font=\scriptsize, align=left}
        ]
    %============================================================
    % Z2 Wilson loop
    %============================================================
    \begin{groupplot}[
        group style={
            group name=wilson2,
            group size=1 by 2,
            vertical sep=3pt,
            x descriptions at=edge bottom%,
            % every plot/.style={thick}
        },
        width=6.5cm,
        height=4.5cm,
        no marks,
        table/col sep=comma,
        table/x=coupling,
        xtick pos=left,
        ytick pos=left,
        ylabel={$W$},
        xlabel={$\lambda$},
        try min ticks=5,
        cycle list name=exotic,
        legend style={font=\tiny, draw=gray!40},
        legend image post style={scale=0.5}
        ]
        % Sector n=0
        \ZTwoWilsonGraph{00}
        \legend{
            {$L = 10$},
            {$L = 12$},
            {$L = 14$},
            {$L = 16$},
            {$L = 18$}
        }
        % notes on the graphs
        \draw (axis cs:0,1) node (deconf) [circle, fill=gray!40!black, inner sep=0pt, minimum size=4pt] {};
        \draw[<-, shorten <=2pt, gray!40!black] (deconf) -- +(0.3,-0.1) node [right, notes, anchor=west] {deconfined\\point};
        \draw[black] (axis cs:1,0) node [above, notes] {confined};

        % Sector n=1
        \ZTwoWilsonGraph{10}
        % notes on the graphs
        \draw (axis cs:0.3,0) node [above=10pt, notes] {deconfined};
        \draw (axis cs:1.7,0) node [above=10pt, notes] {confined};
        \draw[dashed, gray] (axis cs:1,0) -- (axis cs:1,1);
    \end{groupplot}
    % Sector labels
    \draw (wilson2 c1r1.east) node[above, rotate=-90] {$n=0$};
    \draw (wilson2 c1r2.east) node[above, rotate=-90] {$n=1$};
    \draw (wilson2 c1r1.north) node[above, font=\normalsize] {\textbf{(a)} $\Z_2$ Wilson loop};


    %============================================================
    % Z3 Wilson loop
    %============================================================
    \begin{scope}[xshift=6.5cm]
        \begin{groupplot}[
                group style={
                    group name=wilson3,
                    group size=1 by 2,
                    vertical sep=3pt,
                    x descriptions at=edge bottom%,
                    % every plot/.style={thick}
                },
                width=6.5cm,
                height=4.5cm,
                no marks,
                table/col sep=comma,
                table/x=coupling,
                xtick pos=left,
                ytick pos=left,
                ylabel={$W$},
                xlabel={$\lambda$},
                try min ticks=5,
                cycle list name=exotic,
                legend style={font=\tiny, draw=gray!40},
                legend image post style={scale=0.5}
            ]

            % Sector n=0
            \ZThreeWilsonGraph{00}
            \legend{
                {$L = 7$},
                {$L = 9$},
                {$L = 11$},
                {$L = 13$}
            }
            \draw (axis cs:0,1) node (deconf) [circle, fill=gray!40!black, inner sep=0pt, minimum size=4pt] {};
            \draw[<-, shorten <=2pt, gray!40!black] (deconf) -- +(0.3,-0.07) node [right, notes, align=left, anchor=west] {deconfined\\point};
            \draw[black] (axis cs:1.2,0) node [above, notes] {confined};

            % Sector n=1
            \ZThreeWilsonGraph{10}
            \draw (axis cs:0,1) node (deconf) [circle, fill=gray!40!black, inner sep=0pt, minimum size=4pt] {};
            \draw[<-, shorten <=2pt, gray!40!black] (deconf) -- +(0.15,-0.4) node [below, notes,align=center] {deconfined\\point};
            \draw (axis cs:1.65,0) node [above=3pt, notes] {double degeneracy};
            \draw (axis cs:0.65,0.5) node [above, rotate=-45, notes] {crossover};

        \end{groupplot}

        \begin{axis}[
                no marks,
                width=3.5cm,
                height=2.75cm,
                table/col sep=comma,
                table/x=coupling,
                xtick={0.5,0.75,1.0,1.25},
                xtick align=center,
                ytick align=center,
                xtick pos=left,
                ytick pos=left,
                ymax=3.5,
                at={(wilson3 c1r2.north east)},
                anchor={north east},
                font=\tiny
            ]
            \addplot[thick, blue] table [y=DeltaE1] {graphs/data/Z3_gap_sec_1.csv}
            node [pos=0.8, black, above] {$\Delta E_1$};
            \addplot[thick, red]  table [y=DeltaE2] {graphs/data/Z3_gap_sec_1.csv}
            node [pos=0.6, black, left=2pt] {$\Delta E_2$};
        \end{axis}
        \draw (wilson3 c1r1.east) node[above, rotate=-90] {$n=0$};
        \draw (wilson3 c1r2.east) node[above, rotate=-90] {$n=1, 2$};
        \draw (wilson3 c1r1.north) node[above, font=\normalsize] {\textbf{(b)} $\Z_3$ Wilson loop};
    \end{scope}



    %============================================================
    % Z4 Wilson loop
    %============================================================
    \begin{scope}[xshift=13cm, yshift=1cm]
        \begin{groupplot}[
                group style={
                    group name=wilson4,
                    group size=1 by 3,
                    vertical sep=3pt,
                    horizontal sep=3pt,
                    x descriptions at=edge bottom,
                    y descriptions at=edge left% ,
                    % every plot/.style={thick}
                },
                width=6.5cm,
                height=3.5cm,
                no marks,
                table/col sep=comma,
                table/x=coupling,
                xtick pos=left,
                ytick pos=left,
                ylabel={$W$},
                xlabel={$\lambda$},
                try min ticks=5,
                cycle list name=exotic,
                legend style={font=\tiny, draw=gray!40},
                legend image post style={scale=0.5}
            ]
            % Sector n=0
            \ZFourWilsonGraph{00}
            \legend{{$L = 6$}, {$L = 8$}, {$L = 10$}}
            \draw (axis cs:0,1) node (deconf) [circle, fill=gray!40!black, inner sep=0pt, minimum size=4pt] {};
            \draw[<-, shorten <=2pt, gray!40!black] (deconf) -- +(0.35,-0.075) node [right, notes, align=left] {deconfined\\point};
            \draw[black] (axis cs:1.4,0) node [above, notes] {confined};

            % Sector n=1
            \ZFourWilsonGraph{10}
            \draw (axis cs:0,1) node (deconf) [circle, fill=gray!40!black, inner sep=0pt, minimum size=4pt] {};
            \draw[<-, shorten <=2pt, gray!40!black] (deconf) -- +(0.35,-0.075) node [right, notes, align=left] {deconf.\\point};
            \draw (axis cs:1.7,0) node [above=3pt, notes] {double degen.};

            % Sector n=2
            \ZFourWilsonGraph{20}
            \draw (axis cs:0.3,1) node [below=12pt, notes] {deconfined};
            \draw (axis cs:1.7,0) node [above=15pt, notes] {confined};
            \draw[dashed, gray] (axis cs:1.0,0) -- (axis cs:1.0,1);
    \end{groupplot}


    \begin{axis}[
            no marks,
            width=4cm,
            height=2.5cm,
            table/col sep=comma,
            table/x=coupling,
            xtick={0.5,0.75,1.0,1.25},
            xtick align=center,
            ytick align=center,
            xtick pos=left,
            ytick pos=left,
            ymax=2.5,
            at={(wilson4 c1r2.north east)},
            anchor={north east},
            font=\tiny
        ]
        \addplot[thick, blue] table [y=DeltaE1] {graphs/data/Z4_gap_sec_1.csv}
        node [pos=0.8, black, above] {$\Delta E_1$};
        \addplot[thick, red]  table [y=DeltaE2] {graphs/data/Z4_gap_sec_1.csv}
        node [pos=0.4, black, above] {$\Delta E_2$};
    \end{axis}

    \draw (wilson4 c1r1.east) node[above, rotate=-90] {$n=0$};
    \draw (wilson4 c1r2.east) node[above, rotate=-90] {$n=1, 3$};
    \draw (wilson4 c1r3.east) node[above, rotate=-90] {$n=2$};

    % title
    \draw (wilson4 c1r1.north) node[above, font=\normalsize] {\textbf{(c)} $\Z_4$ Wilson loop};
    \end{scope}
\end{tikzpicture}

%     \vspace*{-10pt}
%     \caption{
%         \textbf{(a)} $\Z_2$ Wilson loop in the sectors $n=0$ (\emph{top}) and $n=1$ (\emph{bottom}), for sizes $L=10,12, \dots,18$.
%         The sector $n=0$ presents only a deconfined point at $\lambda=0$ and then decays rapidly into a confined phase, while the sector $n=1$ has a phase transition for $\lambda \simeq 1$.
%         \textbf{(b)} $\Z_3$ Wilson loop for the sectors $n=0$ (\emph{top}) and $n=1,2$ (\emph{bottom}, which are equivalent), for sizes $L = 7,9,11$ and $13$.
%         Inset: energy differences $\Delta E_i = E_i - E_0$ for $i=1,2$, as a function of the coupling $\lambda$, in the sectors $n=1,2$, showing the emergence of a double-degenerate ground state for $\lambda > 1$.
%         \textbf{(c)} $\Z_4$ Wilson loop for sectors $n=0, \dots, 3$ and sizes $L=6, \dots, 10$.
%         Only the sector $n = 2$ has a clear deconfined-confined phase transition, as expected from the duality with the $4$-clock model.
%     }
%     \label{fig:wilson_loops}
% \end{figure*}



\paragraph{Results for \texorpdfstring{$N=2$}{N=2}}
As a warm up, we consider the $\Z_2$ ladder LGT, with lengths $L=10,12,\dots,18$.
This model is equivalent to a $p=2$ clock model, which is just the quantum Ising chain, with only two superselection sectors for $n=0$ and $n=1$.
When $n=1$, the Hamiltonian  contains only the transverse filed and is integrable \cite{baxter1982exactlysm}.
Thus, we expect a critical point for $\lambda \simeq 1$, which will be a DCPT in the gauge model language.
This is clearly seen in the behaviour of the half-ladder Wilson loop, as shown in the lower panel of Fig.~\ref{fig:z2_wilson}.
For $n=0$, both the transverse and longitudinal fields  are present, the model is no longer integrable  \cite{banuls2011thermalization, kormos2017confinement, pomponio2022bloch} and we expect to always see a confined phase, except for $\lambda = 0$.
This is indeed confirmed by the behaviour of the half-ladder Wilson loop shown in the upper panel of Fig.~\ref{fig:z2_wilson}.

We can further characterize the phases of the two sectors by looking at the structure of the ground state, for $\lambda<1$ and $\lambda>1$, which is possible thanks to the exact diagonalization.
In particular, in the deconfined phase of the sector $n=1$, the ground state is a superposition of the deformations of the non-contractible electric string that makes the $n=1$ vacuum $\ket{\Omega_1}$.
For this reason, this phase can be thought as a \emph{kink condensate} \cite{fradkin1978order} (which is equivalent to a paramagnetic phase), where each kink corresponds to a deformation of the string.
Instead, for $\lambda > 1$, where we have confinement (as in the $n=0$ sector), the ground state is essentially a product state, akin to a ferromagnetic state. %(see \cite{Note1})


\begin{figure}[t]
    \centering
    \newcommand{\ZThreeWilsonGraph}[1]{
        \nextgroupplot
        \addplot+[thick] table [y=7x2]  {assets/graphs/data/Z3_wilson_#1.csv};
        \addplot+[thick] table [y=9x2]  {assets/graphs/data/Z3_wilson_#1.csv};
        \addplot+[thick] table [y=11x2] {assets/graphs/data/Z3_wilson_#1.csv};
        \addplot+[thick] table [y=13x2] {assets/graphs/data/Z3_wilson_#1.csv};
}

\begin{tikzpicture}[
        notes/.style={gray!20!black, font=\scriptsize},
        font=\small
    ]
    \begin{groupplot}[
        group style={
            group name=wilson3,
            group size=1 by 2,
            vertical sep=3pt,
            x descriptions at=edge bottom%,
            % every plot/.style={thick}
        },
        width=7.5cm,
        height=5cm,
        no marks,
        table/col sep=comma,
        table/x=coupling,
        xtick pos=left,
        ytick pos=left,
        ylabel={$W$},
        xlabel={$\lambda$},
        try min ticks=5,
        cycle list name=exotic,
        legend style={font=\tiny, draw=gray!40},
        legend image post style={scale=0.5}
        ]

        % Sector n=0
        \ZThreeWilsonGraph{00}
        \legend{
            {$L = 7$},
            {$L = 9$},
            {$L = 11$},
            {$L = 13$}
        }
        \draw (axis cs:0,1) node (deconf) [circle, fill=gray!40!black, inner sep=0pt, minimum size=4pt] {};
        \draw[<-, shorten <=2pt, gray!40!black] (deconf) -- +(0.3,-0.07) node [right, notes, align=left, anchor=west] {deconfined\\point};
        \draw[black] (axis cs:1.2,0) node [above, notes] {confined};

        % Sector n=1
        \ZThreeWilsonGraph{10}
        \draw (axis cs:0,1) node (deconf) [circle, fill=gray!40!black, inner sep=0pt, minimum size=4pt] {};
        \draw[<-, shorten <=2pt, gray!40!black] (deconf) -- +(0.15,-0.4) node [below, notes,align=center] {deconfined\\point};
        \draw (axis cs:1.65,0) node [above=3pt, notes] {double degeneracy};
        \draw (axis cs:0.65,0.5) node [above, rotate=-50, notes] {crossover};

    \end{groupplot}

    \begin{axis}[
            no marks,
            width=4cm,
            height=2.75cm,
            table/col sep=comma,
            table/x=coupling,
            xtick={0.5,0.75,1.0,1.25},
            xtick align=center,
            ytick align=center,
            xtick pos=left,
            ytick pos=left,
            ymax=3.5,
            at={(wilson3 c1r2.north east)},
            anchor={north east},
            font=\tiny
        ]
        \addplot[thick, blue] table [y=DeltaE1] {assets/graphs/data/Z3_gap_sec_1.csv}
            node [pos=0.8, black, above] {$\Delta E_1$};
        \addplot[thick, red]  table [y=DeltaE2] {assets/graphs/data/Z3_gap_sec_1.csv}
            node [pos=0.6, black, left=2pt] {$\Delta E_2$};
    \end{axis}

    \draw (wilson3 c1r1.east) node[above, rotate=-90] {sector $n=0$};
    \draw (wilson3 c1r2.east) node[above, rotate=-90] {sector $n=1, 2$};
    \draw (wilson3 c1r1.north) node[above, font=\normalsize] {$\Z_3$ Wilson loop};

\end{tikzpicture}

    \vspace*{-10pt}
    \caption{$\Z_3$ Wilson loop for the sectors $n=0$ (\emph{top}) and $n=1,2$ (\emph{bottom}, which are equivalent), for sizes $L = 7,9,11$ and $13$.
       Inset: energy differences $\Delta E_i = E_i - E_0$ for $i=1,2$, as a function of the coupling $\lambda$, in the sectors $n=1,2$, showing the emergence of a double-degenerate ground state for $\lambda > 1$.
}
    \label{fig:z3_wilson}
\end{figure}


\smallskip

\paragraph{Results for \texorpdfstring{$N=3$}{N=3}}%
The $\Z_3$ LGT is studied for lengths $L=7,9,11$ and $13$.
This model can be mapped to a $3$-clock model, which is equivalent to a $3$-state quantum Potts model with a longitudinal field, which is present in all sectors, as one can see from \eqref{eq:dual_ladder_hamiltonian_real}.
This field is expected to disrupt any ordered state and thus it is not possible to observe a phase transition, as it is confirmed  by the behaviour of the half-ladder Wilson loops  shown in Fig.~\ref{fig:z3_wilson}.
In addition, all the sectors present a deconfined point at $\lambda = 0$.
In the case $n=0$, for $\lambda > 0$ we recognize a quick transition to a confined phase, similar to what happens in \cite{burrello2021ladder}.
While for $n=1$ and $2$ (which are equivalent), the model exhibits a smoother \emph{crossover} to an ordered phase characterized by a doubly-degenerate ground state, for $\lambda > 1$.
Notice that, as discussed above,  the presence of the ``skew'' longitudinal field breaks the three-fold degeneracy expected in the ordered phase of the $3$-clock model into a two-fold degeneracy only.

\smallskip

\paragraph{Results for \texorpdfstring{$N=4$}{N=4}}%
The $\Z_4$ ladder LGT have four superselection sectors.
The behaviour of half-ladder Wilson loops as function of $\lambda$ is shown in Fig.~\ref{fig:z4_wilson}.
As in the previous models, for $n=0$ we see a deconfined point at $\lambda = 0$, followed by a sharp transition to a confined phase.
The sector $n=2$, which has no longitudinal field, is the only one to present a clear DCPT for $\lambda \approx 1$, as it is expected from the fact that the $4$-clock model is equivalent to two decoupled Ising chains \cite{ortiz2012dualities}.
In the two equivalent sectors $n=1$ and $3$, where the longitudinal field is complex, the Wilson loop shows a peculiar behaviour, at least for the largest size ($L=10$) of the chain: it decreases fast as soon $\lambda > 0$, to stabilize to a finite value in the region $0.5 \lesssim \lambda \lesssim 1$, before tending to zero.
The characteristics of this phase would deserve a deeper analysis, that we plan to do in a future work.
For $\lambda \gtrsim 1$, the system enters a deconfined phase with a double degenerate ground state, as for the $\Z_3$ model.


\begin{figure}[t]
    \centering
    \newcommand{\ZFourWilsonGraph}[1]{
    \nextgroupplot
    \addplot+ [thick] table [y=6x2]  {assets/graphs/data/Z4_wilson_#1.csv};
    \addplot+ [thick] table [y=8x2]  {assets/graphs/data/Z4_wilson_#1.csv};
    \addplot+ [thick] table [y=10x2] {assets/graphs/data/Z4_wilson_#1.csv};
}

\begin{tikzpicture}[
        font=\small,
        notes/.style={gray!20!black, font=\scriptsize}
    ]
    \begin{groupplot}[
            group style={
                group name=wilson4,
                group size=1 by 3,
                vertical sep=3pt,
                horizontal sep=3pt,
                x descriptions at=edge bottom,
                y descriptions at=edge left% ,
                % every plot/.style={thick}
            },
            width=7.5cm,
            height=5cm,
            no marks,
            table/col sep=comma,
            table/x=coupling,
            xtick pos=left,
            ytick pos=left,
            ylabel={$W$},
            xlabel={$\lambda$},
            try min ticks=5,
            cycle list name=exotic,
            legend style={font=\tiny, draw=gray!40},
            legend image post style={scale=0.5}
        ]
        % Sector n=0
        \ZFourWilsonGraph{00}
        \legend{{$L = 6$}, {$L = 8$}, {$L = 10$}}
        \draw (axis cs:0,1) node (deconf) [circle, fill=gray!40!black, inner sep=0pt, minimum size=4pt] {};
        \draw[<-, shorten <=2pt, gray!40!black] (deconf) -- +(0.35,-0.075) node [right, notes, align=left] {deconfined\\point};
        \draw[black] (axis cs:1.4,0) node [above, notes] {confined};

        % Sector n=1
        \ZFourWilsonGraph{10}
        \draw (axis cs:0,1) node (deconf) [circle, fill=gray!40!black, inner sep=0pt, minimum size=4pt] {};
        \draw[<-, shorten <=2pt, gray!40!black] (deconf) -- +(0.35,-0.075) node [right, notes, align=left] {deconfined\\point};
        \draw (axis cs:1.7,0) node [above=5pt, notes] {double degen.};

        % Sector n=2
        \ZFourWilsonGraph{20}
        \draw (axis cs:0.3,1) node [below=12pt, notes] {deconfined};
        \draw (axis cs:1.7,0) node [above=15pt, notes] {confined};
        \draw[dashed, gray] (axis cs:1.0,0) -- (axis cs:1.0,1);
    \end{groupplot}

    %
    % Energy gap
    %
    \begin{axis}[
            no marks,
            width=4cm,
            height=2.75cm,
            table/col sep=comma,
            table/x=coupling,
            xtick={0.5,0.75,1.0,1.25},
            xtick align=center,
            ytick align=center,
            xtick pos=left,
            ytick pos=left,
            ymax=2.5,
            at={(wilson4 c1r2.north east)},
            anchor={north east},
            font=\tiny
        ]
        \addplot[thick, blue] table [y=DeltaE1] {assets/graphs/data/Z4_gap_sec_1.csv}
        node [pos=0.8, black, above] {$\Delta E_1$};
        \addplot[thick, red]  table [y=DeltaE2] {assets/graphs/data/Z4_gap_sec_1.csv}
        node [pos=0.4, black, above] {$\Delta E_2$};
    \end{axis}

    \draw (wilson4 c1r1.east) node[above, rotate=-90] {sector $n=0$};
    \draw (wilson4 c1r2.east) node[above, rotate=-90] {sector $n=1, 3$};
    \draw (wilson4 c1r3.east) node[above, rotate=-90] {sector $n=2$};

    % title
    \draw (wilson4 c1r1.north) node[above, font=\normalsize] {$\Z_4$ Wilson loop};
\end{tikzpicture}

    \vspace*{-10pt}
    \caption{$\Z_4$ Wilson loop for sectors $n=0, \dots, 3$ and sizes $L=6, \dots, 10$.
        Only the sector $n = 2$ has a clear deconfined-confined phase transition, as expected from the duality with the $4$-clock model.
    }
    \label{fig:z4_wilson}
\end{figure}

\smallskip

% \paragraph{Conclusions and outlooks}
% In this work, we proposed an exact gauge preserving duality transformation that maps the  $\mathbb{Z}_N$ lattice gauge theory on a ladder onto a 1D $N-$clock model in a transversal field, coupled to a possibly complex longitudinal field which depends on the superselection sector.
%
% This map allowed us to perform numerical simulations with an exact diagonalization algorithm with sizes up to $L=18, 13, 10$ for $N=2,3,4$ respectively.
% To study the phases of the model and a possible DCPT transition, we calculated the Wilson loops in the different topological sectors, finding an unusual behaviour in the sectors with $n$ odd (mod $N$), possibly suggesting the emergence of a new phase, such as for example the incommensurate phase appearing in chiral clock models \cite{huse1983chiral, whitsitt2018clock, zhuang2015clock}, whose characterization requires however to consider longer sizes of the chain in order to evaluate the asymptotic behaviour of correlators.
% This will be the subject of future work, in which we can also consider the possibility to include static and dynamical matter in the lattice gauge model.

% \medskip

% \begin{acknowledgments}
%     Numerical simulations have been performed with the QuSpin library for exact diagonalization \cite{weinberg2017quspin, weinberg2019quspin}.
%     We thank M.~Burrello and O.~Pomponio for useful discussions.
%     This research is partially supported by INFN through the project “QUANTUM”, the project “SFT" and the project “QuantHEP" of the QuantERA ERA-NET Co-fund in Quantum Technologies (GA No. 731473).
% \end{acknowledgments}
