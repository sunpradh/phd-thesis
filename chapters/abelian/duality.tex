%----------------------------------------
% SECTION: Dualities in physics
%----------------------------------------
\section{Dualities in physics}%
\label{sec:dualities_in_physics}


Duality is a simple yet powerful idea in physics.
They can be intended as specific mathematical transformations connecting seemingly unrelated physical phenomena.
They have been know for a long time, indeed a first example would be the duality of the electromagnetic field in the absence of sources, noticed by Heaviside in 1884.
Generally in physics, the concept of duality is connected to ideas, like symmetries, mappings between different coupling regimes, perturbative expansions for strongly correlated systems, and the wave-particle duality of quantum mechanics \cite{savit1980duality, cobanera2011bond}.

They play a major role in statistical physics and condensed matter.
In statistical mechanics, dualities were introduced for the first time by Kramers and Wannier \cite{kramers1941statistics}, who found a relation between the high temperature and low temperature regimes of the two-dimensional Ising mode.
In this way, they were able to find the critical temperature years before Onsager solution \cite{onsager1944ising}.
In this case we speak of self-dualities, where the same model is mapped onto itself but in a different coupling regime.
The essential legacy of Kramers and Wannier is the fact that self-dualities can put constraints on the phase boundaries and the exact location of critical points.

Not all dualities are self-dualities.
In fact, it also possible to relate two apparently different physical models with a duality transformation.
A known example is the Jordan-Wigner transformation \cite{schultz1964ising, jordan1928pauli}, where spin \ac{dof} (which are bosonic in nature) are mapped onto fermionic \ac{dof} in one-dimension \todo{elaborare}.
This duality shows that, in fact, there is not much difference between bosonic and fermionic \ac{dof}.

\todo{forse aggiungere qualcosa in più}

%
% SUBSECTION: The bond-algebraic approach
%
\subsection{The bond-algebraic approach}
\label{sub:the_bond_algebraic_approach}

In the following section we will quickly review the bond-algebraic approach to dualities \cite{cobanera2011bond}, because it offers a powerful and convenient way for dealing with duality transformations, in particular when gauge symmetries are involved.
The concept of \emph{bond-algebra} was first introduced in \cite{nussinov2009bond} and it exploits the fact that most \emph{Hamiltonian are a sum of simple and (quasi)local terms}:
\begin{equation}
    H = \sum_{\Gamma} \lambda_{\Gamma} h_{\Gamma},
\end{equation}
where $\Gamma$ is a set of indices (e.g.~the lattice sites but can be completely general) and $\lambda_{\Gamma}$ are numbers (usually the couplings).
Roughly speaking, by quasi-local we mean that a operator $h_{\Gamma}$ involves a small number of \ac{dof} that are spatially near each other (for example nearest neighbour).
The terms $h_{\Gamma}$ are called \emph{bond operators} (or simply \emph{bonds}).
From the bonds $h_{\Gamma}$ we obtain a \emph{bond algebra} $\algebra\{h_{\Gamma}\}$, which is the algebra of all the operators generated by all the possible products and sums of the bonds $h_{\Gamma}$ and their Hermitian conjugates.
In practical terms, given a set of bonds $\{h_{\Gamma}\}$, the bond-algebra $\algebra\{h_{\Gamma}\}$ is the algebra spanned by
\begin{equation*}
    \{
        \identity, h_{\Gamma}, \,
        h_{\Gamma}^{\dagger}, \,
        h_{\Gamma} h_{\Gamma^{\prime}}, \,
        h_{\Gamma}^{\dagger}  h_{\Gamma^{\prime}}, \,
        h_{\Gamma} h_{\Gamma^{\prime}}^{\dagger}, \,
        h_{\Gamma}^{\dagger}  h_{\Gamma^{\prime}}^{\dagger}, \,
        h_{\Gamma} h_{\Gamma^{\prime}} h_{\Gamma^{\prime\prime}},
        \dots
    \}
\end{equation*}
By construction, $\algebra\{h_{\Gamma}\}$ is closed under the operation Hermitian conjugation, but since an Hamiltonian $H$ is Hermitian then $h_{\Gamma}^{\dagger} = h_{\Gamma^{\prime}}$ for some $\Gamma^{\prime}$.
Therefore, $\algebra \{h_{\Gamma}\}$ is simply spanned by
\begin{equation*}
    \{
        \identity,
        h_{\Gamma}, \,
        h_{\Gamma} h_{\Gamma^{\prime}}, \,
        h_{\Gamma} h_{\Gamma^{\prime}} h_{\Gamma^{\prime\prime}}, \,
        \dots
    \}
\end{equation*}
Notice that the bonds $h_{\Gamma}$ that generate $\algebra\{ h_{\Gamma} \}$ do not need to be independent.


It is important to point out that a single Hamiltonian $H$ can have different bond algebras associated to it.
In fact, a bond algebra is determined by the partitioning of the bonds in $H$.
In principle, given any two decomposition of the same Hamiltonian,
\begin{equation*}
    H
    = \sum_{\Gamma} \lambda_{\Gamma} h_{\Gamma}
    = \sum_{\Sigma} \lambda^\prime_{\Sigma} h^\prime_{\Sigma},
\end{equation*}
one should expect $\algebra\{h_{\Gamma}\} \neq \algebra\{ h^{\prime}_{\Sigma} \}$ in general (see \cite{cobanera2011bond}).
To make an example, consider the Hamiltonian
\begin{equation*}
    H = \sum_{i} \qty( h_x \sigma^x_i + h_y \sigma^z_i ).
\end{equation*}
We can either partition the bonds by taking $\sigma^x_i$ and $\sigma^z_i$ as generators separately or by taking $h_x \sigma^x_i + h_z \sigma^z_i$ as a single bond.
In the former case we would obtain $\algebra\{\sigma^x, \sigma^z\} $, while in the latter we would have $\algebra\{h_x \sigma^x_i + h_z \sigma^z_i\}$.
These two algebras are clearly different,
\begin{equation*}
    \algebra\{\sigma^x, \sigma^z\}
    \neq
    \algebra\{h_x \sigma^x_i + h_z \sigma^z_i\},
\end{equation*}
because $\algebra\{h_x \sigma^x_i + h_z \sigma^z_i\}$ is commutative, while $\algebra\{\sigma^x, \sigma^z\} $ is not.

In the framework of bond-algebra, quantum dualities can be formulated as \emph{homomorphisms of bonds-algebras}.
By homomorphism we intend a map $\Phi$ between two algebras $\algebra_1$ and $\algebra_2$ that preserves the linear and multiplicative structure of the algebras.
In mathematical terms, given any $u,v \in \algebra_1$ and any complex number $\lambda$ we have
\begin{equation*}
    \Phi(u + \lambda v) = \Phi(u) + \lambda \Phi(v)
    \qand
    \Phi(u v) = \Phi(u) \Phi(v).
\end{equation*}

To be more precise with our definition of quantum duality,
consider two Hamiltonians $H_1$ and $H_2$ that act on Hilbert spaces of the same dimensions.
They are said to be \emph{dual} if there is some bond-algebra $\algebra_{H_1}$ of $H_1$ that is homomorphic to some bond-algebra $\algebra_{H_2}$ of $H_2$ and if the homomorphism $\Phi : \algebra_{H_1} \to \algebra_{H_2}$ maps $H_1$ onto $H_2$, $\Phi(H_1) = H_2$.
These mappings do not need to be isomorphisms, especially when gauge symmetries are involved, and we will explain why later.

In a traditional approach to quantum dualities, one tries to map each degree of freedom of $H_1$ onto a degree of freedom of $H_2$.
This can be rather cumbersome, because in this way most duality transformations appear to be non-local.
In other words, one degree of freedom on one side may correspond to a large number of \ac{dof} on the other side.
This is apparent, for example, with the Jordan-Wigner transformation, where a single spin is dual to a whole chain of fermions.

Quantum dualities in the bond-algebraic approach are instead \emph{local}, meaning that each single \emph{bond} $h_{\Gamma_1}$ of $H_1$ is mapped onto a single bond $h_{\Gamma_2}$ of $H_2$.
This may translates in non-locality when treating elementary \ac{dof} and is due to the fact that the generators of a bond algebra are usually two- (or more) body operators and expressing the elementary \ac{dof} with these operators may require large (if not infinite) products.

An isomorphism like $\Phi$ is physically sound if it is \emph{unitarily implementable} \cite{cobanera2011bond}, which means that there is a unitary matrix $\mathcal{U}$ such that the duality isomorphism reads
\begin{equation}
    \Phi(\mathcal{O}) =
    \mathcal{U} \mathcal{O} \mathcal{U}^{\dagger}, \quad
    \forall \mathcal{O} \in \algebra,
\end{equation}
where $\algebra$ is the operator algebra of the model under investigation.
\todo{elaborate}


To make the bond-algebraic approach more clear we will consider one example: the \emph{\acf{ising}}.
In this model we will see an example of self-duality through the use of disorder variables.
Our intent is not to shine new physics but to show how the use of bond-algebras offers a clear \emph{formalism} for treating dualities.

\todo{parlare anche della Jordan-Wigner?}
% approach clearer we now apply it to the 1D quantum Ising model with transverse field.


% To make the bond-algebraic approach more clear we will consider two examples:
% the \emph{quantum Ising model} and the \emph{XY model}.
% In the first model we will see an examples of self-duality through the use of disorder variables, while in the latter an example of duality that relates two very different models using the Jordan-Wigner transformation.
% Our intent is not to shine new physics but to show how the use of bond-algebras offers a clear \emph{formalism} for treating dualities of different kinds.
% % approach clearer we now apply it to the 1D quantum Ising model with transverse field.


\subsection{The quantum Ising model}
\label{sub:the_quantum_ising_model}

The \acl{ising} with a transverse field is a chain of spin-\onehalf described by the Hamiltonian
\begin{equation}
    \HamilIsing(h) = \sum_{i} \qty( \sigma^z_i \sigma^z_{i+1} + h \sigma^x_i ),
    \label{eq:hamiltonian_ising}
\end{equation}
where the sums runs over the sites of the chain and $h$ is the transverse field strength.
Notice that the Hamiltonian $\HamilIsing$ is indeed a sum of quasi-local terms.
In particular we have two types of terms: the interaction term $\sigma^z_i \sigma^z_{i+1}$ and the transverse field $\sigma^x_i$.
They are local or quasi-local because they involve at most two neighbouring sites.
These two sets of terms are the bonds of the Hamiltonian $\HamilIsing$, therefore bond-algebra $\algebra \{\sigma^z_i \sigma^z_{i+1}, \sigma^x_i\}$ is spanned by:
\begin{equation*}
    \{
        \identity, \,
        \sigma^z_i \sigma^z_{i+1}, \,
        \sigma^z_i \sigma^z_{i+1} \sigma^z_j \sigma^z_{j+1}, \,
        \dots, \,
        \sigma^x_i, \,
        \sigma^x_i \sigma^x_j, \,
        \dots, \,
        \sigma^z_i \sigma^z_{i+1} \sigma^x_i, \,
        \dots
    \}.
\end{equation*}
We consider an infinite chain in order to avoid subtleties with the boundaries conditions, which can have major effects on a duality transformation.

% We recognize as bonds the operators $\{\sigma^x_i\}$ and $\{\sigma^z_i \sigma^z_{i+1}\}$, which generates $\algebra \{\sigma^z_i \sigma^z_{i+1}, \sigma^x_i\} \equiv \algebra^{\ising}$.
The algebraic relations that defines the generators of $\algebra^{\ising}$ can be summarized as follows:
\begin{enumerate}
    \item each bonds square to the identity operator
        \begin{equation*}
            (\sigma^z_i \sigma^z_{i+1})^2 = (\sigma^x_i)^2 = \identity.
        \end{equation*}

    \item the bonds $\sigma_i^x$ anticommutes only with $\sigma^z_i \sigma^z_{i+1}$ and $\sigma^z_{i-1} \sigma^z_i$
        \begin{equation*}
            \acomm{\sigma_i^x}{\sigma^z_i \sigma^z_{i+1}} =
            \acomm{\sigma_i^x}{\sigma^z_{i-1} \sigma^z_i} = 0.
        \end{equation*}

    \item the bonds $\sigma^z_i \sigma^z_{i+1}$ anticommutes only with $\sigma^x_i$ and $\sigma^x_{i+1}$
        \begin{equation*}
            \acomm{\sigma^z_i \sigma^z_{i+1}}{\sigma^x_i} =
            \acomm{\sigma^z_i \sigma^z_{i+1}}{\sigma^x_{i+1}} = 0.
        \end{equation*}
\end{enumerate}

Given the symmetric roles that the basic bonds $\sigma^x_i$ and $\sigma^z_i \sigma^z_{i+1}$ play with each other, we can set up a mapping $\Phi^{\ising}$ that exchange their roles:
\begin{equation}
    \Phi^{\ising}(\sigma^z_i \sigma^z_{i+1}) = \sigma^x_i, \qquad
    \Phi^{\ising}(\sigma^x_i) = \sigma^z_{i-1} \sigma^z_{i}.
    \label{eq:duality_ising}
\end{equation}
This transformation can be extended to the whole $\algebra^{\ising}$ through the homomorphic property of $\Phi^{\ising}$.
It preserves all the important algebraic relationship and is one-to-one, hence it is an \emph{isomorphism} of $\algebra^{\ising}$ onto itself.
The Hamiltonian $\HamilIsing$ is just an element of $\algebra^{\ising}$.
We can apply $\Phi^{\ising}$ to $\HamilIsing$ and use its homomorphic property, which yields
\begin{equation}
    \begin{split}
        \Phi^{\ising}( \HamilIsing(h) )
        & = \sum_{i} \pqty{ \Phi^{\ising} (\sigma^z_i \sigma^z_{i+1}) + h \Phi^{\ising}(\sigma^x_i)} \\
        & = \sum_{i} \pqty{ \sigma^x_i + h \sigma^z_{i-1} \sigma^z_{i}} \\
        & = h \sum_{i} \pqty{ \sigma^z_i \sigma^z_{i+1} + h^{-1} \sigma^x_i}.
    \end{split}
\end{equation}
Notice that the indices in the sum can be freely shifted because we are working with an infinite number of sites.
We have thus obtained
\begin{equation}
    \Phi^{\ising} (\HamilIsing(h)) = h \HamilIsing(h^{-1}),
\end{equation}
henceforth $\Phi^{\ising}$ is a \emph{self-duality} of \eqref{eq:hamiltonian_ising}.
Notice that $\HamilIsing(h)$ is mapped onto itself but with the inverted coupling, $h \mapsto h^{-1}$, meaning that we can map the strongly coupled phase $h \gg 1$ into the weakly coupled phase $h \ll 1$, and vice versa.
This is basically the quantum version the \emph{Kramers-Wannier duality} \cite{kramers1941statistics, fradkin1978order}.
% but, unlike the usual formulation of this duality, we did not need to introduce \emph{disorder variables} to write down the transformation.

If we think of the term $\sigma^x_i$ as living on the site $i$ and of $\sigma^z_i \sigma^z_{i+1}$ as of living on the \emph{link} between the site $i$ and $i+1$, then we can think of $\Phi^{\ising}$ as mapping \eqref{eq:hamiltonian_ising} onto the \emph{dual lattice}.
In fact, the dual lattice of a chain is still a chain and the site term $\sigma^x_i$ is mapped onto a link term $\sigma^z_i \sigma^z_{i+1}$, and vice versa.


\begin{figure}[t]
    \SideFigure[label=fig:kramers_wannier, desc={Self-duality map of the \ac{ising}}]{
        \begin{tikzpicture}[scale=0.6]
    \draw[lattice] (-1,0) grid (9,0);
    \draw[lattice, xshift=-1cm] (0,-2) grid (10,-2);

    % spin operators
    \draw[X] (2, 0) -- (4, 0) node [midway, black, above] {$\sigma^z_{i} \sigma^z_{i+1}$};
    \foreach \x in {0,2,...,8} \draw (\x,0) node [site] {};
    \draw (6,0) node [Z site] {} node [black, above] {$\sigma^x_j$};

    % disorder operators
    \draw[X] (5, -2) -- (7, -2) node [midway, black, below] {$\sigma^z_{j} \sigma^z_{j+1}$};
    \foreach \x in {-1,1,...,9} \draw (\x,-2) node [site] {};
    \draw (3, -2) node [Z site] {} node [black, below] {$\sigma^x_i$};

    \draw[freccia] (3,0) -- (3,-2) node [pos=0.5, right] {$\Phi^\ising$};
    \draw[freccia] (6,0) -- (6,-2) node [pos=0.5, right] {$\Phi^\ising$};
\end{tikzpicture}

    }{
        Pictorial representation of the duality map $\Phi^{\ising}$, that maps the \ac{ising} on the same model on the dual lattice
    }
\end{figure}


We want to have a clearer physical picture of the duality map $\Phi^{\ising}$ and build a bridge with the traditional approach to dualities for the \ac{ising}.
For this reason we want to find the \emph{elementary \ac{dof} of the dual model}.
The \ac{dof} of the dual model lives on the sites of the dual lattice, which corresponds to the links of the original lattice.
On these dual sites we again have spin-\onehalf \ac{dof} and, for more clarity, we use $\mu^x$ and $\mu^z$ for referring to the Pauli matrices acting on these new spins.
The dual site $i$ corresponds to the link $(i, i+1)$, while the dual link $(i-1, i)$  corresponds to the site $i$.

From \eqref{eq:duality_ising}, we already know that
\begin{equation}
    \sigma^z_i \sigma^z_{i+1} = \mu^x_i
    \qand
    \sigma^x_i = \mu^z_{i-1} \mu^z_i.
\end{equation}
The role of $\mu^x_i$ is evident, it measure the alignment of two neighbouring spins on the sites $i$ and $i+1$, while the meaning of $\mu^z_i$ is still opaque.
We can arrive at the definition of $\mu^z_i$ by exploiting the map $\Phi^{\ising}$.
The bond $\mu^z_{i-1} \mu^z_i$ corresponds to the image of $\sigma^x_i$ through $\Phi^{\ising}$, so we know how they are mapped.
If we isolate $\mu^z_i$ with an \emph{infinite product}, we then obtain
\begin{equation}
    \mu^z_i
    = \prod_{j = -\infty}^{i} \mu^z_{j-1} \mu^z_j
    = \prod_{j = -\infty}^{i} \sigma^x_j.
    \label{eq:def_mu_z}
\end{equation}
We see that $\mu^z_i$ flips all the spins before the $i$-th site.
From \eqref{eq:def_mu_z}, we can see the \emph{non-local} origin of the dual \ac{dof} in traditional dualities.
When working with two, or more, body terms, in order to isolate a single body term the use of large (or even infinite) product is necessary.

To understand the role of $\mu^x_i$ and $\mu^z_i$, consider now the ferromagnetic ground states $\ket{\Omega_{\rho}}$ of \eqref{eq:hamiltonian_ising}, where $\rho=\uparrow, \downarrow$.
Say we start from $\ket{\Omega_{\uparrow}}$, without loss of generality.
The action of $\mu^z_i$ on $\ket{\Omega_{\rho}}$ is to create a \emph{kink}, which is a domain wall between two ordered regions.
From this point of view, a single spin-flip $\sigma^x_i \ket{\Omega_{\uparrow}}$ creates a \emph{kink-antikink pair}.

\begin{figure}[t]
    \SideFigure[label=fig:kink_states, desc={ferromagnetic ground states and kink states in the \ac{ising}}]{%
        \begin{tikzpicture}[scale=0.5]
    \draw[ladder] (-0.5, 0) -- (6.5, 0);
    \DrawSites{0,1,...,6}{0,0}
    \foreach \x in {0,1,...,6} \Spin{up}{\x}{0};
    \node[right] at (7, 0) { $\ket{\Omega_{\uparrow}}$};
    \node[left] at (-1, 0) {(a)};

    \begin{scope}[yshift=-2.5cm]
        \draw[ladder] (-0.5, 0) -- (6.5, 0);
        \DrawSites{0,1,...,6}{0,0}
        \foreach \x in {0,1,...,6} \Spin{down}{\x}{0};
        \node[right] at (7, 0) { $\ket{\Omega_{\downarrow}}$};
        \node[left] at (-1, 0) {(b)};
    \end{scope}

    \begin{scope}[yshift=-5cm]
        \draw[ladder] (-0.5, 0) -- (6.5, 0);
        \DrawSites{0,1,...,6}{0,0}
        \foreach \x in {0,1,2,3} \Spin{down}{\x}{0};
        \foreach \x in {4,5,6} \Spin{up}{\x}{0};
        \draw[very thick, dashed] (3.5, -0.9) -- +(0, 1.8);
        \node[right] at (7, 0) { $\mu^z_i \ket{\Omega_{\uparrow}}$ };
        \node[left] at (-1, 0) {(c)};
    \end{scope}

    \begin{scope}[yshift=-7.5cm]
        \draw[ladder] (-0.5, 0) -- (6.5, 0);
        \DrawSites{0,1,...,6}{0,0}
        \foreach \x in {0,1,2} \Spin{up}{\x}{0};
        \Spin{down}{3}{0}
        \foreach \x in {4,5,6} \Spin{up}{\x}{0};
        \draw[very thick, dashed] (3.5, -0.9) -- +(0, 1.8);
        \draw[very thick, dashed] (2.5, -0.9) -- +(0, 1.8);
        \node[right] at (7, 0) { $\sigma^x_i \ket{\Omega_{\uparrow}}$ };
        \node[left] at (-1, 0) {(d)};
    \end{scope}
\end{tikzpicture}

    }{%
        \emph{(a)} and \emph{(b)} ferromagnetic ground states $\ket{\Omega_{\uparrow}}$ and $\ket{\ket{\Omega_{\downarrow}}}$.
        \emph{(c)} kink created on the link between site $i$ and $i+1$ by the operator $\mu^z_i$.
        \emph{(d)} kink-antikink pairs created around the site $i$ by the spin flip $\sigma^x_i$.
    }
\end{figure}

\todo{serve aggiungere altro o è già abbastanza lungo il discorso?}

% The disorder variables are defined as
% \begin{equation}
%     \mu^x_i = \sigma^z_i \sigma^z_{i+1}, \qquad
%     \mu^z_i = \prod_{j=i}^{\infty} \sigma^x_j.
%     \label{eq:disorder_variables}
% \end{equation}
% and satisfy all the relations that characterize the Pauli matrices,
% they square to the identity and anticommutes with each other:
% \begin{equation}
%     (\mu^{\rho}_i)^2 = \identity
%     \qand
%     \acomm{\mu^x_i}{\mu^z_i} = 0,
% \end{equation}
% where $\rho=x,z$.
% The first operator, $\mu^x_i$, measures the alignment of two neighbouring spins while the second, $\mu^z_i$, flips all the spins from the $i$-th spin.
% Instead of defining a priori these new variables we can obtain them through the map $\Phi^{\ising}$ applied to the elementary \ac{dof} $\sigma^z_i$ and $\sigma^x_i$.
% In other to do so, have to be part of the algebra $\algebra^{\ising}$.
% We already know that the operators $\sigma^x_i$ are in $\algebra^{\ising}$ and they transform as
% \begin{equation}
%     \Phi^{\ising} (\sigma^x_i) =
% \end{equation}, because they appear in the Hamiltonian \eqref{eq:hamiltonian_ising}, but not $\sigma^z_i$ but these can be obtained with the product
% \begin{equation}
%     \sigma^z_i = \prod_{j=i}^{\infty} \sigma^z_i \sigma^z_{i+1}.
%     \label{eq:infinite_product_sigma_z}
% \end{equation}
% Applying $\Phi^\ising$ to \eqref{eq:infinite_product_sigma_z} we obtain
% \begin{equation}
%     \Phi^{\ising}(\sigma^z_i)
%     = \Phi^{\ising} \qty(\prod_{j = i}^{\infty} \sigma^z_j \sigma^z_{j+1})
%     = \prod_{j = i}^{\infty} \Phi^{\ising}(\sigma^z_j \sigma^z_{j+1})
%     = \mu^x_{i+1}
% \end{equation}


%
% SUBSECTION: Gauge-reducing dualities
%
\subsection{Gauge-reducing dualities}%
\label{sub:gauge_reducing_dualities}

In this section we will review the notion of \emph{gauge-reducing dualities},
In order to do so we start by highlighting the difference between ordinary symmetries and quantum symmetries.
% but it is best first to start with the difference between between ordinary symmetries and quantum symmetries.

Following the statement of Wigner's theorem\citneeded, a quantum symmetry is a unitary or anti-unitary mapping that commute with the Hamiltonian.
This does not mean that all symmetries have the same physical meaning or mathematical consequences.
By the term ``\emph{ordinary symmetries}'' we refer to the most common types of symmetry that we encounter in physical systems that usually correspond to \emph{global transformation} of the physical apparatus or system, like for example rotational invariance.
These symmetries have a direct physical impact, since they can influence the level degeneracy of an Hamiltonian and force strict selection rules.

On the other hand, gauge symmetries are \emph{local symmetries} of the model that signal the presence of \emph{redundant \ac{dof}}.
In fact, it is better to think of gauge symmetries as \emph{local constraints} on the elementary \ac{dof} of the gauge model.
As a result, the state space of a gauge model is larger than set of physical states, which are exactly the states that are invariant under the action of the gauge symmetries.
%, which mean that they satisfies the local constraints of the model.
% in fact gauge invariance can be thought as a set of \emph{local constraints} on the elementary \ac{dof} of the model.
% This means that the state space of the model is larger than set of physical states and these are exactly the states that are invariant under the action of the gauge symmetries, which mean that they satisfies the local constraints of the model.
The same reasoning applies to the observables of the gauge model.
An observables is represented by an Hermitian operator and a \emph{physical observable} is represented by an Hermitian operator that commutes with gauge symmetries.

So, if physical states and physical observables already satisfies the local constraints of the gauge symmetries, this means that the physical impact of the latter is already encoded in the former.
% For example, ordinary symmetries add new information on the structure of the state space, i.e.~selection rules.
% While on the other hand, gauge symmetries \emph{do not add} new information on the structure of the \emph{physical state space}, because the latter is obtained by resolving the local constraints.
It is clear that the ordinary symmetries and gauge symmetries are very different and is better them conceptually far apart as possible \cite{cobanera2011bond}.


% represent a physical observable only if it commutes with the gauge symmetries, which makes them gauge-invariant.
% If

When dealing with a gauge model, it would be natural to assume that, in order to establish a duality with any gauge symmetries, these have to be eliminated from the former model first.
In other terms, that it would be necessary to project the operator content on the subspace of physical states first or, alternatively, proceed with gauge-fixing.
Although this is common in traditional approach to dualities, with bond algebras this is not strictly necessary.
As stated in \cite{cobanera2011bond}, with the bond-algebraic approach one can find mappings to models without any gauge symmetry that preserve all the important algebraic properties, without the need to projection or gauge-fixing.

The procedure goes as follows: consider a gauge model and let $H^{\text{G}}$ be its Hamiltonian and $G_{\Gamma}$ its gauge symmetries.
An operator $\mathcal{O}$ is gauge-invariant if and only if it commutes with all the $G_{\Gamma}$:\begin{equation*}
    \mathcal{O} \text{~physical}
    \quad \Longleftrightarrow \quad
    \comm{\mathcal{O}}{G_{\Gamma}} = 0 \quad \forall \Gamma.
\end{equation*}
Clearly, the Hamiltonian has to be gauge-invariant, hence $[H^{\text{GR}}, G_{\Gamma}] = 0$.
Now let $H^{\text{GR}}$ be the dual Hamiltonian of a non-gauge, or gauge-reduced, model.
Furthermore, let $\algebra^{\text{G}}$ and $\algebra^{\text{GR}}$ be the bond-algebra of the gauge and gauge-reduced models, respectively.
A \emph{gauge-reducing duality} is a map
\begin{equation*}
    \Phi^{\text{GR}} : \algebra^{\text{G}} \to \algebra^{\text{GR}}
\end{equation*}
such that $H^{\text{G}}$ is mapped onto $H^{\text{GR}}$ while making all the gauge symmetries of the gauge model trivial:
\begin{equation}
    \Phi^{\text{GR}}(H^{\text{G}}) = H^{\text{GR}}
    \qand
    \Phi^{\text{GR}}(G_{\Gamma}) = \identity \quad \forall \Gamma.
\end{equation}

Unlike the dualities in Sec.~\ref{sub:the_bond_algebraic_approach}, a gauge-reducing duality like $\Phi_{\text{GR}}$ has to be implementable as a \emph{projective unitary operator} $\mathcal{U}$.
Formally, this can be written as
\begin{equation}
    \Phi_{\text{GR}} ( \mathcal{O} ) = \mathcal{U} \mathcal{O} \mathcal{U}^\dagger, \quad
    \mathcal{U} \mathcal{U}^\dagger = \identity, \quad
    \mathcal{U}^\dagger \mathcal{U} = P_{\text{GI}}
\end{equation}
where $P_{\text{GI}}$ is the projector of the subspace of gauge-invariant states, i.e.~$G_{\Gamma} \ket{\psi} = \ket{\psi}$ for all $\Gamma$.
Roughly speaking, this projective unitary operator can be represented as rectangular matrix that preserves the norm of gauge-invariant states while projecting out all the other states.

In the next section we will use an example of gauge-reducing duality, which will be instrumental for the rest of the chapter.


\subsection{Dualities in two dimensions}
\label{sub:dualities_in_two_dimensions}

% \todo{arricchire}

As an example of gauge-reducing duality, we will apply the technology introduced in Sec.~\ref{sub:gauge_reducing_dualities} to the $\Z_2$ \ac{lgt} in two-dimensions.
% A clear example of a gauge-reducing duality is provided by the $\Z_N$, $d=2$ gauge model
We resume the Hamiltonian \eqref{eq:z2_lgt_hamiltonian}
\begin{equation*}
    H^{\Z_2}
    = - \sum_{p} B_p - \lambda \sum_{\link} Z_{\link}
    = - \sum_{p} B_p - \lambda \sum_{x} \qty(Z_{(x, +\hat{1})} + Z_{(x, +\hat{2})}),
\end{equation*}
and its Gauss (or vertex) operators
\begin{equation}
    A_v = \prod_{\link \in v}  Z_{\link},
\end{equation}
which generate the gauge symmetries and commute with the Hamiltonian
\begin{equation}
    \comm{H^{\Z_2}}{A_v} = 0 \quad \forall v \in \Lattice.
\end{equation}
% \begin{equation}
% 	H_{\text{G}} = \sum_r \left( V_{(r,\hat{1})} + V_{(r,\hat{2})} + \lambda U_r \right).
% \end{equation}

% Its group of gauge symmetries is generated by \eqref{eq:gauss_operator}.
% In the simplest case where $N=2$, the $V$'s can be represented by the Pauli operators $\sigma^z$ and the $U$'s by $\sigma^x$.
% In so doing, the Gauss operator becomes
% \begin{equation}
% 	G_r =
% 	\sigma^z_{(r, \hat{1})}
% 	\sigma^z_{(r, \hat{2})}
% 	\sigma^z_{(r, -\hat{1})}
% 	\sigma^z_{(r, -\hat{2})},
% 	\label{eq:gauss_operator_Z2}
% \end{equation}
% and commutes with $H_{\text{G}}$ and with the bonds $\left\{\sigma^z_{(r,\hat{1})},\ \sigma^z_{(r,\hat{2})},\ U_r\right\} $.

In particular, each term of the Hamiltonian commutes with the Gauss operators, which means that the bond algebra they generate is gauge-invariant,
This bond-algebra satisfy three simple relations:
\begin{enumerate}[label=(\roman*)]
    \item all the bonds square to the identity,
        % \begin{equation*}
        %     B_p^2 = \identity, \quad
        %     Z_{(x, +\hat{1})}^2 = \identity, \qand
        %     Z_{(x, +\hat{2})}^2 = \identity
        % \end{equation*}
    \item each spin $Z$ anti-commutes with two adjacent plaquettes operators $U$
    \item each plaquette operator $U$ anti-commutes with four spins $Z$.
\end{enumerate}

The model $H^{\Z_2}$ is dual to the $d=2$ \ac{ising}.
% set of relations is identical to those satisfied by the bonds of the $d=2$ \ac{ising}:
The Hamiltonian of the latter in two-dimensions is
\begin{equation}
    \HamilIsing =
    - \sum_{i} \qty(
        \sigma^z_i \sigma^z_{i+\hat{1}} +
        \sigma^z_i \sigma^z_{i+\hat{2}} +
        h \sigma^x_i
    ),
\end{equation}
where the index $i$ runs over the sites.
One recognizes as separate bonds the terms $\sigma^z_i \sigma^z_{i+\hat{1}}$, $\sigma^z_i \sigma^z_{i+\hat{2}}$, and $\sigma^x_i$.
It is immediate to see that these bonds satisfy the same relations of the bonds of $H^{\Z_2}$.

The dual model of $H^{\Z_2}$ lives on the dual lattice.
Therefore we identify a plaquette $p$ in the gauge model with a site $i$ of the \ac{ising}, while $x$ will refer to the lower left site of the plaquette $p$.
With this notation, we can now build the duality mapping $\Phi^{\text{2d}}$ as follows:
\begin{equation}
    \Phi^{\text{2d}}(Z_{(x, \hat{1})} )  = \sigma^z_{(i-\hat{2})} \sigma^z_i, \quad
    \Phi^{\text{2d}}(Z_{(x, \hat{2})} )  = \sigma^z_{(i-\hat{1})} \sigma^z_i, \quad
    \Phi^{\text{2d}}(U_p) = \sigma^x_i.
    \label{eq:duality_2d}
\end{equation}
Applying to $\Phi^{\text{2d}}$ to $H^{\Z_2}$ we obtain
\begin{equation}
    \Phi^{2d} (H^{\Z_2}) =
    - \sum_{i} \sigma^x_i - \lambda \sum_{i} \qty(
        \sigma^z_{(i-\hat{2})} \sigma^z_i +
        \sigma^z_{(i-\hat{1})} \sigma^z_i
    ) =
    \lambda \HamilIsing(\lambda^{-1})
\end{equation}
Thus, $\Phi^{\text{2d}}$ maps $H^{\Z_2}$ to $\HamilIsing$, up to a multiplicative constant, if we identify the constants $\lambda \leftrightarrow h^{-1}$.

From \eqref{eq:duality_2d}, one readily obtains
\begin{equation*}
    \Phi^{\text{2d}}(G_x) = \identity,
\end{equation*}
which means that $\Phi^{\text{2d}}$ is in fact a \emph{gauge-reducing duality}.
% Moreover, $\Phi^{\text{2d}}$ is a gauge-reducing duality homomorphism, since $\Phi(G_x) = \identity$.
Therefore, $\HamilIsing$ represents all the physics contained in $H_{\text{gauge}}$, but without all the redundant \ac{dof}.
In the general case, i.e.~for a generic $\Z_N$ symmetry, the duality leads to an $N$-clock model \cite{radicevic2019spin}.

The reason why it is possible to encode the physical content of the gauge model in a simpler \ac{ising} is the following.
The physical states of a pure gauge model is made of closed electric loops and each electric loop can be thought as containing magnetic flux.
So, each physical state can be fully described by indicating which plaquettes contains magnetic flux and which do not.
The electric lines naturally arises as domain walls between plaquettes with different flux.

Basically, the duality mapping $\Phi^{\text{2d}}$ assigns to each plaquette a spin-\onehalf{} \ac{dof}, indicating the flux state.
Everything else readily follows.
The plaquette operator $U_p$ flips the state of the plaquette, therefore it should be mapped to an operator that flips the spin in $p$, thus $\sigma^x$.
The electric fields $V_{(x, \hat{1})}$ and $V_{(x, \hat{2})}$ are just domain walls between plaquettes, therefore they should be mapped to interaction terms like $\sigma^z_{i-\hat{1}} \sigma^z_{i}$ and $\sigma^z_{i-\hat{2}} \sigma^z_i$.


\todo{cosa altro c'è da aggiungere?}

% vim: spelllang=en
