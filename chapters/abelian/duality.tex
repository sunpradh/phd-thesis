%----------------------------------------
% SECTION: Dualities in physics
%----------------------------------------
\section{Dualities in physics}%
\label{sec:dualities_in_physics}


Duality is a simple yet powerful idea in physics.
They can be intended as specific mathematical transformations connecting seemingly unrelated physical phenomena.
They have been know for a long time, indeed a first example would be the duality of the electromagnetic field in the absence of sources, noticed by Heaviside in 1884.
Generally in physics, the concept of duality is connected to ideas, like symmetries, mappings between different coupling regimes, perturbative expansions for strongly correlated systems, and the wave-particle duality of quantum mechanics \cite{savit1980duality, cobanera2011bond}.

They play a major role in statistical physics and condensed matter.
In statistical mechanics, dualities were introduced for the first time by Kramers and Wannier \cite{kramers1941statistics}, who found a relation between the high temperature and low temperature regimes of the two-dimensional Ising mode.
In this way, they were able to find the critical temperature years before Onsager solution \cite{onsager1944ising}.
In this case we speak of self-dualities, where the same model is mapped onto itself but in a different coupling regime.
The essential legacy of Kramers and Wannier is the fact that self-dualities can put constraints on the phase boundaries and the exact location of critical points.

Not all dualities are self-dualities.
In fact, it also possible to relate two apparently different physical models with a duality transformation.
A known example is the Jordan-Wigner transformation \cite{schultz1964ising, jordan1928pauli}, where spin degrees of freedom (which are bosonic in nature) are mapped onto fermionic degrees of freedom in one-dimension \todo{elaborare}.
This duality shows that, in fact, there is not much difference between bosonic and fermionic degrees of freedom.

\todo{forse aggiungere qualcosa in più}

%
% SUBSECTION: The bond-algebraic approach
%
\subsection{The bond-algebraic approach}
\label{sub:the_bond_algebraic_approach}

In the following section we will quickly review the bond-algebraic approach to dualities \cite{cobanera2011bond}, because it offers a powerful and convenient way for dealing with duality transformations, in particular when gauge symmetries are involved.
The concept of \emph{bond-algebra} was first introduced in \cite{nussinov2009bond} and it exploits the fact that most \emph{Hamiltonian are a sum of simple and (quasi)local terms}:
\begin{equation}
    H = \sum_{\Gamma} \lambda_{\Gamma} h_{\Gamma},
\end{equation}
where $\Gamma$ is a set of indices (e.g.~the lattice sites but can be completely general) and $\lambda_{\Gamma}$ are numbers (usually the couplings).
The terms $h_{\Gamma}$ are the \emph{bond operators} (or simply \emph{bonds}).
They involve at most few degrees of freedom which are locally near.
From the bonds $h_{\Gamma}$ we obtain a \emph{bond algebra} $\algebra\{h_{\Gamma}\}$, which is the algebra of all the operators generated by all the possible products and sums of the bonds $h_{\Gamma}$ and their Hermitian conjugates.
In practical terms, given a set of bonds $\{h_{\Gamma}\}$, the bond-algebra $\algebra\{h_{\Gamma}\}$ is the algebra spanned by
\begin{equation*}
    \{
        \identity, h_{\Gamma}, \,
        h_{\Gamma}^{\dagger}, \,
        h_{\Gamma} h_{\Gamma^{\prime}}, \,
        h_{\Gamma}^{\dagger}  h_{\Gamma^{\prime}}, \,
        h_{\Gamma} h_{\Gamma^{\prime}}^{\dagger}, \,
        h_{\Gamma}^{\dagger}  h_{\Gamma^{\prime}}^{\dagger}, \,
        h_{\Gamma} h_{\Gamma^{\prime}} h_{\Gamma^{\prime\prime}},
        \dots
    \}
\end{equation*}
By construction, $\algebra\{h_{\Gamma}\}$ is closed under the operation Hermitian conjugation, but since an Hamiltonian $H$ is Hermitian then $h_{\Gamma}^{\dagger} = h_{\Gamma^{\prime}}$ for some $\Gamma^{\prime}$.
Therefore, $\algebra \{h_{\Gamma}\}$ is simply spanned by
\begin{equation*}
    \{
        \identity,
        h_{\Gamma}, \,
        h_{\Gamma} h_{\Gamma^{\prime}}, \,
        h_{\Gamma} h_{\Gamma^{\prime}} h_{\Gamma^{\prime\prime}}, \,
        \dots
    \}
\end{equation*}
Notice that the bonds $h_{\Gamma}$ that generate $\algebra\{ h_{\Gamma} \}$ do not need to be independent.

To make an example, consider the quantum Ising model with a transverse field, which is a chain of spin-\onehalf with the Hamiltonian
\begin{equation}
    \HIsing(h) = \sum_{i} \qty( \sigma^z_i \sigma^z_{i+1} + h \sigma^x_i ),
    \label{eq:hamiltonian_ising}
\end{equation}
where the sums runs over the sites of the chain and $h$ is the transverse field strength.
Notice that the Hamiltonian is sum of quasi-local terms.
In particular two types of terms: the interaction term $\sigma^z_i \sigma^z_{i+1}$ and the transverse field $\sigma^x_i$.
They are local or quasi-local because they involve at most two neighbouring sites.
These two set of terms are the bonds of the Hamiltonian $\HIsing$ and our bond-algebra $\algebra \{\sigma^z_i \sigma^z_{i+1}, \sigma^x_i\}$ is spanned by
\begin{equation*}
    \{
        \identity, \,
        \sigma^z_i \sigma^z_{i+1}, \,
        \sigma^z_i \sigma^z_{i+1} \sigma^z_j \sigma^z_{j+1}, \,
        \dots, \,
        \sigma^x_i, \,
        \sigma^x_i \sigma^x_j, \,
        \dots, \,
        \sigma^z_i \sigma^z_{i+1} \sigma^x_i, \,
        \dots
    \}.
\end{equation*}


It is important to point out that a single Hamiltonian $H$ can have different bond algebras associated to it.
In fact, a bond algebra is determined by the partitioning of the bonds in $H$.
In principle, given any two decomposition of the same Hamiltonian,
\begin{equation*}
    H
    = \sum_{\Gamma} \lambda_{\Gamma} h_{\Gamma}
    = \sum_{\Sigma} \lambda^\prime_{\Sigma} h^\prime_{\Sigma},
\end{equation*}
one should expect $\algebra\{h_{\Gamma}\} \neq \algebra\{ h^{\prime}_{\Sigma} \}$ in general (see \cite{cobanera2011bond}).
To make an example, consider the Hamiltonian
\begin{equation*}
    H = \sum_{i} \qty( h_x \sigma^x_i + h_y \sigma^z_i ).
\end{equation*}
We can either partition the bonds by taking $\sigma^x_i$ and $\sigma^z_i$ as generators separately or by taking $h_x \sigma^x_i + h_z \sigma^z_i$ as a single bond.
In the former case we would obtain $\algebra\{\sigma^x, \sigma^z\} $, while in the latter we would have $\algebra\{h_x \sigma^x_i + h_z \sigma^z_i\}$.
These two algebras are clearly different,
\begin{equation*}
    \algebra\{\sigma^x, \sigma^z\}
    \neq
    \algebra\{h_x \sigma^x_i + h_z \sigma^z_i\},
\end{equation*}
because $\algebra\{h_x \sigma^x_i + h_z \sigma^z_i\}$ is commutative, while $\algebra\{\sigma^x, \sigma^z\} $ is not.

In this framework of bond-algebra, quantum dualities can be formulated as \emph{homomorphisms of bonds-algebras}.
By homomorphism we intend a map $\Phi$ between two algebras $\algebra_1$ and $\algebra_2$ that preserves the linear and multiplicative structure of the algebras.
In mathematical terms, given any $u,v \in \algebra_1$ and any complex number $\lambda$ we have
\begin{equation*}
    \Phi(u + \lambda v) = \Phi(u) + \lambda \Phi(v)
    \qand
    \Phi(u v) = \Phi(u) \Phi(v).
\end{equation*}

To be more precise with our definition of quantum duality,
Consider two Hamiltonians $H_1$ and $H_2$ that act on Hilbert spaces of the same dimensions.
They are said to be \emph{dual} if there is some bond-algebra $\algebra_{H_1}$ of $H_1$ that is homomorphic to some bond-algebra $\algebra_{H_2}$ of $H_2$ and if the homomorphism $\Phi : \algebra_{H_1} \to \algebra_{H_2}$ maps $H_1$ onto $H_2$, $\Phi(H_1) = H_2$.
These mappings do not need to be isomorphisms, especially when gauge symmetries are involved, and we will explain why later.

In a traditional approach to quantum dualities, one tries to map each degree of freedom of $H_1$ onto a degree of freedom of $H_2$.
This can be rather cumbersome, because in this way the duality transformation appear to be non-local.
In other words, one degree of freedom on one side may correspond to a large number of degrees of freedom on the other side.
This is apparent, for example, with the Jordan-Wigner transformation, where a single spin is dual to a whole chain of fermions.

Quantum dualities with bond-algebra approach are instead \emph{local}, meaning that each single bond $h_{\Gamma_1}$ of $H_1$ is mapped onto a single bond $h_{\Gamma_2}$ of $H_2$.
This may translates in non-locality when treating elementary degrees of freedom and is due to the fact that the generators of a bond algebra are usually two- (or more) body operators and expressing the elementary degrees of freedom with these operators may require large (if not infinite) products.

An isomorphism like $\Phi$ is physically sound if it is \emph{unitarily implementable} \cite{cobanera2011bond}, which means that there is a unitary matrix $\mathcal{U}$ such that the duality isomorphism reads
\begin{equation}
    \Phi(\mathcal{O}) =
    \mathcal{U} \mathcal{O} \mathcal{U}^{\dagger}, \quad
    \forall \mathcal{O} \in \algebra,
\end{equation}
where $\algebra$ is the operator algebra of the model under investigation.
\todo{elaborate}


To make the bond-algebraic approach more clear we will consider two examples:
the \emph{quantum Ising model} and the \emph{XY model}.
In the first model we will see an examples of self-duality through the use of disorder variables, while in the latter an example of duality that relates two very different models using the Jordan-Wigner transformation.
Our intent is not to shine new physics but to show how the use of bond-algebras offers a clear \emph{formalism} for treating dualities of different kinds.
% approach clearer we now apply it to the 1D quantum Ising model with transverse field.


\paragraph{Quantum Ising model}
We have already written the Hamiltonian of quantum Ising model with transverse field in \eqref{eq:hamiltonian_ising}.
We recognize as bonds the operators $\{\sigma^x_i\}$ and $\{\sigma^z_i \sigma^z_{i+1}\}$, which generates $\algebra \{\sigma^z_i \sigma^z_{i+1}, \sigma^x_i\} \equiv \algebra^{\text{Ising}}$.
The relations that defines the generators of $\algebra^{\text{Ising}}$ can be summarized as follows:
\begin{enumerate}
    \item each bonds square to the identity operator
        \begin{equation*}
            (\sigma^z_i \sigma^z_{i+1})^2 = (\sigma^x_i)^2 = \identity.
        \end{equation*}

    \item the bonds $\sigma_i^x$ anticommutes only with $\sigma^z_i \sigma^z_{i+1}$ and $\sigma^z_{i-1} \sigma^z_i$
        \begin{equation*}
            \acomm{\sigma_i^x}{\sigma^z_i \sigma^z_{i+1}} =
            \acomm{\sigma_i^x}{\sigma^z_{i-1} \sigma^z_i} = 0.
        \end{equation*}

    \item the bonds $\sigma^z_i \sigma^z_{i+1}$ anticommutes only with $\sigma^x_i$ and $\sigma^x_{i+1}$
        \begin{equation*}
            \acomm{\sigma^z_i \sigma^z_{i+1}}{\sigma^x_i} =
            \acomm{\sigma^z_i \sigma^z_{i+1}}{\sigma^x_{i+1}} = 0.
        \end{equation*}
\end{enumerate}

Given the symmetric roles that the basic bonds $\sigma^x_i$ and $\sigma^z_i \sigma^z_{i+1}$ play with each other, one can set up a mapping $\Phi^{\text{Ising}}$ that exchange their roles:
\begin{equation}
    \Phi^{\text{Ising}}(\sigma^z_i \sigma^z_{i+1}) = \sigma^x_i, \qquad
    \Phi^{\text{Ising}}(\sigma^x_i) = \sigma^z_{i-1} \sigma^z_i.
    \label{eq:duality_ising}
\end{equation}
This transformation can be extended to the whole $\algebra^{\text{Ising}}$ through the homomorphic property of $\Phi^{\text{Ising}}$.
It preserves all the important algebraic relationship and is one-to-one, hence it is an \emph{isomorphism} of $\algebra^{\text{Ising}}$ onto itself.
Therefore, it is a \emph{self-duality} of the model.
The Hamiltonian $\HIsing$ is just an element of $\algebra^{\text{Ising}}$ and through $\Phi^{\text{Ising}}$ it becomes
\begin{equation}
    \Phi^{\text{Ising}}( \HIsing(h) )
    = \sum_{i} \pqty{ \sigma_i^x + h \sigma^z_{i} \sigma^z_{i+1}} \\
    = h \HIsing(h^{-1}),
\end{equation}


%
% SUBSECTION: Gauge-reducing dualities
%
\subsection{Gauge-reducing dualities}%
\label{sub:gauge_reducing_dualities}

In this section we will review the notion of \emph{gauge-reducing dualities}.
Gauge symmetries are \emph{local symmetries} of the model that signal the presence of \emph{redundant degrees of freedom}, in fact gauge invariance can be thought as a set of \emph{local constraints} on the elementary degrees of freedom of the model.
This means that the state space of the model is larger than set of physical states and these are exactly the states that are invariant under the action of the gauge symmetries, which would mean that they satisfies the local constraints of the model.
The same can be applied to the Hermitian operators and the observables of the model.
An Hermitian operator represent a physical observable only if it commutes with the gauge symmetries, which makes them gauge invariant.

When dealing with a gauge model, it would be natural to assume that, in order to establish a duality with any gauge symmetries, these have to be eliminated from the former model.
In other terms, that it would be necessary to project out the operator content on the subspace of physical states first or proceed with gauge-fixing.
Although this is a common and traditional approach to dualities, with bond algebras this is not strictly necessary.
As stated in \cite{cobanera2011bond}, with the bond-algebraic approach one can find mappings to models without any gauge symmetry that preserve all the important algebraic properties.

The procedure goes as follows: consider a gauge model and let $H_{\text{G}}$ be its Hamiltonian and $G_{\Gamma}$ its gauge symmetries.
Naturally, an operator is said to be gauge-invariant only if it commutes with all the $G_{\Gamma}$ and
clearly the Hamiltonian has to be gauge-invariant, hence $[H, G_{\Gamma}] = 0$.
Now let $H_{\text{GR}}$ be the dual Hamiltonian of a non-gauge model.
A \emph{gauge-reducing duality} $\Phi_{\text{GR}}$ maps $H_{\text{G}}$ onto $H_{\text{GR}}$ while making all the gauge symmetries of the former model trivial, which means:
\begin{equation}
    \Phi_{GR}(H_{\text{G}}) = H_{\text{GR}}, \qquad
    \Phi_{GR}(G_{\Gamma}) = \identity, \quad \forall \Gamma.
\end{equation}

Unlike the dualities in Sec.~\ref{sub:the_bond_algebraic_approach}, a gauge-reducing duality like $\Phi_{\text{GR}}$ has to be implementable as a \emph{projective unitary operator} $\mathcal{U}$.
Formally, this can be written as
\begin{equation}
    \Phi_{\text{GR}} ( \mathcal{O} ) = \mathcal{U} \mathcal{O} \mathcal{U}^\dagger, \quad
    \mathcal{U} \mathcal{U}^\dagger = \identity, \quad
    \mathcal{U}^\dagger \mathcal{U} = P_{\text{GI}}
\end{equation}
where $P_{\text{GI}}$ is the projector of the subspace of gauge-invariant states, i.e.~$G_{\Gamma} \ket{\psi} = \ket{\psi}$ for all $\Gamma$.
Roughly speaking, this projective unitary operator can be represented as rectangular matrix that preserves the norm of gauge-invariant states while projecting out all the other states.

A clear example of a gauge-reducing duality is provided by the $\Z_N$, $d=2$ gauge model

\begin{equation}
	H_{\text{G}} = \sum_r \left( V_{(r,\hat{1})} + V_{(r,\hat{2})} + \lambda U_r \right).
\end{equation}

Its group of gauge symmetries is generated by \eqref{eq:gauss_operator}. In the simplest case where $N=2$, the $V$'s can be represented by the Pauli operators $\sigma^z$ and the $U$'s by $\sigma^x$. In so doing, the Gauss operator becomes
\begin{equation}
	G_r =
	\sigma^z_{(r, \hat{1})}
	\sigma^z_{(r, \hat{2})}
	\sigma^z_{(r, -\hat{1})}
	\sigma^z_{(r, -\hat{2})},
	\label{eq:gauss_operator_Z2}
\end{equation}
and commutes with $H_{\text{G}}$ and with the bonds $\left\{\sigma^z_{(r,\hat{1})},\ \sigma^z_{(r,\hat{2})},\ U_r\right\} $.
In other words, the bond algebra they generate is gauge-invariant, and satisfy three simple relations: (i) all the bonds square to the identity, (ii) each spin $\sigma^z$ anti-commutes with two adjacent plaquettes operators $U$, and (iii) each plaquette operator $U$ anti-commutes with four spins $\sigma^z$.
This set of relations are identical to those satisfied by the bonds of the $d=2$ quantum Ising model, and the mapping is the following:

\begin{equation}
    \begin{split}
        \Phi\left( \sigma^z_{(r,\hat{1})} \right) & = \sigma^x_{(r-\hat{2})} \sigma^x_r, \\
        \Phi\left( \sigma^z_{(r,\hat{2})} \right) & = \sigma^x_{(r-\hat{1})} \sigma^x_r, \\
        \Phi(U_r) & = \sigma^z_r.
    \end{split}
    \label{eq:duality_2d}
\end{equation}

Thus, $\Phi$ maps $H_{\text{G}}$ to $H_{\text{Ising}}$, if we identify the constants $\lambda \leftrightarrow h$ and $1 \leftrightarrow J$.
Moreover, $\Phi$ is a gauge-reducing duality homomorphism, since $\Phi(G_r) = \identity$.
Therefore, $H_{\text{Ising}}$ represents all the physics contained in $H_{\text{gauge}}$, but without all the gauge redundancies.
In the general case, i.e.~for a generic $\Z_N$ symmetry, the duality leads to an $N$-clock model \cite{radicevic2019spin}.

% vim: spelllang=en
