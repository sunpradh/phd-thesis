\section{Dualities of the ladder LGT}%
\label{sec:dualities_of_the_ladder}

In this section we discuss the main result in \cite{pradhan2022ladder}, which is a construction of a duality map between \acp{lgt} on a \emph{ladder geometry} and \emph{\aclp{clock}}.
Before proceeding with construction of the duality map, we briefly describe what are \acp{clock}.


%
% SECTION: Clock models
%
\subsection{Clock models}%
\label{sub:clock_models}

In this section we will deal with a class of generalizations of the quantum Ising model known as \emph{clock models} \cite{fendley2014parafermions, baxter1989clock}, which shows a resemblance to the $\Z_N$ LGT models we introduced previously.
This similarity will later be exploited in order to obtain a complete description of the LGT models without any redundant gauge-symmetry.

For a discussion about clock models we start from the Hamiltonian of the quantum Ising model with a transverse field, which can simply be written as
\begin{equation}
    H = - \sum_{i} \sigma^z_i \sigma^z_{i+1} - h \sum_{i} \sigma^x_i,
    \label{eq:ising_hamiltonian_duality}
\end{equation}
where $\sigma^{x,z}_i$ are the usual $2 \times 2$ Pauli matrices for each site $i$:
\begin{equation}
    \sigma^x_i = \pmqty{ 0 & 1 \\ 1 &  0 }, \quad
    \sigma^z_i = \pmqty{ 1 & 0 \\ 0 & -1 }.
\end{equation}
They are a set of unitary matrices that commute on different sites, while on the same site they anticommute $\sigma^x \sigma^z = - \sigma^z \sigma^x$.
Another way to put it is to say that the exchange of $\sigma_x$ and $\sigma_z$ on the same site produces a phase $e^{i \pi} = -1$.

Clock models can be thought as generalizations of the quantum Ising model, but not to higher spins.
A $p$-state clock model (or simply a $p$-clock model) utilizes a set of unitary operators that generalize the algebra of Pauli matrices in the following sense:
the operators $\sigma_x$ and $\sigma_z$ get promoted to the \emph{clock operators} $X$ and $Z$, respectively, which are $p \times p$ unitary matrices whose exchange produces a phase $\omega = e^{i 2 \pi / p}$, instead of $-1$.
The algebraic properties of these clock operators $X$ and $Z$ can be summarized as follows:
\begin{equation}
    \begin{aligned}
        X Z & = \omega Z X, &
        X^p & =  Z^p = \One_p, \\
        X^\dagger & = X^{-1} = X^{p-1}, &
        Z^\dagger & = Z^{-1} = Z^{p-1}
    \end{aligned}
    \label{eq:clock_operator_algebra}
\end{equation}

We see that the Schwinger-Weyl algebra in \eqref{eq:schwinger_weyl_algebra} and the clock operator algebra in \eqref{eq:clock_operator_algebra} are basically the same, but there are some key differences to point out betweens a $\Z_N$ LGT and a $p$-clock model.

The degrees of freedom of a $\Z_N$ LGT live on the links of the lattice while in a $p$-clock model they live on the sites.
But the most important aspect is that we don't have any gauge symmetry in a $p$-clock model, hence we do not have to impose any local constraints or physical conditions.
These models can be derived as the quantum Hamiltonians of the classical 2D vector Potts model, which is a discretization of the 2D planar XY model \cite{ortiz2012dualities}.

A typical $p$-clock model Hamiltonian with transverse field has the form
\begin{equation}
    \Hclock(\lambda) = - \sum_{i} Z_i Z_{i+1} - \lambda \sum_{i} X_i + \hc
    \label{eq:clock_hamiltonian}
\end{equation}
which is, as expected, very similar to the quantum Ising Hamiltonian in \eqref{eq:ising_hamiltonian_duality}.
Furthermore, just like the latter, $p$-clock models with only transverse field are \emph{self-dual}:
the clocks can be mapped into the kinks (or domain walls) and one would obtain the same exact Hamiltonian description but with inverted transverse field \cite{ortiz2012dualities}.
For $p < 5$, the clock models presents a self dual point in $\lambda = 1$, that separates an ordered phase from a disordered one.
On the other hand, for $p \geq 5$ we have an intermediate continuous critical phase between the ordered and disordered phase with two BKT transition points, which are related to each other through the self-duality \cite{sun2019phase}.

These models have been thoroughly studied, even with the addition of a longitudinal field $\propto Z_i$ \cite{baxter1982exactlysm} or chiral interactions.
In particular, in the case of chiral interactions, it was shown \cite{fendley2012parafermions} that the Hamiltonian \eqref{eq:clock_hamiltonian} can be mapped to a parafermionic chain through a Fradkin-Kadanoff transformation, and in presence of a $\mathbb{Z}_3$ symmetry, it shows three different phases \cite{zhuang2015clock}, if open boundaries are implemented: a trivial, a topological and an incommensurate (IC) phase.
The case which presents a real longitudinal field term was considered in \cite{huang2019clock},  where some of the critical exponents have been estimated.
The general case, where chiral interactions are included in a $\mathbb{Z}_N$ model, has been studied in \cite{fendley2012parafermions}.
Here, the author considered the model as an extension of the Ising/Majorana chain and found the edge modes of the theory.
He also calculated the points, in the parameter space, where the model is integrable or `superintegrable'.
All these studies are motivated by theoretical interest and recent experiments, which can be analysed by the above models \cite{bernien2017probing}.


% vim: spelllang=en



%--------------------------------------------------
% SUBSECTION: Duality on the clock models
%--------------------------------------------------
\subsection{Gauge-reducing duality onto clock models}
\label{sub:duality_onto_clock_models}

In this section we will show one of the main result of \cite{pradhan2022ladder}: how to construct a mapping of the $\Z_N$ ladder \ac{lgt} onto a $N$-clock model on a chain with a transversal field and a longitudinal field, the latter depending on the super-selection sector of the ladder \ac{lgt}.

The first step is the decomposition of the set of bonds in \eqref{eq:hamiltonian_base}.
Obviously, the magnetic terms $U_{\square}$ have to be separated from the electric terms $V_\ell$, but the latter cannot be all treated the same.
It is clear from the geometry of the ladder, that the links $\runglink$ have a different role when compared with the links $\toplink$ and $\botlink$, because the former are \emph{domain walls} while the latter are not.
Therefore, the duality transformation has to distinguish between the vertical links and horizontal links.
Furthermore, also the top links $\toplink$ and bottom links $\botlink$ have to be treated separately because the electric operator on them have different commutation relations with the plaquette operators.
In fact, using the notation introduced in Sec.~\ref{sec:abelian_models_on_the_ladder}, we have
\begin{equation}
    U_i \Vdown_i = \omega \Vdown_i U_i, \qquad
    U_i \Vup_i = \omega^{-1} \Vup_i U_i.
    \label{eq:comm_rel_ladder}
\end{equation}
and indeed they acquire different phases.

The plan is to associate to each plaquette a clock degree of freedom, hence we identify a plaquette $\square_i$ with a site $i$ of a clock chain and the magnetic flux of a plaquette becomes the ``fundamental gauge invariant degree of freedom'' of the \ac{lgt} ladder model.
Given the fact that we are working in the electric basis, we chose for convenience to map the $\Z_N$ magnetic operator $U_i$ to the ``momentum'' operator $X_i$ of the $N$-clock chain.
The electric field on a vertical link $\runglink$ is the result of the flux difference between the two plaquettes that it separates, which suggests that the operator $\Vrung$ have to be mapped to a kinetic-type term like $Z_i^\dagger Z_{i-1}$.
This can be readily verified.
From \eqref{eq:plaq_op_ladder} we get
\begin{equation*}
    V^0_i U_i = \omega^{-1} U_i V_i^0, \qquad
    V^0_i U_{x-1} = \omega U_{x-1} V_i^0,
\end{equation*}
therefore the maps
\begin{equation*}
    U_i \mapsto X_i, \qquad
    V^0_i \mapsto Z_i^\dagger Z_{i-1},
    \label{eq:elec_h_and_plaq_op_map}
\end{equation*}
clearly conserves the commutation relations of $U_i$ and $V^0_i$.

For now we are left with task of finding a suitable mapping of $\Vup$ and $\Vdown$.
With respect to the other bonds of the theory, both of them commute with $V^0$ while for \eqref{eq:comm_rel_ladder} holds for $U_i$.
Hence, a suitable and general mapping of $\Vup$ and $\Vdown$ can be:
\begin{equation}
    \Vdown_i \mapsto \coeffdown_i Z_i, \qquad
    \Vup_i \mapsto \coeffup_i Z_i^\dagger,
    \label{eq:elec_op_horiz_ladder_map}
\end{equation}
where $\coeffdown_i$ and $\coeffup_i$ are complex numbers.
Although, they cannot be any complex number.
Both $\Vdown_i$ and $\Vup_i$ have to be mapped onto unitary operators, which limits the numbers $\coeffdown_i$ and $\coeffup_i$ to be \emph{complex phases}.

To further constraint the value of these coefficients, we can use the Gauss law.
In particular, given the fact that we are looking for a gauge-reducing duality, the aim is to make the Gauss law trivial.
Using the mappings \eqref{eq:elec_h_and_plaq_op_map} and \eqref{eq:elec_op_horiz_ladder_map} in \eqref{eq:gauss_law_ladder} yields
\begin{equation}
    \begin{split}
        G^\uparrow_i & \mapsto
        (\coeffup_i Z_i^\dagger) (\coeffup_{i-1} Z_{i-1}^\dagger) (Z_i^\dagger Z_{i-1})^\dagger
        = \coeffup_i (\coeffup_{i-1})^*, \\
        G^\downarrow_i & \mapsto
        (\coeffdown_i Z_i^\dagger) (Z_i^\dagger Z_{i-1}) (\coeffdown_{i-1} Z_{i-1}^\dagger)
        = \coeffdown_i (\coeffdown_{i-1})^*
    \end{split}
    \label{eq:gauss_law_map_ladder}
\end{equation}

Gauss law have to be satisfied in a pure gauge theory, which mean that we have to impose $G^\uparrow_i = \identity$ and $G^\downarrow_i = \identity$ for all $i$.
This is only possible if
\begin{equation}
    \coeffdown_i = \coeffdown, \qquad
    \coeffup_i = \coeffup, \qquad
    \forall i.
\end{equation}

Furthermore, thanks to \eqref{eq:gauss_law_map_ladder}  we also know how to introduce static matter into this duality, because it can be thought as a violation of the Gauss law.
We just have to change the phases $\coeffup_i$ and $\coeffdown_i$.

The last factor to consider is how the $\coeffup$ and $\coeffdown$ are related on the same site $i$.
In this regard, the super-selection sectors of the theory come to the rescue.
As established in Sec.~\ref{sec:abelian_models_on_the_ladder}, the super-selection sectors are identified by the eigenvalue of $S_2$ in \eqref{eq:nonlocal_op_ZN}, which in the ladder geometry becomes
\begin{equation}
    S_2 = \Vup_i \Vdown_i
    \label{eq:top_string_op_ladder}
\end{equation}
for any fixed $x$.
Its eigenvalue are simply $\omega^k$, for $k = 0, \dots, N-1$.

Given a super-selection sector $\omega^k$, using the mapping \eqref{eq:elec_op_horiz_ladder_map} on \eqref{eq:top_string_op_ladder} yields
\begin{equation}
    S_2 \; \longmapsto \; ( \coeffup Z^\dagger_i ) ( \coeffdown Z_i ) = \coeffup \coeffdown = \omega^k.
\end{equation}
Here, one is free to choose $\coeffup$ and $\coeffdown$, given that their product has to be equal to $\omega^k$.
This freedom corresponds to a \emph{global} symmetry of the system and it has nothing to due with the gauge symmetries, because the latter has already been solved.
We choose to fix $\coeffup$ to $1$ and $\coeffdown$ to $\omega^k$:
\begin{equation}
    \coeffup = 1, \qquad
    \coeffdown = \omega^k.
\end{equation}

In conclusion, we summarize the duality mapping for the super-selection sector $\omega^k$ of the $\Z_N$ \ac{lgt} on a ladder:
\begin{equation}
    \begin{aligned}
       U_i      & \; \longmapsto \; X_i, \quad &
        V^0_i    & \; \longmapsto \; Z^\dagger_i Z_{i-1}, \\
        \Vup_i   & \; \longmapsto \; Z_i^\dagger, \quad &
        \Vdown_i & \; \longmapsto \; \omega^k Z_i.
    \end{aligned}
    \label{eq:ladder_duality}
\end{equation}

With the duality \eqref{eq:ladder_duality}, from \eqref{eq:ladder_hamiltonian} in the sector $(\omega^k, 1)$ we obtain
\begin{equation}
    \HamilLadder(\lambda) \; \longmapsto \; \lambda \HamilDual(\lambda^{-1})
\end{equation}
where $\HamilDual$ is the dual $N$-clock Hamiltonian:
\begin{equation}
    \HamilDual(\lambda^{-1}) =
    - \sum_{i} \qty(
        Z_i^\dagger Z_{i-1}
        + \lambda^{-1}  X_i
        + (1 + \omega^k)  Z_i
        + \text{h.c.}
    )
    \label{eq:dual_ladder_hamiltonian}
\end{equation}

We see that \eqref{eq:dual_ladder_hamiltonian} is a clock model with both \emph{transversal} and \emph{longitudinal} fields.
In particular, the longitudinal field carries the information of the super-selection sector of the ladder model.

Interestingly, for $N$ even the sector $k = N/2$ has a special role.
Within this sector $\omega^k = -1$, for which the \emph{longitudinal field disappears} and $\HamilClock$ reduces to self-dual quantum clock models with a known quantum phase transition.
This phase transitions for $k = N/2$ can be put in correspondence with a \emph{confined-deconfined} transition, which will be discussed in much more detail in the next section.

Let us remark that the complex coupling $(1 + \omega^n)$ does not make the Hamiltonian  (\ref{eq:dual_ladder_hamiltonian}) necessarily chiral \cite{fendley2012parafermions, whitsitt2018clock}.
In fact, one can get the real Hamiltonian
\begin{equation}
    \HamilDual(\lambda^{-1}) = \HamilClock(\lambda^{-1}) - 2 \cos \pqty{\frac{\pi n}{N}} \sum_{i} \pqty{Z_i + Z_i^{\dagger}}.
    \label{eq:dual_ladder_hamiltonian_real}
\end{equation}
by absorbing the complex phase in the $Z_i$-operators, with the transformation $Z_i \mapsto \omega^{-n/2} Z_i$. This transformation globally rotates the eigenvalues of the $Z_i$-operators, while preserving the algebra relations.
For $n$ even, this is just a permutation of the eigenvalues, meaning that it does not affect the Hamiltonian spectrum.
Instead, for $n$ odd, up to a reorder, the eigenvalues are shifted by an angle $\pi/N$, i.e.~half the phase of $\omega$.
In the latter case we will denote the rotated $Z_i$ operator with $\tilde{Z}_i$.
The energy contribution of the extra term in \eqref{eq:dual_ladder_hamiltonian_real}  depends on the real part of these eigenvalues and for $n$ odd we obtain that the lowest energy state is no longer unique, in fact it is doubly degenerate.
This means that for $\lambda \to \infty$, where the extra term becomes dominant, we expect an ordered phase with a doubly degenerate ground state.
Finally, one can easily prove that the sectors $n$ and $N-n$ are equivalent
\footnote{For the sector $N-n$ we have that the overall factor $\cos(\pi(N-n)/N)$ is just $-\cos(\pi n/N)$.
The minus sign can then be again absorbed into the $Z$'s operators.
This overall operation is equivalent to the mapping $Z \mapsto \omega^{-n/2} Z$ for the sector $N-n$.}.


\begin{figure}[t]
    \centering
    \begin{tikzpicture}

    % ladder
    \draw[ladder] (-0.5,0) grid (7.5,1);

    % Plaquette operator
    \draw[U] (1,0) -- (2,0);
    \draw[U] (2,0) -- (2,1);
    \draw[U] (2,1) -- (1,1);
    \draw[U] (1,1) -- (1,0);
    \draw[box] (0.75,-0.25) rectangle (2.25, 1.25) node [above left, Blue] {$U_x$};

    % horizontal electric operators
    \draw[V] (4,0) -- (5,0);
    \draw[V] (4,1) -- (3,1);
    \draw[box] (3.75,-0.25) rectangle (5.25, 0.25) +(-0.25,0) node [above left, Red] {$\Vdown_x$};;
    \draw[box] (2.75, 0.75) rectangle (4.25, 1.25) node [above left, Red] {$\Vup_x$};

    % vertical electric operator
    \draw[V] (6,0) -- (6,1);
    \draw[box] (5.75,-0.25) rectangle (6.25, 1.25) node [above, Red] {$V^0_x$};

    % ladder sites
    \DrawSites{0,...,7}{0,1};

    % chain
    \draw[ladder] (-0,-2) -- (7,-2);

    % clock operators
    \draw (1.5,-2) node [X site] {} node [text=Blue, below=5pt] {$X_i$};
    \draw (3.5,-2) node [Z site] {} node [text=Red, below=5pt] {$Z^\dagger_i$};
    \draw (4.5,-2) node [Z site] {} node [text=Red, below=5pt] {$\omega^k Z_i$};
    \draw [Z] (5.5,-2) -- (6.5,-2) node [pos=0.5, below=5pt] {$Z^\dagger_i Z_{i-1}$};

    % chain sites
    \foreach \x in {0,2,5,6} \draw (\x+0.5,-2) node [dual site] {};

    % arrows
    \draw[freccia] (1.5,-0.25) -- (1.5, -2);
    \draw[freccia] (3.5, 0.75) -- (3.5, -2);
    \draw[freccia] (4.5,-0.25) -- (4.5, -2);
    \draw[freccia] (6.0,-0.25) -- (6.0, -2);

    % labels
    \draw (-0.75,0.5) node [left, align=right] {$\Z_N$ ladder \\ LGT};
    \draw (-0.75, -2) node [left, align=right] {$N$-clock \\ chain};
\end{tikzpicture}

    \caption[Duality map of $\Z_N$ ladder \ac{lgt}]{Visual representation of the duality transformation from the $\Z_N$ ladder \ac{lgt} to the $N$-clock model.
    The plaquette operator $U_i$ and the electric operators $\Vup$ and $\Vdown$ map to one-site operators in the clock model, while
    the remaining electric operator $V^0$ maps to a hopping term between nearest neighbouring sites.}
    \label{fig:ladder_duality}
\end{figure}

