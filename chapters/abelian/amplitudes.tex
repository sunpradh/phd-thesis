\subsubsection{Distribution of the amplitudes of the ground state}
\label{ssub:amplitudes_distribution}

In the $N=2$ case, we further differentiate the phase diagrams of the two sectors by looking at the ground state amplitudes distribution, for $\lambda<1$ and $\lambda>1$.
Obviously, the ground state can be written as a superposition of the gauge invariant states of $\Hphys$ in the given sector
\begin{equation}
    \ket{\Psi_{\text{g.s.} }}= \sum_n c_n \ket{n},
    \label{eq:gs_amplitudes}
\end{equation}
The basis $\ket{n}$ and the amplitudes $c_n$ are sorted in a decreasing order with respect to the modulus of the latter.
The first state of the list, with amplitude $c_1$, is always the Fock vacuum $\ket{\Omega}$ of the sector, hence we consider the distribution of the ratios $\abs{c_n / c_1}$, which are plotted in Fig.~\ref{fig:gs_ampl_distr_0.1_Z2}--\ref{fig:gs_ampl_distr_1.5_Z2} for $\lambda=0.1$ and $\lambda=1.5$, respectively.
The most interesting one is at $\lambda = 0.1$, where the difference between the deconfined phase in the sector $n=1$ and the confined one in the sector $n=0$ can be seen.
In particular, in the deconfined phase the ground state is a superposition of deformations of the Fock vacuum, i.e~the non-contractible electric string, which can be thought as a \emph{kink condensate} \cite{fradkin1978order} (or as a paramagnetic phase), where each kink corresponds to a deformation of the string.
Meanwhile, for $\lambda > 1$, where we have confinement in both sectors, the ground state is essentially a product state, akin to a ferromagnetic state.
This is explained in Fig.~\ref{fig:gs_ampl_distr_0.1_Z2} and Fig.~\ref{fig:gs_ampl_distr_1.5_Z2}.



\begin{figure}[h]
    \centering
    \hspace{3em}$\Z_2$ g.s.~amplitudes distribution, $\lambda=0.1$ \\[5pt]
    \begin{tikzpicture}[
    lattice/.style = {Gray, thin, solid},
    on/.style = {Green, very thick, solid},
    lab/.style = {scale=0.35},
    box/.style = {draw=black, dotted, inner sep=4pt},
    arr/.style = {<-, black},
    flux/.style = {fill=Green, fill opacity=0.1}
    ]

\begin{axis}[
        height=7cm,
        width=8cm,
        tick align=outside,
        tick pos=left,
        xmajorticks=false,
        ylabel={$|c_n/c_1|$},
        ymin=0, ymax=1.1,
        ytick style={color=black},
        ytick={0,0.2,0.4,0.6,0.8,1,1.2},
        yticklabels={$0.0$, $0.2$, $0.4$, $0.6$, $0.8$, $1.0$, $1.2$}
]
\addplot [very thick, blue!70]
table {%
0 1
1 0.72670304775238
12 0.72670304775238
13 0.581751346588135
24 0.581751346588135
25 0.529128313064575
36 0.529128313064575
37 0.52813982963562
78 0.52809739112854
79 0.466958403587341
90 0.466958403587341
91 0.423632383346558
114 0.423632383346558
115 0.422796964645386
186 0.422760725021362
187 0.385271906852722
198 0.385271906852722
199 0.384550213813782
};

% Title
\node [anchor=north east, draw=gray] at (axis description cs: 0.99, 0.99) (title) {sector $(0,0)$};

\node at (0, 1) (vacuum) {};
\node at (6, 0.72670304775238)  (1-1-loop) {};
\node at (16, 0.581751346588135) (1-2-loop) {};
\node at (50, 0.529128313064575) (2-1-loop) {};
\node at (83, 0.466958403587341) (1-3-loop) {};
\node at (130, 0.423632383346558) (1-2-1-1-loop) {};
\node at (192, 0.385271906852722) (3-1-loop) {};

\end{axis}

% Vacuum
\draw[arr] (vacuum) node  [circle, fill=black, inner sep=0pt, minimum size=3pt] {}
    -- +(1,0) node (vacuum-label) {};
\draw (vacuum-label) node [right, box] {
    \begin{tikzpicture}[lab]
        \draw[lattice]  (-0.5, 0) grid (5.5, 1);
    \end{tikzpicture}
};

% 1 single plaquette loop
\draw[arr] (1-1-loop) -- +(1,0.7) node (1-1-loop-label) {};
\draw (1-1-loop-label) node [right, box] {
    \begin{tikzpicture}[lab]
        \draw[lattice]  (-0.5, 0) grid (5.5, 1);
        \fill[flux] (2,0) rectangle (3,1);
        \draw[on] (2,0) rectangle (3,1);
    \end{tikzpicture}
};

% 1 double plaquette loop
\draw[arr] (1-2-loop) -- +(0.6,0.6) node (1-2-loop-label) {};
\draw (1-2-loop-label) node [right, box] {
    \begin{tikzpicture}[lab]
        \draw[lattice]  (-0.5, 0) grid (5.5, 1);
        \fill[flux] (1,0) rectangle (3,1);
        \draw[on] (1,0) rectangle (3,1);
    \end{tikzpicture}
};


% 2 single plaquette loops
\draw[arr] (2-1-loop) -- +(-0.2, -0.8) node (2-1-loop-label) {};
\draw (2-1-loop-label) node [below, box] {
    \begin{tikzpicture}[lab]
        \draw[lattice]  (-0.5, 0) grid (5.5, 1);
        \fill[flux] (0,0) rectangle (1,1);
        \fill[flux] (3,0) rectangle (4,1);
        \draw[on] (0,0) rectangle (1,1);
        \draw[on] (3,0) rectangle (4,1);
    \end{tikzpicture}
};

% 1 triple plaquette loops
\draw[arr] (1-3-loop) -- +(1,0.6) node (1-3-loop-label) {};
\draw (1-3-loop-label) node [right, box] {
    \begin{tikzpicture}[lab]
        \draw[lattice]  (-0.5, 0) grid (5.5, 1);
        \fill[flux] (1,0) rectangle (4,1);
        \draw[on] (1,0) rectangle (4,1);
    \end{tikzpicture}
};

% 1 double 1 single plaquette loops
\draw[arr] (1-2-1-1-loop) -- +(-0.1,-0.6) node (1-2-1-1-loop-label) {};
\draw (1-2-1-1-loop-label) node [below, box] {
    \begin{tikzpicture}[lab]
        \draw[lattice]  (-0.5, 0) grid (5.5, 1);
        \fill[flux] (1,0) rectangle (2,1);
        \fill[flux] (3,0) rectangle (5,1);
        \draw[on] (1,0) rectangle (2,1);
        \draw[on] (3,0) rectangle (5,1);
    \end{tikzpicture}
};

% 3 single plaquette loops
\draw[arr] (3-1-loop) -- +(-0.1,-1.2) node (3-1-loop-label) {};
\draw (3-1-loop-label) node [below left, box] {
    \begin{tikzpicture}[lab]
        \draw[lattice]  (-0.5, 0) grid (5.5, 1);
        \fill[flux] (0,0) rectangle +(1,1);
        \fill[flux] (2,0) rectangle +(1,1);
        \fill[flux] (4,0) rectangle +(1,1);
        \draw[on] (0,0) rectangle +(1,1);
        \draw[on] (2,0) rectangle +(1,1);
        \draw[on] (4,0) rectangle +(1,1);
    \end{tikzpicture}
};



\end{tikzpicture}
\\[-2pt]\hspace{0.4pt}
    \begin{tikzpicture}[
    lattice/.style = {Gray, thin, solid},
    on/.style = {Green, very thick, solid},
    lab/.style = {scale=0.35},
    box/.style = {draw=black, dotted, inner sep=4pt},
    arr/.style = {<-, black},
    flux/.style = {fill=Green, fill opacity=0.1}
    ]

    \begin{axis}[
        height=7cm,
        width=9cm,
        tick align=outside,
        tick pos=left,
        % title={sector $(0, 0)$},
        xlabel={$n$},
        ylabel={$|c_n/c_1|$},
        ymin=0, ymax=1.1,
        ytick style={color=black},
        ytick={0,0.2,0.4,0.6,0.8,1,1.2},
        yticklabels={$0.0$, $0.2$, $0.4$, $0.6$, $0.8$, $1.0$, $1.2$}
        ]
        \addplot [very thick, blue!70]
        table {%
            0 1
            1 1
            2 0.902501583099365
            25 0.902501583099365
            26 0.90012514591217
            133 0.900000095367432
            134 0.816542267799377
            157 0.816542267799377
            158 0.814608097076416
            199 0.814515113830566
        };

        % Title
        \node [anchor=north east, draw=gray] at (axis description cs: 0.99, 0.99) (title) {sector $n=1$};

        \node at (0.5, 1) (vacuum) {};
        \node at (10, 0.902501583099365) (1-1-loop) {};
        \node at (70, 0.900000095367432) (1-big-loop) {};
        \node at (160, 0.816542267799377) (2-1-loop) {};
    \end{axis}

    % Vacuum
    \draw[arr] (vacuum) node  [circle, fill=black, inner sep=0pt, minimum size=3pt] {}
    -- +(1,0) node (vacuum-label) {};
    \draw (vacuum-label) node [right, box] {
        \begin{tikzpicture}[lab]
            \draw[lattice]  (-0.5, 0) grid (5.5, 1);
            \draw[on] (-0.5,0) -- (5.5, 0);
        \end{tikzpicture}
    };

    % 1 single plaquette loop
    \draw[arr] (1-1-loop) -- +(0.5,-0.5) node (1-1-loop-label) {};
    \draw (1-1-loop-label) node [below, box] {
        \begin{tikzpicture}[lab]
            \draw[lattice]  (-0.5, 0) grid (5.5, 1);
            \fill[flux] (2,0) rectangle (3,1);
            \draw[on] (-0.5, 0) -- (2, 0) -- (2, 1) -- (3, 1) -- (3, 0) -- (5.5, 0);
        \end{tikzpicture}
    };



    % 1 big loop
    \draw[arr] (1-big-loop) -- +(0.0,-1.5) node (1-big-loop-label) {};
    \draw (1-big-loop-label) node [below, box] {
        \begin{tikzpicture}[lab]
            \draw[lattice]  (-0.5, 0) grid (5.5, 1);
            \draw[on] (-0.5, 0) -- (2, 0) -- (2, 1) -- (4, 1) -- (4, 0) -- (5.5, 0);
            \fill[flux] (2,0) rectangle (4,1);
            \begin{scope}[yshift=-1.5cm]
                \draw[lattice]  (-0.5, 0) grid (5.5, 1);
                \draw[on] (-0.5, 0) -- (1, 0) -- (1, 1) -- (4, 1) -- (4, 0) -- (5.5, 0);
                \fill[flux] (1,0) rectangle (4,1);
            \end{scope}
            \begin{scope}[yshift=-3.0cm]
                \draw[lattice]  (-0.5, 0) grid (5.5, 1);
                \draw[on] (-0.5, 0) -- (1, 0) -- (1, 1) -- (5, 1) -- (5, 0) -- (5.5, 0);
                \fill[flux] (1,0) rectangle (5,1);
            \end{scope}
            \node at (2.5, -3) [font=\footnotesize, inner sep=0pt] {$\vdots$};
        \end{tikzpicture}
    };


    % 2 single plaquette loops
    \draw[arr] (2-1-loop) -- +(-0.2,-1.0) node (2-1-loop-label) {};
    \draw (2-1-loop-label) node [below, box] {
        \begin{tikzpicture}[lab]
            \draw[lattice]  (-0.5, 0) grid (5.5, 1);
            \fill[flux] (1,0) rectangle (2,1);
            \fill[flux] (3,0) rectangle (4,1);
            \draw[on] (-0.5, 0) -- (1, 0) -- (1, 1) -- (2, 1) -- (2, 0) -- (3,0) -- (3,1) -- (4,1) -- (4,0) -- (5.5, 0);
        \end{tikzpicture}
    };


\end{tikzpicture}

    \caption[$\Z_2$ ground state amplitude distribution for $\lambda = 0.1$]{
        $\Z_2$ ground state amplitude distribution for $\lambda=0.1$ of the first 200 states and with lattice size $12 \times 2$.
        \emph{Top}: distribution of the ratios $|{c_n/c_1}|$ for the sector $n=0$ (see \eqref{eq:gs_amplitudes}).
        We see that the heaviest states that enters the ground state, apart from the vacuum that sets the scale, are made of small electric loops, typical of a confined phase.
        \emph{Bottom}: the same distribution of ratios for the sector $n=1$.
        We see that the heaviest states are made of bigger and bigger deformations of the electric string that goes around the ladder.
        This happens because the energy contributions depends only on the domain walls between two plaquettes with different flux content.
        This behaviour is similar to the so-called \emph{kink condensation} in spin chains \cite{fradkin1978order}, where the disordered state can be expressed as a superposition of all possible configuration of kinks (i.e.~domain walls between two differently ordered regions).
        In the language of the duality, this deconfined phase then maps to the paramagnetic phase of the quantum Ising model with \emph{only} transverse field.
    }
    \label{fig:gs_ampl_distr_0.1_Z2}
\end{figure}



\begin{figure}[h]
    \centering
    \hspace{3em}$\Z_2$ g.s.~amplitudes distribution, $\lambda=1.5$\\[5pt]
    \begin{tikzpicture}[
    lattice/.style = {Gray, thin, solid},
    on/.style = {Green, very thick, solid},
    lab/.style = {scale=0.35},
    box/.style = {draw=black, dotted, inner sep=4pt},
    arr/.style = {<-, black},
    flux/.style = {fill=Green, fill opacity=0.1},
    font=\small,
    ]

    \begin{axis}[
        height=5cm,
        width=8cm,
        tick align=outside,
        tick pos=left,
        xmajorticks=false,
        ylabel={$|c_n/c_1|$},
        ymin=0, ymax=1.1,
        ytick style={color=black},
        ytick={0,0.2,0.4,0.6,0.8,1,1.2},
        yticklabels={$0.0$, $0.2$, $0.4$, $0.6$, $0.8$, $1.0$, $1.2$}
        ]
        \addplot [very thick, blue!70]
        table {%
            0 1
            1 0.0831366777420044
            12 0.0831366777420044
            13 0.00918400287628174
            24 0.00918400287628174
            25 0.00691556930541992
            78 0.0069117546081543
            79 0.0010453462600708
            186 0.000763535499572754
            189 0.00057530403137207
            199 0.000574946403503418
        };

        % Title
        \node [anchor=north east, draw=gray] at (axis description cs: 0.99, 0.99) (title) {sector $(0,0)$};

        \node at (0, 1) (vacuum) {};
    \end{axis}

    % Vacuum
    \draw[arr] (vacuum) node  [circle, fill=black, inner sep=0pt, minimum size=3pt] {}
    -- +(1,0) node (vacuum-label) {};
    \draw (vacuum-label) node [right, box] {
        \begin{tikzpicture}[lab]
            \draw[lattice]  (-0.5, 0) grid (5.5, 1);
        \end{tikzpicture}
    };

\end{tikzpicture}
\\[-2pt]\hspace{0.4pt}
    \begin{tikzpicture}[
    lattice/.style = {Gray, thin, solid},
    on/.style = {Green, very thick, solid},
    lab/.style = {scale=0.35},
    box/.style = {draw=black, dotted, inner sep=4pt},
    arr/.style = {<-, black},
    flux/.style = {fill=Green, fill opacity=0.1}
    ]

    \begin{axis}[
        height=5cm,
        width=8cm,
        tick align=outside,
        tick pos=left,
        % title={sector $(0, 0)$},
        xlabel={$n$},
        ylabel={$|c_n/c_1|$},
        ymin=0, ymax=1.1,
        ytick style={color=black},
        ytick={0,0.2,0.4,0.6,0.8,1,1.2},
        yticklabels={$0.0$, $0.2$, $0.4$, $0.6$, $0.8$, $1.0$, $1.2$}
        ]
        \addplot [very thick, blue!70]
        table {%
            0 1
            1 1
            2 0.172052025794983
            25 0.172052025794983
            26 0.0594711303710938
            49 0.0594711303710938
            50 0.0305156707763672
            73 0.0305156707763672
            74 0.0297093391418457
            157 0.0296221971511841
            158 0.0258673429489136
            181 0.0258673429489136
            182 0.0130573511123657
            199 0.0130573511123657
        };

        % Title
        \node [anchor=north east, draw=gray] at (axis description cs: 0.99, 0.99) (title) {sector $(1,0)$};

        \node at (0, 1) (vacuum) {};

    \end{axis}

    % Vacuum
    \draw[arr] (vacuum) node  [circle, fill=black, inner sep=0pt, minimum size=3pt] {}
    -- +(1,0) node (vacuum-label) {};
    \draw (vacuum-label) node [right, box] {
        \begin{tikzpicture}[lab]
            \draw[lattice]  (-0.5, 0) grid (5.5, 1);
            \draw[on] (-0.5,0) -- (5.5, 0);
        \end{tikzpicture}
    };

\end{tikzpicture}

    \caption[$\Z_2$ ground state amplitude distribution for $\lambda = 1.5$]{
        $\Z_2$ ground state amplitude distribution for $\lambda=1.5$ of the first 200 states and with lattice size $12 \times 2$.
        \emph{Top}: distribution of the ratios $|{c_n/c_1}|$ for the sector $n=0$ (see \eqref{eq:gs_amplitudes}).
        \emph{Bottom}: the same distribution of ratios for the sector $n=1$.
        For both sectors $n=0$ (\emph{top}) and $n=1$ (\emph{bottom}) we are in a confined phase, which corresponds to a ferromagnetic phase in the dual model (the \ac{ising}).
        Here we see a polarized state where the domain walls are suppressed and the ground state is essentially a product state.
    }
    \label{fig:gs_ampl_distr_1.5_Z2}
\end{figure}


