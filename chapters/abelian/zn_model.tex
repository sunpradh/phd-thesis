%----------------------------------------
% SECTION:
%----------------------------------------
\section{Generalization to \texorpdfstring{$\Z_N$}{Z\_N}}%
\label{sec:generalization_to_zn}
In this section we are going to generalize the previous case of a $\Z_2$ LGT to a class of Abelian LGT with a discrete symmetry $\Z_N$.
This class is of particular interest because they approximate a $U(1)$ gauge theory.



%
% SUBSECTION: Schwinger-Weyl algebra
%
\subsection{Schwinger-Weyl algebra}%
\label{sub:schwinger_weyl_algebra}

According to Wilson's Hamiltonian approach to lattice gauge theories \cite{wilson1974confinement}, $U(1)$ gauge fields are defined on the links of a lattice $\lattice$  either in a pair of conjugate variables, the electric field  $E_\link$ and either the vector potential $A_\link$, satisfying
\begin{equation}
    \comm{E_\link}{A_{\link^{\prime}}}  = i \delta_{\link , \link^{\prime} },
\end{equation}
or equivalently the \emph{magnetic operator}, also called \emph{comparator},
$U_\link = e^{-i A_{\link} }$, such that
\begin{equation}
    \comm{E_\link}{U_{\link^{\prime}}} =  \delta_{\link , \link^{\prime} } \, U_{\link},
\end{equation}
all acting on an infinite dimensional Hilbert space defined on each link.
This form of the canonical commutation relations represents the infinitesimal version of the relations:
\begin{equation}
     e^{i\xi E} e^{-i\eta A } e^{-i\xi E} = e^{i\xi \eta} e^ {-i\eta A },
\end{equation}
for any $\xi, \eta \in \mathbb{R}$,
that define the Schwinger-Weyl group \cite{notarnicola2015discrete, ercolessi2018znmodels, schwinger1960unitary}.

For a discrete group like $\Z_N$, the notion of infinitesimal generators loses any meaning and we are led to directly consider, for each link $\link \in \lattice$, two unitary operators
$V_\link, \, U_\link$, such that \cite{schwinger1960unitary, schwinger2001symbolism}
\begin{equation}
    V_\link U_\link V_\link^{\dagger}=e^{2\pi i/N}U_\link, \qquad
    U_\link^N=\identity_N, \qquad
    V_\link^N=\identity_N.
    \label{eq:schwinger_weyl_algebra}
\end{equation}
while they commute on different links.
Thus, by representing $\Z_N$  with the set of the $N$ roots of unity $e^{i 2 \pi k/N}$\, ($k=1, \cdots, N$), commonly referred to as the discretized circle,
we see that $V$ plays the role of a ``position operator'' on the discretized circle, while $U$ that of a ``momentum operator''.


\begin{figure}[t]
    \FigureSideCaption[eq:link_operator_ZN]{%
        \begin{tikzpicture}[scale=1.1]
    \draw (1,0) arc (0:360:1);
    \foreach \angle in {72, 144, ..., 360}
    \node[site, fill=Yellow, minimum size=7pt] (state \angle) at (\angle:1) {};
    \draw (state 360.east) node[right] {$1$};
    \draw (state 72.north)  node[above] {$\omega$};
    \draw (state 144.west) node[above left] {$\omega^2$};
    \draw (state 216.west) node[left] {$\omega^{3}$};
    \draw (state 288.south) node[below] {$\omega^{4}$};

    \draw[Z, -stealth, shorten >=5pt] (0,0) -- (72:1) node [pos=0.5, left, black] {$V$};

    \draw[X, -stealth] (1.5, 0) arc (0:40:1.5) node [pos=0.5, right, black] {$U$};

    \draw (0,0) node[site, fill=black, draw=black] {};

\end{tikzpicture}

    }{%
        The operators $U$ and $V$ of a single link, in the $\Z_5$ case. The $V$ plays the role of a position operator, while $U$ that of a shift operator.
    }
\end{figure}


These algebraic relations admit a faithful finite-dimensional representation of dimension $N$ \cite{weyl1950theory}, for any integer $N$, which is obtained as follows.
To each link $\link$, we can associate an $N$-dimensional Hilbert space $\mathcal{H}_\link$ generated by an orthonormal basis $\{\ket{v_{k,\link}}\}$ ($k=1, \dots,N$), called the \emph{electric basis}, that diagonalizes $V_\link$.
With this choice, we can promptly write the actions of $U_\link$ and $V_\link$:
\begin{equation}
    \begin{split}
        U\ket{v_{k,\link}}           = \ket{v_{k+1,\link}}, & \qquad
        U\ket{v_{N,\link}}           = \ket{v_{1,\link}}\\
        U^{\dagger}\ket{v_{k,\link}} = \ket{v_{k-1,\link}}, & \qquad
        U^\dagger\ket{v_{1,\link}}   = \ket{v_{N,\link}}\\
        V\ket{v_{k,\link}}           = \omega^k \ket{v_{k,\link}}, & \qquad
        V^\dagger\ket{v_{k,\link}}   = \omega^{-k} \ket{v_{k,\link}}.
    \end{split}
    \label{eq:elect_basis_op_action}
\end{equation}
where $\omega = e^{2 \pi i / N}$ and $k = 0, \dots, N-1$.
We choose to work in this particular basis and the various $k$ can be interpreted as the quantized values of the electric field on the links.
With this choice, $U$ and $V$ in matrix form are written as
\begin{equation}
    U =
    \setlength\arraycolsep{5pt}
    \begin{pmatrix}
        0      & 0      & \cdots & \cdots & 1      \\
        1      & 0      & \cdots & \cdots & 0      \\[-5pt]
        0      & 1      & \ddots &        & . \\[-5pt]
        \vdots & \vdots & \ddots & \ddots & \vdots \\
        0      & 0      & \cdots & 1      & 0
    \end{pmatrix}
    \qand
    \setlength\arraycolsep{2pt}
    V =
    \begin{pmatrix}
        1 \\
        & \omega \\
        &         & \omega^2 \\[-7pt]
        &         &           & \ddots \\[-2pt]
        &         &           &         & \omega^{N-1}
    \end{pmatrix}.
\end{equation}

On a  two-dimensional square lattice of size $L \times L$, the links $\link$ of the lattice can also be labeled with $(x, \pm\hat{i})$, where $x \in \lattice$ is a site and
$\hat{i}=\hat{1}, \hat{2}$ the two independent unit vectors.
In this way, $(x, \pm\hat{i})$ will refer to the link that start in $x$ and goes in the positive (negative) direction~$\hat{i}$ (see Fig.~\ref{fig:link_labels}).
This notation will be simplified when we reduce to the ladder case.


\begin{figure}
    \FigureSideCaption[fig:link_labels]{%
    \input{assets/figures/link_labels.tex}
    }{%
        Labelling of the sites and the links in the two dimensional lattice.
        A site is labeled simply with $x = (x_1, x_2)$, while $\hat{1} = (1,0)$ and $\hat{2} = (0,1)$ stand for the unit vectors of the lattice.
        A link $\link$ is denoted with a pair $(x, \pm\hat{i})$, with $\hat{i} = \hat{1}, \hat{2}$.
    }
\end{figure}


\subsection{Gauge invariance and physical states}%
\label{sub:gauge_invariance}

Gauge transformations act on vector potentials while preserving the electric field.
For a $U(1)$ gauge theory, a local phase transformation is induced by a real function $\alpha_x$
defined on the vertices $x\in \mathbb L$, such that  $A_{\link} \rightarrow A_{\link} + (\alpha_{x_2} - \alpha_{x_1})$ or equivalently $U_{\link} \rightarrow  e^{i(\alpha_{x_2} - \alpha_{x_1})
E_{\link}}  U_{\link}   e^{-i(\alpha_{x_2} - \alpha_{x_1} )E_{\link}} $, where $x_1,x_2$ are the initial and final vertices of the (directed) link $\link$.
In the case of a discrete symmetry, a gauge transformation at a site $x \in \lattice$ is a product of $V$'s (and $V^\dagger$'s) defined on the links which comes out (and enters) the vertex.
More specifically, for a two dimensional lattice,
if the link $\link$ at site $x$ is oriented in the positive direction, i.e.~either $(x, +\hat{1})$ or $(x, +\hat{2})$, then $V$ is used, otherwise $V^\dagger$.
Thus,
the single local gauge transformation at the site $x$ is enforced by the operator:
\begin{equation}
    G_x =
    V_{(x, \hat{1})}^{\phantom{\dagger}}
    V_{(x, \hat{2})}^{\phantom{\dagger}}
    V^\dagger_{(x, -\hat{1})}
    V^\dagger_{(x, -\hat{2})},
    \label{eq:gauss_operator}
\end{equation}
as shown in the left part of in Fig. \ref{fig:star_plaq_operators}.

The whole operator algebra $\mathcal{A}$ of the theory is generated by the set of all $U_\link$ and $V_\link$ (and their Hermitian conjugates) of all the links of the lattice $\lattice$, while
the  \emph{gauge-invariant subalgebra} $\mathcal{A}_{\gi}$ consists of operators that commutes with all the $G_x$:
\begin{equation}
    \mathcal{A}_\gi = \{ O_{\gi} \in \mathcal{A} \;:\; [O_{\gi}, G_x] = 0 \quad \forall x \in \lattice \}.
\end{equation}
Using \eqref{eq:gauss_operator} and recalling \eqref{eq:schwinger_weyl_algebra}, it is possible to see that the $V_\link$'s commute with $G_x$ (as expected), while the $U_\link$'s do not.
In spite of that, we can build gauge-invariant operators out of the comparators $U_\link$.
Consider a \emph{plaquette} $\square$ of the lattice $\lattice$ at $x$, by which we mean the face of the lattice with vertices $\{x, x+\hat{1}, x+\hat{1}+\hat{2}, x+\hat{2}\}$ in the counterclockwise order, as shown in the right part of Fig.~\ref{fig:star_plaq_operators}.
On this plaquette, the operator $U_{\square}$ is defined as
\begin{equation}
    U_{\square} =
    U_{(x, \hat{1})}
    U_{(x + \hat{1}, \hat{2})}
    U_{(x + \hat{1} + \hat{2}, -\hat{1})}^\dagger
    U_{(x + \hat{2}, -\hat{2})}^\dagger.
    \label{eq:plaq_operator}
\end{equation}
and one finds out that the $U_{\square}$'s commute with $G_x$, for all $x\in \mathbb L$, thus giving a generator of $\mathcal{A}_\gi$.

The set $\{U_{\square}, V_\link\}$ (for all plaquettes $\square$ and all links $\link$) may not be enough to generate the whole algebra $\mathcal{A}_\gi$, in case of periodic boundary conditions.
In order to prove this, consider a lattice $\lattice$, periodic in both dimensions, and denote with $\mathcal{C}_1$ and $\mathcal{C}_2$  any two \emph{non-contractible loops} around the lattice, that extends along the $\hat{1}$ and $\hat{2}$ direction respectively.
Then, define the (Wilson loop) operators $\overline{W}_1$ and $\overline{W}_2$ (pictured in blue in Fig.~\ref{fig:nonlocal_operators}):
\begin{equation}
    \overline{W}_i = \prod_{\link \in \mathcal{C}_i} U_\link, \qquad i=1,2.
    \label{eq:top_wilson_loop}
\end{equation}
A simple calculation shows that both $\overline{W}_1$ and $\overline{W}_2$ commute with all $G_x$, thus they are gauge-invariant, but one also finds out that none of them can be written as a product of $U_{\square}$ nor $V_\link$.
Therefore they have to be added explicitly to the set of generators of $\mathcal{A}_\gi$ in order to obtain the whole algebra.
These operators $\overline{W}_1$ and $\overline{W}_2$ play a fundamental role in the model to define topological sectors of the theory, as we will see later.

The total Hilbert space $\mathcal{H}^{\text{tot}}$ is given by the $\otimes_{\link} \mathcal{H}_{\link}$.
A state of the whole lattice $\ket{\Psi_{\text{ph}}} \in \mathcal{H}^{\text{tot}}$ is said to be \emph{physical} if it is a \emph{gauge-invariant state}:
\begin{equation}
    G_x \ket{\Psi_{\text{ph}}} = \ket{\Psi_{\text{ph}}}, \qquad \forall x \in \lattice
    \label{eq:gauss_law}
\end{equation}
This condition can be translated into a constraint on the eigenvalues $v_{(x, \pm \hat{i})}= \omega^{k_{(x, \pm \hat{i})} }$  of the operators $V_\link$ on the links $\link = (x, \pm \hat{i})$ of the vertex $x$:
\begin{equation}
    v_{(x, \hat{1})}^{\phantom{\ast}}
    v_{(x, \hat{2})}^{\phantom{\ast}}
    v_{(x, -\hat{1})}^\ast
    v_{(x, -\hat{2})}^\ast = 1,
\end{equation}
or, because of (\ref{eq:elect_basis_op_action}):
\begin{equation}
    \sum_{i=1,2} \pqty{ k_{(x, \hat{i})} - k_{(x, -\hat{i})} } = 0 \quad \text{mod $N$}.
    \label{eq:gauss_law_elec_eigvals}
\end{equation}
Given the fact that the $k$ in \eqref{eq:schwinger_weyl_algebra} represent the values of the electric field, one can see that \eqref{eq:gauss_law_elec_eigvals} can be interpreted as a discretized version of the Gauss law $\nabla \cdot \vec{E} = 0$ in two dimensions,
for a pure gauge theory where there are no electric charges.


\begin{figure}[t]
    \centering
    \begin{tikzpicture}[scale=1.2]
    % Lattice
    \draw[Gray,thin] (-0.5,-0.5) grid (4.5,2.5);

    % Plaquette operator
    \draw[U] (3,1) -- (4,1) node [pos=0.5, below] {$U$};
    \draw[U] (4,1) -- (4,2) node [pos=0.5, right] {$U$};
    \draw[U] (4,2) -- (3,2) node [pos=0.5, above] {$U^\dagger$};
    \draw[U] (3,2) -- (3,1) node [pos=0.5, left]  {$U^\dagger$};
    \draw[Blue, ultra thick, pattern=north east lines, pattern color=Blue] (3,1) rectangle (4,2);
    \draw (3.5,1.5) node [fill=white, rounded corners] {$U_{\square}$};

    % Gauss operator
    \draw[V] (1, 1) -- (2, 1) node [pos=0.5, below right] {$V$};
    \draw[V] (1, 1) -- (1, 2) node [pos=0.5, above left] {$V$};
    \draw[V] (0, 1) -- (1, 1) node [pos=0.5, below left] {$V^\dagger$};
    \draw[V] (1, 0) -- (1, 1) node [pos=0.5, below right] {$V^\dagger$};

    \foreach \y in {0,1,2} \foreach \x in {0,1,...,4} \draw (\x,\y) node [site] {};

    \draw (1,1) node [above right] {$G_x$};
\end{tikzpicture}

    \caption{Pictorial representation of the Gauss operators $G_x$ in \eqref{eq:gauss_operator} (\emph{left}) and plaquette operator $U_{\square}$ in \eqref{eq:plaq_operator} (\emph{right}).}
    \label{fig:star_plaq_operators}
\end{figure}


\subsection{\texorpdfstring{$\Z_N$}{Z\_N} Hamiltonian and the Toric Code}%
\label{sub:hamiltonian}

The class of models we consider are described by the Hamiltonian \cite{tagliacozzo2011entanglement, hamma2008adiabatic, trebst2007topological}:
\begin{equation}
    H_{\Z_N}(\lambda) = - \sum_{\square} U_{\square} - \lambda \sum_{\link} V_{\link} + \text{h.c.},
    \label{eq:hamiltonian_base}
\end{equation}
where the first sum is over the plaquettes $\square$ of the lattice while the second sum is over the links $\link$.
One can easily see that this Hamiltonian is local and gauge-invariant, hence the dynamics it describes it is fully contained in $\Hphys$.
Furthermore, the operator $U_{\square}$ plays the role of a \emph{magnetic} term, to be more precise it is the magnetic flux inside the plaquette $\square$, while $V$ is the \emph{electric} term.
The coupling $\lambda$ tunes the relative strength of the electric and magnetic energy contribution.

These models are akin to the Toric Code \cite{kitaev2003fault}, which can be thought as a prime example of a $\Z_2$ lattice gauge theory.
More precisely, $H_{\Z_2}$ in \eqref{eq:hamiltonian_base} can be thought as a \emph{deformation} of the former, where an external ``transverse'' field is added to it.
Indeed, using the notation used so far, the Toric Code can be formulated as:
\begin{equation}
    H_{\text{TC}} = - J_m \sum_{\square} U_{\square} - J_e \sum_{x} G_x.
    \label{eq:hamiltoniana_toric_code}
\end{equation}
whose ground states $\ket{\Psi}$ satisfies the constraints
\begin{equation}
    U_{\square} \ket{\Psi} = \ket{\Psi} \;\; \forall \; \square, \quad
    G_x \ket{\Psi} = \ket{\Psi} \;\; \forall x.
    \label{eq:constraints_gs_toric_code}
\end{equation}
Only elementary excitations above the ground state can violate these constraints and they can be of two type: a \emph{magnetic vortex} (which violates the plaquette constraint) or a \emph{electric charge} (which violates the Gauss law).
If one imposes $J_e \gg J_m$ to enforce Gauss law, in the low-energy sector there are no electric charges and one recovers the pure gauge $\Z_2$ model of \eqref{eq:hamiltonian_base} for $\lambda = 0$.
Therefore, in general the $\Z_N$ models described in \eqref{eq:hamiltonian_base} can be considered as generalization of the Toric Code, from the point of view of lattice gauge theories.


\subsection{Superselection sectors}
\label{sub:superselection_sectors}



\begin{figure}[t]
    \centering
    \begin{tikzpicture}
    % lattice grid
    \draw[lattice, step=1] (-0.5,-0.5) grid (4.5,4.5);

    % Wilson loops
    \draw[X] (-0.5, 1) -- (0, 1);
    \draw[X] (4, 1) -- (4.5, 1)
        node [pos=1, right] {$\mathcal{L}_1$}
        node [pos=0, above left, black] {$\Wilson_1$};
    \foreach \x in {0,...,3} \draw[U] (\x, 1) -- ++(1, 0);

    \draw[X] (3, -0.5) -- (3, 0)
        node [pos=0, below] {$\mathcal{L}_2$}
        node [pos=0.15, left, black] {$\Wilson_2$};
    \draw[X] (3, 4) -- (3, 4.5);
    \foreach \y in {0,...,3} \draw[U] (3, \y) -- ++(0, 1);

    % 't Hooft strings
    \draw[Z, dashed, ->-=0.4]
        (-0.5,3.5) -- (4.5,3.5)
        node[pos=0, left] {$\mathcal{C}_1$}
        ;
    \foreach \x in {0,...,4} { \draw[Z, ->-=0.40] (\x, 3) -- +(0, 1); }
    \draw (2,3) node [below right] {$\tHooft_1$};

    \draw[Z, dashed, ->-=0.2]
        (0.5,-0.5) -- (0.5,4.5)
        node[pos=1, above] {$\mathcal{C}_2$}
        ;
    \foreach \y in {0,...,4} { \draw[Z, ->-=0.40] (0, \y) -- +(1, 0); }
    \draw (1, 2) node [below right] {$\tHooft_2$};

    % Sites
    \DrawSites{0,...,4}{0,...,4};
\end{tikzpicture}

    \caption{Graphical representation of the non-local order parameters $\overline{W}_{1,2}$ (in blue) and $\overline{S}_{1,2}$ (in red) and their respective paths $\mathcal{C}_{1,2}$ and $\tilde{\mathcal{C}}_{1,2}$.}
    \label{fig:nonlocal_operators}
\end{figure}

Let us consider the Toric Code.
One of its main features is the presence of topologically protected degenerate ground states \cite{kitaev2003fault}.
In order to illustrate this, besides $\overline{W}_1$ and $\overline{W}_2$, defined in \eqref{eq:top_wilson_loop}, another type of non-local operators have to be introduced.
They are defined on \emph{cuts} of the lattice $\lattice$, i.e.~paths on the dual lattice $\tilde{\lattice}$.
Consider \emph{non-contractible} cuts $\tilde{\mathcal{C}}_1$ and $\tilde{\mathcal{C}}_2$ along the directions $\hat{1}$ and $\hat{2}$, respectively.
On this cuts, the ('t Hooft string) operators $\overline{S}_1$ and $\overline{S}_2$ are constructed as
\begin{equation}
    \overline{S}_i = \prod_{\link \in \tilde{\mathcal{C}}_i} V_\link, \qquad i=1,2,
    \label{eq:top_string_operators}
\end{equation}
in a similar fashion to \eqref{eq:top_wilson_loop}.
This is shown in red in Fig.~\ref{fig:nonlocal_operators}.
The operators $\overline{W}_i$ and $\overline{S}_i$ ($i=1,2$) commutes with all the operators $U_{\square}$ and $G_x$ in the Toric Code Hamiltonian $H_{\text{TC}}$ of \eqref{eq:hamiltoniana_toric_code}, but do not commute with each other.
In fact, we have $\overline{W}_i \overline{S}_j = - \overline{S}_j \overline{W}_i$ if $i \neq j$.
This means that $H_{\text{TC}}$ can be block-diagonalized with respect to the eigenvalues of $\overline{S}_i$ (or $\overline{W}_i$), while $\overline{W}_j$ (or $\overline{S}_j$) connects one block to the other.
Furthermore, since in the case of the  $\Z_2$ symmetry,
$\overline{S}_i$ (or $\overline{W}_i$) has only two eigenvalues (equal to $\pm 1$), there are a total of $2 \times 2 = 4$ degenerates ground states, which are topologically protected, thanks to the fact
that $\overline{W}_j$ (or $\overline{S}_j$) cannot be expressed in terms of the local operators $U_{\square}$ and $G_x$.
Notice that, as it can be easily seen, in the Toric Code the role of $\overline{W}_i$ and $\overline{S}_i$ can be interchanged.

Let us now turn to $\Z_N$ LGT models.
The operators $\overline{W}_i$ no longer commute with the Hamiltonian \eqref{eq:hamiltonian_base} which now contains an electric field term.
Thus, $\lambda \neq 0$, we have no degenerate ground states.
But we can still use the $\overline{S}_i$ operators to decompose the Hilbert space $\Hphys$, since they still commute with all the \emph{local operators} $U_{\square}$ and $V_{\link}$ (thus also with $H_{\Z_N}$).
Now one can see that the operator $\overline{S}_i$ ($i=1,2$) of \eqref{eq:top_string_operators} has $N$ eigenvalues $\omega^n$, with $n=1, \dots, N-1$.
Hence, one can decompose $\Hphys$ as sum of superselection sectors
\begin{equation}
    \Hphys = \bigoplus_{n, m=0}^{N-1} \Hphys^{(n, m)},
    \label{eq:decomposizione_Hphys}
\end{equation}
where for each $\ket{\phi} \in \Hphys^{(n, m)}$ we have:
\begin{equation}
    \overline{S}_1 \ket{\phi} = \omega^{m}\ket{\phi}, \quad
    \overline{S}_2 \ket{\phi} = \omega^{n}\ket{\phi}.
\end{equation}
Let us consider now the role of the Wilson loops $\overline{W}_i$.
One can easily see that:
\begin{equation}
    \overline{W}_2 \overline{S}_1 = \omega \overline{S}_1 \overline{W}_2, \qquad
    \overline{W}_1 \overline{S}_2 = \omega \overline{S}_2 \overline{W}_1.
    \label{eq:algebra_op_nonlocali}
\end{equation}
It follows that $\overline{W}_{1,2}$ acts a shift operators for the eigenspaces of $\overline{S}_{2,1}$:
\begin{equation}
    \overline{W}_1 : \Hphys^{(n, m)} \to \Hphys^{(n + 1, m)}, \quad
    \overline{W}_2 : \Hphys^{(n, m)} \to \Hphys^{(n, m + 1)},
    \label{eq:azione_wilson_loop}
\end{equation}
where the integers $n + 1$ and $m + 1$ have to be taken $\mathrm{mod}\; N$.

From a physical point of view, the Wilson loops operators $\overline{W}_1$ and $\overline{W}_2$ create non-contractible electric loops around the lattice, while  the 't Hooft strings $\overline{S}_2$ and $\overline{S}_1$ detect the presence and the strength of these electric loops.
Therefore, it is clear that the Hilbert subspace $\Hphys^{(n, m)}$ is the subspace of all the states that contains an electric loop of strength $\omega^n$ and $\omega^{m}$ along the $\hat{1}$ and $\hat{2}$ direction, respectively.
Furthermore, the evolution of a state in $\Hphys^{(n,m)}$ with the Hamiltonian in \eqref{eq:hamiltonian_base} is confined in $\Hphys^{(n,m)}$.
