\subsection{Quantum clock models}%
\label{sub:clock_models}

\Acp{clock} are a class of models that generalizes the \ac{ising} \cite{fendley2014parafermions, baxter1989clock}.
They show a resemblance to the $\Z_N$ \ac{lgt} models we introduced previously, in Sec.~\ref{sec:generalization_to_zn}.
In fact, this similarity will later be exploited in order to obtain a gauge-reducing duality of the $\Z_N$ \ac{lgt} ladder models.

% For a discussion about clock models we start from the Hamiltonian of the \ac{ising} with transverse field, which can simply be written as
% \begin{equation}
%     H = - \sum_{i} \sigma^z_i \sigma^z_{i+1} - h \sum_{i} \sigma^x_i,
%     \label{eq:ising_hamiltonian_duality}
% \end{equation}
% where $\sigma^{x,z}_i$ are the usual $2 \times 2$ Pauli matrices for each site $i$:
% \begin{equation}
%     \sigma^x_i = \pmqty{ 0 & 1 \\ 1 &  0 }, \quad
%     \sigma^z_i = \pmqty{ 1 & 0 \\ 0 & -1 }.
% \end{equation}
The Hamiltonian \eqref{eq:hamiltonian_ising}  of \ac{ising}, with trasverse field, uses Pauli matrices $\sigma^z$ and $\sigma^x$ as basic operators and they have the fundamental property that they \emph{anticommutes} on the same site, $\acomm{\sigma^z_i}{\sigma^x_i} = 0$
This relation rewritten as
\begin{equation}
    \sigma^z_i \sigma^x_i = - \sigma^x_i \sigma^z_i,
    \label{eq:anticommutation_Pauli_matrices}
\end{equation}
which be read as follows:\emph{if the two operators are exchanged, then a phase $-1$ is acquired}.
Another important fact about Pauli matrices we want highlight is that they \emph{square to the identity}:
\begin{equation}
    (\sigma^x_i)^2 = (\sigma^z_i)^2 = \identity.
    \label{eq:identity_Pauli_matrices}
\end{equation}
% In a \ac{clock}, these Pauli matrices
% They are a set of unitary matrices that commute on different sites, while on the same site they anticommute $\sigma^x \sigma^z = - \sigma^z \sigma^x$.
% Another way to put it is to say that the exchange of $\sigma_x$ and $\sigma_z$ on the same site produces a phase $e^{i \pi} = -1$.

\Ac{clock} are generalizations of the \ac{ising}, but not to higher spins.
A $p$-state \ac{clock} (or simply a $p$-clock model) utilizes a set of unitary operators that generalizes \eqref{eq:anticommutation_Pauli_matrices} and \eqref{eq:identity_Pauli_matrices} in the following sense:
the operators $\sigma^x$ and $\sigma^z$ are promoted to the \emph{clock operators} $X$ and $Z$, respectively;
they are $p \times p$ unitary matrices whose exchange produces a phase $\omega = e^{i 2 \pi / p}$ and their $p$-th power is equal to the identity.
The algebraic properties of these clock operators $X$ and $Z$ can be summarized as follows:
\begin{equation}
    \begin{aligned}
        X Z & = \omega Z X, &
        X^p & =  Z^p = \identity_p, \\
        X^\dagger & = X^{-1} = X^{p-1}, &
        Z^\dagger & = Z^{-1} = Z^{p-1}
    \end{aligned}
    \label{eq:clock_operator_algebra}
\end{equation}

We see that the Schwinger-Weyl algebra in \eqref{eq:schwinger_weyl_algebra} and the clock operator algebra in \eqref{eq:clock_operator_algebra} are basically the same, but there are some key differences to point out betweens a $\Z_N$ \ac{lgt} and a $p$-clock model.

The \ac{dof} of a $\Z_N$ \ac{lgt} live on the links of the lattice while in a $p$-clock model they live on the sites.
But the most important aspect is that we don't have any gauge symmetry in a $p$-clock model, hence we do not have to impose any local constraints or physical conditions.
These models can be derived as the quantum Hamiltonians of the classical 2D vector Potts model, which is a discretization of the 2D planar XY model \cite{ortiz2012dualities}.

A typical $p$-clock model Hamiltonian with transverse field has the form
\begin{equation}
    \HamilClock(\lambda) = - \sum_{i} Z_i Z_{i+1} - \lambda \sum_{i} X_i + \hc
    \label{eq:clock_hamiltonian}
\end{equation}
which is, as expected, very similar to the quantum Ising Hamiltonian in \eqref{eq:hamiltonian_ising}.
Furthermore, just like the latter, $p$-clock models with only transverse field are \emph{self-dual}:
the clocks can be mapped into the kinks (or domain walls) and one would obtain the same exact Hamiltonian description but with inverted transverse field \cite{ortiz2012dualities}.
For $p < 5$, the clock models presents a self dual point in $\lambda = 1$, that separates an ordered phase from a disordered one.
On the other hand, for $p \geq 5$ we have an intermediate continuous critical phase between the ordered and disordered phase with two BKT transition points, which are related to each other through the self-duality \cite{sun2019phase}.

These models have been thoroughly studied, even with the addition of a longitudinal field $\propto Z_i$ \cite{baxter1982exactlysm} or chiral interactions.
In particular, in the case of chiral interactions, it was shown \cite{fendley2012parafermions} that the Hamiltonian \eqref{eq:clock_hamiltonian} can be mapped to a parafermionic chain through a Fradkin-Kadanoff transformation, and in presence of a $\mathbb{Z}_3$ symmetry, it shows three different phases \cite{zhuang2015clock}, if open boundaries are implemented: a trivial, a topological and an incommensurate (IC) phase.
The case which presents a real longitudinal field term was considered in \cite{huang2019clock},  where some of the critical exponents have been estimated.
The general case, where chiral interactions are included in a $\mathbb{Z}_N$ model, has been studied in \cite{fendley2012parafermions}.
Here, the author considered the model as an extension of the Ising/Majorana chain and found the edge modes of the theory.
He also calculated the points, in the parameter space, where the model is integrable or `superintegrable'.
All these studies are motivated by theoretical interest and recent experiments, which can be analysed by the above models \cite{bernien2017probing}.


% vim: spelllang=en
