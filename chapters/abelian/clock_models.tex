\subsection{Clock models}%
\label{sub:clock_models}

In this section we will deal with a class of generalizations of the quantum Ising model known as \emph{clock models} \cite{fendley2014parafermions, baxter1989clock}, which shows a resemblance to the $\Z_N$ \ac{lgt} models we introduced previously.
This similarity will later be exploited in order to obtain a complete description of the \ac{lgt} models without any redundant gauge-symmetry.

For a discussion about clock models we start from the Hamiltonian of the quantum Ising model with a transverse field, which can simply be written as
\begin{equation}
    H = - \sum_{i} \sigma^z_i \sigma^z_{i+1} - h \sum_{i} \sigma^x_i,
    \label{eq:ising_hamiltonian_duality}
\end{equation}
where $\sigma^{x,z}_i$ are the usual $2 \times 2$ Pauli matrices for each site $i$:
\begin{equation}
    \sigma^x_i = \pmqty{ 0 & 1 \\ 1 &  0 }, \quad
    \sigma^z_i = \pmqty{ 1 & 0 \\ 0 & -1 }.
\end{equation}
They are a set of unitary matrices that commute on different sites, while on the same site they anticommute $\sigma^x \sigma^z = - \sigma^z \sigma^x$.
Another way to put it is to say that the exchange of $\sigma_x$ and $\sigma_z$ on the same site produces a phase $e^{i \pi} = -1$.

Clock models can be thought as generalizations of the quantum Ising model, but not to higher spins.
A $p$-state clock model (or simply a $p$-clock model) utilizes a set of unitary operators that generalize the algebra of Pauli matrices in the following sense:
the operators $\sigma_x$ and $\sigma_z$ get promoted to the \emph{clock operators} $X$ and $Z$, respectively, which are $p \times p$ unitary matrices whose exchange produces a phase $\omega = e^{i 2 \pi / p}$, instead of $-1$.
The algebraic properties of these clock operators $X$ and $Z$ can be summarized as follows:
\begin{equation}
    \begin{aligned}
        X Z & = \omega Z X, &
        X^p & =  Z^p = \identity_p, \\
        X^\dagger & = X^{-1} = X^{p-1}, &
        Z^\dagger & = Z^{-1} = Z^{p-1}
    \end{aligned}
    \label{eq:clock_operator_algebra}
\end{equation}

We see that the Schwinger-Weyl algebra in \eqref{eq:schwinger_weyl_algebra} and the clock operator algebra in \eqref{eq:clock_operator_algebra} are basically the same, but there are some key differences to point out betweens a $\Z_N$ \ac{lgt} and a $p$-clock model.

The \ac{dof} of a $\Z_N$ \ac{lgt} live on the links of the lattice while in a $p$-clock model they live on the sites.
But the most important aspect is that we don't have any gauge symmetry in a $p$-clock model, hence we do not have to impose any local constraints or physical conditions.
These models can be derived as the quantum Hamiltonians of the classical 2D vector Potts model, which is a discretization of the 2D planar XY model \cite{ortiz2012dualities}.

A typical $p$-clock model Hamiltonian with transverse field has the form
\begin{equation}
    \Hclock(\lambda) = - \sum_{i} Z_i Z_{i+1} - \lambda \sum_{i} X_i + \hc
    \label{eq:clock_hamiltonian}
\end{equation}
which is, as expected, very similar to the quantum Ising Hamiltonian in \eqref{eq:ising_hamiltonian_duality}.
Furthermore, just like the latter, $p$-clock models with only transverse field are \emph{self-dual}:
the clocks can be mapped into the kinks (or domain walls) and one would obtain the same exact Hamiltonian description but with inverted transverse field \cite{ortiz2012dualities}.
For $p < 5$, the clock models presents a self dual point in $\lambda = 1$, that separates an ordered phase from a disordered one.
On the other hand, for $p \geq 5$ we have an intermediate continuous critical phase between the ordered and disordered phase with two BKT transition points, which are related to each other through the self-duality \cite{sun2019phase}.

These models have been thoroughly studied, even with the addition of a longitudinal field $\propto Z_i$ \cite{baxter1982exactlysm} or chiral interactions.
In particular, in the case of chiral interactions, it was shown \cite{fendley2012parafermions} that the Hamiltonian \eqref{eq:clock_hamiltonian} can be mapped to a parafermionic chain through a Fradkin-Kadanoff transformation, and in presence of a $\mathbb{Z}_3$ symmetry, it shows three different phases \cite{zhuang2015clock}, if open boundaries are implemented: a trivial, a topological and an incommensurate (IC) phase.
The case which presents a real longitudinal field term was considered in \cite{huang2019clock},  where some of the critical exponents have been estimated.
The general case, where chiral interactions are included in a $\mathbb{Z}_N$ model, has been studied in \cite{fendley2012parafermions}.
Here, the author considered the model as an extension of the Ising/Majorana chain and found the edge modes of the theory.
He also calculated the points, in the parameter space, where the model is integrable or `superintegrable'.
All these studies are motivated by theoretical interest and recent experiments, which can be analysed by the above models \cite{bernien2017probing}.


% vim: spelllang=en
