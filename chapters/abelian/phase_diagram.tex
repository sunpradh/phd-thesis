\subsection{Investigating the phase diagram}%
\label{sub:investigating_the_phase_diagram}

We wish to study the phase diagram of the $\Z_N$ \ac{lgt} phase diagram, in particular we are interested in any \ac{dcpt}.
In a pure gauge theory, these phases are investigated with non-local order parameters like the \emph{\ac{wl}} (not be confused with the non-contractible \ac{wl}s in \eqref{eq:nonlocal_op_ZN}).
This is because we expect the deconfined phase to be a topological phase, which can be investigated only with non-local order parameters.

Given a closed region $\mathcal{R}$, a \ac{wl} operator $W_{\mathcal{R}}$ is defined as
\begin{equation}
    W_{\mathcal{R}} = \prod_{\square \in \mathcal{R}} U_{\square}.
    \label{eq:closed_wilson_loop}
\end{equation}
Alternatively, considering the oriented boundary $\partial \mathcal{R}$ one can write
\begin{equation}
    W_{\mathcal{R}} = \prod_{\ell \in \partial \mathcal{R}} U_{\ell},
\end{equation}
where the Hermitian conjugate is implied every time we move in the negative directions.
It is also implied that the curve $\partial \mathcal{R}$ is a contractible loop.
Wilson showed in \cite{wilson1974confinement} that quark confinement is related to the expectation value $\ev{W_{\mathcal{R}}}$ of a \ac{wl}, which can be thought as a gauge field average on a region.
In particular, in the presence of quark confinement the gauge field average follows an \emph{area law}, where it decays exponentially with the area enclosed by $\mathcal{R}$.
On the other hand, in the deconfined phase we have a \emph{perimeter law}, where the gauge field average decays exponentially with the perimeter of $\mathcal{R}$.

Unfortunately on a ladder geometry there is not much difference between the area and the perimeter of a \ac{wl}.
In fact, in units of the lattice spacing, the area of a \ac{wl} over $n$ plaquettes is $n$ while its perimeter is just $2n+2$.
They both grow linearly.
Nonetheless, we can still look at the behaviour of the \ac{wl}, for a fixed length, at different couplings $\lambda$.

When the coupling $\lambda$ in \eqref{eq:hamiltonian_base} is equal to zero, the \ac{tc} is recovered and in any of its topological sector the ground state is the equal superposition of all the states with any number of closed electrical loops, in a similar fashion to coherent states.
This makes the \ac{tc} a \emph{quantum loop gas}, which is a \emph{deconfined phase}.
Furthermore, the operator $W_{\mathcal{R}}$ in \eqref{eq:closed_wilson_loop} creates an electrical loop around the region $\mathcal{R}$.
From the constraints
% TODO aggiustare ref
% \eqref{eq:constraints_gs_toric_code},
it can easily be proved that $W_{\mathcal{R}}$ leaves the \ac{tc} ground states unchanged, showing in fact that they behaves as coherent states, which leads to $\ev{W_{\mathcal{R}}} = 1$.

Therefore, $\ev{W_{\mathcal{R}}} \approx 1$ signals a deconfined phase and on the other hand a vanishing $\ev{W_{\mathcal{R}}} \approx 0$ corresponds to confined phase.
For this reason, even tough we lack an area/perimeter law on the ladder geometry it is still sensible to look at the behaviour of the \ac{wl}.
Analogous models in two-dimensions show a transition for non-zero $\lambda$ \cite{trebst2007topological, hamma2008adiabatic, tagliacozzo2011entanglement}.


In the dual clock model picture, the \ac{wl} translates to a disorder operator \cite{fradkin1978order}, which means that a deconfined phase can be thought of as a paramagnetic (or disordered) phase, while the confined phase is like a ferromagnetic (or ordered) phase.
Moreover, the longitudinal field breaks the $N$-fold symmetry of the ferromagnetic phase into a one-fold or two-fold degeneracy, depending on the parity ($n$ even/odd) of the super-selection sector.


% Another possible approach for investigating the phase diagram is to use the \emph{string tension}.
% In two dimensions, given an \emph{open} curve $\tilde{\mathcal{C}}$ on the dual lattice $\tilde{\mathbb{L}}$ we can construct an open \ac{ths} operator $S_{\tilde{\mathcal{C}}}$ as
% \begin{equation}
%     S_{\tilde{\mathcal{C}}} = \prod_{\ell \in \tilde{\mathcal{C}}} V_{\ell}
% \end{equation}
% with the usual caveat: we have to take the Hermitian conjugate everytime the path goes in the negative direction.
% Then the string tension is just the expectation value $\ev{S_{\tilde{\mathcal{C}}}}$ it is called in this way because it related to the potential energy (tension) between two magnetic fluxes created at the ends of the curve $\tilde{\mathcal{C}}$.
% Henceforth, in a deconfined phase $\ev{S_{\tilde{\mathcal{C}}}} \approx 0$, which means that the magnetic fluxes can be moved freely with no cost in energy, like in the \ac{tc}.

We study the $\Z_N$ \ac{lgt} on a ladder numerically through \emph{\ac{ed}}, by evaluating the half-ladder \ac{wl}, i.e.~
\begin{equation}
    W = U_1 U_2 \cdots U_{L/2},
\end{equation}
and working in the restricted physical Hilbert space $\Hphys^{(n)}$ ($n=0,\dots,N-1$), which has dimension $N^L$, much smaller than $N^{3L}$ (the dimension of the total Hilbert space).


\begin{figure}[t]
    \centering
    \begin{tikzpicture}[
    scale=0.9,
    font=\small,
    site/.style = {circle, inner sep=0 pt, minimum size=3.5pt, draw=black, fill=white},
    string/.style={{Circle[length=4pt, width=4pt]}-{Circle[length=4pt, width=4pt]}, very thick, dashed, Red}
    ]
    %%% Wilson loop

    % Lattice
    \draw[Gray, thin] (-0.5,0) grid (8.5,1);
    % Loop interior
    \draw[Blue, ultra thick, pattern=north east lines, pattern color=Blue] (0,0) rectangle (4,1);
    % Labels
    \draw (2,0.5) node [fill=white, rounded corners] {$W$};
    \draw (0,0) node [below] {$0$};
    \draw (4,0) node [below] {$L/2$};
    \draw (8,0) node [below] {$L$};
    \draw (4,1.5) node {Wilson loop};
    \foreach \y in {0,1} \foreach \x in {0,...,8} \draw (\x,\y) node [site] {};

    %%% 't Hooft string
    \begin{scope}[yshift=-3cm]
        % Lattice
        \draw[Gray, thin] (-0.5,0) grid (8.5,1);
        % String
        \draw [string] (0.5,0.5) -- (3.5,0.5) node [black, above, pos=0.65] {$S$};
        \foreach \x in {1,2,3} \draw [Red, ultra thick] (\x,0) -- (\x,1);
        % Labels
        \draw (0,0) node [below] {$0$};
        \draw (4,0) node [below] {$L/2$};
        \draw (8,0) node [below] {$L$};
        \draw (4,1.5) node {'t Hooft string};
        \foreach \y in {0,1} \foreach \x in {0,...,8} \draw (\x,\y) node [site] {};
    \end{scope}
\end{tikzpicture}

    \caption[Non-local order parameters on the ladder]{The non-local order parameters that have been used for investigating the phase diagram of $\Z_N$ ladder \ac{lgt}.
    \emph{Top}: half-ladder \ac{wl}.
    % \emph{Bottom}: half-ladder \ac{ths} operator.}%
    }%
    \label{fig:nlop_ladder}
\end{figure}


In the next section we show how to exploit the duality in Sec.~\ref{sub:duality_onto_clock_models} in the contest of \ac{ed}.
