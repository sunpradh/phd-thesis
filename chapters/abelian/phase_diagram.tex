\subsection{Investigating the phase diagram}%
\label{sub:investigating_the_phase_diagram}

We wish to study the phase diagram of the $\Z_N$ \ac{lgt} phase diagram, in particular we are interested in any \ac{dcpt}.
In a pure gauge theory, these phases are investigated with non-local order parameters like the \emph{\ac{wl}} (\ac{wl}) (not be confused with the non-contractible \ac{wl}s in \eqref{eq:nonlocal_op_ZN}).
This is because we expect the deconfined phase to be a topological phase, which can be investigated only with non-local order parameters.

Given a closed region $\mathcal{R}$, a \ac{wl} operator $W_{\mathcal{R}}$ is defined as
\begin{equation}
    W_{\mathcal{R}} = \prod_{\square \in \mathcal{R}} U_{\square}.
    \label{eq:closed_wilson_loop}
\end{equation}
Alternatively, considering the oriented boundary $\partial \mathcal{R}$ one can write
\begin{equation}
    W_{\mathcal{R}} = \prod_{\ell \in \partial \mathcal{R}} U_{\ell},
\end{equation}
where the Hermitian conjugate is implied every time we move in the negative directions.
It is also implied that the curve $\partial \mathcal{R}$ is a contractible loop.
As explained in Sec.~\ref{sub:order_parameters_and_gauge_invariance}, quark confinement is related to the expectation value $\ev{W_{\mathcal{R}}}$ of a \ac{wl}, which can be thought as a gauge field average over the region $\mathcal{R}$.
In particular, in the presence of quark confinement the gauge field average follows an \emph{area law}, where it decays exponentially with the area enclosed by $\mathcal{R}$.
On the other hand, in the deconfined phase we have a \emph{perimeter law}, where the gauge field average decays exponentially with the perimeter of $\mathcal{R}$.

Unfortunately on a ladder geometry there is not much difference between the area and the perimeter of a \ac{wl}.
In fact, in units of the lattice spacing, the area of a \ac{wl} over $n$ plaquettes is $n$ while its perimeter is just $2n+2$.
They both grow linearly.
Nonetheless, we can still look at the behaviour of the \ac{wl}, for a fixed length, at different couplings $\lambda$, for the following reason.


\begin{figure}[t]
    \centering
    \input{assets/figures/nonlocal_order_parameters.tex}
    \caption[Half-ladder \ac{wl} on the ladder]{%
        Half-ladder Wilson loop on the ladder.
        Notice that $W$ can only grow in one direction, meaning there is no difference in scaling between the area and the perimeter.
        Nonetheless, it still can be used for distinguishing phases.%
    }
    \label{fig:nlop_ladder}
\end{figure}


When the coupling $\lambda$ in \eqref{eq:hamiltonian_base} is equal to zero, the \ac{tc} is recovered and in any of its topological sector the ground state is the equal superposition of all the states with any number of closed electrical loops, in a similar fashion to coherent states.
This makes the \ac{tc} a \emph{quantum loop gas}, which is a \emph{deconfined phase}.
Furthermore, the operator $W_{\mathcal{R}}$ in \eqref{eq:closed_wilson_loop} creates an electrical loop around the region $\mathcal{R}$.
From the ground state constraints of the \ac{tc} \eqref{eq:ground_state_constraints},
it can easily be proved that $W_{\mathcal{R}}$ leaves its ground states unchanged, showing in fact that they behaves as coherent states, which leads to $\ev{W_{\mathcal{R}}} = 1$.

Therefore, $\ev{W_{\mathcal{R}}} \approx 1$ signals a deconfined phase and, on the other hand, a vanishing $\ev{W_{\mathcal{R}}} \approx 0$ would correspond to confined phase.
This is what is expected in the opposite limit, $\lambda \to \infty$, when only the electric term survives.
For this reason, even tough we lack an area/perimeter law on the ladder geometry it is still sensible to look at the behaviour of the \ac{wl}.
Analogous models in two-dimensions show that there is indeed a transition for non-zero $\lambda$ \cite{trebst2007topological, hamma2008adiabatic, tagliacozzo2011entanglement}.


In the dual clock model picture, the \ac{wl} translates to a disorder operator \cite{fradkin1978order}, which means that a deconfined phase can be thought of as a paramagnetic (or disordered) phase, while the confined phase is like a ferromagnetic (or ordered) phase.
Moreover, the longitudinal field breaks the $N$-fold symmetry of the ferromagnetic phase into a one-fold or two-fold degeneracy, depending on the parity ($n$ even/odd) of the super-selection sector.

For the reasons showed above, we have decided to study the $\Z_N$ \ac{lgt} on a ladder by evaluating the half-ladder \ac{wl} (see Fig.~\ref{fig:nlop_ladder}):
\begin{equation}
    W = U_1 U_2 \cdots U_{L/2}.
\end{equation}
Additionally, we have also decided to study each physical subspace $\Hphys^{(n)}$ (for $n=0,\dots,N-1$) separately, because we wanted to investigate if the choice of super-selection sector has any effect on the phase diagram.
This motivated us to use \ac{ed}, where the state space can be implemented exactly.
The physical subspace $\Hphys^{(n)}$ has dimension $N^L$, much smaller than $N^{3L}$ (the dimension of the total Hilbert space), hence it allows for larger lattice sizes for \ac{ed}.
