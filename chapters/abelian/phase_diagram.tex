\subsection{Investigating the phase diagram}%
\label{sub:investigating_the_phase_diagram}

We wish to study the phase diagram of the $\Z_N$ \ac{lgt} phase diagram, in particular we are interested in any \emph{confined} or \emph{deconfined} phase.
In a pure gauge theory, these phases are investigated non-local order parameters like the \emph{Wilson loop} (not be confused with the non-contractible Wilson loops in \eqref{eq:nonlocal_op_ZN}) or \emph{string tension}.
This is because we expect the deconfined phase to be a topological phase, which can be investigated only with non-local order parameters.

Given a closed region $\mathcal{R}$, a Wilson loop operator $W_{\mathcal{R}}$ is defined as
\begin{equation}
    W_{\mathcal{R}} = \prod_{\square \in \mathcal{R}} U_{\square}.
    \label{eq:closed_wilson_loop}
\end{equation}
Alternatively, considering the oriented boundary $\partial \mathcal{R}$ one can write
\begin{equation}
    W_{\mathcal{R}} = \prod_{\ell \in \partial \mathcal{R}} U_{\ell},
\end{equation}
where the Hermitian conjugate is implied everytime we move in the negative directions.
It is also implied that the curve $\partial \mathcal{R}$ is a contractible loop.
Wilson showed in \cite{wilson1974confinement} that quark confinement is related to the expectation value $\ev{W_{\mathcal{R}}}$ of a Wilson loop, which can be thought as a gauge field average on a region.
In particular, in the presence of quark confinement the gauge field average follows an \emph{area law}, where it decays exponentially with the area enclosed by $\mathcal{R}$.
On the other hand, in the deconfined phase we have a \emph{perimeter law}, where the gauge field average decays exponentially with the perimeter of $\mathcal{R}$.

Unfortunately on a ladder geometry there is not much difference between the area and the perimeter of a Wilson loop.
In fact, in units of the lattice spacing, the area of a Wilson loop over $n$ plaquettes is $n$ while its perimeter is just $2n+2$.
They both grow linearly.
Nonetheless, we can still look at the behaviour of the Wilson loop, for a fixed length, at different couplings $\lambda$.

When the coupling $\lambda$ in \eqref{eq:hamiltonian_base} is equal to zero, the Toric Code is recovered and in any of its topological sector the ground state is the equal superposition of all the states with any number of closed electrical loops, in a similar fashion to coherent states.
This makes the Toric Code a \emph{quantum loop gas}, which is a \emph{deconfined phase}.
Furthermore, the operator $W_{\mathcal{R}}$ in \eqref{eq:closed_wilson_loop} creates an electrical loop around the region $\mathcal{R}$.
From the constraints
% TODO aggiustare ref
% \eqref{eq:constraints_gs_toric_code},
it can easily be proved that $W_{\mathcal{R}}$ leaves the Toric Code ground states unchanged, showing in fact that they behaves as coherent states, which leads to $\ev{W_{\mathcal{R}}} = 1$.

Therefore, $\ev{W_{\mathcal{R}}} \approx 1$ signals a deconfined phase and on the other hand a vanishing $\ev{W_{\mathcal{R}}} \approx 0$ corresponds to confined phase.
For this reason, even tough we lack an area/perimeter law on the ladder geometry it is still sensible to look at the behaviour of the Wilson loop.


Another possible approach for investigating the phase diagram is to use the \emph{string tension}.
In two dimensions, given an \emph{open} curve $\tilde{\mathcal{C}}$ on the dual lattice $\tilde{\mathbb{L}}$ we can construct an open 't Hooft string operator $S_{\tilde{\mathcal{C}}}$ as
\begin{equation}
    S_{\tilde{\mathcal{C}}} = \prod_{\ell \in \tilde{\mathcal{C}}} V_{\ell}
\end{equation}
with the usual caveat: we have to take the Hermitian conjugate everytime the path goes in the negative direction.
Then the string tension is just the expectation value $\ev{S_{\tilde{\mathcal{C}}}}$ it is called in this way because it related to the potential energy (tension) between two magnetic fluxes created at the ends of the curve $\tilde{\mathcal{C}}$.
Henceforth, in a deconfined phase $\ev{S_{\tilde{\mathcal{C}}}} \approx 0$, which means that the magnetic fluxes can be moved freely with no cost in energy, like in the Toric Code.


