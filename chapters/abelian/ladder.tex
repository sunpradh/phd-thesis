%--------------------------------------------------
% SECTION: Abelian models on the ladder
%--------------------------------------------------
\section{Abelian models on the ladder}
\label{sec:abelian_models_on_the_ladder}

In this short chapter we will introduce $\Z_N$ \ac{lgt} on a \emph{ladder geometry}.
This type of lattice can be considered as a strip of a two-dimensional square lattice.
The peculiarity of this geometry is that it allows the existence of magnetic terms in a quasi one-dimensional lattice, which usually are not possible in a pure one-dimensional systems.
Moreover, since the Hilbert space is highly constrained, it allows the possibility to study systems of moderate size through exact diagonalization.
The latter will be analyzed in the last section.

A \emph{ladder} is a lattice $\mathbb{L}$ made of two parallels chains, the \emph{legs}, coupled to each other by the \emph{rungs} to form square plaquettes.
On the ladder, each rung is identified by a coordinate $i=1,\dots,L$, where $L$ is the length of the ladder, and the two vertices on the rung are denoted with $i^{\uparrow}$ and $i^{\downarrow}$ in the upper and lower leg, respectively.
Links, as usual, will be denoted by $\link$.
On the legs they are labelled as $\toplink_i$ (upper leg) or $\botlink_i$ (lower leg), while on the rungs they are labelled $\runglink_i$.

We preserve the same formulation of $\Z_N$ \ac{lgt} but in order to lighten our notation,
we use the symbols $V^0_i, \; U^0_i$ for the operators defined on the rung $i$, and  $V^{\rho}_i, \; U^{\rho}_i$ with $\rho = \uparrow, \downarrow$ for the operators on the horizontal links of the upper and lower leg, respectively, to the right of the rung.
In synthesis:
\begin{equation}
    \begin{split}
        U_{\runglink_i} & \equiv \Urung_i, \quad
        U_{\botlink_i}   \equiv \Udown_i, \quad
        U_{\toplink_i}   \equiv \Uup_i \\
        V_{\runglink_i} & \equiv \Vrung_i, \quad
        V_{\botlink_i}   \equiv \Vdown_i, \quad
        V_{\toplink_i}   \equiv \Vup_i,
    \end{split}
\end{equation}
and see Fig.~\ref{fig:ladder_operators}.
The plaquette operator on the right of the rung $i$ will be labeled as $U_i$:
\begin{equation}
    U_i = \Udown_i \Urung_{i+1} (\Uup_i)^{\dagger} (\Urung_i)^{\dagger}.
    \label{eq:plaq_op_ladder}
\end{equation}
Moreover, on a ladder the vertices are three-legged, so the Gauss operators are slightly modified:
\begin{equation}
    \GaussUp_i
    = \Vup_i ( \Vup_{i-1} )^\dagger ( \Vrung_i )^\dagger \text{~and~}
    \GaussDown_i
    = \Vdown_i \Vrung_i ( \Vdown_{i-1} )^\dagger,
    \label{eq:gauss_law_ladder}
\end{equation}
where $\GaussUp_i$ and $\GaussDown_i$ refers, respectively, to the Gauss operators on the vertices $i^{\uparrow}$ and $i^{\downarrow}$.
As a reference see Fig.~\ref{fig:ladder_operators}.

\begin{figure}
    \centering
    \begin{tikzpicture}[
    scale=0.5,
    ]

    %
    % Link operators
    %

    % ladder
    \draw[lattice] (1,0) grid (9,2);

    % label
    \node [font=\small] at (5, 3.6) {Link operators};

    % Magnetic field operator field operators
    \node[below] at (2,0) {$i^{\downarrow}$};
    \node[above] at (2,2) {$i^{\uparrow}$};
    \draw[U] (2,0) -- (2,2) node [pos=0.5, right] {$\Urung_i$};
    \draw[U] (2,2) -- (4,2) node [pos=0.5, above] {$\Uup_i$};
    \draw[U] (2,0) -- (4,0) node [pos=0.5, below] {$\Udown_i$};

    % Electric field operators
    \node[below] at (6,0) {$j^{\downarrow}$};
    \node[above] at (6,2) {$j^{\uparrow}$};
    \draw[V] (6,0) -- (6,2) node [pos=0.5, right] {$\Vrung_j$};
    \draw[V] (6,2) -- (8,2) node [pos=0.5, above] {$\Vup_j$};
    \draw[V] (6,0) -- (8,0) node [pos=0.5, below] {$\Vdown_j$};

    % Sites
    \DrawSites{2,4,...,8}{0,2}

    %
    % Local operators
    %
    \begin{scope}[xshift=12cm]
        % ladder
        \draw[lattice] (-1,0) grid (13,2);

        % labels
        \node[font=\small] at (8.3, 3.6) {Gauss op.};
        \node[font=\small] at (1, 3.6) {Plaq.~op.};

        % Plaquette operator
        \draw[Grey80] (0,0) node [below] {$i^{\downarrow}$};

        % \draw[Grey80] (0,2) node [above] {$i^{\uparrow}$};
        \draw[Blue80, ultra thick, pattern=north east lines, pattern color=Blue80] (0,0) rectangle +(2,2);
        \draw[U] (0,0) -- (2,0);
        \draw[U] (2,0) -- (2,2);
        \draw[U] (2,2) -- (0,2);
        \draw[U] (0,2) -- (0,0);
        \node at (1,1) [fill=white, rounded corners, text=Blue80] {$U_i$};

        % Gauss law Up
        \draw[V] (6,0) -- (6,2);
        \draw[V] (4,2) -- (6,2);
        \draw[V] (6,2) -- (8,2);
        \node[Red, above] at (6,2) {$\GaussUp_j$};
        \node[below left] at (6, 2) {$j^{\uparrow}$};

        % Gauss law down
        \draw[V] (8,0) -- (10,0);
        \draw[V] (10,0) -- (10,2);
        \draw[V] (10,0) -- (12,0);
        \node[Red][below] at (10,0) {$\GaussDown_k$};
        \node[above left] at (10, 0) {$k^{\downarrow}$};

        % Sites
        \DrawSites{0,2,...,12}{0,2}
    \end{scope}

\end{tikzpicture}

    \caption[Operators of a $\Z_N$ ladder \ac{lgt}]{%
        Picture of the different ladder operators.
        \emph{Left}: the magnetic and electric link operators.
        \emph{Right}: plaquette operator $U_i$ and the Gauss operators $\GaussUp_j$ and $\GaussDown_k$.
        Notice that operators and sites on the upper leg are indicated with an up arrow, on the lower leg with a down arrow and on the rungs with a superscript $0$.
    }
    \label{fig:ladder_operators}
\end{figure}


Finally, we write explicitly the Hamiltonian for a $\Z_N$ \ac{lgt} on a ladder:
\begin{equation}
    \Hladder(\lambda) =
    - \sum_{i} \bqty{ U_i + \lambda \pqty{ \Vup_i + \Vdown_i + V^0_i } + \text{h.c.} }.
    \label{eq:ladder_hamiltonian}
\end{equation}


For what concerns the super-selection sectors of the theory,
non-contractible loops are possible now only in the $\hat{2}$ direction.
Therefore, out of the \ac{wl} operators in \eqref{eq:nonlocal_op_ZN} only $\overline{W}_1$ is well defined, meaning that we can create non-contractible electric loops along the $\hat{1}$.
Hence, only $\overline{S}_2$ in \eqref{eq:nonlocal_op_ZN} (the \ac{ths} conjugate to $W_1$) can be used as a mean for distinguishing these different sectors.
Explicitly, the \ac{wl} $\Wilson_1$ and $\tHooft_2$ can be written as
\begin{equation}
    \Wilson_1 = \prod_{i} \Udown_i \qand
    \tHooft_2 = \Vup_{i_0} \Vdown_{i_0},
\end{equation}
where $i_0$ is any chosen rung (see Fig.~\ref{fig:nonlocal_operators_ladder}).
Furthermore, it does not make sense to consider the \ac{ths} $S_1$ because it is equal to the product of all the Gauss operators on either one of the legs,
\begin{equation}
    \tHooft_1 = \prod_{i} \GaussDown_i = \prod_{i} \GaussUp_i,
\end{equation}
so it always equal to the identity on physical states, signaling the obvious fact that we do not have non-contractible electric loops around the $\hat{2}$ direction.
We can conclude that the physical Hilbert space can be decomposed in only $N$ sectors as
\begin{equation}
    \Hphys = \Hphys^{(0)} \oplus \Hphys^{(1)} \oplus \dots \oplus \Hphys^{(N-1)},
\end{equation}
and in each sector we have that
\begin{equation}
    S \ket{\phi} = \omega^n \ket{\phi} \quad \text{if} \quad \ket{\phi} \in \Hphys^{(n)}.
\end{equation}

\begin{figure}[t]
    \SideFigure[label=fig:nonlocal_operators_ladder, desc={Non-local operators on the ladder}]{%
        \begin{tikzpicture}[
        scale=0.5,
        font=\small
    ]
    % ladder
    \draw[lattice] (1,0) grid (11,2);

    % S string
    \draw[Z, dashed] (5, -0.8) -- (5, 2.8)
        node [above, black] {$\tHooft_2 = \Vup_{i_0} \Vdown_{i_0}$};
    \draw[Z, ->-=0.5] (4, 2) -- (6, 2);

    % W string
    \draw[X] (1, 0) -- (11, 0);
    \node [below, black] at (9, 0) {$\Wilson_1 = \prod_{i} U^{\downarrow}_i $};
    \foreach \x in {2, 4, ..., 8} \draw[U] (\x, 0) -- +(2,0);

    % sites
    \DrawSites{2,4,...,10}{0,2}
\end{tikzpicture}

    }{%
        Picture of the non-local string operators $\Wilson_1$ and $\tHooft_2$ on the ladder.%
    }
\end{figure}


Due to the fact that the ladder is quasi one-dimensional, the presence of non-contractible electric loops can highly affects the physical states.
Take the case of a $\Z_2$ theory, which is pictured in Fig.~\ref{fig:states_different_sectors}.
It has just two sectors: $n=0$ and $n=1$.
In the former all the physical configuration are made of closed loop, distributed along the $\hat{1}$ direction.
While in the latter, the physical configurations are just deformations of of one single electric loop that goes around the ladder.
This can make us reasonably believe that the two sectors might have completely different physical content.


\begin{figure}[t]
    \SideFigure[label=fig:states_different_sectors, desc={Physical states in different sectors in the $\Z_2$ ladder \ac{lgt}}]{%
        \begin{tikzpicture}[scale=0.4]
    %
    % Sector n=0
    %

    % ladder
    \draw[lattice] (-1,0) grid (13,2);
    \node[font=\small, above] at (2,2.5) {$\Z_2$ sector $n=0$};

    % loops
    \draw[up] (2, 0) rectangle (4,2);
    \draw[up] (6, 0) rectangle (10,2);
    \draw[up] (13, 0) -- (12, 0) -- (12, 2) -- (13, 2);

    % sites
    \DrawSites{0,2,...,12}{0,2}

    %
    % Sector n=1
    %
    \begin{scope}[yshift=-5cm]
        % ladder
        \draw[lattice] (-1,0) grid (13,2);
        \node[font=\small, above] at (2,2.5) {$\Z_2$ sector $n=1$};

        % big loop
        \draw[up]
        (-1, 0) -- (2, 0) -- (2, 2) -- (4, 2) -- (4, 0) --
        (6, 0) -- (6, 2) -- (10, 2) -- (10, 0) --
        (12, 0) -- (12, 2) -- (13, 2);

        % sites
        \DrawSites{0,2,...,12}{0,2}
    \end{scope}
\end{tikzpicture}

    }{%
        Example of two physical configurations (in the electric basis) in a $\Z_2$ theory in the two different super-selection sectors.
        This shows that states belonging to two different sectors can be quite different.%
    }
\end{figure}


Like in the two-dimensional case, the Hamiltonian can be reduced to a single super-selection sector.
One of the main features of this is that once the sector is fixed, it is possible to write a duality transformation of the  Hamiltonian to a pure one-dimensional \emph{quantum clock model}, resolving entirely the \emph{gauge symmetries}.
Thanks to this duality map, we will see how that the different sectors have very different behaviour and each can have its own unique phase diagram.
The latter is the object of discussion of the second part of this chapter, but before doing so we need to introduce the notion of \emph{dualities} and, in particular, \emph{the bond-algebraic approach to dualities}.
