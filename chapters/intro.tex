%----------------------------------------
% CHAPTER: Introduction
%----------------------------------------
\newpage

\chapter*{Introduction}
\label{chap:introduction}
\addcontentsline{toc}{chapter}{Introduction}

Gauge theories are essential ingredients in our microscopical description of physical fundamental laws and are a cornerstone of contemporary scientific research.
In high-energy physics, they underlie the Standard Model that describe the elementary particles and their interaction.
In condensed matter physics, gauge theories emerge as effective descriptions of strongly correlated phenomena, such as superconductivity or the fraction Hall effect.

One of the most important open questions in high-energy physics is confinement in \ac{qcd}, which is an example of non-Abelian gauge theory and it is believed to be the correct theory that describe the strong interaction.
The best evidence for confinement comes from the Wilson formulation of gauge theories on a lattice \cite{wilson1974confinement}, which, at first glance, can appear odd because the vacuum is not a crystal \cite{creutz1985book}.
Indeed, there have not been experimental proofs so far that show any deviations from the symmetries of the Lorentz group.

From the point of view of particle physics, the lattice represents a mathematical trick.
It provides a cutoff, which removes the ultraviolet infinities that infest \ac{qft}.
It is just a regulator and as such it must be removed after renormalization.
Physical results can only be extracted in the continuum limit, where the lattice spacing goes to zero.

But why do we need such a regulator?
Infinities has always been present in quantum field theories since its conception.
Consider the case of \ac{qed}.
It had an immense success without ever using a discrete space-time, thanks to \emph{perturbation theory}.
The most conventional calculation schemes are based on Feynman expansions,
where a given observable is expressed as a power series in the interaction coupling.
The terms are computed until a divergence is met in a particular diagram.
These divergences can then be removed with some regularization method.

The reason why this methodology does not work in non-Abelian theories lies in the fact that some phenomena, like confinement, are inherently \emph{non-perturbative}.
Roughly speaking, perturbation theory relies on the fact that the true interacting theory is just a slight modification of the free theory.
In other words, it works only when the coupling constants are small.
In the case of \ac{qcd}, the free theory with vanishing coupling constant has no resemblance to the observed phenomenon.

In order to go beyond the diagrammatic approach of Feynman expansions, one needs a non-perturbative cutoff.
This is the main strength of the lattice, it eliminates all the wavelengths smaller than the lattice spacing before any kind of expansions is done.
Furthermore, on a lattice a field theory is \emph{mathematically well-defined}, in contrast with many standard formulations of \ac{qft}s (like the path-integral approach).

\Ac{lgt} is just a reformulation of \ac{qft} on a lattice, which exposes a close connection with \emph{\ac{sm}}.
In fact, it can be showed that a path-integral in \ac{qft} is equivalent to a partition function in \ac{sm}.
The square of the coupling constant in \ac{qft} corresponds directly to the temperature, and a strong coupling expansion becomes equivalent to a high temperature expansion.
Thus, with a lattice formulation of \ac{qft}s allows a particle physicist to use the full technology of \ac{sm} and condensed matter theory.

\todo{finire}
