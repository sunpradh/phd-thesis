%----------------------------------------
% CHAPTER: Introduction
%----------------------------------------
\newpage

\chapter*{Introduction}
\label{chap:introduction}
\addcontentsline{toc}{chapter}{Introduction}

Gauge theories are an essential ingredients in our microscopical description of physical fundamental laws and are a cornerstone of contemporary scientific research.
In high-energy physics, they underlie the Standard Model that describe the elementary particles and their interaction.
While in \ac{cm}, gauge theories emerge as effective descriptions of strongly correlated phenomena, such as superconductivity or the fractional Hall effect.

\emph{Lattice Gauge Theories} (LGTs) are one of the most promising methods for non-perturbative studies of gauge theories \cite{banuls2020simulating}.
First developed by Wilson in 1974 for investigating quark confinement \cite{wilson1974confinement}, it is now going through a renaissance period due to the growing field of \emph{Quantum Simulation} (QS) \cite{cirac2012goals}.
The work by Wilson opened up the possibility of simulating Quantum Field Theories, in particular QCD, in regimes that are not accessible through perturbative methods.
Unfortunately, the numerical simulations can also be limited, due to some intrinsic issues of the methodologies used.
One infamous example is the sign problem in the presence of a finite chemical potential.

Nonetheless, the simulation on classical devices of quantum systems is inherently limited, due to the exponential growth of the Hilbert space.
For this reason, in 1982, Feynman \cite{feynman1982simulation} suggested that the best way to simulate a quantum model is through a controllable experimental quantum device, which is able to mimic or emulate the degrees of freedom and dynamical laws of another system.
In short, Feynman was the first physicist to put forward the idea of \emph{quantum simulators}.
These devices can either be a specialized machine, able to simulate a specific class of models or a universal machine that can simulate any model, i.e., a quantum computer.

\medskip

What characterizes a gauge theory is the presence of local symmetries, which can be regarded as local constraints, that signal the presence of redundant degrees of freedom.
In order to make them approachable via QS, they have to be reformulated on a lattice in a Hamiltonian framework.
In this way, they can be fully treated as quantum many-body systems.
However, the extensive number of local gauge constraints can complicate the implementation and simulation process.
Some kind of scheme has to be employed in order to either eliminate redundant degrees of freedom, or make unphysical configurations inaccessible.

Another point of debate regarding LGTs is the digitalization of gauge fields.
In a typical gauge theory, like QED or QCD, the degrees of freedom live on a compact manifold (the gauge group manifold) making the number of states de facto infinite for each point in space.
This is not compatible with a typical quantum simulator, where only a finite register of states is available, especially on a quantum computer.
Therefore, some care has to be adopted in order to find a set of states and operators that can effectively reproduce a gauge theory in the continuum.
There are many methods available: quantum link models, finite subgroups, representations truncations, etc.

\bigskip

In this work we focus on finite group gauge theories.
We argue that choosing a finite subgroup of a Lie group (like $\mathrm{U}(1)$ or $\mathrm{SU}(N)$) offers a more natural way of truncating the number of degrees of freedom.
It preserves the unitarity of the parallel transporters, associated with the gauge fields.
This property is lost, for example, if one chooses to truncate the irreducible representations of the Lie group instead.
We focus on both Abelian \cite{pradhan2022ladder} and non-Abelian \cite{pradhan_unpublished} finite groups, and examined different aspects.

\bigskip

\noindent This manuscript is structured as follow:
\begin{description}[labelsep=1.1em]
    \item[Chap.~\ref{chap:introduction_to_lattice_gauge_theories}]
        As a starting point, in the first part of the chapter we review Yang-Mills theory.
        It is a gauge field theory based on the compact Lie groups like $\SU(N)$.
        They can be regarded as generalizations of QED and are the basis for theories like QCD.
        In order to create a clear context for LGTs, we also review Yang-Mills theory in Euclidean space-time and its Hamiltonian formulation.

        In the second part of the chapter we move onto Wilson's lattice gauge formulation \cite{wilson1974confinement}.
        This is going to be useful in showcasing some fundamental concepts: the discretization on the lattice, how to construct interaction terms for the gauge fields, what kind of order parameters can be used, etc.


    \item[Chap.~\ref{chap:quantum_simulation_of_lattice_gauge_theories}]
        In this chapter, we change subject and introduce in more detail the topic of Quantum Simulation.
        After a general exposition and explanation of its context, we review the current landscape through the possible paths that can be taken in Quantum Simulation: digital, analog and quantum-inspired.
        Then, we focus more on the state of the art of Quantum Simulation of LGTs.


    \item[Chap.~\ref{chap:dualities_in_abelian_models}]
        The contents of this chapter is based on \cite{pradhan2022ladder}.
        %, where we focused on a class of Abelian gauge models on a ladder geometry.
        We considered LGTs with gauge group $\mathbb{Z}_N$, in order to have a discretized $U(1)$ theory.
        Then, we formulated these models on a ladder geometry, because it is an almost one-dimensional lattice that allows for magnetic terms, which are not possible in a pure one-dimensional chain.

        With QS in mind, we wanted to find an effective description of these models that was able to resolve all the gauge constraints.
        One of the main achievements of the work is the construction of a \emph{duality map} between these $\mathbb{Z}_N$ models and \emph{quantum clock models} (QCMs) \cite{ortiz2012dualities}.
        Thanks to this duality, we were able to show that the super-selection sectors of the gauge model map to a different class of QCMs.
        % This is because a longitudinal field appears in the Hamiltonian which directly depends on the super-selection.
        This leads to the fact that each super-selection sector has its own distinguished phase diagram, in particular regarding deconfinement-confinement phase transitions.


    \item[Chap.~\ref{chap:finite_group_gauge_theories}]
        The contents of this chapter is based on \cite{pradhan_unpublished}, where we consider lattice gauge theories with finite gauge group, focusing more on the non-Abelian case,
        inspired by the Hamiltonian formulation of Kogut and Susskind \cite{kogut1975hamiltonian},

        In the case of a Lie group, the electric Hamiltonian is given by the Casimir element of the Lie algebra, which can be reinterpreted as the Laplacian on the group manifold.
        In this chapter we show that in the case of a finite group, an analogous construction is possible, even though there is no equivalent of a Lie algebra for a finite group.
        In particular, we show that the electric Hamiltonian can still be written as a natural Laplacian, but, this time, on the \emph{Cayley graph} of the finite group.
        % Given this fact, it is possible to apply graph-theoretical notions to derive some consequence on the spectrum.
        % Furthermore, given the freedom of choice in constructing the Cayley graph, it is possible to have an entire family of Hamiltonians associated with the same group.

        Another important result of this work is the full description of the physical, gauge-invariant Hilbert space, regardless of the choice of group or Hamiltonian.
        This was possible thanks to the use of \emph{spin network states}, which exploits the subspace of invariant states of a vertex.
        % In summary, the physical Hilbert space can be constructed as a sum over all the possible configurations of irreducible representations.
        % Each term of the sum corresponds to a specific product of subspaces, once the irreducible representations have been fixed.
        Additionally, we show also how to compute the dimension of the physical Hilbert space for any lattice size.

\end{description}

