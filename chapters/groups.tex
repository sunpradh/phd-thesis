\chapter{Some groups of interest}
\label{sec:some groups of interest}

%
% SECTION: The cyclic groups ZN
%
\section{The cyclic groups \texorpdfstring{$\Z_N$}{ZN}}
\label{app:the_cyclic_groups}


The cyclic group of order $N$, which we denote as $\Z_N$, consists of the powers $1, r, \dots, r^{N-1}$ of an element $r$ such that $r^N = 1$.
This can be written formally as
\begin{equation}
    \Z_N = \langle r | r^N = 1 \rangle.
\end{equation}
The group $\Z_N$ can be realized as the group fo rotations through angles $2 \pi k / N$ around an axis.

It is an Abelian group, therefore all its irreps are of degree $1$.
Such a representation associates with $r$ a complex number $\chi(r) = \omega$, and with $r^k$ the number $\chi(r^k) = \omega^k $.
Since $r^N = 1$, we have $\omega^N = 1$, that is, $\omega$ is a root of unity of degree $N$, which means
\begin{equation}
    \omega = e^{i 2 \pi j / N}, \qquad
    \text{for~} j = 0, 1, \dots, N-1.
\end{equation}
Thus, all the irreps are labeled by an integer $j = 0, \dots, N-1$ and are all of degree $1$.
So we do not need to specify the matrix elements obviously.
The characters $\chi_0, \chi_1, \dots, \chi_{N-1}$ are given by
\begin{equation}
    \chi_j(r^k) = e^{i 2 \pi k j / N}.
    \label{eq:character_ZN}
\end{equation}
It is immediate to see that $\chi_n \, \chi_{n^{\prime}} = \chi_{n + n^{\prime}}$, where $n + n^{\prime}$ is taken modulo $N$.
Because $\Z_N$ is Abelian, all its conjugacy classes are singlets, i.e.~contains only one element, hence we have $N$ conjugacy classes.

Regarding the basis of the group algebra $\C[\Z_N]$, the group basis $\{\ket{r^k}\}$ and the irreps basis $\ket{j}$ are related by the trasformation
\begin{equation}
    \ket{j}
    = \sum_{k = 0}^{N-1} \ket*{r^k} \braket*{r^k}{j}
    = \frac{1}{\sqrt{N}} \sum_{k = 0}^{N-1} \omega^{kj} \ket*{r^k},
\end{equation}
which is just the \emph{quantum Fourier transform}.


%
% SECTION: The dihedral groups ZN
%
\section{The dihedral groups \texorpdfstring{$D_N$}{DN}}
\label{app:the_dihedral_groups}


The dihedral group $D_N$ of order $N$ is the group of rotations and reflections of the plane which preserve a regular polygon with $n$ vertices.
It contains $N$ rotations, which forma a subgroup isomorphic to $\Z_N$, and $N$ reflections.
Its order is $2N$.
If we denote by $r$ the rotation through an angle $2 \pi / N$ and if $s$ is any of the reflections, we have:
\begin{equation}
    r^N = 1, \qquad s^2 = 1, \qquad srs = r^{-1}.
\end{equation}
Each dihedral group $D_N$ can be regarded as a finite subgroup of the Lie group $O(2)$.
Each element of $D_N$ can be written uniquely, either in the form $r^k$ (with $k=0, \dots, N-1$) if it is just a rotation or in the form $r^k s$ (with $k = 0, \dots, N-1$).
Notice that $srs = r^{-1}$ implies $s r^k s = r^{-k}$, hence $(r^k s)^2 = 1$.

It is useful to note that $D_N$ may be written as the semi-direct product of two cyclic groups,
\begin{equation}
    \label{eq:semidirect product}
    D_N = \Z_2 \ltimes \Z_N \ , \quad \quad (h_1,g_1) (h_2,g_2) = (h_1h_2, g_1\varphi_{h_1}(g_2)) \ .
\end{equation}
Here $\Z_N$ is the subgroup of rotations, and the $\Z_2$ factor gives the action of the reflection.
Setting $\Z_2 = \{e, h\}$, the twist $\varphi$ acts as $\phi_e(g)=g$ and $\phi_h(g)=g^{-1}$.


\paragraph*{Irreps for $N$ even}

First, there are 4 irreps of degree $1$, obtained by letting $\pm 1$ correspond to $r$ and $s$ in all possible ways.
The characters of the one-dimensional irreps will be denoted with $\psi_0, \dots, \psi_3$ and are given by the following table:
% Their characters $\chi_0, \chi_1, \chi_2$ and $\chi_3$ are given by the following table:

\begin{center}
    \begin{tabular}{ccccc}
        \toprule
               & $r$ & $r^k$ & $s$ & $r^k s$ \\
        \midrule
        $\psi_0$ & $+1$ & $+1$     & $+1$ & $+1$         \\
        $\psi_1$ & $+1$ & $+1$     & $-1$ & $-1$         \\
        $\psi_2$ & $-1$ & $(-1)^k$ & $+1$ & $(-1)^k$     \\
        $\psi_3$ & $-1$ & $(-1)^k$ & $-1$ & $(-1)^{k+1}$ \\
        \bottomrule
    \end{tabular}
\end{center}

Next we consider representations of degree $2$.
Put $\omega = e^{i 2 \pi / N}$ and let $h$ be an arbitrary integer.
We define a representation $\rho^h$ of $D_N$ by setting:
\begin{equation}
    \rho^h(r) =
    \begin{pmatrix}
        \omega^{h} & 0 \\
        0 & \omega^{-h}
    \end{pmatrix}
    \qand
    \rho^h(s) =
    \begin{pmatrix}
        0 & 1 \\ 1 & 0
    \end{pmatrix}.
    \label{eq:irreps_2d_DN}
\end{equation}
A direct calculation shows that this is indeed a representation and for generic elements $r^k$ and $r^k s$ we have:
\begin{equation}
    \rho^h(r^k) =
    \begin{pmatrix}
        \omega^{hk} & 0 \\
        0 & \omega^{-hk}
    \end{pmatrix}
    \qand
    \rho^h(r^k s) =
    \begin{pmatrix}
        0 & \omega^{hk} \\ \omega^{-hk} & 0
    \end{pmatrix}.
\end{equation}
% This representation is \emph{induced} by the representation of $\Z_N$ with character $\chi_h$.
Moreover, $\rho^h$ and $\rho^{N-h}$ are isomorphic.
Hence we may assume $0 \leq h \leq N/2$.
The extreme cases $h=0$ and $h=N/2$ are uninteresting:
The former corresponds to the one-dimensional irrep with character $\psi_0 + \psi_1$, while the latter to $\psi_2 + \psi_3$.
On the other hand, for $0 < h < N/2$, the representation $\rho^h$ are \emph{irreducible}.
The corresponding characters $\chi_h$ are given by:
\begin{equation}
    \chi_h(r^k) = \omega^{hk} + \omega^{-hk} = 2 \cos \frac{2 \pi hk}{N}
    \qand
    \chi_h(r^k s) = 0
    \label{eq:chars_2d_DN}
\end{equation}

The irreducible representations of degree 1 and 2 constructed above are the \emph{only irreducible representations} of $D_N$, up to isomorphism.
In fact, thesum of the squares of their degrees
\begin{equation*}
    4 \times 1 + \qty( \frac{N}{2} - 1  ) \times 4 = 2N
\end{equation*}
equals to the order of $D_N$ (see \eqref{eq:burnside_th_dim}).



\paragraph*{Irreps for $N$ odd}

In the case of odd $N$ we only have \emph{two one-dimensional irreps}, with character table

\begin{center}
    \begin{tabular}{ccc}
        \toprule
        & $r$ & $s$ \\
        \midrule
        $\psi_0$ & $+1$ & $+1$ \\
        $\psi_1$ & $+1$ & $-1$ \\
        \bottomrule
    \end{tabular}
\end{center}

We are missing the irreps $\psi_2$ and $\psi_3$ of the previous case because $(-1)^N = +1$ is true only for $N$ even.

The representations $\rho^h$ of degree 2 are defined by the same formulas \eqref{eq:irreps_2d_DN} as in the case where $N$ is even.
Those corresponding to $0 < h < N/2$ are irreducible and pairwise non-isomorphic.
Observe that, since $N$ is odd, the condition $h < n/2$ can also be written as $h \leq (n-1)/2$.
The formulas of their characters is the same as \eqref{eq:chars_2d_DN}.

These representations are the \emph{only} ones.
Indeed, it can be readily verified with formula \eqref{eq:burnside_th_dim} of Th.~\ref{th:burnside}.
The sum of the squares of their degrees is equal to
\begin{equation*}
    2 \times 1 + \frac{N-1}{2} \times 4 = 2N,
\end{equation*}
which is in fact the order of $D_N$.


\paragraph*{The case $N=4$}

We describe in more detail the dihedral group $D_4$ of order $8$.
Its elements are
\begin{equation*}
    D_4 = \{1,\, r,\, r^2,\, r^3,\, s,\, r s,\, r^2 s,\, r^3 s \}
\end{equation*}
and it has $5$ conjugacy classes:
\begin{equation*}
    \{e\}, \quad
    \{r,r^3\}, \quad
    \{r^2\}, \quad
    \{s,r^2\}, \quad
    \{rs,r^3s\}.
\end{equation*}

We also have $5$ irreps $\{\pi_j\}$, which we number them with $j=0, \dots, 4$.
For $j=0, \dots,3$ we have the one-dimensional irreps, while for $j=4$ we have the only two-dimensional irrep:
\begin{equation}
    \pi_4(r) =
    \begin{pmatrix}
        i & 0 \\ 0 & -i
    \end{pmatrix}
    \qand
    \pi_4(s) =
    \begin{pmatrix}
        0 & 1 \\ 1 & 0
    \end{pmatrix}.
\end{equation}
Alternatively, we can choose another two-dimensional representation $\overline{\pi}_4$ which uses only real matrices
\begin{equation}
    \overline{\pi}_4(r) =
    \begin{pmatrix}
        0 & -1 \\ 1 & 0
    \end{pmatrix}
    \qand
    \overline{\pi}_4(s) =
    \begin{pmatrix}
        1 & 0 \\ 0 & -1
    \end{pmatrix}
\end{equation}
and isomorphic to $\pi_4$.
% All the irreps are one-dimensional except $j=4$, which is two-dimensional.
The character table is shown in Table \ref{tab:chars_D4}.

\begin{table}[t]
    \centering
    \begin{tabular}{cccccc}
        \toprule
         &  $\{e\}$ & $\{r,r^3\}$ & $\{r^2\}$ & $\{s,r^2s\}$ & $\{rs,r^3s\}$ \\
        \midrule
        $\chi_0$ & $+1$ & $+1$ & $+1$  & $+1$ & $+1$ \\
        $\chi_1$ & $+1$ & $-1$ & $+1$  & $+1$ & $-1$ \\
        $\chi_2$ & $+1$ & $+1$ & $+1$  & $-1$ & $-1$ \\
        $\chi_3$ & $+1$ & $-1$ & $+1$  & $-1$ & $+1$ \\
        $\chi_4$ & $+2$ & $0$  & $-2$  & $0$  & $0$ \\
        \bottomrule
    \end{tabular}
    \caption{Character table of $D_4$}
    \label{tab:chars_D4}
\end{table}

As the $j=4$ is the only faithful representation, it is a natural choice for the magnetic Hamiltonian.
\todo{forse ci sono un po' di notazione inconsistenti}
