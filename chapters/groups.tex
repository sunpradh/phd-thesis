\chapter{Some groups of interest}
\label{sec:some groups of interest}

%
% SECTION: The cyclic groups ZN
%
\section{The cyclic groups \texorpdfstring{$\Z_N$}{ZN}}
\label{app:the_cyclic_groups}


The cyclic group of order $N$, which we denote as $\Z_N$, consists of the powers $1, r, \dots, r^{N-1}$ of an element $r$ such that $r^N = 1$.
This can be written formally as
\begin{equation}
    \Z_N = \langle r | r^N = 1 \rangle.
\end{equation}
The group $\Z_N$ can be realized as the group fo rotations through angles $2 \pi k / N$ around an axis.

It is an Abelian group, therefore all its irreps are of degree $1$.
Such a representation associates with $r$ a complex number $\chi(r) = \omega$, and with $r^k$ the number $\chi(r^k) = \omega^k $.
Since $r^N = 1$, we have $\omega^N = 1$, that is, $\omega$ is a root of unity of degree $N$, which means
\begin{equation}
    \omega = e^{i 2 \pi j / N}, \qquad
    \text{for~} j = 0, 1, \dots, N-1.
\end{equation}
Thus, all the irreps are labeled by an integer $j = 0, \dots, N-1$ and are all of degree $1$.
So we do not need to specify the matrix elements obviously.
The characters $\chi_0, \chi_1, \dots, \chi_{N-1}$ are given by
\begin{equation}
    \chi_n(r^k) = e^{i 2 \pi k n / N}.
\end{equation}
It is immediate to see that $\chi_n \, \chi_{n^{\prime}} = \chi_{n + n^{\prime}}$, where $n + n^{\prime}$ is taken modulo $N$.
Because $\Z_N$ is Abelian, all its conjugacy classes are singlets, i.e.~contains only one element, hence we have $N$ conjugacy classes.

Regarding the basis of the group algebra $\C[\Z_N]$, the group basis $\{\ket{r^k}\}$ and the irreps basis $\ket{j}$ are related by the trasformation
\begin{equation}
    \ket{j}
    = \sum_{k = 0}^{N-1} \ket*{r^k} \braket*{r^k}{j}
    = \frac{1}{\sqrt{N}} \sum_{k = 0}^{N-1} \omega^{kj} \ket*{r^k},
\end{equation}
which is just the \emph{quantum Fourier transform}.


%
% SECTION: The dihedral groups ZN
%
\section{The dihedral groups \texorpdfstring{$D_N$}{DN}}
\label{app:the_dihedral_groups}


The dihedral group $D_N$ of order $N$ is the group of rotations and reflections of the plane which preserve a regular polygon with $n$ vertices.
It contains $N$ rotations, which forma a subgroup isomorphic to $\Z_N$, and $N$ reflections.
Its order is $2N$.
If we denote by $r$ the rotation through an angle $2 \pi / N$ and if $s$ is any of the reflections, we have:
\begin{equation}
    r^N = 1, \qquad s^2 = 1, \qquad srs = r^{-1}.
\end{equation}
Each dihedral group $D_N$ can be regarded as a finite subgroup of the Lie group $O(2)$.
Each element of $D_N$ can be written uniquely, either in the form $r^k$ (with $k=0, \dots, N-1$) if it is just a rotation or in the form $r^k s$ (with $k = 0, \dots, N-1$).
Notice that $srs = r^{-1}$ implies $s r^k s = r^{-k}$, hence $(r^k s)^2 = 1$.


\paragraph*{Irreps for $N$ even}

First, there are 4 irreps of degree $1$, obtained by letting $\pm 1$ correspond to $r$ and $s$ in all possible ways.
Their characters $\chi_0, \chi_1, \chi_2$ and $\chi_3$ are given by the following table:

\begin{center}
    \begin{tabular}{ccc}
        \toprule
               & $r$ & $s$ \\
        \midrule
        $\chi_0$ & $+1$ & $+1$ \\
        $\chi_1$ & $+1$ & $-1$ \\
        $\chi_2$ & $-1$ & $+1$ \\
        $\chi_3$ & $-1$ & $-1$ \\
        \bottomrule
    \end{tabular}
\end{center}

\todo{finire}

\paragraph*{Irreps for $N$ odd}

\todo{fare}


\bigskip

We describe in more detail the dihedral group $D_4$ of order $8$.
Its elements are $D_N = \{1, r, r^2, r^3, r^4, s, r s, r^2 s, r^3 s \}$ and it has $5$ conjugacy classes, $\{e\},\{r,r^3\},\{r^2\},\{s,r^2\},\{rs,r^3s\}$.
We also have $5$ irreducible representations, which we number them from $j=0$ (trivial representation) to $j=4$.
All the irreps are one-dimensional except $j=4$, which is two-dimensional.
The character table is shown in Table \ref{tab:char}.
\begin{table}[h]
    \centering
    \begin{tabular}{c|c|c|c|c|c}
     & $\{e\}$&$\{r,r^3\}$&$\{r^2\}$&$\{s,r^2s\}$&$\{rs,r^3s\}$ \\
     \hline
        $\chi_0 $ & 1 & 1  & 1  & 1  & 1  \\
        $\chi_1 $ & 1 & -1 & 1  & 1  & -1 \\
        $\chi_2 $ & 1 & 1  & 1  & -1 & -1 \\
        $\chi_3 $ & 1 & -1 & 1  & -1 & 1  \\
        $\chi_4 $ & 2 & 0  & -2 & 0  & 0
    \end{tabular}
    \caption{Character table of $D_4$}
    \label{tab:char}
    \vspace{-3mm}
\end{table}
As the $j=4$ is the only faithful representation, it is a natural choice for the magnetic Hamiltonian.
It is useful to note that $D_N$ may be written as the semi-direct product of two cyclic groups,
\begin{equation}
    \label{eq:semidirect product}
    D_N = \Z_2 \ltimes \Z_N \ , \quad \quad (h_1,g_1) (h_2,g_2) = (h_1h_2, g_1\varphi_{h_1}(g_2)) \ .
\end{equation}
Here $\Z_N$ is the subgroup of rotations, and the $\Z_2$ factor gives the action of the reflection.
Setting $\Z_2 = \{e, h\}$, the twist $\varphi$ acts as $\phi_e(g)=g$ and $\phi_h(g)=g^{-1}$.
