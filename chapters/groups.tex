\chapter{Some groups of interest}
\label{sec:some groups of interest}

\section{The cyclic groups \texorpdfstring{$\Z_N$}{ZN}}
\label{sec:ZN groups}

The cyclic groups $\Z_N$ are Abelian groups of order $N$.
They have a single generator $g$, which thus satisfies $g^N=1$.
Thus $\Z_N = \{1, g, g^2, \ldots, g^{N-1}\}$.
Since $\Z_N$ is Abelian, its conjugacy classes are singlets, i.e.
it has $N$ conjugacy classes of one element; moreover, it has $N$ inequivalent irreps, all of which are one-dimensional,
\begin{equation}
    \rho_j (g^k) = \omega_N^{kj}  \ , \,\,\,\,\,\,\, j = 0,1,\ldots, N-1 \ ,
\end{equation}
with $\omega_N= e^{2\pi i /N}$.
The bases $\{\ket{g^k}\}$ and $\{\ket{j}\}$ are related by
\begin{equation}
    \ket{j} = \sum_{k=0}^{N-1} \braket{g^k|j} \ket{g^k} = \frac{1}{\sqrt{N}}\sum_{k=0}^{N-1} \omega_N^{kj} \ket{g^k}
\end{equation}
which is just the discrete Fourier transform.


\section{The dihedral groups \texorpdfstring{$D_N$}{DN}}
\label{sec:DN groups}

The dihedral groups $D_N$ are non-Abelian groups of order $2N$.
They are subgroups of $\Or(2)$, and they are generated by a rotation $r$ and a reflection $s$, which satisfy $r^N=s^2=1$ and $srs = r^{-1}$.
We describe in more detail the dihedral group $D_4$ of order $8$.
Its elements are $D_N = \{1, r, r^2, r^3, r^4, s, r s, r^2 s, r^3 s \}$ and it has $5$ conjugacy classes, $\{e\},\{r,r^3\},\{r^2\},\{s,r^2\},\{rs,r^3s\}$.
We also have $5$ irreducible representations, which we number them from $j=0$ (trivial representation) to $j=4$.
All the irreps are one-dimensional except $j=4$, which is two-dimensional.
The character table is shown in Table \ref{tab:char}.
\begin{table}[h]
    \centering
    \begin{tabular}{c|c|c|c|c|c}
     & $\{e\}$&$\{r,r^3\}$&$\{r^2\}$&$\{s,r^2s\}$&$\{rs,r^3s\}$ \\
         \hline
    $\chi_0 $ & 1 & 1 & 1 & 1 & 1\\
    $\chi_1 $ & 1 & -1 & 1 & 1 & -1\\
    $\chi_2 $ & 1 & 1 & 1 & -1 & -1\\
    $\chi_3 $ & 1 & -1 & 1 & -1 & 1\\
    $\chi_4 $ & 2 & 0 & -2 & 0 & 0
    \end{tabular}
    \caption{Character table of $D_4$}
    \label{tab:char}
    \vspace{-3mm}
\end{table}
As the $j=4$ is the only faithful representation, it is a natural choice for the magnetic Hamiltonian.
It is useful to note that $D_N$ may be written as the semi-direct product of two cyclic groups,
\begin{equation}
    \label{eq:semidirect product}
    D_N = \Z_2 \ltimes \Z_N \ , \quad \quad (h_1,g_1) (h_2,g_2) = (h_1h_2, g_1\varphi_{h_1}(g_2)) \ .
\end{equation}
Here $\Z_N$ is the subgroup of rotations, and the $\Z_2$ factor gives the action of the reflection.
Setting $\Z_2 = \{e, h\}$, the twist $\varphi$ acts as $\phi_e(g)=g$ and $\phi_h(g)=g^{-1}$.
