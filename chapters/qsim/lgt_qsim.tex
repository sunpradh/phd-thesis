%----------------------------------------
% SECTION: Quantum simulation of gauge theories
%----------------------------------------
\section{Simulating gauge theories}
\label{sec:simulating_gauge_theories}


\ac{lgt}s are arguably one of the most computationally intensive quantum many-body problem, due to the large numbers of \ac{dof} and, especially, the local gauge constraints.
Typically, \ac{lgt} computations uses the Quantum \ac{mc} method in the Euclidean path integral formulation, due to the natural presence of a Boltzmann factor in the action, that act as a statistical weight.
\Ac{mc} methods applied to \ac{qcd}, which from here it will be referred as Lattice \ac{qcd}, already yielded a number of very interesting results, that includes:
calculations of fundamental properties of \ac{qcd} (such as quark masses and the running coupling); masses of QCD bound states (such as protons, neutron, pions, etc.); structure of hadrons (for example how quarks and gluons interact with one another inside the proton); flavour physics (leading to constraints on the CKM matrix elements) and much more \cite{lin2018qcd, banuls2020lgtreview}.

However, \ac{mc} methods suffer from some issues, mostly due to the Euclidean space-time formulation and the sign problem.
The Euclidean space-time, due to its imaginary time, makes it impossible to compute observables that explicitly depend on real time.
They are accessible only in Minkowski space-time, where time has a different signature.
An example of these observables are light-cone correlations, in terms of which partonic properties are formulated and expressed \cite{cichy2019lightcone}.
As already mentioned, another severe shortcoming of Lattice \ac{qcd} is posed by the sign problem, which has already been briefly discussed in Sec.~\ref{sub:the_sign_problem}.
In summary, when chemical potential is introduced, the Boltzmann weight acquire a complex phase and can no longer be interpreted as a statistical weight.
It should be noted that the sign problem in Lattice \ac{qcd} is not strictly of fermionic origin, as it usually happens in quantum many-body physics, but is more fundamental and is due to the signature of space-time \cite{banuls2020lgtreview}.

\medskip

Furthermore, it has been proven that the sign problem is $\mathsf{NP}$--hard\footnote{
    $\mathsf{NP}$--hard problems are, informally, at least as hard as the hardest problem in $\mathsf{NP}$.
    In very layman terms, $\mathsf{NP}$ is the class of problems that are \emph{verifiable in polynomial time}, and is set against the $\mathsf{P}$ class where the problems are \emph{solvable in polynomial time} \cite{arora2009computational}.
    If it is true that $\mathsf{NP} \neq \mathsf{P}$, then $\mathsf{NP}$ problems would not admit an \emph{efficient} algorithms for finding solution, ``efficient'' would mean in polynomial time.
    Although it is widely believed that $\mathsf{NP} \neq \mathsf{P}$, this conjecture it is still yet not proven and it is one of the Millennium Prize Problems.
}
for the three-dimensional Ising spin glass \cite{troyer2005fermionioc}, so, most probably, is also $\mathsf{NP}$--hard for Lattice \ac{qcd}.
$\mathsf{NP}$--hardness should not discourage a scientist from trying to solve a problem, rather it is what makes a problem interesting.
$\mathsf{NP}$--hardness only prevents a \emph{generic} and \emph{efficient} solution from existing (if $\mathsf{NP} \neq \mathsf{P}$ is in fact true).
However, this does not mean that there are not ways to alleviate the hardness of a problem, in particular if one uses some physical insight.

One way where physical insight has been used for classical simulation is represented by \acp{tn} methods, which completely avoids the sign problem for \acp{lgt}.
They achieve such a result by not relying on \ac{mc} methods and importance sampling, but by working directly in the Hilbert space of the model under investigation.
As with every computational technique, it still has its shortcomings (see the discussion in Sec.~\ref{sub:quantum_inspired_algorithms}).

\medskip

On the other hand, one can try to avoid any computational complexity by means of \ac{qs}, either digital or analog.
Recently, different approaches have been proposed for the \ac{qs} of \acp{lgt}, from different communities \cite{banuls2020lgtreview, dalmonte2016lgtreview, banuls2020simulating}.

For analog \ac{qs}, the options ranges from ultracold atoms in optical lattices \cite{wiese2013ultracold, zohar2013ultracold, zohar2015quantum, zohar2011qed, banerjee2012atomic}, trapped ions \cite{hauke2013trappedions, yang2016trappedions}, or superconducting qubits \cite{marcos2013sclgt, marcos2014sclgt2d, brennen2016lgt}.
The proposals have addressed \acp{lgt} of at different levels (Abelian or non-Abelian, with or without dynamical matter, etc.)
For a complete review, see \cite{banuls2020simulating, banuls2020lgtreview, dalmonte2016lgtreview}.

Regarding digital \ac{qs} of \ac{lgt}, some kind of \emph{digitalization} of the fields is necessary.
By digitalization, we mean the task of formulating, representing, and encoding \ac{qft} in ways that are useful for computations.
The path-integral formulation, presented in Sec.~\ref{sec:wilson_approach_to_lft}, is the most straightforward digitalization scheme for non-perturbative field theory but it is not feasible for quantum computing.
It relies on resources far beyond near-term quantum computers.
For example, in gauge theories with Lie groups, like \ac{qcd} or \ac{qed}, the \ac{dof} lives on a compact manifold.
This means that there an infinite number of states for each point in space-time (or lattice).
In other terms, the bosons that represent the gauge fields have an infinite-dimensional Hilbert space.
This is obviously not feasible on quantum computer with a finite quantum register, for the same reason that real numbers cannot be represented on a classical computers.
Hence, some kind of truncation scheme is necessary.
Additionally, in the path-integral formulation of \ac{lgt}, fermionic fields are integrated out, leaving a non-local action.
A direct application of this procedure to quantum computers would require a high connectivity between qubits, which again is not feasible for near-term quantum computers.

\bigskip

% In the following, we will showcase the different tactics for the digitalization of gauge fields in \ac{qs}.
The starting point for the digital simulation of \acp{lgt} is the Hamiltonian formulation.
This has been worked out by Kogut and Susskind in their seminal paper \cite{kogut1975hamiltonian}, but it has been done with compact Lie groups in mind, like $\SU(N)$.
So, some extra steps are necessary in order to have a formulation of \acp{lgt} that are implementable on a quantum computer.
Several proposal for digitized \acp{lgt} have been put out
\cite{zohar2015latticegauge, zohar2017digital, zohar2017z2gauge, milstead2018qyangmills, bender2018lgt3d, cui2020circuit, byrnes2006lgt, chandrasekharan1997linkmodels, hackett2019lgt, lamm2019lgt}
and different paths are possible.
In the following list we highlight the main approaches for digitalized \acp{lgt}:
\begin{description}[labelsep=1.1em]
    \item[Quantum Link Models]
        Essentially, this proposal digitizes $\U(N)$ and $\SU(N)$ gauge fields with spin operators in extra dimensions, while preserving gauge symmetry \cite{chandrasekharan1997linkmodels, brower1999linkmodel}.
        This makes them suitable for digital simulations, because spins have finite \ac{dof}, while the continuum limit is reached with dimensional reduction techniques.

    \item[Dual variables]
        Many gauge theories have compact \ac{dof}.
        This compactness means that the lattice action admits a (possibly infinite) character expansion in the irreducible representations of the gauge group.
        These irreducible representations are the dual variables \cite{savit1980duality, kaplan2020lgt}.
        The discreteness of these dual variables makes them adapt for \ac{qs}.
        Then, the irreducible representations can be further truncated in order to have a finite number of \ac{dof}.

    \item[Finite subgroups]
        The simplest way to approximate a continuous group would be by substituting it with one of its discrete subgroup.
        This reduces the number of \ac{dof} to a finite number, one for each element of the subgroup.
        This can greatly simplifies theoretical analysis \cite{fradkin1979phase} and the development of algorithms \cite{lamm2019lgt}.
        On the other hand, it introduces issues when considering the continuum limit, due to a ``freezing transition'', that can be mitigated with the addition of some extra terms \cite{ji2020digitization, lamm2019gluon}.

\end{description}

In the works presented in this manuscript \cite{pradhan2022ladder, pradhan_unpublished} we chose to focus on finite groups, because they allow a formulation of \acp{lgt} where the unitarity of the gauge fields is preserved\footnote{To be more specific, the parallel transporters associated with the gauge fields and their conjugate operators are unitary.}.
With digital \ac{qs} as the main goal, unitarity is particularly convenient.
In quantum computation only unitary operations are directly implementable as sequence of quantum gates.
Hence, when a basis is fixed, the action of both the gauge fields and the electric fields will be translatable in terms of sequence of quantum gates.
This greatly simplifies the development of quantum algorithms for \acp{lgt}.

However, this comes with a cost.
The need to preserve unitarity may require the modification of the algebraic relations between the gauge fields and its conjugates.
This is extremely relevant if one cares about the continuum limit, which is an important topic for someone that is interested in simulations of \ac{qcd} for example.
But if one is just interested in the theoretical exploration of finite group gauge theories, then this issue can be put aside, which is what we intend to do.
