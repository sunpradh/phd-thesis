\chapter{Conclusions}
\label{chap:conclusions}

In this thesis two works by the author have been presented \cite{pradhan2022ladder, pradhan_unpublished}, both about \acp{lgt} with finite groups.
Before moving to their discussion, a proper context has been given by first introducing the general topic of \acp{lgt} (Chap.~\ref{chap:introduction_to_lattice_gauge_theories}) and \acl{qs} (Chap.~\ref{chap:quantum_simulation_of_lattice_gauge_theories}).

\medskip

The first work \cite{pradhan2022ladder}, discussed in Chap.~\ref{chap:dualities_in_abelian_models}, focuses on the use of dualities in Abelian \acp{lgt}.
Using a systematic formal approach, that makes use of bond-algebras, it was possible to obtain a duality transformation from Abelian models on a quasi one-dimensional lattice (the ladder) to a class of one-dimensional non-gauge models, called \aclp{clock}.
This map highlights how the physics of a gauge model can depends on super-selection rules, which is often an overlooked aspect.
Depending on the selected rules we can have a \acl{dcpt}, or none.
Or even a new intermediate phase for which not much is known, as it happened for the $\Z_3$ case within the sector $n \neq 0$ or the $\Z_4$ case for $n \neq 0, 2$.

\smallskip

The second work \cite{pradhan_unpublished}, discussed in Chap.~\ref{chap:finite_group_gauge_theories}, treats a general framework for formulating non-Abelian \acp{lgt}, in the Hamiltonian formulation, focusing on two key aspects: the electric term and the physical Hilbert space.
Regarding the former, it introduces a novel perspective where the electric term can be interpreted as a Laplacian on the Cayley graph of the group.
This new interpretation can be considered quite natural, because it generalize the Lie group case where the electric term is a Laplacian on the group manifold.
Furthermore, it has been shown that a complete description of the physical Hilbert space, for any group, is possible through the use of spin-network states.
Such a achievement can have major consequences, as it can lead to more efficients implementations of \acp{lgt} where no resources has to be wasted on redundant degrees of freedom.

\smallskip

In conclusion, both works show some novel point of view on the subject of \acp{lgt} for \acl{qs}, that the author hopes to be beneficial for the whole community working on these topics.
