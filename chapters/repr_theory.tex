%----------------------------------------
% APPENDIX: Some results in representation theory
%----------------------------------------
\chapter{Some results in representation theory}
\label{app:some_results_in_representation_theory}

In this appendix we sum up some of the basic results of representation theory of finite and compact Lie groups.
All the representation theory here presented is taken over the complex field $\C$.

\section{Basic results}
\label{sec:basic_results}

A finite group only has finitely many representations up to equivalence, and they are all unitary:

\begin{theorem}
    Let $G$ be a finite group and $\Sigma$ the set of equivalence classes of irreducible representations (irreps) of $G$.
    Then, $\Sigma$ is finite, and the representative of each class can be chosen to be unitary.
\end{theorem}

We can then state the following:

\begin{theorem}[Burnside]\label{th:burnside}
    Let $G$ be a finite group.
    Then:
    \begin{enumerate}
        \item If $\dim(j) $ is the dimension of the $j$-th inequivalent irreps of $G$, and there are $M$ such irreps, then
            \begin{equation}
                \sum_{j = 1}^{M} \dim(j)^2 = \abs{G},
                \label{eq:burnside_th_dim}
            \end{equation}
        where $\abs{G}$ is the order of the group.

        \item The number of inequivalent irreducible representations of $G$ is equal to the number of conjugacy classes of $G$.
    \end{enumerate}
\end{theorem}

An immediate consequence follows:

\begin{corollary}
    If $G$ is a finite Abelian group, then it has precisely $\abs{G}$  inequivalent irreps.
\end{corollary}


Similar results apply to compact groups.
First of all,

\begin{theorem}
    The irreps of a compact Lie group are finite-dimensional.
\end{theorem}

Moreover,

\begin{theorem}
    Let $G$ be a compact Lie group and $\Sigma$ the set of equivalence classes of irreducible representations of $G$.
    Then, $\Sigma$ is countable, and the representative of each class can be chosen to be unitary.
\end{theorem}

Given the irreps $\{\pi_j\}$ of a group (compact Lie or finite), these satisfy the so-called \emph{orthogonality theorem}.
The statement for compact Lie groups is the following:

\begin{theorem}[Orthogonality theorem for compact Lie groups]\label{th:orthogonality_compact_groups}
    Let $\{\pi_j\}$ be unitary irreps of $G$.
    Then
    \begin{equation}
        \int \dd g [\pi_j(g)]_{nm}^{\ast} [\pi_{j^{\prime}}(g)]_{n^{\prime} m^{\prime}}
        = \frac{\text{Vol}(G)}{\dim(j)} \delta_{j j^{\prime} } \delta_{n n^{\prime}} \delta_{m m^{\prime}},
    \end{equation}
    where the volume of the group is the volume corresponding to the chosen Haar measure.
\end{theorem}

While the statement for finite groups needs some little adjustments:

\begin{theorem}[Orthogonality theorem for finite groups]
    Let $\{\pi_j\}$ be unitary irreps of $G$.
    Then
    \begin{equation}
        \sum_{g \in G} [\pi_j(g)]_{nm}^{\ast} [\pi_{j^{\prime}}(g)]_{n^{\prime} m^{\prime}}
        = \frac{\abs{G}}{\dim(j)} d_{j j^{\prime}} \delta_{n n^{\prime}} \delta_{m m^{\prime}}
    \end{equation}
\end{theorem}

A useful corollary is that the sum of all matrices of a non-trivial irrep $j$ is zero:
\begin{equation}
    \int \dd g \pi_j (g) = 0
    \qor
    \sum_{g \in G} \pi_j(g) = 0,
    \label{eq:corollary_orthogonality_th}
\end{equation}
where the former equation correspond to compact groups while the latter to finite groups.
This follows by taking $j^{\prime}$ equal to the trivial representation, whose matrice elements are all equal to the identity.
Then if $j$ is non-trivial, the right-hand side of the orthogonality theorem is always zero.
Taking $m^{\prime} = n^{\prime}$ on the left-hand side gives the claim


\section{Character theory}
\label{sec:character_theory}

In this section, we will only be concerned with finite groups.
The irreps of a finite group $G$ are the function $\chi : G \to \C$ defined as the traces of irreps of $G$:
\begin{equation}
    \chi_j(g) = \tr \pi_j (g).
\end{equation}
There are as many irreducible charactes as there airreducible representations.
We will use the the following result:

\begin{theorem}
    The characters $\{\chi_j\} $ of a group $G$ form a basis for the space of class functions on $G$.
\end{theorem}

A class function $f$ satisfies
\begin{equation}
    f(a x a^{-1}) = f(x)
    \qquad \text{for all $x, a \in G$},
\end{equation}
which means that it is constant on conjugacy classes.
We will also needthe following

\begin{theorem}[Orthogonality theorem for characters]
    The irreducible characters of a finite group are orthonormal, in the sense that
    \begin{equation}
        \frac{1}{\abs{G}} \sum_{g \ in G} \chi_i^{\ast}(g) \chi_j(g) = \delta_{ij}.
    \end{equation}
\end{theorem}

The characters also satisfy a different kind of orthogonality relation, where one sums over characters rather than over group elements:

\begin{theorem}
    The irreducible characters $\{\chi_i\}$ of a finite group satisfy
    \begin{equation}
        \sum_{i} \chi^{\ast}(g) \chi_i(h) =
        \begin{cases}
            \abs{G} / \abs{C(g)} & \text{$g$ and $h$ are conjugate} \\
            0 & otherwise
        \end{cases}
    \end{equation}
    where $i$ indices the irreducible characters and $\abs{C(g)}$ is the sice of the conjugacy class of $g$.
\end{theorem}

Finally, we can define the \emph{convolution} of two class functions $\phi$ and $\psi$:
\begin{equation}
    (\phi \ast \psi)(g) = \sum_{h \in G} \phi(g h^{-1}) \psi(h).
\end{equation}
The convolution is symmetric, $\phi \ast \psi = \psi \ast \phi$.
We will use the fact that the convolution of two characters is again a character,
\begin{equation}
    \chi_i \ast \chi_j = \frac{\abs{G}}{\dim(j)}  \delta_{ij} \chi_j.
\end{equation}
This can be showed with a direct computation.


\section{Peter-Weyl theorem}
\label{sec:peter_weyl_theorem}

The Peter-Weyl theorem is instrumental in the formulation of the Hamiltonian in Sec.~\ref{sub:the_representation_basis}.
See \cite{knapp1996lie, tao2011peterweyl} for the Lie group case and \cite{serre1967representations} for the finite group case.
The statement for compact Lie groups is:

\begin{theorem}[Peter-Weyl for compact Lie groups]
    Let $G$ be a compact Lie group.
    Then
    \begin{enumerate}[label=(\roman*)]
        \item  The space of square-integrable functions on $G$ can be decomposed as a sum of representation spaces.
            More precisely, $V_{j}$ is the vector space for the irreps $\pi_j$, then
            \begin{equation}
                L^2(G) = \bigoplus_{j \in \Sigma} V_j^{\ast} \otimes V_j.
            \end{equation}

        \item The matrix elements of all the inequivalent irreps of $G$ form an orthogonal basis for $L^2(G)$.

        \item If $\{\ket{g}\}$ is the orthonormal group element basis for $L^2(G) $, then the orthonormal matrix element basis $\ket{jmn}$ satisfies
            \begin{equation}
                \braket{g}{jmn} =
                \sqrt{\frac{\dim(j)}{\mathrm{Vol}(G)}} [\pi_j(g)]_{mn}.
            \end{equation}
    \end{enumerate}
\end{theorem}

Note that there are multiple ways of writing the Peter-Weyl decomposition, as
\begin{equation}
    V^{\ast} \otimes V \simeq \mathrm{End}V \simeq V^{\oplus \dim V}
\end{equation}

As we will see later, these correspond to different ways of seeing $L^2(G)$ as a representation space.

Note that part \emph{(i)} can be understood as a generalisation of the Fourier decomposition.
In fact, since $\U(1)$ is Abelian, all of its irreps are one-dimensional and are given by matrix elements of the form $\{e^{i n x}\}$ for $x \in S^{1} = [0, 2 \pi)$.
Then the Peter-Weyl theorem states that any square-integrable function on $U(1) \simeq S^1$ can be written as a Fourier series.

Recall that matrix elements are defined as follows.
Consider the examples of $\SU(2) $ \cite{milstead2018qyangmills}, but the generalisation is easy.
As we know, the irreps of $\SU(2) $ are labeled by a half integer $j \in \frac{1}{2} \Z^+$.
Then in this case
\begin{equation}
    L^2 (G) = \bigoplus_{j \ in \frac{1}{2} \Z^+} V_j^{\ast} \otimes V_j,
\end{equation}
where $V_j = \C^{2j + 1}$
We have irreps $\pi_j$ for each $j$ and the matrix elements are literally the elements of the matrices representing a certain $U \in \SU(2)$ as a function of $U$.
More precisely, they are functions
\begin{equation}
    [\pi_j( \cdot )]_{mn} = \SU(2) \to \C, \qquad
    g \mapsto [\pi_j(g)]_{mn},
\end{equation}
where $-j \leq m,n \leq j$ in integer steps.
In the general case, it is more natural to take $1 \leq m,n \leq \dim(j) $.

In part \emph{(iii)} $\mathrm{Vol}(G) $ is the volume of the group given by the chosen Haar measure.
The resul of part \emph{(iii)} can be readily derived as a consequence of \emph{(ii)} and the orthogonality theorems for representations.
The non-trivial statement is that the matrix elements of representations space $L^2(G)$, while the orthogonality is an algebraic statement.
In fact, by \emph{(ii)} the matrix elements $[\pi_j]_{mn}$ for a basis for the space of wave-functions $L^2(G)$.
The corresponding states are then given by
\begin{equation}
    \ket{jmn} = C_{jmn} \int \dd U [\pi_j(U)]_{mn} \ket{U},
\end{equation}
where the constant $C_{jmn}$ can be chosen to ensure that the $\ket{jmn}$ are normalized.
Then we can compute their inner product,
\begin{equation*}
    \braket{j^{\prime} m^{\prime} n^{\prime}}{jmn} =
    C_{j^{\prime} m^{\prime} n^{\prime}}^{\ast} C_{jmn}
    \frac{\mathrm{Vol}(G)}{\dim(j)}
    \delta_{j j^{\prime}} \delta_{m m^{\prime}} \delta_{n n^{\prime}}.
\end{equation*}

It follows that the representation basis $\{\ket{jmn}\}$ is orthonormal with an appropriate choice of constants,
\begin{equation*}
    C_{jmn} = \sqrt{\frac{\dim(j)}{\mathrm{Vol}(G)} }
\end{equation*}
for compact Lie groups.
Everything we have said here also holds for finite groups, with the replacement $\mathrm{Vol}(G) \to \abs{G} $.

Crucially, the Peter-Weyl theorem also holds for finite groups \cite[Sec.~6.2]{serre1967representations}:

\begin{theorem}[Peter-Weyl for finite groups]
    Let $G$ be a finite groups.
    Then
    \begin{enumerate}[label=(\roman*)]
        \item The group algebra on $G$ can be decomposed as a sum of representation spaces.
            More precisely, if $V_j$ is the vector space for the $j$-th irrep $\pi_j$, then
            \begin{equation}
                \C[G] = \bigoplus_{j \in \Sigma} V_{j}^{\ast} \otimes V_{j}.
            \end{equation}

        \item The matrix elements of all the inequivalent irreps of $G$ form an orthogonal basis for $\C[G] $.

        \item If $\{\ket{g}\}$ is the orthonormal group element basis for $\C[G] $, then the orthonormal matrix element basis $\{\ket{jmn}\}$ can be chosen to satisfy
            \begin{equation}
                \braket{g}{jmn} =
                \sqrt{\frac{\dim(j)}{\abs{G}} } [\pi_j(g)]_{mn}.
            \end{equation}
    \end{enumerate}
\end{theorem}

The result is essentially the same as in the compact case.
Note that in the finite group case there is no issue of convergence, and as such we do not need to specify further information on the group algebra.
The duality relation can be shown to hold in the same manner as for compact Lie groups.
