%----------------------------------------
% SECTION: YANG MILLS THEORY
%----------------------------------------
\section{Review of Yang-Mills theory}
\label{sec:yang_mills_theory}

\todo{Ovviamente da riscrivere}.
A Yang-Mills theory is a gauge field theory on Minkowski space $\R^{1,d}$, where the gauge group $U(1)$ or $SU(N)$, with matter fields, which are defined by a representation of the gauge group.
For example, \emph{Quantum Chromodynamics} (QCD) is an $SU(3)$ gauge theory with Dirac spinors in the fundamental representation.
Keeping in mind the example of QCD, the Lagrangian of the theory is
\begin{equation}
    \mathcal{L} = - \frac{1}{2 g^2} \tr(F_{\mu \nu} F^{\mu \nu}) + \overline{\psi} \qty( i \gamma^{\mu} D_{\mu} - m ) \psi,
\end{equation}
where the fermions $\psi$ are taken in the fundamental representation of $SU(N)$ and the covariant derivative is $D_{\mu} = \partial_{\mu} - i A_{\mu}$.
We choose the convention where the Lie algebra generators $T^a$ are Hermitian and $[T^a, T^b] = i f^{abc} T^c$, with real structure constants $f^{abc}$.
Furthermore, the generators are such that $\tr(T^a T^b) = \frac{1}{2} \delta^{ab}$.
The strength-field tensor $F_{\mu \nu}$ is given by
\begin{equation}
    F_{\mu \nu} = \partial_{\mu} A_{\nu} - \partial_{\nu} A_{\mu} - i [A_{\mu}, A_{\nu}]
\end{equation}
and transforms in the adjoint representation of $SU(N)$.
Both the gauge field $A_{\mu}$ and the curvature tensor $F_{\mu \nu}$ live in the Lie algebra $\mathfrak{su}(N)$.

Under a gauge transformation given by a group-valued function $g(x) \in SU(N)$, such that $\psi \mapsto g(x) \psi(x)$, then
\begin{equation}
    A_{\mu}(x) \mapsto g(x) A_{\mu}(x) g(x)^{-1} + i g(x) \partial_{\mu} g(x)^{-1},
\end{equation}
so that $D_{\mu} \psi(x) \mapsto g(x) D_{\mu} \psi(x)$, while
\begin{equation}
    F_{\mu \nu} \mapsto g(x) F_{\mu \nu} g(x)^{-1},
\end{equation}
leaving the action invariant.

The action of the theory in $d+1$ dimensions is the given by
\begin{equation}
    S[A, \psi, \overline{\psi}] = \int \dd^{d+1} \mathcal{L}
\end{equation}
and the path integral
\begin{equation}
    Z = \int \mathcal{D} A \mathcal{D} \overline{\psi} \mathcal{D} \psi \, e^{i S[A, \psi, \overline{\psi}]}.
\end{equation}


%----------------------------------------
% SUBSECTION: EUCLIDEAN FIELD THEORY
%----------------------------------------
\subsection{Euclidean field theory}
\label{sub:euclidean_field_theory}

In order to work in a Euclidean space-time, we need first to perform a \emph{Wick rotation}, where the time coordinate $x_0$ is mapped a forth space coordinate $x_4$, through $x_0 = - i x_4$.
The has the effect of changing the path-integral integrand from $e^{i S}$, which is oscillatory, to $e^{-S}$, which is positive and can be interpreted as a probability distribution of the configurations of the fields.

The Euclidean path integral is
\begin{equation}
    Z_E = \int \mathcal{D} A \mathcal{D} \overline{\psi} \mathcal{D} \psi e^{-S},
\end{equation}
so that the Minkowski action and the Euclidean action satisfy $i S_M = - S_E$.
The respective actions are given by
\begin{equation}
    S_M = \int \dd^{d+1} x_M \mathcal{L}_M, \qquad
    S_E = \int \dd^{d+1} x_E \mathcal{L}_E.
\end{equation}
The rotation $x_0 = -i x_4$ leads to $\mathcal{L}_E = - \mathcal{L}_M$.

The Wick rotation does not change the form of the gauge kinetic term, i.e.,
\begin{equation}
    - \frac{1}{2 g^2} \tr(F_{\mu \nu} F^{\mu \nu}).
\end{equation}
The sum is a simple Euclidean sum, where there are no minus signs when raising or lowering indices and $\mu = 1, \dots, d+1$.

Considering now the fermionic part of the Yang-Mills Lagrangian, we need to perform the Wick rotation on the Dirac operator.
In Minkowski space-time
\begin{equation}
    \overline{\psi} \qty( i \gamma^{\mu}_M D_{\mu} - m ) \psi
    = \overline{\psi} \qty( i \gamma^{\mu}_M \partial_{\mu} + \gamma^{\mu}_M A_{\mu} - m ) \psi,
\end{equation}
where $\gamma^{\mu}_M$ are the gamma matrices of the Clifford algebra, and they satisfy $\{\gamma^{\mu}_M, \gamma^{\nu}_M\} = 2 \eta^{\mu \nu}$.
In the Euclidean Clifford algebra instead the gamma matrices $\gamma^{\mu}_E$ satisfy $\{\gamma^{\mu}_E, \gamma^{\nu}_E\} = 2 \delta^{\mu \nu}$.
Given the fact that we have $\partial_0 = i \partial_4$ and $A_0 = i A_4$, in order to obtain the correct form we have to put $\gamma^0_M = \gamma^4_E$.
This procedure yields
\begin{equation}
    \overline{\psi} \qty( i \gamma^{\mu}_M \partial_{\mu} + \gamma^{\mu}_M A_{\mu} - m ) \psi
    =
    - \overline{\psi} \qty( \gamma^{\mu}_E \partial_{\mu} + i\gamma^{\mu}_E A_{\mu} + m ) \psi.
\end{equation}

Since $\mathcal{L}_E = - \mathcal{L}_M$, we finally arrive at
\begin{equation}
    \mathcal{L}_E
    = \frac{1}{2 g^2} \tr (F_{\mu \nu} F^{\mu \nu}) + \overline{\psi} \qty( \gamma^{\mu} D_{\mu} ) \psi,
\end{equation}
where the indices are all Euclidean and $D_{\mu} = \partial_{\mu} + i A_{\mu}$.



%----------------------------------------
% SUBSECTION: HAMILTONIAN FORMULATION
%----------------------------------------
\subsection{Hamiltonian formulation}
\label{sub:hamiltonian_formulation}

The Hamiltonian formulation of a Yang-Mills theory can be tricky, especially the part about the gauge field.
Usually, one has to procede by computing the conjugate momenta and performing a Legendre transform.
The main issue here is that the gauge field component $A_0$, does not have a conjugate momentum:
\begin{equation}
    \pdv{\mathcal{L}}{\dot{A}_0} = 0.
\end{equation}
Hence, the transformation is not invertible.
The easiest way to remedy to the situation is to impose the gauge condition $A_0 = 0$,
which is called \emph{canonical gauge} or \emph{temporal gauge}.
With this condition, the kinetic term for the gauge fields can be written as
\begin{equation}
    \mathcal{L}
    = - \frac{1}{2 g^2} \tr(F_{\mu \nu} F^{\mu \nu})
    = \frac{1}{g^2} \qty( \mathbf{E}^2 - \mathbf{B}^2 )
    = \frac{1}{g^2} \qty(E^a_i E^a_i - B^a_i B^a_i),
    \label{eq:YM_lagrang_temporal_gauge}
\end{equation}
where $\mathbf{E}$ and $\mathbf{B}$ are, respectively, the ``chromoelectric'' and the ``chromomagnetic'' fields.
In the temporal gauge $\mathbf{E} = \dot{\mathbf{A}}$, the time derivative of the spatial components $\mathbf{A}$ of the gauge field $A_{\mu}$, while $\mathbf{B}$ corresponds to the spatial components of the strength-field tensor $F^{\mu \nu}$ and does not involve any time derivative.
From the Legendre transformation of \eqref{eq:YM_lagrang_temporal_gauge} we obtain the Hamiltonian density:
\begin{equation}
    \mathcal{H}
    = \frac{1}{g^2} E^a_i \dot{A}^a_i - \frac{1}{2 g^2} \qty( E^a_i E^a_i - B^a_i B^a_i )
    = \frac{1}{2g^2} \tr ( \mathbf{E}^2 + \mathbf{B}^2 ),
\end{equation}
hence the Hamiltonian in $d$ spatial dimensions is
\begin{equation}
    H = \int \dd^{d} x \frac{1}{2 g^2} \tr(\mathbf{E}^2 + \mathbf{B}^2).
\end{equation}

In the Hamiltonian formulation, the fields $\mathbf{A}$ and $\mathbf{E}$ are now operators, satisfying the commutation relations
\begin{equation}
    \begin{split}
        \comm{A^a_i(x)}{E^b_j(y)} & = i g^2 \delta_{ij} \delta_{ab} \delta(x-y) \\
        \comm{E^a_i(x)}{E^b_j(y)} & = \comm{A^a_i(x)}{A^b_j(y)} = 0.
    \end{split}
    \label{eq:comm_rel_E_A_continuum}
\end{equation}
The $\mathbf{B}$ operator is defined from $\mathbf{A}$.

In order to impose the canonical gauge $A_0 = 0$, the equation of motion for $A_0$ has to be satisfied.
This leads \citneeded to the fact that one must have
\begin{equation}
    D_i E_i = 0,
\end{equation}
where $D_i$ are the spatial components of the covariant derivative and $E_i$ the spatial components of the chromoelectric field.
Unfortunately, the equation above is inconsistent with the commutation relation \eqref{eq:comm_rel_E_A_continuum} and so it cannot be implemented as an operator equation.
The easiest solution is to impose it on states that are considered \emph{physical}:
\begin{equation}
    D_i E_i \ket{\psi_{\phys}} = 0,
\end{equation}
for each component of $D_i E_i$.
The constraints select a subspace of the overall Hilbert space, which will be labeled as the \emph{physical Hilbert space}.
The condition above for a $U(1)$ theory reduces to the well known \emph{Gauss law} $\nabla \cdot \mathbf{E} = 0$.


%----------------------------------------
% SUBSECTION: The Sign Problem
%----------------------------------------
\subsection{The Sign Problem}
\label{sub:the_sign_problem}

Even though interesting phases have been predicted for QCD in the $\mu - T$ plane\citneeded, such as quark-gluon plasma\citneeded o superconductivity\citneeded, detailed quantitative analysis has been limited to the $\mu = 0$ region only.
This is due mainly to the difficulty of studying QCD in the low energy regime, where the perturbative approach fails\citneeded.
Moreover, even \emph{lattice gauge theories}, a non-perturbative approach to Yang-Mills theory (and gauge theories in general), is not applicable for $\mu \neq 0$ due to the \emph{sign problem}.

In the Hamiltonian formulation, the chemical potential in the same manner as statistical mechanics.
If $\hat{H}$ is the Hamiltonian operator and $\hat{N}$ a number operator, then one can simply replace $\hat{H}$ with $\hat{H} - \mu \hat{N}$.
In the case of a Yang-Mills theory, the number operator corresponds to fermion number, i.e., $\hat{N} = \psi^{\dagger} \psi$.

In the path integral formalism, fermions are Grassmann variables, which we can integrate over:
\begin{equation}
    \int \mathcal{D} \psi \mathcal{D} \overline{\psi} \exp(-\int d^{d+1} \overline{\psi} K \psi) = \det K,
\end{equation}
where $K$ is the kinetic operator for the fermions.
If the fermions are coupled to gauge field, as one would expect from a Yang-Mills theory, then $K$ has some complicated dependence on the fields $A_{\mu}$.
If one includes the chemical potential term $\mu \overline{\psi}^{\dagger} \psi$ in the Lagrangian, then the fermion determinant $\det K$ turns out to be complex\citneeded, with a non-trivial phase factor.

As a result, the integrand of the path-integral is no longer positive, and it canot be interpreted as a probability distribution.
This is the infamous \emph{sign problem} and poses severe limitation to Monte Carlo simulations in the finite $\mu$ region.
