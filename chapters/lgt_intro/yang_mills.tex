%----------------------------------------
% SECTION: YANG MILLS THEORY
%----------------------------------------
\section{Review of Yang-Mills theory}
\label{sec:yang_mills_theory}

A \ac{ym} theory is a gauge field theory on Minkowski space $\R^{1,d}$ coupled to matter.
    The gauge group $G$ is usually chosen to be a compact Lie group like $\U(1)$ or $\SU(N)$ and the matter fields are defined by a representation of $G$.
    For example, \ac{qcd} is an $SU(3)$ gauge theory with Dirac spinors in the fundamental representation.
We choose to keep the dimension $d$ of space completely general and to use $D$ to denote the full dimension of space-time, i.e., $D=d+1$.
% Following the example of \ac{qcd}, the Lagrangian of the theory is

We start from the Lagrangian.
Considering that \ac{ym} theory can be seen as the generalization of \ac{qed}, a $\U(1)$ gauge theory, to any compact Lie group, the Lagrangian looks exactly like the one from \ac{qed}:
\begin{equation}
    \LagrangYM = - \frac{1}{2 g^2} \tr(F_{\mu \nu} F^{\mu \nu}) + \overline{\psi} \qty( i \gamma^{\mu} D_{\mu} - m ) \psi,
    \label{eq:yang_mills_lagrangian}
\end{equation}
with some differences that will be explained later.
Hereafter, the Einstein summation rule is implied.
Notice that in \eqref{eq:yang_mills_lagrangian} we have only considered one fermionic species.
In more realistic cases we would have a some over the different fermion flavors but for simplicity and ease of exposition we will ignore flavors and consider only type of fermion.

% The fermion $\psi$ are taken in the fundamental representation of $G$ and the covariant derivative is $D_{\mu} = \partial_{\mu} - i A_{\mu}$.
\paragraph*{Gauge fields}

The symbol $D_{\mu}$ in \eqref{eq:yang_mills_lagrangian} denote the \emph{covariant derivative}:
\begin{equation}
    D_{\mu} = \partial_{\mu} - i A_{\mu},
\end{equation}
where $A_{\mu}$ are the space-time components of the gauge fields.
Each component is Lie algebra valued function of space-time:
\begin{equation}
    A_{\mu}(x) = \sum_{a} A_{\mu}^a(x) T^a,
\end{equation}
where the sum is over the generators of the Lie algebra $\g$, corresponding to the group $G$.
We choose the convention where the generators $T^a$ are Hermitian with
\begin{equation}
    [T^a, T^b] = i f^{abc} T^c
    \qand
    \tr(T^a T^b) = \frac{1}{2} \delta^{ab},
\end{equation}
with real structure constants $f^{abc}$.

The dynamics of the gauge fields is given by $F_{\mu \nu}$, which is the \emph{strength-field tensor} and defined as
\begin{equation}
    F_{\mu \nu} = \partial_{\mu} A_{\nu} - \partial_{\nu} A_{\mu} - i [A_{\mu}, A_{\nu}].
    \label{eq:strength_field_tensor}
\end{equation}
and transforms in the adjoint representation of $SU(N)$.
Notice that when $G = U(1)$, i.e.~Abelian, the commutator term in \eqref{eq:strength_field_tensor} vanishes and we re-obtain the strength-field tensor of \ac{qed}.
Like the gauge field $A_{\mu}$, also the tensor $F_{\mu \nu}$ lives in the Lie algebra $\g$.
Therefore, the product $F^{\mu \nu} F_{\mu \nu}$ is actually a matrix.
Only scalar terms are allowed in the Lagrangian \eqref{eq:yang_mills_lagrangian}, hence we have to take the trace:
\begin{equation*}
    \tr F^{\mu \nu} F_{\mu \nu} = \sum_{a} F^{\mu \nu a} F_{\mu \nu}^a.
\end{equation*}
% Both the gauge field $A_{\mu}$ and the tensor $F_{\mu \nu}$ live in the Lie algebra $\mathfrak{su}(N)$.

\paragraph*{Fermions}

Regarding the fermion field $\psi$, this is taken in the \emph{fundamental representation} of $G$.
The matrices $\gamma^{\mu}$ form a Clifford algebra, which is defined by the relations
\begin{equation}
    \acomm{\gamma^{\mu}}{\gamma^{\nu}} = 2 \eta^{\mu \nu},
\end{equation}
where $\eta^{\mu \nu}$ the space-time metric.
For the latter we choose the convention $\eta^{\mu \nu} = \mathrm{diag}(+1, -1, -1, -1)$.
The conjugate $\overline{\psi}$ is defined as
\begin{equation*}
    \overline{\psi} = \psi^{\dagger}\gamma^0.
\end{equation*}
\todo{dire qualcosina in più}


\paragraph*{Gauge transformations}

A gauge transformation is defined by a group valued function of space-time $g: \R^{1,d} \to G$.
It transforms the fermion field as
\begin{equation}
    \psi(x) \mapsto g(x) \psi(x).
    \label{eq:fermion_gauge_transf}
\end{equation}
Here we have been a bit sloppy with notation, by writing $g(x) \psi(x)$ in \eqref{eq:fermion_gauge_transf} we actually mean the action of the element $g(x) \in G$ in the same representation of $\psi(x)$.

% Under a gauge transformation given by a group-valued function $g(x) \in SU(N)$, such that $\psi \mapsto g(x) \psi(x)$, then
In order to have an invariant Lagrangian, the gauge fields $A_{\mu}$ have to undergo a transformation induced by the function $g$:
\begin{equation}
    A_{\mu}(x) \mapsto g(x) A_{\mu}(x) g(x)^{-1} + i g(x) \partial_{\mu} g(x)^{-1},
\end{equation}
so that $D_{\mu} \psi(x) \mapsto g(x) D_{\mu} \psi(x)$, while
\begin{equation}
    F_{\mu \nu} \mapsto g(x) F_{\mu \nu} g(x)^{-1}.
\end{equation}


\paragraph*{Path-integral}

For a path integral formulation we need to first define the action.
This is just the integral of the Lagrangian in \eqref{eq:yang_mills_lagrangian}, over the $D$-dimensional space-time:
% The action of the theory in $D=d+1$ dimensions is the given by
\begin{equation}
    \Action[A, \psi, \overline{\psi}] = \int_{\R^{1,d}}  \dd^{D} \Lagrangian.
    \label{eq:action_def}
\end{equation}
The action \eqref{eq:action_def} defines the ``weight'' in the path-integral.
Now we can write down the partition function for a \ac{ym} theory:
\begin{equation}
    \PartFunc = \int \DD A\, \DD \overline{\psi}\, \DD \psi \, e^{i \Action[A, \psi, \overline{\psi}]}.
    \label{eq:partition_function_mink}
\end{equation}


%----------------------------------------
% SUBSECTION: EUCLIDEAN FIELD THEORY
%----------------------------------------
\subsection{Euclidean field theory}
\label{sub:euclidean_field_theory}

In the previous section we introduced \ac{ym} theory in Minkowski space-time.
We now need to move onto Euclidean space-time, which is the starting point for \ac{lgt}.
This is true for various reasons.
As stated previously, Euclidean formulation allows to bridge into the territory of \ac{sm}, meaning we can use its full technology.
Second, there are also some advantages from the operational point of view.
The weight in \eqref{eq:partition_function_mink} is a complex phase, which can be problematic from the computational point of view because a priori convergence is not guaranteed.
We will see that in Euclidean space-time the weight will become a positive-defined function, which makes it clear it is a probability distribution.

In order to pass to a Euclidean space-time $\R^{d+1}$, we need to perform a \emph{Wick rotation},
where the time coordinate $x_0$ is mapped to a forth space coordinate $x_4$:
\begin{equation*}
    x_0 \mapsto - i x_4.
    \label{eq:wick_rotation}
\end{equation*}
To distinguish quantities in Euclidean or Minkowski space-time we use the subscripts $E$ and $M$, respectively.
The rotation \eqref{eq:wick_rotation} affects both space-time measures,
\begin{equation*}
    \dd^D x_M = \dd x_0 \dd^d x_i \qand
    \dd^D x_E = \dd^d x_i \dd x_4,
\end{equation*}
and the time-components of the quantities that enters the Lagrangian, which leads to an overall effect
\begin{equation*}
    \Lagrangian_E = - \Lagrangian_M.
\end{equation*}
The action is defined in the same way in both types of space-time,
\begin{equation*}
    \Action_M = \int \dd^D x_M \Lagrangian_M
    \qand
    \Action_E = \int \dd^D x_E \Lagrangian_E,
\end{equation*}
and due to the Wick rotation \eqref{eq:wick_rotation}, we obtain that they satisfy
\begin{equation}
    i \Action_M = - \Action_E.
\end{equation}
This leads to the following definition of the Euclidean path-integral:
\begin{equation}
    \PartFunc_E = \int \DD A\, \DD \overline{\psi}\, \DD \psi\, e^{-\Action_E[A, \overline{\psi}, \psi]}.
    \label{eq:euclidean_path_integral}
\end{equation}
Notice that the weight $e^{-\Action_E}$ in \eqref{eq:euclidean_path_integral} is now a positive-valued function, given that $\Action_E$ is a real function, and has the form of a \emph{Boltzmann weight}.
In other words, following the spirit of \ac{sm}, we have now a probability distribution $e^{- \Action_E}$ over the configurations of the fields $A_{\mu}$, $\psi$ and $\overline{\psi}$.

% It has the effect of changing the integrand in \eqref{eq:partition_function_mink} from $e^{i S}$, which is oscillatory, to $e^{-S}$, which is positive and can be interpreted as a probability distribution of the configurations of the fields.

% The Euclidean path integral is
% \begin{equation}
%     \PartFunc_E = \int \mathcal{D} A \mathcal{D} \overline{\psi} \mathcal{D} \psi e^{-S},
% \end{equation}
% so that the Minkowski action and the Euclidean action satisfy $i S_M = - S_E$.
% The respective actions are given by
% \begin{equation}
%     \Action_M = \int \dd^{d+1} x_M \Lagrangian_M, \qquad
%     \Action_E = \int \dd^{d+1} x_E \Lagrangian_E.
% \end{equation}
% The rotation $x_0 = -i x_4$ leads to $\Lagrangian_E = - \Lagrangian_M$.
%
The Wick rotation does not change the aspect of the gauge kinetic term,
\begin{equation}
    - \frac{1}{2 g^2} \tr(F_{\mu \nu} F^{\mu \nu}),
\end{equation}
but the sum is now a simple Euclidean sum, where there are minus signs appear when raising or lowering indices, and $\mu = 1, \dots, d+1$.

Considering now the fermionic part of the \ac{ym} Lagrangian, we need to perform the Wick rotation on the Dirac fields $\psi$ and $\overline{\psi}$.
In Minkowski space-time
\begin{equation}
    \overline{\psi} \qty( i \gamma^{\mu}_M D_{\mu} - m ) \psi
    = \overline{\psi} \qty( i \gamma^{\mu}_M \partial_{\mu} + \gamma^{\mu}_M A_{\mu} - m ) \psi,
\end{equation}
where $\gamma^{\mu}_M$ denotes the gamma matrices in Minkowski space-time:
\begin{equation*}
    \{\gamma^{\mu}_M, \gamma^{\nu}_M\} = 2 \eta^{\mu \nu}.
\end{equation*}
The Euclidean Clifford algebra instead uses gamma matrices $\gamma^{\mu}_E$ that instead satisfy
\begin{equation*}
    \{\gamma^{\mu}_E, \gamma^{\nu}_E\} = 2 \delta^{\mu \nu}.
\end{equation*}
Given that we have $\partial_0 = i \partial_4$ and $A_0 = i A_4$, the correct form can only be achieved by putting $\gamma^0_M = \gamma^4_E$.
This procedure yields
\begin{equation}
    \overline{\psi} \qty( i \gamma^{\mu}_M \partial_{\mu} + \gamma^{\mu}_M A_{\mu} - m ) \psi
    =
    - \overline{\psi} \qty( \gamma^{\mu}_E \partial_{\mu} + i\gamma^{\mu}_E A_{\mu} + m ) \psi.
\end{equation}
Since $\Lagrangian_E = - \Lagrangian_M$, we can conclude
\begin{equation}
    \Lagrangian_E
    = \frac{1}{2 g^2} \tr (F_{\mu \nu} F^{\mu \nu}) + \overline{\psi} \qty( \gamma^{\mu} D_{\mu} + m) \psi,
\end{equation}
where the indices are all Euclidean and $D_{\mu} = \partial_{\mu} + i A_{\mu}$.



%----------------------------------------
% SUBSECTION: HAMILTONIAN FORMULATION
%----------------------------------------
\subsection{Hamiltonian formulation}
\label{sub:hamiltonian_formulation}

Even tough Wilson formulation \cite{wilson1974confinement} was in the path-integral and Lagrangian language, we will also review the Hamiltonian formulation of non-Abelian \ac{qft} because its connection to Quantum Simulation.
Expressing a \ac{ym} theory in the Hamiltonian language can be tricky, especially in the presence of gauge symmetries.
Usually, one has to procede by computing the conjugate momenta and perform a Legendre transform in order to obtain the Hamiltonian.
In the presence of gauge symmetries, the time-component $A_0$ of the gauge fields does \emph{not} have a conjugate momentum.
Instead it leads to a \emph{constraint}:
% The main issue here is that the gauge field component $A_0$, does not have a conjugate momentum:
\begin{equation}
    \pdv{\Lagrangian}{\dot{A}_0} = 0,
\end{equation}
which means that the Legendre transform is not invertible.

The easiest remedy is to \emph{fix the gauge} beforehand, by imposing $A_0 = 0$, which is called \emph{canonical} or \emph{temporal gauge}.
With this condition, the gauge fields Lagrangian can be written as
\begin{equation}
    - \frac{1}{2 g^2} \tr(F_{\mu \nu} F^{\mu \nu})
    = \frac{1}{g^2} \qty( \mathbf{E}^2 - \mathbf{B}^2 )
    = \frac{1}{g^2} \qty(E^a_i E^a_i - B^a_i B^a_i),
    \label{eq:YM_lagrang_temporal_gauge}
\end{equation}
where $\mathbf{E}$ and $\mathbf{B}$ are, respectively, the corresponding electric and magnetic fields for a non-Abelian theory.
In the temporal gauge we only have the spatial components $\mathbf{A}$ of the gauge field $A_{\mu}$.
The electric field $\mathbf{E}$ is the time derivative of $\mathbf{A}$, i.e.~$\mathbf{E} = \dv{\mathbf{A}}{t}$,
which means that $\mathbf{E}$ is the conjugate momentum to $\mathbf{A}$.
Meanwhile, the magnetic field $\mathbf{B}$ can be obtained from the spatial components of the strength-field tensor $F^{\mu \nu}$, with $B_i = - \frac{1}{2} \varepsilon_{ijk} F^{jk}$, where $\varepsilon_{ijk}$ is the Levi-Civita symbol.
Once the gauge is fixed, the Hamiltonian can be finally be obtained with a Legendre transform:
% From the Legendre transformation of \eqref{eq:YM_lagrang_temporal_gauge} we obtain the Hamiltonian density:
\begin{equation}
    \mathcal{H}
    = \frac{1}{g^2} E^a_i \dot{A}^a_i - \frac{1}{2 g^2} \qty( E^a_i E^a_i - B^a_i B^a_i )
    = \frac{1}{2g^2} \tr ( \mathbf{E}^2 + \mathbf{B}^2 ).
\end{equation}

In the Hamiltonian formulation, the fields $\mathbf{A}$ and $\mathbf{E}$ have to be elevated to operators, by imposing the following commutation relations:
\begin{equation}
    \begin{split}
        \comm{A^a_i(x)}{E^b_j(y)} & = i g^2 \delta_{ij} \delta_{ab} \delta(x-y) \\
        \comm{E^a_i(x)}{E^b_j(y)} & = \comm{A^a_i(x)}{A^b_j(y)} = 0.
    \end{split}
    \label{eq:comm_rel_E_A_continuum}
\end{equation}
An astute observer will notice that \eqref{eq:comm_rel_E_A_continuum} are completely analogous to a position-momentum commutation relation, similar to $\comm{x_i}{p_j} = i \delta_{ij}$.
In fact, like in the latter case, where the momentum $p_i$ is the generator of translations of $x_i$, the electric field $\mathbf{E}$ is the generator of translation of $\mathbf{A}$.
To be more precise, it is the canonical momentum $\mathbf{E}/g^2$ that generates translations of $\mathbf{A}$.
In other words, $\mathbf{E}/g^2$ \emph{generates infinitesimal gauge transformations}.
This point of view will be rather useful when treating the gauge fields on a lattice.


In order to impose the canonical gauge $A_0 = 0$, the equation of motion for $A_0$ has to be satisfied:
\begin{equation}
    \partial_{\mu} \qty( \pdv{\Lagrangian}{(\partial_{\mu} A_0)} ) - \pdv{\Lagrangian}{A_0} = 0.
\end{equation}
In the absence of sources, this leads to
\begin{equation}
    D_i E_i = 0,
    \label{eq:non_abelian_gauss_law}
\end{equation}
where $D_i$ and $E_i$ are the spatial components of the covariant derivative and electric field, respectively.
What we obtained is basically the generalization of \emph{Gauss law} to non-Abelian theories.
In fact, the condition \eqref{eq:non_abelian_gauss_law} for a $U(1)$ theory reduces to the well known $\nabla \cdot \mathbf{E} = 0$.
Unfortunately, the equation \eqref{eq:non_abelian_gauss_law} is inconsistent with the commutation relations \eqref{eq:comm_rel_E_A_continuum}, so it cannot be implemented as an operator equation.
The easiest solution, or loophole, to this empasse is to impose to consider \emph{physical} or \emph{gauge-invariant} only states that satisfy
\begin{equation}
    D_i E_i \ket{\psi_{\phys}} = 0.
\end{equation}
This constraint select a subspace of the overall Hilbert space $\HilbertSpace$, which will be labeled as the \emph{physical Hilbert space} $\Hphys$.


%----------------------------------------
% SUBSECTION: The Sign Problem
%----------------------------------------
\subsection{The Sign Problem}
\label{sub:the_sign_problem}

A number of interesting phases have been predicted for \ac{qcd} in the $\mu - T$ plane \cite{aarts2016qcd}, where $\mu$ is the chemical potential and $T$ the temperature, such as \emph{quark-gluon plasma} \cite{detar2009qcdthermo} or \emph{color superconductivity} \cite{alford2001coloursc} (see Fig.~\ref{fig:qcd_phase_diagram}).
Unfortunately, detailed quantitative analysis of \ac{qcd} has been limited to the $\mu = 0$ region only \cite{aarts2016qcd}.
This is mainly due to the difficulty of studying \ac{qcd} in the low energy regime, where the perturbative approach fails \cite{peskin1995qft, creutz1985book}.
Moreover, even \ac{lgt}s, at least in the path-integral formulation, is not applicable for $\mu \neq 0$ due to the infamous \emph{sign problem}.


\begin{figure}[t]
    \centering
    \begin{tikzpicture}[
    asse/.style = {-stealth, very thick},
    transizio/.style = {-Circle, very thick},
    align=center
    ]
    % Axis
    \draw[asse] (0, 0) -- (7, 0) node [below left, font=\small] {chemical potential};
    \draw[asse] (0, 0) -- (0, 5) node [above left, rotate=90, font=\small] {temperature};
    % transition lines
    \draw[transizio] (1.5, 0) arc (0:40:1.5);
    \draw[transizio] (3.5, 0) arc (0:60:3)
        node[above] {critical point};
    % etichette
    \node[above] at (0.75, 0) {vacuum};
    \node[above] at (2.25, 0) {nuclear\\ matter};
    \node[above right] at (3.5, 0) {neutron\\ stars};
    \node at (6.5, 1) {color\\ superconductor?};
    \node at (4, 4) {quark-gluon\\ plasma};
    \node[rotate=90] at (0.5, 3.5) {early universe};
\end{tikzpicture}

    \caption{A (very) rough sketch of the phase diagram of \ac{qcd}.}
    \label{fig:qcd_phase_diagram}
\end{figure}


In the Hamiltonian formulation, the chemical potential in introduced in the same manner as standard \ac{sm}.
If $H$ is the Hamiltonian operator and $N$ a number operator, then one can simply replace $H$ with $H - \mu N$.
In the case of a \ac{ym} theory, the number operator would correspond to the fermion number operator $N = \psi^{\dagger} \psi$.

In the path integral formalism, fermions are introduced as Grassmann variables.
This means that they easily be integrated out \cite{peskin1995qft, aarts2016qcd}:
\begin{equation}
    \int \DD \overline{\psi}\, \DD \psi \exp(-\int d^{d+1} \overline{\psi} K \psi) = \det K,
\end{equation}
where $K$ is the kinetic operator for the fermions.
If the fermions are coupled to $A_{\mu}$, as it happens in \ac{ym} theory, then $K$ has some complicated dependence on the fields $A_{\mu}$.
If one includes the chemical potential term $\mu \psi^{\dagger} \psi$ in the Lagrangian, then the fermion determinant $\det K$ turns out to be complex \cite{aarts2016qcd}, with a non-trivial phase factor.

As a result, the integrand of the path-integral is no longer positive-defined, and it cannot be interpreted as a probability distribution.
Furthermore, a complex weight in the path-integral makes the integrand oscillatory, which does not help with convergence.
This is summarizes the infamous \emph{sign problem}, which poses severe limitation to, for example, \ac{mc} simulations in the finite $\mu$ region.
