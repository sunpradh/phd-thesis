%----------------------------------------
% SECTION: Wilson approach to gauge theories
%----------------------------------------
\section{Wilson approach to \acl{lgt}}
\label{sec:wilson_approach_to_lft}

Starting from the path integral formulation, the first step in the formulation of a \ac{lgt} is the discretization of space-time, where a discrete $d+1$-dimensional lattice substitutes the continuum space-time.
The simplest choice in this regard is a hypercubic lattice with lattice spacing $a$, but in theory an \ac{lgt} can be defined on any type of lattice.
An immediate advantage of using a lattice instead of a continuum is the natural ultraviolet cutoff given by the inverse of the lattice spacing.

Formally, a lattice $\Lattice$ is defined as
\begin{equation}
    \Lattice = \qty{
        x \in \R^{D} :
        x = \sum_{\mu = 1}^{D} a n_{\mu} \hat{\mu} \quad
        n_{\mu} \in \Z
    },
\end{equation}
where $\mu = 1, \dots, D$, and $\hat{\mu}$ is the unit vector in the $\hat{\mu}$-th direction.
The edges, or \emph{links}, will be labeled by a pair $(x,\hat{\mu})$, meaning that we are referring to the link in the $\hat{\mu}$ direction from the vertex, or \emph{site}, $x$.
It is important to fix an orientation for each direction in the lattice.
The most natural choice is to choose $+ \hat{\mu}$ for each $\hat{\mu}$.
So, even though $(x,\hat{\mu})$ and $(x + \hat{\mu}, - \hat{\mu})$ refers to the same link, the former is traversed in the positive direction while the latter in the negative direction.

In an \ac{lgt}, both the sites and links host \ac{dof}.
In particular, the matter fields lives on the sites while the gauge fields live on the links.
The definition of these \ac{dof} will need some care, because we have two main requirements, especially if we are interested in \ac{ym} theory:
\begin{itemize}
    \item The lattice action should reduce to the continuum action in the continuum limit, i.e., $a \to 0$;
    \item The lattice action should respect the gauge symmetry.
\end{itemize}
Lorentz invariance is naturally broken on a lattice but we expect to recover it in the continuum limit.

\begin{figure}[t]
    \centering
    \begin{tikzpicture}[
    lattice/.style={Grey70},
    site/.style={circle, inner sep=0pt, minimum size=4pt, draw=black, fill=white}
    ]
    \begin{scope}[scale=1.25]
        \draw[lattice] (-1.5, 0, 0) -- (0, 0, 0);
        \draw[lattice] (0, -1.5, 0) -- (0, 0, 0);
        \draw[lattice] (0, 0, -1.5) -- (0, 0, 0);
        \draw[lattice] (1.5, 0, 0) -- (0, 0, 0);
        \draw[lattice] (0, 1.5, 0) -- (0, 0, 0);

        \draw[axis] (0, 0, 0) -- (1,  0, 0) node [above] {$\hat{1}$};
        \draw[axis] (0, 0, 0) -- (-1, 0, 0) node [above] {$- \hat{1}$};
        \draw[axis] (0, 0, 0) -- (0,  1, 0) node [left] {$\hat{2}$};
        \draw[axis] (0, 0, 0) -- (0, -1, 0) node [right] {$-\hat{2}$};
        \draw[axis] (0, 0, 0) -- (0,  0, -1) node [above] {$\hat{3}$};

        \node[site] at (0, 0) {};
        \node[below right] at (0, 0) {$x$};

        \draw[lattice] (0, 0, 2) -- (0, 0, 0);
        \draw[axis] (0, 0, 0) -- (0,  0, 1) node [below] {$-\hat{3}$};
    \end{scope}

    \begin{scope}[xshift=5cm, yshift=-1cm, scale=1.25]
        \foreach \z in {0, ..., 2} {
            \foreach \y in {0, ..., 2} {
                \foreach \x in {0, ..., 2}
                \node[site] (site \x \y \z) at (\x, \y, \z) {};

                \foreach \xone in {0, 1} {
                    \pgfmathtruncatemacro{\xtwo}{\xone + 1}
                    \path (site \xone \y \z) edge[lattice] (site \xtwo \y \z);
                }
            }

            \foreach \yone in {0, 1} {
                \pgfmathtruncatemacro{\ytwo}{\yone + 1}
                \foreach \x in {0, ..., 2}
                    \path (site \x \yone \z) edge[lattice] (site \x \ytwo \z);

            }
        }

        \foreach \zone in {1, 2} {
            \pgfmathtruncatemacro{\ztwo}{\zone - 1}
            \foreach \y in {0, ..., 2}
                \foreach \x in {0, ..., 2}
                    \path (site \x \y \zone) edge[lattice] (site \x \y \ztwo.center);
        }

        \draw[decorate, decoration=brace] (1, -0.2, 2) -- (0, -0.2, 2) node [midway, below=3pt] {$a$};
    \end{scope}
\end{tikzpicture}


    \caption[three-dimensional lattice]{
        \emph{Left}. Unit vectors $\hat{\mu}$ in three-dimensional lattice.
        \emph{Right}. Example of a three-dimensional lattice.
    }
\end{figure}

%
% SUBSECTION: Gauge fields on a lattice
%
\subsection{Gauge fields on a lattice}
\label{sub:gauge_fields_on_a_lattice}

The simplest way to define a \ac{ym} action on a lattice would be to consider a continuum action, substitute finite-difference approximations for derivatives, and replace the space-time integral by a sum over the lattice sites.
However, the result of this is an action which is not-gauge invariant for non-zero lattice spacing \cite{wilson1974confinement}.
This is likely to mean that the theory would still lack gauge-invariance in the $a \to 0$ limit.
The alternative, outlined in \cite{wilson1974confinement, creutz1985book}, would be to formulate gauge invariance for a lattice theory then modify its action until it is gauge invariant for any $a$.

We start by considering a general group $G$.
We associate an element $U_{\mu}(x) \in G$ to each link $(x, \mu)$.
If one traverse the link in the opposite direction, one should consider the inverse element $U^{-1}$.
In the case of $\SU(N)$, we take $U_{\mu}(x)$ to be the matrices in the fundamental representation and a vector potential can be obtained in the continuum limit by writing
\begin{equation}
    U_{\mu}(x) = e^{i a g A_{\mu}(x)}.
\end{equation}


It is necessary to discuss about gauge invariance before moving to the dynamics of these gauge fields.
% Before moving on the dynamics of these gauge fields we have talk on the meaning of gauge invariance on the lattice.
A gauge transformation is described by a group-valued function $g(x)$ (in the appropriate representation), which acts on the vertices $x$.
The variable $U_{\mu}(x)$ sits in the middle of the site $x$ and $x + \hat{\mu}$, therefore it transforms as
\begin{equation}
    U_{\mu}(x) \mapsto g(x) U_{\mu}(x) g(x + a \hat{\mu})^{\dagger}
    \label{eq:gauge_transf_field_lattice}.
\end{equation}

The action of the gauge fields $U_{\mu}(x)$ has to satisfy two requirements:
% In order to introduce a dynamics for the gauge fields $U_{\mu}(x)$, we need to define their action which need two satisfy two requirements:
it has to be \emph{gauge-invariant} and reduce to the pure gauge \ac{ym} action in the continuum limit.
From \eqref{eq:gauge_transf_field_lattice}, we can immediately deduce that taking the product of $U_{\mu}(x)$ along a closed curve will yield a gauge-invariant quantity.
The simplest close curve we can consider is a \emph{plaquette}, i.e., the smallest square face.

Consider a plaquette sitting in the $(\mu, \nu)$-plane at site $x$ (see Fig.~\ref{fig:single_plaq_wilson_loop}).
We define the \emph{single plaquette \acl{wl}} $\WilsonPlaq(x)$ as
\begin{equation}
    \WilsonPlaq(x) =
    U_{\mu}(x) U_{\nu}(x + a \hat{\mu}) U_{\mu}(x + a \hat{\nu})^{\dagger} U_{\nu}(x)^{\dagger}.
    \label{eq:single_plaquette_Wilson_loop}
\end{equation}
Notice that we do not have any sum in the indices $\mu$ and $\nu$ because they are not Lorentz indices.
% The quantity in \eqref{eq:single_plaquette_Wilson_loop} is called a single plaquette \emph{\ac{wl}}.
Only scalar quantities are allowed in the action, so we need to take the trace of $\WilsonPlaq$.
Then, our lattice action will be defined as the sum over the plaquettes of $\tr \WilsonPlaq$ (and its Hermitian conjugate):
\begin{equation}
    \Action_{\text{W}} = - \frac{1}{g^2} \sum_{x} \sum_{\mu, \nu}  \qty( \tr \WilsonPlaq(x) + \tr \WilsonPlaq^{\dagger}(x) ).
    \label{eq:wilson_action}
\end{equation}
This is known as the \emph{Wilson action} \cite{wilson1974confinement, creutz1985book}.
The quantity $\tr \WilsonPlaq$ behaves as expected in the continuum limit, where we have to work with the strength field $F^{\mu \nu}$:
\begin{equation}
    \tr \WilsonPlaq \approx N - \frac{a^{4}}{2} \tr F_{\mu \nu} F^{\mu \nu} + \mathcal{O}(a^6),
\end{equation}


\begin{figure}[t]
    \SideFigure[label=fig:single_plaq_wilson_loop, desc={Single plaquette Wilson loop}]{%
        \begin{tikzpicture}[scale=1.25]
    % background axis
    \draw[axis] (2, 0, 0) -- (3, 0, 0) node [right] {$\hat{\nu}$};
    \draw[axis] (0, 0, 0) -- (0, 1, 0) node [left] {};

    % plaquette term
    \draw[U] (0, 0, 0) -- (0, 0, 2) node [midway, left] {$U_{\mu}(x)$};
    \draw[U] (0, 0, 2) -- (2, 0, 2) node [midway, below] {$U_{\nu}(x + \hat{\mu})$};
    \draw[U] (2, 0, 2) -- (2, 0, 0) node [midway, right] {$U_{\mu}^{\dagger}(x + \hat{\nu})$};
    \draw[U] (2, 0, 0) -- (0, 0, 0) node [midway, above] {$U_{\nu}^{\dagger}(x)$};
    \fill[Blue, fill opacity=0.1] (0, 0, 0) -- (0, 0, 2) -- (2, 0, 2) -- (2, 0, 0) -- cycle;
    \node at (1, 0, 1) {$W_{\mu \nu}(x)$};

    % Sites at mu=0
    \node[site, label=above left:{$x$}] at (0, 0, 0) {};
    \node[site, label=above:{$x+ \hat{\nu}$}] at (2, 0, 0) {};

    \draw[axis] (0, 0, 2) -- (0, 0, 3) node [below] {$\hat{\mu}$};

    % Sites at mu=2
    \node[site, label=left:{$x + \hat{\mu}$}] at (0, 0, 2) {};
    \node[site] at (2, 0, 2) {};
\end{tikzpicture}

    }{%
        Single plaquette Wilson loop $W_{\mu \nu}(x)$, which is defined on the plaquette in the $(\mu, \nu)$--plane at $x$ (highlighted in blue).
        For convenience we have set $a = 1$.
    }
\end{figure}

The lattice action is not unique.
The Wilson action in \eqref{eq:wilson_action} is the simplest choice that one can make that satisfy our requirement.
Some other modifications are possible, they can include other types of closed loops or other representations.
These modification have their place, for example for improving the continuum limit or errors \cite{bhanot1982modified, edgar1982generalised, creutz1983generalized}.
However, they will not be considered here.

Obviously, in a path-integral formulation of \ac{lgt}s we need to define the path-integral in order to have a quantum theory:
\begin{equation}
    Z = \int \prod_{(x, \hat{\mu})}  \dd U_{\mu}(x) e^{- \Action_{\text{W}}}.
    \label{eq:path_intelgral_Wilson_lgt}
\end{equation}
Here we integrate over all possible values for the gauge variables.
Due to the fact that $U_{\mu}$ are elements of a group $G$ that for most physical applications is compact, the most natural choice is to the invariant group measure also known as \emph{Haar measure} \cite[Chap.~8]{creutz1985book}.
Notice that \eqref{eq:path_intelgral_Wilson_lgt} is now a well defined mathematical quantity, unlike the path-integral in continuum theory where a clear mathematical definition is still lacking.
Now that the path-integral measure has been defined, we can compute the average of an observable $\mathcal{O}$ with
\begin{equation}
    \ev*{\mathcal{O}} = \frac{1}{Z} \int \prod_{(x, \hat{\mu})} \dd U_{\mu}(x) \mathcal{O} e^{- \Action_{\text{W}}}
\end{equation}


%
% SUBSECTION: Order parameters and gauge invariance
%
\subsection{Order parameters and gauge invariance}
\label{sub:wilson_confinement_test}

The Wilson formulation of \ac{lgt}s can resemble spin models studied in statistical mechanics.
The link variables $U_{\mu}(x)$ can be thought as some sort of generalization of the spin \ac{dof}.
They are distributed in a crystal-like structure and interact with their nearest neighbours, in this case through a four-body interaction (the plaquette action), instead of two-body interaction (like the Ising model).
If one wants to pursue this analogy, then it is reasonable to look at order parameters that behaves like the spontaneous magnetization, where a non-vanishing expectation value signals a phase transition.
The analogue of such an order parameter in \ac{lgt} would be something like
\begin{equation}
    \ev*{U_{\mu}(x)} \neq 0,
    \label{eq:non_zero_link_var_expt_value}
\end{equation}
but it has been shown that this is impossible in Wilson theory \cite{elitzur1975theorem}.
Let's argue why.

In standard spin models, a non-zero magnetization represents a spontaneous breaking of the global symmetry of the system.
Consider the simplest case of the classical Ising model, where the \ac{dof} are binary variables $\sigma = \pm 1$.
Without an external field, the energy is given by the interaction of nearest neighbouring spins, i.e., $\sigma_i \sigma_j$.
This system has an obvious global $\Z_2$ symmetry, that corresponds to the inversion $\sigma_i \mapsto -\sigma_i$ off all the spins.
A ferromagnetic phase is, by definition, signaled by $\ev*{\sigma} \neq 0$, which necessarily breaks the global $\Z_2$ symmetry of the model.
Once a direction is selected by $\ev*{\sigma} \neq 0$, it remains stable under thermal fluctuations because they cannot coherently shift the magnetization of a large (or infinite) number of spins.

In a \ac{lgt}, an expectation value like \eqref{eq:non_zero_link_var_expt_value} would \emph{break gauge invariance}, which is a \emph{local symmetry}, not a global one.
As explained previously, gauge invariance means that the action is unchanged under local arbitrary ``rotations'' of the link variables $U_{\mu}(x)$ (see \eqref{eq:gauge_transf_field_lattice}).
Hence, thermal fluctuations will induce such rotations and in the long rung it will average on all the possible gauges, which leads to
\begin{equation}
    \ev*{U_{\mu}(x)} = \int \dd U_{\mu}(x) U_{\mu}(x) = 0
\end{equation}
if $U_{\mu}$ contains only non-trivial irreducible representations of the group (see Th.~\ref{th:orthogonality_compact_groups} in App.~\ref{app:some_results_in_representation_theory}).
This means that ``magnetization'' is always vanishing in a \ac{lgt} and gauge invariance cannot be spontaneously broken, which is the contents of the Elitzur theorem \cite{elitzur1975theorem}.

The conclusion of this brief discussion may seem rather grim, as magnetization in spin models is the most convenient and used order parameter.
But this does not mean that there are no other order parameters in a \ac{lgt}.
We have just showed that the problem when considering something like $\ev*{U_{\mu}}$ is gauge invariance.
So, the most reasonable step forward is to consider \emph{gauge-invariant quantities} as order parameters.
We have already seen that tracing over a product of $U_{\mu}$ variables along a closed curve is a gauge-invariant quantity, called \emph{\acf{wl}}.

\begin{figure}[t]
    \begin{minipage}[t]{0.48\textwidth}
        \centering
        \parbox[t][6cm][c]{\textwidth}{\centering\begin{tikzpicture}[
    x = {(1cm, 0cm)},
    y = {(-3.85mm, -3.85mm)},
    z = {(0cm, 1cm)},
    space/.style = {Green, ultra thick, line join=round},
    time/.style = {Orange, ultra thick, line join=round}
    ]
    \draw[axis] (0, 0, 0) -- (4, 0, 0) node[right] {$x$};
    \draw[axis] (0, 0, 0) -- (0, 4, 0) node[below] {$y$};
    \draw[axis] (0, 0, 0) -- (0, 0, 3) node[above] {$t$};

    % Space-like Wilson loop
    \draw[space, ->-=0.15, ->-=0.65] (1, 1, 0) -- (3, 1, 0) -- (3, 3, 0) -- (1, 3, 0) -- cycle;
    \fill[black, opacity=0.05] (1, 1, 0) -- (3, 1, 0) -- (3, 3, 0) -- (1, 3, 0) -- cycle;
    \node[below right] at (3, 3, 0) {space-like};

    % Time-like Wilson loop
    \draw[time, ->-=0.15, ->-=0.65] (0, 1, 1) -- (2, 1, 1) -- (2, 1, 3) -- (0, 1, 3) -- cycle;
    \fill[black, opacity=0.05] (0, 1, 1) -- (2, 1, 1) -- (2, 1, 3) -- (0, 1, 3) -- cycle;
    \node[right] at (2, 1, 3) {time-like};
\end{tikzpicture}

}
        \caption{Space-like and time-like Wilson loop.}
    \end{minipage}
    \hfill
    \begin{minipage}[t]{0.48\textwidth}
        \parbox[t][6cm][c]{\textwidth}{\centering\begin{tikzpicture}[
    point/.style = {circle, minimum size=1.5mm, inner sep=0pt, fill=Orange, draw=Orange},
    ]
    \node (start) at (3, 0) {};
    \node (end) at (0, 3) {};

    \draw[very thick, Orange, ->-=0.25, ->-=0.75] (start.center)
        .. controls (0, 0) and (1, 1) .. (end.center)
        node [pos=0.5, below left, black] {$+q$}
        .. controls (2, 3.5)  and (3.5, 1) .. (start.center)
        node [pos=0.5, right, black] {$-q$}
    ;
    \fill[black, fill opacity=0.05] (start.center)
        .. controls (0, 0) and (1, 1) .. (end.center)
        .. controls (2, 3.5)  and (3.5, 1) .. (start.center);

    \node[point] at (start) {};
    \node[point] at (end) {};
    \draw[axis] (4, 1) -- (4, 2) node [above] {$t$};
\end{tikzpicture}
}
        \caption[Quark-antiquark pair]{A time-like Wilson loop can be seen as a process where a quark-antiquark pair is created and then annihilated.}
    \end{minipage}
\end{figure}

In so far, we have considered only single plaquette loops but nothing restraints us from considering arbitrary large loops, indeed it serves as a \emph{confinement test} for pure gauge theories.
% One of the most important quantity that can be computed for a pure \ac{lgt} is the \ac{wl} for generic closed paths $\mathcal{C}$.
% It serves as \emph{confinement} test.
It has been shown \cite{wilson1974confinement} that confinement is equivalent to the \emph{area law} behaviour of \ac{wl}s, i.e.,
\begin{equation}
    \ev*{W(\mathcal{C})} \sim \exp(- \sigma A(\mathcal{C})),
    \label{eq:wilson_area_law}
\end{equation}
where $A(\mathcal{C})$ is the minimal area inside the closed path $\mathcal{C}$ and $\sigma$ the \emph{string tension} (the coefficient of the linear potential between two quarks).
On the other hand, in the absence of confinement one finds instead the \emph{perimeter law}:
\begin{equation}
    \ev*{W(\mathcal{C})} \sim \exp(- k P(\mathcal{C})),
    \label{eq:wilson_perimeter_law}
\end{equation}
where $P(\mathcal{C})$ is the perimeter of the curve $\mathcal{C}$ and $k$ just some constant.


The reason behind this behaviour can be seen with a simple qualitative picture \cite{creutz1985book, wilson1974confinement}.
A closed timelike \ac{wl} basically represents a process in which a quark-antiquark pair is produced, moved along the sides of the loop and annihilated.
If we are in a confining phase we can then expect a liner potential between the quark and antiquark.
We can imagine a flux tube \emph{binding} the two charges, which swoops the whole inside area of the loop.
Then, it is easy to image that the energy of this whole process will necessarily depend on the area of the loop.
On the other hand, if we are in a deconfined phase then there is no potential binding the two quarks.
In this case the energy of the whole process depends only on the self-energy of quarks, which move along the sides of the loop.
Therefore, the leading energy contribution of this process depends on the perimeter, instead of the area.
Obviously, this picture is no longer valid when dynamical matter is involved, because in a confining phase pair production is always preferred when separating two quarks at large distances.
% In a confining phase, pair production is always preferred when separating two quarks at large distances.

From \eqref{eq:wilson_area_law} and \eqref{eq:wilson_perimeter_law}, we can deduce that the string tension $\sigma$ can be used as an order parameter.
It is non-zero for a confining phase, while it vanishes for a deconfined phase.
But it is \emph{non-local} in nature, as it involves the asymptotic behaviour of potential, and therefore of the correlation functions of the theory.

\begin{figure}[t]
    \SideFigure[label=fig:quark_binding, desc={Quark binding mechanism}]{
        \begin{tikzpicture}[
    scale=0.5,
    quark/.style = {circle, draw, thick, minimum size=0.3cm, inner sep=0pt, fill=Yellow},
    field/.style = {->-=0.5, thick}
    ]
    \node[quark, label=left:{$+q$}] (q1) at (0, 0) {};
    \node[quark, label=right:{$-q$}] (q2) at (7, 0) {};

    \path (q1) edge[field] (q2);
    \draw[field] (q1) .. controls (1, 1)   and (6, 1)  .. (q2);
    \draw[field] (q1) .. controls (1, -1)  and (6, -1) .. (q2);
    \draw[field] (q1) .. controls (-2, 2)  and (9, 2)  .. (q2);
    \draw[field] (q1) .. controls (-2, -2) and (9, -2) .. (q2);
\end{tikzpicture}

    }{
        Example of a flux tube binding two charges, which gives rise to the quark binding mechanics
    }
\end{figure}



\subsection{Fermions on a lattice}
\label{sub:fermions_on_a_lattice}

The fermionic degrees of freedom lives on the sites of the lattice.
So, for each site $x$ of the lattice $\Lattice$, we have a spinor variable $\psi_x$.
Then, in order to build the lattice action, the derivative $\partial_{\mu} \psi(x)$ has to be substituted with a symmetric finite difference
\begin{equation*}
    \partial_{\mu} \psi(x)
    \quad \longrightarrow \quad
    \frac{1}{2 a} \qty( \psi_{x - \hat{\mu}} - \psi_{x + \hat{\mu}}).
\end{equation*}
One can suppose that this substitution, accompanied with the usual substitution of integrals with sums over the lattice will yield the correct lattice action for fermions.
Unfortunately, this is not the whole story.
Defining lattice fermions is not this straightforward due to the known \emph{doubling problem} \cite{nielsen1981fermions, susskind1977fermions}.
In simple terms, when introducing fermions on a lattice, instead of a continuous space, it leads to an extra number of spurious fermions, which are just lattice artifacts.

In order to get an insight into the fermion doubling issue, consider the correlation function for a single fermionic species.
If $K$ is the kinetic matrix for the fermions, then $G = K^{-1}$ gives their correlation matrix.
One finds that the correlation function between two sites $x$ and $y$ has the form \cite{creutz1985book}
\begin{equation}
    (G)_{x,y} = \frac{1}{a^d L^d} \sum_{k} \tilde{G}_{k} e^{2 \pi i k \cdot (x - y) / L},
    \label{eq:fermionic_corr_func_real_space}
\end{equation}
where $a$ is the lattice spacing, $L^d$ the total volume and $\tilde{G}_k$ the correlation function in momentum space:
\begin{equation}
    \tilde{G}^{-1}_k = m + \frac{i}{a} \sum_{\mu} \gamma_{\mu} \sin(2 \pi k_{\mu} / L).
    \label{eq:fermionic_corr_func_latt}
\end{equation}
It involves a simple trigonometric function because the derivative term involves nearest neighbouring sites.
One can then take the model to a large lattice, which justifies in substituting the discrete sums with integrals:
\begin{equation}
    \frac{2 \pi k_{\mu}}{La}  \; \to \; q_{\mu}
    \qquad \text{and} \qquad
    \frac{1}{a^d L^d} \sum_{k} \; \to \; \int \frac{\dd q^d}{(2 \pi)^d},
    \label{eq:limit_large_lattice}
\end{equation}
where the $q_{\mu}$'s are continuous momentum variables.
This substitution maps \eqref{eq:fermionic_corr_func_latt} into
\begin{equation}
    \tilde{G}_k^{-1} = m + \frac{i}{a} \sum_{\mu} \gamma_{\mu} \sin(a q_{\mu}).
    \label{eq:fermionic_corr_func_large_latt}
\end{equation}

One can naively think of taking the limit $a \to 0$ and expand $\sin(a q_{\mu})$ around the zero and obtain something that look like the correct continuum limit:
\begin{equation}
    \tilde{G}_{\mu}^{-1} = m + i \slashed{q} + \mathcal{O}(a^2).
\end{equation}
But one should not be fooled by this sloppy procedure just because it appears to give the wanted result.
Each component $q_{\mu}$ takes values in the region $[-\pi/a, +\pi/a]$, hence we have to integrate on the whole volume $[-\pi/a, +\pi/a]^d$.
Looking at \eqref{eq:fermionic_corr_func_large_latt}, it is clear that the major contributions to $G$ in \eqref{eq:fermionic_corr_func_real_space} comes from the zeros of $\tilde{G}_k^{-1}$.
This, not only vanishes in the region $q_{\mu} \sim 0$ but also for large momentum $q_{\mu} \sim \pi/a$.
The propagator has no suppression of momentum values near $\pi/a$.
We can isolate the large momenta region by considering
\begin{equation}
    \tilde{q}_{\mu} = q_{\mu} - \pi/a
\end{equation}
for each direction in space.
In this way, we de facto half the integration region,
\begin{equation}
    \int_{-\pi/a}^{\pi/a} \dd q_{\mu} \;\to\;
    \int_{-\pi/2a}^{\pi/2a} (\dd q_{\mu} + \dd \tilde{q}_{\mu})
\end{equation}
and now the limit $a \to 0$ can be taken safely, but it comes with a price to pay.
For each direction in space, we have two independent regions that gives a free fermion contribution to the propagator in the continuum limit.
We have effectively \emph{doubled} the number of fermions for each direction.
In a $d$-dimensional lattice we end up with $2^d$ independent fermions, even though we initially started with just one.

There are many solutions to this fermion doubling problem \cite{susskind1977fermions, tong2018gauge}, but we will focus only on one in this manuscript: \emph{the staggered fermions} \cite{kogut1975hamiltonian, susskind1977fermions}.
We have seen that these fictitious fermions come from the large momenta regions, where $q_{\mu} \sim \pi/a$.
Brutally cutting out this large momenta regions spoils the completeness of the Fourier transform, so it is not a solution, but a smarter solution can give out the same effect.
The idea is to spread the fermionic \ac{dof} over multiple lattice sites, reducing effectively the momenta space.
For example, in two dimensions it would correspond to placing the \emph{particles} on \emph{even} sites and the \emph{antiparticles} on \emph{odd} sites.
A site is considered even or odd when $(-1)^x = (-1)^{x_1 + \dots + x_d} = +1$ or $-1$.

To obtain a staggered fermion, we define a new fermionic species $\chi(x)$ such that
\begin{equation}
    \psi(x) = \prod_{\mu} (\gamma^{\mu})^{n_{\mu}} \chi(x),
\end{equation}
where $x_{\mu} = an_{\mu}$.
Now, if we want to express the discretized covariant derivative, the term $\gamma^{\mu} \psi(x+a n_{\mu})$ have two extra powers of $\gamma^{\mu}$ compared to $\overline{\psi}$.
Since $(\gamma^{\mu})^2 = \pm 1$, we have therefore
\begin{equation}
    \overline{\psi}(x) \gamma^{\mu} \psi(x) = (-1)^{\eta_{\mu}(x)} \chi(x)^{\dagger} \chi(x + a n_{\mu}),
\end{equation}
where $\eta_{\mu}(x)$ is some sign function depending on the site $x$.
This function can be obtained from the commutation relations of the gamma matrices.
In particular, in two dimensions we have
\begin{equation}
    \eta_1(x) = 1
    \quad \text{and} \quad
    \eta_2(x) = (-1)^{n_1},
\end{equation}
while in four (Euclidean) dimension we have instead \cite{tong2018gauge}
\begin{equation}
    \eta_1(x) = 1, \quad
    \eta_2(x) = (-1)^{n_1}, \quad
    \eta_3(x) = (-1)^{n_1 + n_2}, \quad
    \eta_4(x) = (-1)^{n_1 + n_2 + n_3}.
\end{equation}
A similar reasoning can be applied to the mass term $m \overline{\psi}(x) \psi(c)$, where it becomes
\begin{equation}
    m \overline{\psi}(x) \psi(x) = (-1)^{\eta(x)} \chi(x)^{\dagger} \chi(x),
\end{equation}
for some sign function $\eta(x)$ that can be obtained from the commutation relations of the gamma functions.
