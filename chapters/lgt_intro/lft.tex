%----------------------------------------
% SECTION: Quantum simulation
%----------------------------------------
\section{Lattice Field Theory}
\label{sec:lattice_field_theory}

Starting from the path integral formulation, the first step in the formulation of a \emph{lattice field theory} (LFT) is the discretization of space-time, where a discrete $d+1$-dimensional lattice substitutes the continuum space-time.
The simplest choice in this regard is a hypercubic lattice with lattice spacing $a$, but in theory an LFT can be defined on any type of lattice.
An immediate advantage of using a lattice instead of a continuum is the natural ultraviolet cutoff given by the inverse of the lattice spacing.

Formally a lattice $\Lambda$ is defined as
\begin{equation}
    \Gamma = \qty{
        x \in \R^4 :
        x = \sum_{\mu = 1}^{d+1} a n_{\mu} \hat{\mu} \quad
        n_{\mu} \in \Z
    },
\end{equation}
where $\mu = 1, \dots, d+1$ and $\hat{\mu}$ is the unit vector in the direction $\mu$.
The edges will be labeled by a pair $(x,\hat{\mu})$, meaning that we are referring to the edge in the $\hat{\mu}$ direction from the vertex $x$.
It is important to fix an orientation for each direction in the lattice.
The most natural choice is to choose $+ \hat{\mu}$ for each $\hat{\mu}$.
So, even though $(x,\hat{\mu})$ and $(x + \hat{\mu}, - \hat{\mu})$ refers to the same link, the former is traversed in the positive direction while the latter in the negative direction.

In an LFT, both the vertices and edges (also called links) host degrees of freedom (d.o.f).
In particular, the matter fields lives on the vertices while the gauge fields live on the links between vertices.
However, the definition of these d.o.f.~will need some care, because we have two main requirements, especially if we are interested in Yang-Mills theory:
\begin{itemize}
    \item The lattice action should reduce to the continuum action in the continuum limit, i.e., $a \to 0$;
    \item The lattice action should respect the gauge symmetry.
\end{itemize}
Lorentz invariance is naturally broken on a lattice but we expect to recover it in the continuum limit.


%
% SUBSECTION: Gauge fields on a lattice
%
\subsection{Gauge fields on a lattice}
\label{sub:gauge_fields_on_a_lattice}

\todo{spiegazione perché usiamo il gruppo di Lie e non l'algebra di Lie}
Considering a general group $G$, we associate an element $U_{\mu}(x) \in G$ to each link $(x, \mu)$.
If one traverse the link in the opposite direction, one should obtain the inverse element $U^{-1}$.
In the case of $SU(N)$, we take $U_{\mu}(x)$ to be the matrices in the fundamental representation.
We can obtain a vector potential in the continuum limit by writing
\begin{equation}
    U_{\mu}(x) = e^{i a A_{\mu}(x)},
\end{equation}
where $a$ is the lattice spacing.

It is necessary to discuss about gauge invariance before moving to the dynamics of these gauge fields.
% Before moving on the dynamics of these gauge fields we have talk on the meaning of gauge invariance on the lattice.
A gauge transformation is described by a group-valued function $g(x)$ (in the appropriate representation), which acts on the vertices $x$.
The variable $U_{\mu}(x)$ sits in the middle of the site $x$ and $x + \hat{\mu}$, then it is reasonable to think its transformation is
\begin{equation}
    U_{\mu}(x) \mapsto g(x) U_{\mu}(x) g(x + a \hat{\mu})^{\dagger}
    \label{eq:gauge_transf_field_lattice}.
\end{equation}

In order to introduce a dynamics for the gauge fields $U_{\mu}(x)$, we need to define their action which need two satisfy two requirements:
it has to be gauge-invariant and reduce to the pure gauge Yang-Mills action in the continuum limit.
From \eqref{eq:gauge_transf_field_lattice}, we can immediately deduce that taking the product of $U_{\mu}(x)$ along a closed curve will yield a gauge-invariant quantity.
The simplest close curve we can consider is a \emph{plaquette}, i.e., the smallest square face.
Hence, on a plaquette $\square$ we introduce $\W$ to put in the action, defined as
\begin{equation}
    \W =
    U_{\mu}(x) U_{\nu}(x + a \hat{\mu}) U_{\mu}(x + a \hat{\nu})^{\dagger} U_{\nu}(x)^{\dagger}.
    \label{eq:single_plaquette_Wilson_loop}
\end{equation}
Notice that we do not have any sum in the indices $\mu$ and $\nu$ because they are not Lorentz indices.
The quantity in \eqref{eq:single_plaquette_Wilson_loop} is called a single plaquette \emph{Wilson loop}.

We can only have scalar quantities in the action, so we need to take the trace of $\W$.
Then, our lattice action will be defined as the sum over the plaquettes of $\tr \W$ (and its Hermitian conjugate):
\begin{equation}
    \action = - \frac{1}{g^2} \sum_{\square} \qty( \tr \W + \tr \W^{\dagger} ).
    \label{eq:wilson_action}
\end{equation}
This is known as the \emph{Wilson action} \citneeded.
The quantity $\tr \W$ behaves as expected in the continuum limit, where we have to work with the strength field $F^{\mu \nu}$:
\begin{equation}
    \tr \W \approx N - \frac{a^{4}}{2} \tr F_{\mu \nu} F^{\mu \nu} + \mathcal{O}(a^6),
\end{equation}

The lattice action is not unique.
The Wilson action in \eqref{eq:wilson_action} is the simplest choice that one can make that satisfy our requirement.
Some other modification, for example, can include other types of closed loops and these modification can have their place.
However, they will not be considered here.

Obviously, in a path-integral formulation of lattice gauge theories we need to define the path integral, which is immediate.
The partition function is given by
\begin{equation}
    Z = \int \prod_{(x, \hat{\mu})}  \dd U_{\mu}(x) e^{- \action},
\end{equation}
where the integration measure $\dd U_{\mu}(x)$ is understood to be the Haar measure.
In case of a compact group, like $SU(N)$, it is well defined and yields a finite value.
Now that the path integral measure has been defined, the average of an observable $\mathcal{O}$ can be computed as
\begin{equation}
    \ev*{\mathcal{O}} = \frac{1}{Z} \int \prod_{(x, \hat{\mu})} \dd U_{\mu}(x) \mathcal{O} e^{- \action}
\end{equation}


%
% SUBSECTION: Order parameters and gauge invariance
%
\subsection{Order parameters and gauge invariance}
\label{sub:wilson_confinement_test}

The Wilson formulation of lattice gauge theories can resemble spin models studied in statistical mechanics.
The link variables $U_{\mu}(x)$ can be thought as some sort of generalization of the spin degrees of freedom.
They are distributed in a crystal-like structure and interact with their nearest neighbours, in this case through a four-body interaction (the plaquette action), instead of two-body interaction (like the Ising model).
If one wants to pursue this analogy, then it is reasonable to look at order parameters that behaves like the spontaneous magnetization, where a non-vanishing expectation value signals a phase transition.
The analogue of such an order parameter in lattice gauge theory would be something like
\begin{equation}
    \ev*{U_{\mu}(x)} \neq 0,
    \label{eq:non_zero_link_var_expt_value}
\end{equation}
rut it has been shown\citneeded \todo{teorema di Elitzur} that this is impossible in Wilson theory.

In standard spin models, a non-zero magnetization represents a spontaneous breaking of the global symmetry of the system.
Consider the simplest case of the classical Ising model, where the degrees of freedom are binary variables $\sigma = \pm 1$.
Without an external field, the energy is given by the interaction of nearest neighbouring spins, i.e., $\sigma_i \sigma_j$.
This system has an obvious global $\Z_2$ symmetry, that corresponds to the inversion $\sigma_i \mapsto -\sigma_i$ off all the spins.
A ferromagnetic phase is, by definition, signaled by $\ev*{\sigma} \neq 0$, which necessarily breaks the global $\Z_2$ symmetry of the model.
Once a direction is selected by $\ev*{\sigma} \neq 0$, it remains stable under thermal fluctuations because they cannot coherently shift of the magnetization of an infinite (or rather large) number of spins.

In a lattice gauge theory, an expectation value like \eqref{eq:non_zero_link_var_expt_value} would \emph{breaks} gauge invariance, which is a \emph{local symmetry}, not a global one.
As explained previously, gauge invariance means that the action is unchanged under local arbitrary ``rotations'' of the link variables $U_{\mu}(x)$, see \eqref{eq:gauge_transf_field_lattice}.
Hence, thermal fluctuations will induce such rotations and in the long rung it will average on all the possible gauges.
This leads to
\begin{equation}
    \ev*{U_{\mu}} = \int \dd U_{\mu} U_{\mu} = 0
\end{equation}
if $U_{\mu}$ contains only non-trivial irreducible representations of the group.
This means that ``magnetization'' is always vanishing in a lattice gauge theory and gauge invariance cannot be spontaneously broken, which is the contents of the Elitzur theorem\citneeded.

The conclusion of this brief discussion may seem rather grim, as magnetization in spin models is the most convenient and used order parameter.
But this does not mean that there are no other order parameters in a lattice gauge theory.
We have just showed that the problem when considering something like $\ev*{U_{\mu}}$ is gauge invariance.
So, the most reasonable step forward is to consider \emph{gauge-invariant quantities}.
We have already seen that tracing over a product of $U_{\mu}$ variables along a closed curve is a gauge-invariant quantity, called \emph{Wilson loop}.

In so far, we have considered only single plaquette loops but nothing restraints us from considering arbitrary large loops, indeed it serves as a \emph{confinement test} for pure gauge theories.
% One of the most important quantity that can be computed for a pure lattice gauge theory is the Wilson loop for generic closed paths $\mathcal{C}$.
% It serves as \emph{confinement} test.
It has been shown \citneeded that confinement is equivalent to the \emph{area law} behaviour of Wilson loops, i.e.,
\begin{equation}
    \ev*{W(\mathcal{C})} \sim \exp(- \sigma A(\mathcal{C})),
    \label{eq:wilson_area_law}
\end{equation}
where $A(\mathcal{C})$ is the minimal area inside the closed path $\mathcal{C}$ and $\sigma$ the \emph{string tension} (the coefficient of the linear potential between two quarks).
On the other hand, in the absence of confinement one finds instead the \emph{perimeter law}
\begin{equation}
    \ev*{W(\mathcal{C})} \sim \exp(- k P(\mathcal{C})),
    \label{eq:wilson_perimeter_law}
\end{equation}
where $P(\mathcal{C})$ is the perimeter of the curve $\mathcal{C}$ and $k$ just some constant.

The reason behind this behaviour can be seen with a simple qualitative picture.
A closed timelike Wilson loop basically represents a process in which a quark-antiquark pair is produced, moved along the sides of the loop and annihilated.
If we are confining phase we can then expect a liner potential between the quark and antiquark.
We can imagine a flux tube \emph{binding} the two charges, which swoops the whole inside the loop.
Then, it is easy to image that the energy of this whole process will necessarily depend on the area of the loop.
On the other hand, if we are in a deconfined phase then there is no potential binding the two quarks.
In this case the energy of the whole process depends only on the self-energy of quarks, which move along the sides of the loop.
Therefore, the leading energy contribution of this process depends on the perimeter, instead of the area.
Obviously, this picture is no longer valid when dynamical matter is involved.
In a confining phase, pair production is always preferred when separating two quarks at large distances.
\todo{inserire immagine}

From \eqref{eq:wilson_area_law} and \eqref{eq:wilson_perimeter_law}, we can deduce that the string tension $\sigma$ can be used as an order parameter.
It is non-zero for a confining phase, while it vanishes for a deconfined phase.
But it is \emph{non-local} in nature, as it involves the asymptotic behaviour of potential, and therefore of the correlation functions of the theory.



\subsection{Fermions on a lattice}
\label{sub:fermions_on_a_lattice}

Defining fermions is not an easy task due to the known \emph{doubling problem}.
In simple terms, when introducing fermions on a lattice, instead of a continuous space, it leads to a extra spurious fermions, which are just lattice artifacts.

In order to briefly see this, consider the correlation function for a single fermionic species.
If $K$ is the kinetic matrix for the fermions, then $G = K^{-1}$ gives their correlation matrix.
One finds \citneeded that the correlation function between two sites $x$ and $y$ has the form
\begin{equation}
    (G)_{x,y} = \frac{1}{a^d L^d} \sum_{k} \tilde{G}_{k} e^{2 \pi i k \cdot (x - y) / L},
    \label{eq:fermionic_corr_func_real_space}
\end{equation}
where $a$ is the lattice spacing, $L^d$ the total volume and $\tilde{G}_k$ the correlation function in momentum space:
\begin{equation}
    \tilde{G}^{-1}_k = m + \frac{i}{a} \sum_{\mu} \gamma_{\mu} \sin(2 \pi k_{\mu} / L).
    \label{eq:fermionic_corr_func_latt}
\end{equation}
It involves a trigonometric function because the derivative term involves nearest neighbouring sites.
One can then take the model to a large lattice, which justifies in substituting the discrete sums with integrals:
\begin{equation}
    \frac{2 \pi k_{\mu}}{La}  \; \to \; q_{\mu}
    \qquad \text{and} \qquad
    \frac{1}{a^d L^d} \sum_{k} \; \to \; \int \frac{\dd q^d}{(2 \pi)^d},
    \label{eq:limit_large_lattice}
\end{equation}
where the $q_{\mu}$'s are continuous momentum variables.
This substitution maps \eqref{eq:fermionic_corr_func_latt} into
\begin{equation}
    \tilde{G}_k^{-1} = m + \frac{i}{a} \sum_{\mu} \gamma_{\mu} \sin(a q_{\mu}).
    \label{eq:fermionic_corr_func_large_latt}
\end{equation}

One can naively think of taking the limit $a \to 0$ and expand $\sin(a q_{\mu})$ around the zero and obtain something that look like the correct continuum limit:
\begin{equation}
    \tilde{G}_{\mu}^{-1} = m + i \slashed{q} + \mathcal{O}(a^2).
\end{equation}
But one should not be fooled by this sloppy procedure just because it appears to give the wanted result.
Each component $q_{\mu}$ takes values in the region $[-\pi/a, +\pi/a]$, hence we have to integrate on the whole volume $[-\pi/a, +\pi/a]^d$.
Looking at \eqref{eq:fermionic_corr_func_large_latt}, it is clear that the major contributions to $G$ in \eqref{eq:fermionic_corr_func_real_space} comes from the zeros of $\tilde{G}_k^{-1}$.
This, not only vanishes in the region $q_{\mu} \sim 0$ but also for large momentum $q_{\mu} \sim \pi/a$.
The propagator has no suppression of momentum values near $\pi/a$.
We can isolate the large momenta region by considering
\begin{equation}
    \tilde{q}_{\mu} = q_{\mu} - \pi/a
\end{equation}
for each direction in space.
In this way, we de facto half the integration region,
\begin{equation}
    \int_{-\pi/a}^{\pi/a} \dd q_{\mu} \;\to\;
    \int_{-\pi/2a}^{\pi/2a} (\dd q_{\mu} + \dd \tilde{q}_{\mu})
\end{equation}
and now the limit $a \to 0$ can be taken safely, but it comes with a price to pay.
For each direction in space, we have two independent regions that gives a free fermion contribution to the propagator in the continuum limit.
We have effectively \emph{doubled} the number of fermions for each direction.
In a $d$-dimensional lattice we end up with $2^d$ independent fermions, even though we initially started with just one.
\todo{qualcosa a che fare con la chiralità}

There are many solutions to this fermion doubling problem\citneeded, but we will focus only on one in this manuscript: \emph{the staggered fermions}\citneeded.
We have seen that these fictitious fermions come from the large momenta regions, where $q_{\mu} \sim \pi/a$.
Brutally cutting out this large momenta regions spoils the completeness of the Fourier transform, so it is not a solution, but a smarter solution can give out the same effect.
The idea is to spread the fermionic degrees of freedom over multiple lattice sites, reducing effectively the momenta space.
For example, in two dimensions it would correspond to placing the \emph{particles} on \emph{even} sites and the \emph{antiparticles} on \emph{odd} sites.
A site is considered even or odd when $(-1)^x = (-1)^{x_1 + \dots + x_d} = +1$ or $-1$.

To obtain a staggered fermion, we define a new fermionic species $\chi(x)$ such that
\begin{equation}
    \psi(x) = \prod_{\mu} (\gamma^{\mu})^{n_{\mu}} \chi(x),
\end{equation}
where $x_{\mu} = an_{\mu}$.
Now, if we want to express the discredited covariant derivative, the term $\gamma^{\mu} \psi(x+a n_{\mu})$ have two extra powers of $\gamma^{\mu}$ compared to $\overline{\psi}$.
Since $(\gamma^{\mu})^2 = \pm 1$, we have therefore
\begin{equation}
    \overline{\psi}(x) \gamma^{\mu} \psi(x) = (-1)^{\eta_{\mu}(x)} \chi(x)^{\dagger} \chi(x + a n_{\mu}),
\end{equation}
where $\eta_{\mu}(x)$ is some sign function depending on the site $x$.
This function can be obtained from the commutation relations of the gamma matrices.
In particular, in two dimensions we have
\begin{equation}
    \eta_1(x) = 1
    \quad \text{and} \quad
    \eta_2(x) = (-1)^{n_1},
\end{equation}
while in four (Euclidean) dimension we have instead \todo{Citare Tong, gauge theories}
\begin{equation}
    \eta_1(x) = 1, \quad
    \eta_2(x) = (-1)^{n_1}, \quad
    \eta_3(x) = (-1)^{n_1 + n_2}, \quad
    \eta_4(x) = (-1)^{n_1 + n_2 + n_3}.
\end{equation}
A similar reasoning can be applied to the mass term $m \overline{\psi}(x) \psi(c)$, where it becomes
\begin{equation}
    m \overline{\psi}(x) \psi(x) = (-1)^{\eta(x)} \chi(x)^{\dagger} \chi(x),
\end{equation}
for some sign function $\eta(x)$ that can be obtained from the commutation relations of the gamma functions.
