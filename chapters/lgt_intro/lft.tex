%----------------------------------------
% SECTION: Quantum simulation
%----------------------------------------
\section{Lattice Field Theory}
\label{sec:lattice_field_theory}

Starting from the path integral formulation, the first step in the formulation of a \emph{lattice field theory} (LFT) is the discretization of space-time, where a discrete $d+1$-dimensional lattice substitutes the continuum space-time.
The simplest choice in this regard is a hypercubic lattice with lattice spacing $a$, but in theory an LFT can be defined on any type of lattice.
An immediate advantage of using a lattice instead of a continuum is the natural ultraviolet cutoff given by the inverse of the lattice spacing.

Formally a lattice $\Lambda$ is defined as
\begin{equation}
    \Gamma = \qty{
        x \in \R^4 :
        x = \sum_{\mu = 1}^{d+1} a n_{\mu} \hat{\mu} \quad
        n_{\mu} \in \Z
    },
\end{equation}
where $\mu = 1, \dots, d+1$ and $\hat{\mu}$ is the unit vector in the direction $\mu$.
The edges will be labeled by a pair $(x,\hat{\mu})$, meaning that we are referring to the edge in the $\hat{\mu}$ direction from the vertex $x$.
It is important to fix an orientation for each direction in the lattice.
The most natural choice is to choose, obviously, $+ \hat{\mu}$ for each $\hat{\mu}$.
So, even though $(x,\hat{\mu})$ and $(x + \hat{\mu}, - \hat{\mu})$ refers to the same link, the former is traversed in the positive direction while the latter in the negative direction.

In an LFT, both the vertices and edges (also called links) host degrees of freedom (d.o.f).
In particular, the matter fields lives on the vertices while the gauge fields live on the links between vertices.
However, the definition of these d.o.f.~will need some care, because we have two main requirements:
\begin{itemize}
    \item The lattice action should reduce to the continuum action in the continuum limit, i.e., $a \to 0$;
    \item The lattice action should respect the gauge symmetry.
\end{itemize}
Lorentz invariance is naturally broken on a lattice but we expect to recover it in the continuum limit.


\subsection{Gauge fields on a lattice}
\label{sub:gauge_fields_on_a_lattice}


\subsection{Fermions on a lattice}
\label{sub:fermions_on_a_lattice}
