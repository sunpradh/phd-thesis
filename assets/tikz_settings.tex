% TikZ
%----------------------------------------
\usepackage{tikz}
\usetikzlibrary{arrows.meta, decorations.markings, patterns, fit, positioning, backgrounds, math}

\tikzset{
    font=\footnotesize,
    >={stealth},
    ->-/.style={
        decoration={markings, mark=at position #1 with {\arrow{stealth}}},
        postaction={decorate}
    },
    -<-/.style={
        decoration={markings, mark=at position #1 with {\arrow{stealth}}},
        postaction={decorate}
    },
    % Lattice
    site/.style={circle, inner sep=0pt, minimum size=4pt, draw=Grey60, fill=white},
    dual site/.style = {circle, inner sep=0 pt, minimum size=6pt, draw=black, fill=white},
    ladder/.style={Grey40, step=1, thin},
    lattice/.style={Grey40, step=2, thin},
    % Link operators
    X/.style = {Blue, ultra thick},
    Z/.style = {Red,  ultra thick},
    U/.style = {Blue, ultra thick, ->-=0.6},
    V/.style = {Red,  ultra thick, ->-=0.6},
    % Site operators
    X site/.style = {dual site, draw=Blue, fill=Blue},
    Z site/.style = {dual site, draw=Red, fill=Red},
    % electric state
    up/.style = {Green, ultra thick},
    % Up and down spins
    up spin/.style = {-stealth, ultra thick, Green},
    down spin/.style = {stealth-, ultra thick},
    % Plaquette flux
    flux/.style = {fill=Green, fill opacity=0.1},
    % Quantum gates
    gate/.style = {draw=black, fill=white, thick, minimum size=0.6cm, inner sep=0pt, font=\footnotesize},
    control/.style = {draw=black, fill=black, minimum size=3pt, inner sep=0pt, circle},
    cline/.style = {thick},
    % Misc
    freccia/.style={-stealth, very thick, shorten >=5pt, shorten <=5pt},
    box/.style={dotted, rounded corners, thick}
}

% macro for drawing spins
\newcommand{\UpSpin}[2]{\draw[up spin] (#1,#2) ++(0,-0.7) -- ++(0,1.4);}
\newcommand{\DownSpin}[2]{\draw[down spin] (#1,#2) ++(0,-0.7) -- ++(0,1.4);}
\newcommand{\Spin}[3]{%
    \ifthenelse{\equal{#1}{up}}{\UpSpin{#2}{#3}}{}%
    \ifthenelse{\equal{#1}{down}}{\DownSpin{#2}{#3}}{}
}

% macro for drawing a grid of sites
\newcommand{\DrawSites}[3][site]{
    \foreach \y in {#3} \foreach \x in {#2} \draw (\x,\y) node [#1] {};
}

% TikZ caching
\usetikzlibrary{external}
\tikzset{external/system call={lualatex -shell-escape -halt-on-error -interaction=batchmode -enable-write18 -jobname "\image" "\texsource"}}
\tikzexternalize[prefix=assets/cache/] % folder for cache
